\section{Evaluation}\label{sec:evaluation}

% \review{While I found the idea behind section V very interesting, the current version
%   of this section lacks some details that would help in better understanding (1) 
%   how the approach works, and (2) the overall scope of the approach. 
%   $\\$
%   For instance, the authors state that, following [19], they try to "exercise
%   interesting patterns" by adding "admissible but redundant typing rules" like
%   G-ASTR. There are a few points that are unclear here: (1) are these rules
%   discovered manually or automatically (starting from the Redex semantics)?, (2)
%   are there any guiding principles for coming up with rules that lead to
%   interesting cases?
%   $\\$
%   Later, the authors refer to "generation rules modified to be slightly more
%   permissive" to generate "a little" ill-typed terms. Again, are these rules
%   obtained automatically or defined manually? If the latter, did you follow any
%   methodology to derive such rules? Are these rules the same as the "admissible
%   but redundant typing rules" from above?}
% \liyi{Deena? Leo? }

\begin{itemize} 

\item provide evidence that the \systemname compiler is efficient. Compare the compiler with respect to other work, like RLBox, also the previous \checkedc compiler.

\item provide user experience of \systemname. We restrict the use of checked pointers compared to previous checked-c compiler. Is the restriction arrangable. We can say that the tainted shells are auto-matically generated, so we have a mechanism for auto-generating tainted pointers if necessary.

\item if we have space, we can re-introduce random testing a little, saying that how it helps us to develop the compiler.

\item we can then talk about the possible bugs we find in the \checkedc compiler for function pointer or the RLBox bugs.

\end{itemize}

% 
%\leo{The following is extremely
%  weak. ``Most of them'', were there any that weren't? Which ones? For
%  the ones that were, mention github issues.}  The random generator,
%equipped with the conversion tool, successfully found a few minor
%errors in the clang compiler, most of them were already issues in the
%git bug reports. For example, we discovered that while the ternary
%operator is implemented in the compiler it cannot handle complex
%bounds types in the branches. The static analysis is not sophisticated
%enough to properly detect that both branches have the same type. While
%not precisely a bug, the clang compiler does not permit memory for
%null terminated arrays to be allocated with calloc. Although calloc
%fills all spaces in memory with null, the compiler does not recognize
%this and claims that it is an unsafe cast.


% Recall that our
% formal model makes liberal use of bounds annotations in literals and
% the heap. 


% In order to get a better understanding of the formalism we wrote it in
% redex. This allowed us to make sure that expressions were well-typed
% and evaluated to what we expected. It also was helpful for use in
% prototyping; new features could first be added to the redex model to
% see how they interacted with the existing language. This model was
% slightly larger than the Coq model and there are some differences in
% the type systems. We included top level functions and conditional
% expressions. All of these extra expressions are still expressible in
% the coq model, for example functions can be represented as nested let
% expressions. In the Coq model variables are stored on a stack while in
% the Redex model the variables are simply looked up in the context. In
% general the Redex model is easier to modify and slightly closer to the
% actual Checked-C specifications. Instead of using the model for a
% static proof, we used it to increase our certainty of the accuracy of
% the model.


% \item Describe the random testing generator setup and the properties
%   to test.
% \yiyun{Deena's description of the implementation details. I tried
%   integrating the ones that I find relevant/interesting to the text above. Maybe we can
%   add more if we have some space to fill in.}
% In order for our guarantee of safety to hold, we need to know that our
% model acurately reflects the CheckedC clang compiler. Safety is proved
% for the Coq model, but it is significantly smaller than the actual
% language. The Redex model is a combination of both. It is written in
% the same style as the formalism but has slightly more of Checked-C's
% extra features. If expressions from the Redex model display the same
% behavior as equivalent programs in Checked-C then we have greater
% certainty that our model is useful. We built a random testing
% generator to increase this certainty.

  % \item Describe the bug findings from the random testing against the Checked-C compiler.
%   \leo{This is now integrated above}
% The generator was helpful in finding bugs in the redex model. Several things failed to typecheck that should have been well typed, and the generator was able to catch them. The generated code also found a few minor errors in the clang compiler, most of them were already issues in the git bug reports. For example we discovered that while the ternary operator is implemented in the compiler it cannot handle complex bounds types in the branches. The static analysis is not sophisticated enough to properly detect that both branches have the same type. While not precisely a bug, the clang compiler does not permit memory for null terminated arrays to be allocated with calloc. Although calloc fills all spaces in memory with null, the compiler does not recognize this and claims that it is an unsafe cast. In the Redex model there is no issue with this. A few other minor things were brought to light in the implementation of the generator. The main use was to increase certainty that the behavior in the formal model accurately matched the clang compiler.
%  
% % \end{itemize}


% \begin{itemize}
%  
% \item Show that why the formal semantics/type-system defined for Checked-C is useful. 
% Since we have certainty that our model reflects the clang compiler the model is very useful. Proofs are easier on the smaller model, so we can show  that certain things are true for it. Since the Redex model is between the formalism and the clang version we can have certainty that properties we expect are actually true for the clang version.
%  
% \begin{itemize}
% \item Show some bug findings. 
% \item Show the properties that we can guarantee for Checked-C based on the type-system and blame theorem.
% \item Maybe other useful tools that can be extracted from the Redex model.
%  
% \end{itemize}
%  
% \end{itemize}
