\section{\systemname Formalism}\label{sec:formal}

% \begin{itemize}
% \item Describe the types for checked-C as a graph. \dvh{I don't know what this means.}

% \item Describe the subtyping for checked-C types.
% \begin{itemize}
% \item State that the subtypes in Checked-C are transitive.
% \end{itemize}

% \item Describe the syntax of Checked-C

% \item Describe the semantics of Checked-C

% \item Describe the type system of Checked-C

% \item Describe the progress and preservation theorems, and outline the proofs.

% \item Describe the blame theorem and proof.

% \end{itemize}

% \ignore{
% \begin{figure}
%   \begin{center}
%     \includegraphics[height=6in]{syntax.pdf}
%   \end{center}
%   \caption{\lang: Syntax}
% \end{figure}

% \begin{figure}
%   \begin{center}
%     \includegraphics[height=6in]{types.pdf}
%   \end{center}
%   \caption{\lang: Typing}
% \end{figure}
% }
% \liyi{main text begins here. }

% \review{
% While reading the semantics, I found the fact that S-Def and S-DefNull are
%   applicable non-deterministically if n is 0 a bit confusing. Only when
%   reading the meta-theory section I realized that this is not a concrete issue
%   because well-formed heaps are such that $\mathcal{H}(0)$ is never defined. It
%   might be worth pointing this out early on. }
% \mwh{Done.}

\begin{figure}
  \small \centering
  \[ \hspace*{-1.2em}
\begin{array}{l}
\begin{array}{ll}
       \text{Variables:}~ x
& \text{Integers:}~n::=\mathbb{Z} 
\end{array}
\\[0.5em]
\begin{array}{llcllcl}
\text{Context Mode:} & m & ::= & \cmode \mid \umode \\
\text{Pointer Mode:} & \xi & ::= & m \mid \tmode \\
\text{Bound:} & b & ::= & n \mid x \plus n \\
              & \bvar & ::= & (b,b) \\
  
     \text{Word Type:}& \tau &::=& \tint\mid \tptr{\omega}{ \xi}
\\
\text{Type Flag:}&\kappa &::=& nt \mid \cdot
\\
\text{Type:}&\omega &::=& \tau \mid \tallarrayb{\bvar}{\tau} \mid \tfun{\overline{x}}{\overline{\tau}}{\tau}
\\
\text{Expression:}& e & ::= & 
\evalue{n}{\tau} \mid x \mid \ebinop{e}{e}\mid \ecast{\tau}{e} \mid \edyncast{\tau}{e}  \\
&&\mid& \estrlen{x} \mid \emalloc{\xi}{\omega} \mid\estar{e}\mid\eassign{e}{e}  \\
&&\mid& \elet{x}{e}{e} \mid \eif{e}{e}{e} \mid \ecall{e}{\overline{e}}
\\
&&\mid&\eunchecked{\overline{x}}{e}
\mid \echecked{\overline{x}}{e}
\end{array}
    \end{array}
  \]
  \caption{\lang Syntax}
  \label{fig:checkc-syn}
\end{figure}

\ignore{
\begin{figure}[t]
{\small
  \begin{mathpar}

  \inferrule[]
  {}
  {m \vdash \tint}

  \inferrule[]
  {\xi \wedge m\vdash \tau \\ \xi \le m}
  {m \vdash \tptr{\tallarrayb{\bvar}{\tau}}{\xi}}

  \inferrule[]
  {\xi \wedge m \vdash \tau\\ \xi \le m}
  {m \vdash \tptr{\tau}{\xi}}

  \inferrule[]
  {\xi \wedge m \vdash \tau\\ \xi \le m \\\\ \fv(\overline{\tau})\cup\fv(\tau)\subseteq \overline{x}}
  {m \vdash \tptr{(\tfun{\overline{x}}{\overline{\tau}}{\tau}}{\xi})}
  \end{mathpar}
}
{\footnotesize
\[
\begin{array}{l} 
\tmode \wedge \cmode = \umode \qquad \xi \wedge \umode = \umode
\qquad \cmode \wedge m = m 
\qquad  m_1 \wedge m_2 = m_2 \wedge m_1
\\[0.2em]
\xi \le \xi \qquad \tmode \le \xi
\end{array}
\]
}
 \caption{Well-formedness for Types}
\label{fig:wftypes}
\end{figure}
}

%% \dvh{I don't understand the variable grammar.  What is $T$?  What is $\eta$?  I think $\cmode$ and $\umode$ should be in tt font.}
%% \liyi{T and $\eta$ can be moved to the appendix, they are useful only for struct types.}

% \review{
% - Furthermore, inspecting the code also suggests that the expression type (line 155) does not contain constructors for function calls (and I don't see a way to define functions either), conditionals, or strlen, and doesn't distinguish between the two forms of casting. All this contradicts figure 2, and should be clarified
% }
% \liyi{It is in the CheckedC.v file, 392. }
% \mwh{How is this answer helping the reviewer since you've added
%   nothing to the text? Maybe we should add something to an appendix
%   that matches the formalism shown in the paper to definitions in the
%   Coq file?}
% \liyi{The detailed explanation is in the appendix.}

This section describes the formal core model of \systemname, named
\lang, making precise its syntax, semantics, type system, and compilation. It also
develops \lang's meta-theories, including the type soundness, non-exposure, non-crashing, and the compiler simulation theorems.

\subsection{Syntax}\label{sec:syntax}

The \lang syntax of is given in Fig.~\ref{fig:checkc-syn}.
There are two type notions in \lang.  Types $\tau$ classify
word-sized values including integers and pointers, while types
$\omega$ classify multi-word values such as arrays, null-terminated
arrays, functions, and single-word-size values. 
Pointer types ($\tptr{\omega}{\xi}$) include a pointer mode annotation 
($\xi$, the difference between context and pointer modes is introduced shortly below)
that is either checked (\cmode), tainted (\tmode), or unchecked (\umode),
and a type ($\omega$) denoting valid values that can be pointed to.
Array types include both the type of
elements ($\tau$) and a bound ($\bvar$) comprised of an upper and
lower bound on the size of the array ($(b_l,b_h)$). Bounds $b$ are
limited to integer literals $n$ and expressions $x + n$.
Whether an array pointer is null terminated or not is determined by annotation
$\kappa$, which is $nt$ for null-terminated arrays, and $\cdot$
otherwise (we elide $\cdot$ when writing types).

\lang{} function types ($\tfun{\overline{x}}{\overline{\tau}}{\tau}$)
reflect its dependent function declarations,
where $\overline{x}$ represents
a list of \tint{} type variables in a dependent function header
that bind bound variables appearing in $\overline{\tau}$ and $\tau$.
We have a well-formed requirement for a function type;
that is, all variables in $\overline{\tau}$ and $\tau$ are bounded by $\overline{x}$.
Here is the
corresponding \systemname syntax for these types:
\[\hspace*{-0.5em}
\begin{array}{l}
\begin{array}{rcl}
$\code{_t_Array_ptr<}$\tau$\code{> : count(}$n$\code{)}$
&\Leftrightarrow& \tarrayptr{0}{n}{\tau}{\tmode}
\\[0.2em]
$\code{_NT_Array_ptr<}$\tau$\code{> : count(}$n$\code{)}$
&\Leftrightarrow& \tntarrayptr{0}{n}{\tau}{\cmode}
\end{array}
\\[0.2em]
$\code{_t_ptr<(int)(_t_NT_Array_ptr<}$\tau$\code{> : count(}$n$\code{),}$\\
\qquad\qquad$\code{_NT_Array_ptr<}$\tau$\code{>)>: count(}$n$\code{))>}$
\\[0.2em]
\Leftrightarrow\;\; $\tptr{(\tfun{n}{ \tntarrayptr{0}{n}{\tau}{\tmode} \times \tntarrayptr{0}{n}{\tau}{\tmode}}{\tint})}{\tmode}$
\end{array}
\]
As a convention we write $\tptr{\tarrayb{b}{\tau}}{\cmode}$ to mean
$\tptr{\tarray{0}{b}{\tau}}{\cmode}$, so the firs two examples above could
be rewritten $\tptr{\tarrayb{n}{\tau}}{\cmode}$ and
$\tptr{\tntarrayb{n}{\tau}}{\cmode}$, respectively.

\lang expressions include literals ($n\!:\!\tau$), variables ($x$),
 addition ($\ebinop{e_1}{e_2}$), static casts ($\ecast{\tau}{e}$), 
dynamic casts ($\edyncast{\tau}{e}$) \footnote{assumed at compile-time and verified at run-time, see \Cref{app:main}},
the \texttt{strlen} operation ($\estrlen{x}$),
pointer dereference and assignment ($\estar{e}$)
and ($\eassign{e_1}{e_2}$), resp.),
let binding ($\elet{x}{e_1}{e_2}$),
conditionals ($\eif{e}{e_1}{e_2}$),
memory allocation ($\emalloc{\xi}{\omega}$), 
function calls ($\ecall{e}{\overline{e}}$),
unchecked blocks ($\eunchecked{\overline{x}}{e}$), and checked blocks ($\echecked{\overline{x}}{e}$).
% \review{III.A.: the "dynamic cast" terminology may be briefly confused for C++'s
%   RTTI-based dynamic cast feature}
% \mwh{Added reference back to Section 2-B}
Integer literals $n$ are annotated with a type $\tau$ which can be either
$\tint$, or $\tptr{\omega}{\xi}$ in the case $n$ is being used as 
a heap address (this is useful for the semantics);
$\evalue{0}{\tptr{\omega}{\xi}}$ (for any $\xi$ and $\omega$) represents the $\enull$ pointer, as usual. 
The $\texttt{strlen}$ expression operates on variables $x$
rather than arbitrary expressions to simplify managing
bounds information in the type system; the more general case can be
encoded with a \code{let}. We use a less verbose syntax for dynamic bounds
casts; e.g., the following %

\noindent
{\footnotesize
\code{dyn_bounds_cast<_Array_ptr<}$\tau$\code{>>(}$e$\code{, count(}$n$\code{))}
}
becomes 
{\footnotesize$\edyncast{\tptr{\tarrayb{n}{\tau}}{\cmode}}{e}$}
. 

Compared to the former \checkedc model \cite{li22checkedc},
there are four differences.
First, the \systemname type annotations have well-formed restrictions for maintaining non-exposure.
Mainly, in a nested pointer $\tptr{(... \tptr{\tau}{\xi_2} ...)}{\xi_1}$, $\xi_2\le \xi_1$.
It is worth noting that pointer modes are a three point partial order ($\le$),
where $\tmode$ is the infimum, and $\xi\wedge m$ is a special meet operation that projects pointer modes onto context modes,
such that $\tmode$ is projected as $\umode$.
Second, $\emalloc{\xi}{\omega}$ includes a mode flag
$\xi$ for allocating different pointers in different heaps.
We disallow $\omega$ to be a function type ($\tfun{\overline{x}}{\overline{\tau}}{\tau}$).
Third, the first expression $e$ in a function call ($\ecall{e}{\overline{e}}$) represents a function pointer.
Fourth, $\echeckedtext$ blocks are added to the system, 
which permits the nested context-switching between $\echeckedtext$ (represented by context mode $\cmode$)
and $\euncheckedtext$ (represented by context mode $\umode$) code regions.
One example usage of the nested context-switching is the checked function callbacks inside 
an unsafe region in \Cref{lst:humanadjust} and \Cref{lst:humantaint}.
To guarantee the non-exposure safety,
 we extend the $\echeckedtext$ and $\euncheckedtext$ block syntax to be 
$\echecked{\overline{x}}{e}$ and $\eunchecked{\overline{x}}{e}$:
$\overline{x}$ restricts all free variables appearing in $e$, and they cannot be checked pointers.

\lang aims to be simple enough to work with, but powerful enough to
encode realistic \systemname idioms. For example, mutable local
variables can be encoded as immutable locals that point to the heap;
the use of \code{&} can be simulated with \code{malloc};
and loops can be encoded as recursive function calls. \code{struct}s are
not in Fig.~\ref{fig:checkc-syn} for space reasons, but they are
actually in our model, and developed in
\iftr
Appendix~\ref{appx:struct}.
\else
the supplemental report~\cite{checkedc-tech-report}.
\fi
C-style \code{union}s have no safe typing
in \checkedc, so we omit them.


% \mwh{NOTE: COULD WORK THE FOLLOWING INTO THE ABOVE DESCRIPTION OF
%   POINTERS: Array and NT-array types have two relative bounds, whose structures
% can be either an integer or a variable plus an integer. For example,
% if we have the expression
% $x\texttt{=}\emalloc{\tarrayb{(0,10)}{\tint}}$, $x$ then has the type
% $\tptr{\tarrayb{10}{\tint}}{\cmode}$, which represents an array
% pointer of size $10$. The bounds ($(b_l,b_h)$ in
% $\evalue{p}{\tptr{\tarrayb{(b_l,b_h)}{\tau}}{m}}$) indicate relative
% offsets from this pointer of the accessible memory, i.e., $p+b_l$ and
% $p+b_h$. As such, a pointer can only be directly dereferenced if 0 is
% included within the annotated range.  If following the $\emalloctext$
% operation, we execute $*(x-1)$ and $*(x\plus 10)$. The two expressions
% are not valid, because the type of the two expressions $(x-1)$ and
% $(x\plus 10)$ are $\tarrayptr{1}{11}{\tint}{\cmode}$ and
% $\tarrayptr{-10}{0}{\tint}{\cmode}$ and $0$ in these two cases are not
% in the ranges: $[1,11)$ and $[-10,0)$. Thus, in order for a $\cmode$
% pointer with type $\tarrayptr{b_l}{b_h}{\tint}{\cmode}$ to be
% accessible in \checkedc, $0$ must be in the range of $[b_l,b_h)$.}

\begin{figure}
{\small
$\hspace*{-1.2em}
    \begin{array}{l}
    \begin{array}{lll}
e & ::= & \ldots \mid \ret{x}{\evalue{n}{\tau}}{e}\\
r & ::= & e \mid \enull \mid \ebounds\\
E &::=& \Box \mid \ebinop{E}{e} \mid \ebinop{\evalue{n}{\tau}}{E}\mid \ecast{\tau}{E} \mid \edyncast{\tau}{E} \mid\estar{E}\mid\eassign{E}{e}\\[0.2em]
&&\mid\eassign{\evalue{n}{\tau}}{E}\mid \elet{x}{E}{e}\mid \eif{E}{e}{e}\\[0.2em]
&&\mid \ecall{E}{\overline{e}} \mid \ecall{\evalue{n}{\tau}}{\overline{E}} \mid 
\eunchecked{\overline{x}}{E}
\mid \echecked{\overline{x}}{E}


\end{array}
\\ \\
    \end{array} 
$
  \begin{mathpar}
    \inferrule{ m=\mode(E) \\
      e=E[e'] \\
      (\varphi,\heap,e') \longrightarrow (\varphi',\heap',e'')}
    {(\varphi,\heap,e)\longrightarrow_{m} (\varphi',\heap',E[e''])}

    \inferrule{ \umode=\mode(E) \\
      e=E[e'] \\
      \tau=\type(e')}
    {(\varphi,\heap,e)\longrightarrow_{\umode} (\varphi,\heap,E[\evalue{0}{\tau}])}

  \end{mathpar}
}
  \caption{\lang Semantics: Evaluation}
  \label{fig:c-context}
\end{figure}

\subsection{Semantics and Typing}\label{sec:semantics}

% The semantics
% gives an independent account of spatial safety in \lang by
% checking pointer bounds based on the annotations carried on types at
% run-time.  While this account makes clear that bounds checking occurs
% as expected, it suggests an implementation that uses fat pointers to
% carry bounds.  We resolve this tension in the subsequent section on
% compilation and show that an implementation faithful to the semantics
% can be obtained without fat pointers.  
% \review{repeat that the stack is immutable at this point?}
% \liyi{Is it? Is the stack immutable? What does the immutable mean? 
%   In a stack, the variable values can be changed? Right?
%   The pointer address itself cannot be changed once it is created, but the stack variable content can be updated?  }
% \mwh{It certainly seems to be immutable: Your create stack frames
%   using let binding, and the let-bound variables will always be bound
%   to the same things. I.e., stack cells are immutable.}

% \review{this raises a fair amount of questions regarding the treatment of the
%   NULL pointer at this stage of the paper... is it modeled as 0, as returned by
%   `malloc`? are dynamic checks inserted by CheckedC to guarantee that no NULL
%   pointer is dereferenced?}
% \mwh{Yes, it is modeled as 0, and the semantics checks for
%   dereferences of 0. }

Here, we discuss the \lang type system (\Cref{fig:type-system-1}) and operational semantics (\Cref{fig:semantics}); 
mainly focusing on the new changes on top of \checkedc in \cite{li22checkedc} with function pointers, modes, and function calls.
The other type and semantic rules about (NT)-arrays are given in \cite{li22checkedc} and \Cref{sec:literal-pointer-typing}.

The typing judgment has the form $\Gamma;\Theta\vdash_m e : \tau$,
which states that in a type environment $\Gamma$ (mapping variables to
their types) and a predicate environment $\Theta$ (mapping integer-typed
variables to Boolean predicates), expression $e$ will have type $\tau$ if evaluated
in context mode $m$. Key rules for this judgment are given in
Fig.~\ref{fig:type-system-1},
while the operational semantics for \lang is defined as a small-step
transition relation with the judgment $ (\varphi,\heap,e)
\longrightarrow_m (\varphi',\heap',r)$.
 Here, $\varphi$ is a
\emph{stack} mapping from variables to values $\evalue{n}{\tau}$ and
$\heap$ is a \emph{heap} that is partitioned into two parts ($\cmode$ and $\umode$ heaps), each of which
maps addresses (integer literals) to values $\evalue{n}{\tau}$.

\myparagraph{Pointers, Contexts, and Modes}
A $\cmode$ pointer is mapped to a heap location in the $\cmode$ heap, 
while a $\tmode$ and $\umode$ pointer represents a $\umode$ heap location.
We wrote $\heap(m,n)$ to retrieve the $n$-location heap value in the $m$ heap,
and $\heapup{m}{n}{\evalue{n'}{\tau}}$ 
to update location $n$ with the value $\evalue{n'}{\tau}$ in the $m$ heap.
It is worth noting that \systemname is not a fat-pointer system;
thus, in every heap update, the value type annotation remains the same through program executions.
Additionally, for both stack and heap, 
we ensure $\fv(\tau)=\emptyset$ for all the value type annotations $\tau$.

While heap bindings can change, stack bindings are immutable---once
variable $x$ is bound to $\evalue{n}{\tau}$ in $\varphi$, that binding will not
be updated. 
We can model mutable stack variables as pointers into the
mutable heap.
As mentioned, value $\evalue{0}{\tau}$
represents a $\enull$ pointer when $\tau$ is a pointer type.
Correspondingly, $\heap(m,0)$ should always be undefined.
The relation steps to a \emph{result} $r$,
which is either an expression or a $\enull$ or $\ebounds$ failure,
represent a null-pointer dereference or out-of-bounds access,
respectively. Such failures are a \emph{good} outcome; stuck states
(non-value expressions that cannot transition to a result $r$)
characterizing undefined behavior.
%
The context mode $m$ (in $\longrightarrow_{m}$) indicates whether the
stepped redex within $e$ was in a $\cmode$ or $\umode$ region.

The rules for the main operational semantics
judgment \emph{evaluation} are given at the bottom of
Fig.~\ref{fig:c-context}.
The first rule takes an expression $e$, decomposes
it into an \emph{evaluation context} $E$ and a sub-expression $e'$
(such that replacing the hole $\Box$ in $E$ with $e'$ would yield
$e$), and then evaluates $e'$ according to the \emph{computation}
  relation $(\varphi,\heap,e') \longrightarrow (\varphi,\heap,e'')$,
whose rules are given in Fig.~\ref{fig:semantics}, discussed
shortly.
The $\mode$ function  at the bottom of Fig.~\ref{fig:c-context}
determines the context mode that the expression $e'$ locates based on the context $E$.
In \Cref{lst:humantaint}, the function call expression \code{read_msg} has $\umode$ mode since it is inside a tainted function.
The second rule describes the exception handling 
for possible crashing behaviors in unchecked regions.
A $\umode$ mode operation can non-deterministically crash
and the \systemname sandbox mechanism recovers
the program to a safe point ($\evalue{0}{\tau}$)
and continues with the existing program state.
Evaluation contexts $E$ define a standard left-to-right evaluation order. (We explain the
$\ret{x}{\mu}{e}$ syntax shortly.)
%There are other rules for describing the halts of evaluation to $\enull$ and $\ebounds$ states in \Cref{app:main}.


\begin{DIFnomarkup}
\begin{figure*}[t]
{\small
  \begin{mathpar}
        \inferrule[S-DefC]{\heap(\cmode,n)=\evalue{n_a}{\tau_a} }
    {(\varphi,\heap,\estar{\evalue{n}{\tptr{\tau}{c}}}) \longrightarrow (\varphi,\heap,\evalue{n_a}{\tau})}
\quad
        \inferrule[S-DefT]{\heap(\umode,n)=\evalue{n_a}{\tau_a}
         \\  \emptyset;\heap ; \emptyset \vdash_{\umode}\evalue{n_a}{\tau} }
    {(\varphi,\heap,\estar{\evalue{n}{\tptr{\tau}{\tmode}}}) \longrightarrow (\varphi,\heap,\evalue{n_a}{\tau})}
\quad
    \inferrule[S-DefNull]{}{(\varphi,\heap,\estar{\evalue{0}{\tptr{\omega}{\cmode}}}) \longrightarrow (\varphi,\heap,\enull)}
\quad
    \inferrule[S-Cast]
              {}
              {(\varphi,\heap,\ecast{\tau}{\evalue{n}{\tau'}}) \longrightarrow (\varphi,\heap,\evalue{n}{\varphi(\tau)})}

    \inferrule[S-RetEnd]{}{(\varphi,\heap,\ret{x}{\evalue{n}{\tau}}{\evalue{n'}{\tau'}}) \longrightarrow (\varphi,\heap,\evalue{n'}{\tau'})}

        \inferrule[S-Let]{}{(\varphi,\heap,\elet{x}{\evalue{n}{\tau}}{e}) \longrightarrow (\varphi,\heap,\ret{x}{\evalue{n}{\tau}}{e})}

    \inferrule[S-Unchecked]{}{(\varphi,\heap,\eunchecked{\overline{x}}{\evalue{n}{\tau}}) \longrightarrow (\varphi,\heap,\evalue{n}{\tau})}

    \inferrule[S-RetCon]{ (\varphi[x\mapsto \evalue{n}{\tau}],\heap,e) \longrightarrow (\varphi',\heap',e')}{(\varphi,\heap,\ret{x}{\evalue{n}{\tau}}{e}) \longrightarrow (\varphi'[x\mapsto \varphi(x)],\heap',\ret{x}{\varphi'(x)}{e'})}

    \inferrule[S-Checked]{}{(\varphi,\heap,\echecked{\overline{x}}{\evalue{n}{\tau}}) \longrightarrow (\varphi,\heap,\evalue{n}{\tau})}

    \inferrule[S-FunC]{ \Xi(\cmode,n) = \tau\;(\evalue{\overline{x}}{\overline{\tau}})\;(\cmode,e)}
        {(\varphi,\heap,\ecall{\evalue{n}{(\tptr{\tau}{\cmode})}}{{\evalue{\overline{n_a}}{\overline{\tau_a}}}}) \longrightarrow
   (\varphi,\heap, \mathtt{let}\;\overline{x}={\evalue{\overline{n}}{(\overline{\tau}[\overline{n} / \overline{x}])}}\;\mathtt{in}\;\ecast{\tau[\overline{n} / \overline{x}]}{e})}
\quad
    \inferrule[S-FunT]{ \Xi(\umode,n) = \tau\;(\evalue{\overline{x}}{\overline{\tau}})\;(\tmode,e)
                  \\ \emptyset;\heap ; \emptyset \vdash_{\umode}\evalue{n}{\tptr{\tau}{\tmode}}}
        {(\varphi,\heap,\ecall{\evalue{n}{(\tptr{\tau}{\tmode})}}{{\evalue{\overline{n_a}}{\overline{\tau_a}}}}) \longrightarrow
   (\varphi,\heap, \mathtt{let}\;\overline{x}={\evalue{\overline{n}}{(\overline{\tau}[\overline{n} / \overline{x}])}}\;\mathtt{in}\;\ecast{\tau[\overline{n} / \overline{x}]}{e})}


 % \inferrule[S-IfNTNot]{\varphi(x)=\evalue{n}{\tntarrayptr{n_l}{n_h}{\tau}{\cmode}} \\ \heap(n)\neq 0\\ 0 < n_h}
 %            {(\varphi,\heap,\eif{\estar{x}}{e_1}{e_2}) \longrightarrow (\varphi,\heap,e_1)}

\end{mathpar}
}
% {\footnotesize
% \begin{center}
% $
% \begin{array}{l}
% \tau[\overline{n} / \overline{x}]\texttt{(with types }\evalue{\overline{x}}{\overline{\tau}}\texttt{)}\triangleq \forall n_i\in\overline{n}\;x_i\in\overline{x}\;\tau_i\in\overline{\tau}\;.\;\tau_i = \tint \Rightarrow \tau[n_i / x_i]\\[0.2em]
% \mathtt{let}\;\overline{x}=\overline{e}\;\mathtt{in}...\triangleq \mathtt{let}\;x_0=e_0\;\mathtt{in}\;\mathtt{let}\;x_1=e_1\;\mathtt{in}...
% \end{array}
% $
% \end{center}
% }
\caption{\lang Semantics: Computation (Selected Rules)}
\label{fig:semantics}
\end{figure*}
\end{DIFnomarkup}

Fig.~\ref{fig:semantics} shows selected rules for the computation relation.
The rules for pointer related operations---\textsc{S-DefC},
\textsc{S-DefT}, \textsc{S-DefNull}, and \textsc{S-Cast}.
The type rule for deference operations is given as rule \rulelab{T-Def} in \Cref{fig:type-system-1}.
The first three define the semantics of deference and assignment operations.
Rule \textsc{S-DefNull} transitions attempted null-pointer
dereferences to $\enull$, whereas \textsc{S-DefC} dereferences a $\cmode$-mode
non-null (single) pointer.
When $\enull$ is returned by the
computation relation, the evaluation relation halts the entire
evaluation with $\enull$ (using a rule not shown in Fig.~\ref{fig:c-context}); it
does likewise when $\ebounds$ is returned (see \Cref{sec:rem-semantics}).
%\textsc{S-AssignArrC} assigns to an array as long as 0 (the point of
%dereference) is within the bounds designated by the pointer's annotation
%and strictly less than the upper bound. 
\textsc{S-DefT} is similar to \textsc{S-DefC} for tainted pointers.
Any dynamic heap access of a tainted pointer requires a \textit{verification}.
Performing such a verification equates to performing a literal type check for a pointer constant in \Cref{fig:const-type}.
We explain this shortly below for \emph{constant validity checks}.
For now, the verification step, e.g. $\emptyset;\heap ; \emptyset \vdash_{\umode}\evalue{n_a}{\tau}$ in \textsc{S-DefC},
refers to that the value $n_a$ is well-defined in $\heap(m,n_a)$ and has type $\tau$, if $\tau$ is a pointer.
Static casts of a literal $n\!:\!\tau'$ to a type $\tau$ are handled
by \textsc{S-Cast}. In a type-correct program, such casts are
confirmed safe by the type system no matter
if the target is a $\tmode$ or $\cmode$ pointer. To evaluate a cast, the rule
updates the type annotation on $n$. Before doing so, it must
``evaluate'' any variables that occur in $\tau$ according to their
bindings in $\varphi$. For example, if $\tau$ was
$\tarrayptr{0}{x+3}{\tint}{\cmode}$, then $\varphi(\tau)$ would
produce $\tarrayptr{0}{5}{\tint}{\cmode}$ if $\varphi(x) = 2$.
%The full formalism, including \kw{struct}
%and null-terminated bound widening pointer operations, is given in \Cref{app:main}.

%\footnote{This approach is that of the PLT Redex model of \lang; the Coq
%development uses a slightly simpler syntax to achieve the same
%effect.}
% \review{the special case raises questions, e.g. why is this syntax-driven and
%   not type-driven? }
% \liyi{This describes the semantic transition rules. We are using context evaluation framework to define the transition rules as the $E$ definition in Fig.3. like $\frac{x \Rightarrow y}{x+z \Rightarrow y + z}$, I don't know how type-driven can help us define translation rules.  }
% \mwh{Don't follow the above. I don't see this ``context transition
%   rule'' anywhere, and I'm not sure how it would fire, if we had it.}
% \liyi{The comment seems to confuse the meaning of the text about the if-then-else rules. Making the rule specific will help. }



\begin{DIFnomarkup}
\begin{figure*}[t]
{\small
  \begin{mathpar}
   \inferrule[T-ConstU]
       { \neg \cmode(\tau)}
       {\Gamma;\Theta\vdash_u \evalue{n}{\tau} : \tau}

   \inferrule[T-ConstC]
       {\Theta;\heap;\emptyset \vdash_c n : \tau}
       {\Gamma;\Theta\vdash_c \evalue{n}{\tau} : \tau}

    \inferrule[T-Def]
              {\xi \leq m \\\\\Gamma;\Theta \vdash_m e : \tptr{\tau}{\xi}}
              {\Gamma;\Theta \vdash_m \estar{e} : \tau}

     \inferrule[T-CastPtr]
               {\Gamma;\Theta \vdash_m e : \tau' \\
                 \tau' \sqsubseteq_{\Theta} \tptr{\tau}{\xi}}
               {\Gamma;\Theta \vdash_m \ecast{\tptr{\tau}{\xi}}{e} : \tptr{\tau}{\xi}}
                
   \inferrule[T-Let]
    { x\not\in \fv(\tau') \\
        \Gamma;\Theta \vdash_m e_1 : \tau \\\\
          \Gamma[x\mapsto \tau];\Theta \vdash_m e_2 : \tau'
             }
    {\Gamma;\Theta \vdash_m \elet{x}{e_1}{e_2} : \tau'}

   \inferrule[T-LetInt]
    { x\in \fv(\tau') \Rightarrow e_1 \in \text{Bound} \\
        \Gamma;\Theta \vdash_m e_1 : \tint \\\\
           \Gamma[x\mapsto \tint];\Theta[x\mapsto \teq{e_1}] \vdash_m e_2 : \tau'
             }
    {\Gamma;\Theta \vdash_m \elet{x}{e_1}{e_2} : \tau'[e_1 / x]}
\qquad
   \inferrule[T-RetInt]
    { \Gamma[x\mapsto \tint];\Theta[x\mapsto \teq{n}] \vdash_m e : \tau}
    {\Gamma;\Theta \vdash_m \eret{x}{\evalue{n}{\tint}}{e} : \tau}

    \inferrule[T-Checked]
              {\forall x\in\overline{x}\;.\;\neg\cmode(\Gamma(x))\\\neg\cmode(\tau)
                     \\\\\fv(e)\in\overline{x}\\\Gamma;\Theta \vdash_c e : \tau}
              {\Gamma;\Theta \vdash_m \echecked{\overline{x}}{e} : \tau}

    \inferrule[T-Unchecked]
              {\forall x\in\overline{x}\;.\;\neg\cmode(\Gamma(x))\\\neg\cmode(\tau)
                \\\\ \fv(e)\in\overline{x}\\\Gamma;\Theta \vdash_u e : \tau}
              {\Gamma;\Theta \vdash_m \eunchecked{\overline{x}}{e} : \tau}

\inferrule[T-Fun]
    {\Gamma;\Theta \vdash_m e : \tptr{\tfun{\overline{x}}{\overline{\tau}}{\tau}}{\xi} \\
        \Gamma; \Theta \vdash_m \overline{e} : \overline{\tau'} \\
         \overline{e'}=\{e'|(e',\tint)\in (\overline{e} : \overline{\tau'})\}\\\\
         \forall e'\;.\;e' \in \overline{e'} \Rightarrow e'\in \text{Bound}\\
             \overline{\tau'} \sqsubseteq_{\Theta}
               \overline{\tau}[\overline{e'} / \overline{x}]}
    {\Gamma; \Theta \vdash_m e(\overline{e}) : \tau[\overline{e'} / \overline{x}]}




  \end{mathpar}
}
% {\footnotesize
% \begin{center}
% $
% \begin{array}{l}
% \fm(e)\triangleq(\exists x\; n\; \tau. e=x+\evalue{n}{\tau}) \vee (\exists n\;\tau. e = \evalue{n}{\tau})
% \\[0.2em]
% \tau[\overline{e} / \overline{x}]\texttt{(with types }\evalue{\overline{x}}{\overline{\tau}}\texttt{)}\triangleq \forall e_i\in\overline{e}\;x_i\in\overline{x}\;\tau_i\in\overline{\tau}\;.\;\tau_i = \tint \wedge (x_i \in \fv(\tau) \Rightarrow \fm(e_i)) \Rightarrow \tau[e_i / x_i]
% \end{array}
% $
% \end{center}
% }
{\footnotesize
$
\cmode(\tint)=\texttt{false}
\qquad
\cmode(\tptr{\omega}{\cmode})=\texttt{true}
\qquad
\cmode(\tptr{\omega}{\xi})=\texttt{false}\;\;{[\emph{owise}]}
$
}
\caption{Selected type rules}
\label{fig:type-system-1}
\end{figure*}
\end{DIFnomarkup}

\myparagraph{Type Equality and Subtyping and Casting}
%
In \lang, the type equability ($=_{\Theta}$) and subtype ($\sqsubseteq$) relations are given in \Cref{fig:checkc-subtype}.
We provide some example descriptions here.
Type equality $\tau=_{\Theta}\tau'$
is a type construct equivalent relation defined by the bound equality ($=_{\Theta}$) in (NT-)array pointer types
and the alpha equivalence of two function types;
i.e., two (NT-)array pointer types $\tallarrayb{\bvar}{\tau} $ and $ \tallarrayb{\bvar'}{\tau'}$ are equivalent, if 
$\bvar =_{\Theta} \bvar'$ and $\tau=_{\Theta}\tau'$; two function types 
$\tfun{\overline{x}}{\overline{\tau}}{\tau} $ and $ \tfun{\overline{y}}{\overline{\tau'}}{\tau'}$
are equivalent, if we can find a same length (as $\overline{x}$ and $\overline{y}$) variable list $\overline{z}$ that is substituted for $\overline{x}$ and $\overline{y}$ in $\overline{\tau} \to {\tau}$ and $\overline{\tau'} \to {\tau'}$, resp.,
and the substitution results are equal.

The \textsc{T-CastPtr} rule in \Cref{fig:type-system-1}
permits casting from an expression of type $\tau'$ to a checked pointer when
$\tau' \sqsubseteq \tptr{\tau}{\cmode}$. This subtyping relation
$\sqsubseteq$ is built on the type equality ($\tau =_{\Theta} \tau'\Rightarrow\tau \sqsubseteq_{\Theta} \tau'$). 
The rule  ($0\le b_l \wedge b_h \le 1 \Rightarrow \tptr{\tau}{m}\sqsubseteq
\tarrayptr{b_l}{b_h}{\tau}{m}$) permits treating a singleton
pointer as an array pointer with $b_h\le 1$ and $0 \le b_l$.
Two function pointer types are subtyped ($\tptr{\tfun{\overline{x}}{\overline{\tau}}{\tau}}{\xi} \sqsubseteq_{\Theta} \tptr{\tfun{\overline{x}}{\overline{\tau'}}{\tau'}}{\xi}$), 
if the output type are subtyped ($\tau\sqsubseteq_{\Theta}\tau'$) and the argument types are reversely subtyped ($\overline{\tau'}\sqsubseteq_{\Theta}\overline{\tau}$).
%There is another casting rule in \Cref{app:main} stating that
% users are free to cast types in unchecked code regions, since unchecked regions can contain C code.

\begin{DIFnomarkup}
 \begin{figure}[t]
 {\small

 \begin{mathpar}
   \inferrule
       {}
       {\Theta;\heap;\sigma \vdash_m n : \tint}

   \inferrule
       {}
       {\Theta;\heap;\sigma \vdash_m 0 : \tptr{\omega}{\xi}}

   \inferrule
       {(m = \cmode \Rightarrow \xi \neq \cmode) \\\\ (m=\umode \Rightarrow \xi = \umode)}
       {\Theta;\heap;\sigma \vdash_{\cmode} n : \tptr{\omega}{\tmode}}
  
   \inferrule
       {(\evalue{n}{\tptr{\omega}{\xi}})\in \sigma}
       {\Theta;\heap;\sigma \vdash_m n : \tptr{\omega}{\xi}}


   \inferrule
       {\tptr{\omega'}{\xi'} \sqsubseteq_{\Theta} \tptr{\omega}{\xi} 
            \\ \Theta;\heap;\sigma \vdash_m n : \tptr{\omega'}{\xi'}}
       {\Theta;\heap;\sigma \vdash_m n : \tptr{\omega}{\xi}}

   \inferrule
       { \xi \le m 
     \\\Xi(m,n)=\tau\;(\evalue{\overline{x'}}{\overline{\tau}})\;(\xi,e)
       \\  \overline{x} = \{x|(x:\tint) \in (\overline{x'}:\overline{\tau}) \}}
       {\Theta;\heap;\sigma \vdash_m n : \tptr{(\tfun{\overline{x}}{\overline{\tau}}{\tau})}{\xi}}
  
   \inferrule
       {\neg\funptr(\omega)\\ \xi \le m\\
        \forall i \in [0,\size(\omega)) \;.\;
            \Theta;\heap;(\sigma \cup \{(n:\tptr{\omega}{\xi})) \}\vdash_m \heap(m,n+i)}
       {\Theta;\heap;\sigma \vdash_m n : \tptr{\omega}{\xi}}
 \end{mathpar}
 }
{\footnotesize
\[
\begin{array}{l} 
\funptr(\tfun{\overline{x}}{\overline{\tau}}{\tau}) = \texttt{true}
\qquad
\funptr(\omega) = \texttt{false}\;\;{[\emph{owise}]}
\end{array}
\]
}
 \caption{Verification/Type Rules for Constants}
 \label{fig:const-type}
 \end{figure}
\end{DIFnomarkup}

\myparagraph{Constant Validity}
Rules \textsc{T-ConstU} and \textsc{T-ConstC} in \Cref{fig:type-system-1}
describe type assumptions for constants appearing in a program.
$\neg \cmode(\tau)$ judges that a constant pointer 
in an unchecked region cannot be of a checked type.
The restriction ensures that programmers 
cannot guess a checked pointer address and utilize it in an unchecked region in \systemname.
In rule \textsc{T-ConstC}, we requires a static 
verification procedure for validating a constant pointer in \Cref{fig:const-type}. 

The verification process $\Theta;\heap;\sigma \vdash_m n : \tau$
validates the constant $\evalue{n}{\tau}$, 
where $\heap(m)$ is the initial heap that the constant resides on and
$\sigma$ is a set of constant assumed to be checked.
A global function store $\Xi(m)$ is also required to check the validity of a function pointer.
A valid function pointer should appear in the right store region ($\cmode$ or $\umode$)
and the address stores a function with the right type.
The last rule in \Cref{fig:const-type} describes the validity check for a non-function pointer, 
where every element in the pointer range ($[0,\size(\omega))$) should be well
typed.
A checked pointer checks validity in type step as rule \textsc{T-ConstC},
while a tainted/unchecked pointer does not check for such during the type checking.
Tainted pointers are validated through the validity check in dynamic execution as we mentioned in rule \rulelab{S-DefT}.

% \review{Fig4: it was hard to tell which cases were stuck states, or could reduce
%   owing to a rule that was not shown}
% \mwh{Stuck states are those where the expression is a non-value and
%   not $\enull$ or $\ebounds$. Updated III-B and III-D.}
% \review{Fig4: can the rules be presented in the same order they were introduced in
%   the paper?}
% \liyi{ Reordered }

% \review{Fig4: S-FUN: $\vec\tau_a$ seems unused; why?}
% \liyi{the list of $\tau_a$ is a list of input argument types. These
%   are used during type checking, but not during evaluation (as is
%   typical).} 

% LEO: This has an overfull line...?



% Below, we introduce low-level transition semantics for some case operations. The design of the low-level individual operation semantics is carefully engineered to perform match our compiler's behavior, such as correctly characterizing the bound widening behaviors for NT-array pointers, even though it is written in terms of fat-pointer formalization.
% \review{- on page 6, section IIIC, paragraph "Pointer Access" mentions that checked pointers cannot be dereferenced in unchecked blocks - this looks funny, shouldn't it be the other way around? The Coq code contains the hypothesis m'=Unchecked -> m Unchecked in various rules of definition well-typed (BoundCheckedC, line 669; rule TyDeref in the code seems closest to the figure's T-DefArr and T-Def in the appendix, although it's a bit concerning that there's no 1-to-1 correspondence of the rules in the code and the paper).
% }
% \liyi{Yeah. It is another typo. It should be the other way around.  }

\myparagraph{Unchecked and Checked Blocks}
%
During the type checking,
Both $\echecked{\overline{x}}{e}$ and $\eunchecked{\overline{x}}{e}$
check all free variables in $e$ are within $\overline{x}$;
the types for $\overline{x}$ and the final return type $\tau$ of $e$ have no checked pointers.
Otherwise, it violates the non-exposure safety.
For example, \code{read_msg} in the \code{handle_request} function is tainted in \Cref{lst:humantaint},
if any argument for \code{handle_request} is a checked pointer, 
it means that we are exposing a checked pointer address to unsafe regions.
A $\echeckedtext$ or $\euncheckedtext$ block represents 
the context switching from a checked to an checked region, or vice versa.
We need to make sure no checked pointers are information exposed to unsafe code regions.
as rules \textsc{S-Unchecked} and \textsc{S-Checked} in \Cref{fig:semantics}.

\myparagraph{Let Bindings and Dependent Function Pointers}
%
Rules \textsc{T-Let} and \textsc{T-LetInt} in in \Cref{fig:type-system-1} type a $\elettext$ expression, which also admits
type dependency. 
In particular, the result of evaluating a $\elettext$ expression
may have a type that refers to one of its bound variables (e.g., if
the result is a checked pointer with a variable-defined bound). 
If so, we must substitute away this variable once it goes out of scope (\textsc{T-LetInt}). 
Note that we restrict the expression $e_1$ to syntactically match the
structure of a Bounds expression $b$ (see Fig.~\ref{fig:checkc-syn}).
Rule \textsc{T-RetInt} types a $\erettext$ expression when $x$ is of type $\tint$.
$\erettext$ does not appear in source programs but is introduced by the semantics when
evaluating a let binding (rule \textsc{S-Let} in
Fig.~\ref{fig:semantics}). 
%\liyi{why? }
% After the evaluation of a let binding a variable $x$ concludes,
%we need to restore any prior binding of $x$, which is either
%$\bot$ (meaning that there is no $x$ originally) or some value
%$\evalue{n}{\tau}$.

Rule \textsc{T-Fun} in \Cref{fig:type-system-1} is the dependent function call rule. 
Given a function pointer type ($\tptr{\tfun{\overline{x}}{\overline{\tau}}{\tau}}{\xi}$)
from a type-check for $e$ and the types $\overline{\tau'}$ from the argument type checks for $\overline{e}$,
we confirm that each of $\overline{\tau'}$ is
a subtype of the corresponding one in $\overline{\tau}[\overline{e'} / \overline{x}]$,
which replaces possible integer bound variables $\overline{x}$ with bound expressions $\overline{e'}$.
The final result type is the defined target type $\tau$ appearing in the function pointer type
also with such replacement, written as $\tau[\overline{e'} / \overline{x}]$.
Consider the \code{process_req2} function in
Fig.~\ref{lst:final}; its parameter type for \code{msg} 
depends on \code{m_1}. The \textsc{T-Fun} rule will substitute 
\code{n} with the argument at a call-site.
The semantics manages variable scopes using the special $\erettext$
form. \textsc{S-Let} evaluates to a configuration whose expression is
$\ret{x}{\evalue{n}{\tau}}{e})$. We keep $\varphi$ unchanged
and remember $x$ and its new value $\evalue{n}{\tau}$
in $e$'s scope that is defined by the $\erettext$ operation.
Every time when evaluation proceeds on $e$ (rule \textsc{S-RetCon}),
we install the stack value $\evalue{n}{\tau}$ for $x$ in $\varphi$ for the current scope.
After one-step evaluation is completed, 
we store $x$'s change in the result $\erettext$ operation $\ret{x}{\varphi'(x)}{e'})$,
and restore $x$'s outer score value $\varphi(x)$ in $\varphi'$. 
This procedure continues until $e'$ becomes a literal
$n\!:\!\tau$, in which case \textsc{S-RetEnd} removes the $\kw{ret}$ frame and returns
the literal. 

\textsc{S-FunC} and \textsc{S-FunT} are
for $\cmode$ and $\tmode$ mode function pointers, respectively. 
A call to a function pointer $n$ retrieves
 the function definition in $n$'s location in the global function store $\Xi$,
which maps function pointers to
function data $\tau\;(\evalue{\overline{x}}{\overline{\tau}})\;(\xi,e)$, where
$\tau$ is the return type, $(\evalue{\overline{x}}{\overline{\tau}})$
is the parameter list of variables and their types, 
$\xi$ determines the mode of the function, and $e$ is the
function body. 
Similar to \heap, the global function store $\Xi$ is also partitioned into
two parts ($\cmode$ and $\umode$ stores), each of which
maps addresses (integer literals) to the function data described above.

\systemname{} has dependent functions, whose semantic explaination is given in \Cref{appx:add-type-sem}.
Note that the \textsc{S-FunC} and \textsc{S-FunT} rules replace the
  annotations $\overline{\tau_a}$ with
  $\overline{\tau}$ (after instantiation) from the function's
  signature. Using $\overline{\tau_a}$ when executing the body of
the function has no impact on the soundness of \lang, but will violate
Theorem~\ref{simulation-thm}, which we introduce in Sec.~\ref{sec:compilation}.
Rule \textsc{S-FunT} defines the tainted version of function call semantics.
In such case, the verification process 
$\emptyset;\heap ; \emptyset \vdash_{\umode}\evalue{n}{\tptr{\tau}{\tmode}}$
makes sure that the function in the global store is well-defined and has the right type.

 %  For this rule and
% \textsc{S-StrWiden}, this widening persists in the current stack
% frame. When $x$ goes out of scope, .

% \textsc{S-IfNTF} does not widen when seeing null; rule
% \textsc{S-IfNTNot} sees a non-null character, but the pointer is not
% at its upper bound, so the bounds cannot be widened. 

% \ignore{
% Fig.~\ref{fig:semantics} provides the low-level semantic rules for operations involving NT-array pointers, mainly, the $\estrlentext$ and $\eiftext$ operations. The semantics has concurred the ambiguity in the \checkedc specification, e.g., we define the exact behavior of the $\estrlentext$ operation to return the length between the current pointer position and the first null-character.
% We also utilize new technique in our compiler so that the scope of the bound widening behavior in our formalization is a little longer. More details are in Sec.~\ref{sec:compilation}.

% The first rule defines the evaluation behavior of a $\estrlentext$ operation. Given a pointer $x$ with its type $\tntarrayptr{0}{n_h}{\tau}{m}$, the application of such operation takes the address of the pointer $x$, and search incrementally the heap positions next to the address $x$ until we find a $0$ value (representing a null character). We return the value $n_a$ as the length, and update the bound information in the stack for $x$. In the compilation, we use a ghost variable to record such bound changes without using fat-pointer implementations.

% The last three rules in Fig.~\ref{fig:semantics} describe the semantic behaviors of an $\eiftext$ branching operation when the Boolean guard is a dereference of an NT-array pointer. The first one states that if the type upper bound of the pointer $x$ is $0$, and the pointer data value $n_a$ is not $0$, we can conclude that the upper bound is not the last position of the NT-array pointer, so we can then update $1$ in the upper bound while jump to the $\etrue$ branch. The second rule describes that we do not extend the upper bound if the upper-bound of the type of $x$ is not zero because we know that we are not in the NT-array's last position. The third rule describes the behavior of jumping to the $\efalse$ branch when the pointer content is $0$. In this case, we also do not need to increase the upper-bound of the type of $x$.}

\subsection{Compilation}\label{sec:compilation}

As we have shown in \Cref{fig:overview}, the \systemname compiler utilizes the sandbox mechanism \cite{rul2009towards} and the \checkedc compiler \cite{li22checkedc} to compile programs. Here, we introduce how \systemname compiles a program into these two components.

\begin{figure}[t!]
{\small
\hspace*{-0.5em}
\begin{tabular}{|c|c|c|c|}
\hline
& \cmode & \tmode & \umode \\
\hline
& \textsc{CBox} / \textsc{Core} & \textsc{CBox} / \textsc{Core} & \textsc{CBox} / \textsc{Core} \\
\hline
\cmode & $\estar{x}$ / $\getstar{\cmode}{x}$ 
 & $\texttt{sand\_get}(x)$ / $\getstar{\umode}{x}$ &  $\times$ \\
\hline
\umode & $\times$
 & $\estar{x}$ / $\getstar{\umode}{x}$ &  $\estar{x}$ / $\getstar{\umode}{x}$ \\
\hline
\end{tabular}

}
\caption{Compiled Targets for Dereference}
\label{fig:flagtable}
\end{figure}

In \systemname, context and pointer modes determine the particular heap/function store that a pointer points to,
i.e., $\cmode$ pointers point to checked regions, while $\tmode$ and $\umode$ pointers point to unchecked regions.
Unchecked regions are associated with a sandbox mechanism that permits exception handling of potential memory failures.
In the compiled LLVM code, pointer access operations have different syntaxes when the modes are different. 
\Cref{fig:flagtable} lists the different compiled syntaxes of a deference operation ($\estar{x}$) for the compiler implementation (\textsc{CBox}, stands for \systemname) and formalism (\textsc{Core}, stands for \lang). The columns represent different pointer modes and the rows represent context modes.
For example, when we have a $\tmode$-mode pointer in a $\cmode$-mode region, we compile a deference operation to the sandbox pointer access function ($\texttt{sand\_get}(x)$) accessing the data in the \systemname implementation. In \lang, we create a new deference data-structure on top of the existing $\estar{x}$ operation (in LLVM): $\getstar{m}{x}$. If the mode is $\cmode$, it accesses the checked heap/function store; otherwise, it accesses the unchecked one.

We now show how \lang deals with pointer modes, mode switching and function pointer compilations, 
with no loss of expressiveness
as the \checkedc contains the erase of annotations in \cite{li22checkedc} and \Cref{appx:comp1}.
For the compiler formalism, 
we present a compilation algorithm that converts from
\lang to \elang, an untyped language without metadata
annotations, which represents an intermediate layer we build on LLVM for simplifying compilation. 
In \elang, the syntax for deference, assignment, malloc, function calls are: $\getstar{m}{e}$, $\elassign{m}{e}{e}$, 
$\emalloc{m}{\omega}$, and $\elcall{m}{e}{\overline{e}}$.
The algorithm sheds
  light on how compilation can be implemented in the real Checked C
  compiler, while eschewing many vital details (\elang has many 
  differences with LLVM IR).


%This section shows how \systemname deals with 
%annotations can be safely erased: using static information a compiler
%can insert code to manage and check bounds metadata, with no loss of
%expressiveness. We present a compilation algorithm that converts from
%\lang to \elang, an untyped language without metadata
%annotations. The syntax and semantics \elang
  %closely mirrors that of \lang; it differs only in that literals lack
  %type annotations and its operational rules perform no
  %bounds and null checks, which are instead inserted during
  %compilation. Our compilation algorithm is evidence that \lang's
  %semantics, despite its apparent use of fat pointers, faithfully
  %represents Checked C's intended behavior. The algorithm also sheds
  %light on how compilation can be implemented in the real Checked C
  %compiler, while eschewing many important details (\elang has many 
  %differences with LLVM IR).

Compilation is defined by extending \lang's
typing judgment as follows:
\[\Gamma;\Theta;\rho \vdash_m e \gg \dot e:\tau\]
There is now a \elang output $\dot e$ and an input $\rho$, which maps
each (NT-)array pointer variable to its mode and
each variable \code{p} to a pair of \emph{shadow
  variables} that keep \code{p}'s up-to-date upper and lower bounds. 
These may differ from the bounds in \code{p}'s type due to bounds
widening.\footnote{Since lower bounds are never widened, the
  lower-bound shadow variable is unnecessary; we include it for uniformity.} 

% When $\Gamma$,$\Theta$ and $\rho$ are all empty, we write $e \gg \dot e$ rather than the
% complete judgment, implicitly assuming that $e$ is a well-typed and closed
% term.

We formalize rules for this judgment in PLT Redex~\cite{pltredex},
following and extending our Coq development for \lang. To give
confidence that compilation is correct, we use Redex's property-based
random testing support to show that compiled-to $\dot e $ simulates
$e$, for all $e$.

\myparagraph{Checked and Unchecked Blocks}
%
In the \systemname implementation,
$\euncheckedtext$ and $\echeckedtext$ blocks 
are compiled as context switching functions provided by the sandbox mechanism.
We compile $\eunchecked{\overline{x}}{e}$ to 
$\texttt{sandbox\_call}(\overline{x},e)$, where we call the sandbox 
to execute expression $e$ with the arguments $\overline{x}$.
$\echecked{\overline{x}}{e}$ is compiled to 
$\texttt{callback}(\overline{x},e)$, where we perform 
a \texttt{callback} to a checked block code $e$ inside a sandbox.
In \systemname, we adopt an aggressive execution scheme that
directly learns pointer addresses from compiled assembly to make the $\texttt{callback}$ happen.
In the formalism, we rely on the type system to 
guarantee the context switching without creating the extra function calls for simplicity.

%Fig.~\ref{fig:compilationexample} shows how an invocation of
%\code{strlen} on a null-terminated string is compiled into C
%code. Each dereference of a checked pointer requires a null check
%(See \textsc{S-DefNull} in Fig.~\ref{fig:semantics}), which the
%compiler makes explicit: Line~$3$ of the generated code has the null
%check on pointer \code{p} due to the \code{strlen},
%  and a similar check happens
%  at line~$8$ due to the pointer arithmetic on \code{p}.
%Dereferences also require bounds checks: line~$2$ checks \code{p} is
%in bounds before computing \code{strlen(p)}, while line~$10$ does
%likewise before computing \code{*(p+1)}.

\myparagraph{Function Pointers and Calls}
%
Function pointers are managed similarly to normal pointers,
but we insert checks to check if the pointer address is not null in 
the function store instead of heap, and whether or not the type is correctly represented, 
for both $\cmode$ and $\tmode$ mode pointers 
\footnote{$\cmode$-mode pointers are checked once in the beginning and $\tmode$-mode pointers are checked every time when use}.
For example, in compiling the \code{read_msg} function in \Cref{lst:humantaint},
we place a check \code{verify_fun(read_msg, not_null(c, p_lo, p_hi) && type_match)},
The compilation of function calls (compiling to $\elcall{m}{e}{\overline{e}}$) 
is similar to the manipulation of pointer access operations in \Cref{fig:flagtable}.
The other compilation rules are given in \Cref{appx:add-type-sem}.

\subsection{Meta Theories}\label{sec:theorem}

% Before we present our main theorems, we need to first
% discuss the meaning what a pointer being well-typed in a given heap
% snapshot $\heap$ means, which is captured by rules in
% Fig.~\ref{fig:const-type}. The variable type rule ($\textsc{T-Var}$)
% simply checks if a given variable has the defined type in $\Gamma$;
% the constant rule ($\textsc{T-Const}$) is slightly more involved.
% First, it ensures that the type annotation $\tau$ does not contain any
% free variables. More importantly, it ensures that the pointer points
% to a location that makes sense in a given heap.
%  
%  
%  The $\size$ function in Fig.~\ref{fig:const-type}
% refers to the \code{sizeof} function in C computing the number of
% bytes for a type.
%  
%  
%  Second, we
% require that any constant ($\evalue{n}{\tau}$) should make sense in
% $\heap$. We develop a recursive predicate $\sigma \vdash n : \tau$ to
% verify if $n$ has $\tau$ in a heap snapshot $\heap$. $\sigma$ is a
% constant set containing the constants that have been verified by the
% relation. For every constant $\evalue{n}{\tau}$, it is either an
% integer $\tint$, an unchecked pointer $\tptr{\omega}{\umode}$,
% zero-valued number ($n=0$), checked in $\sigma$
% ($\evalue{n}{\tptr{\omega}{\cmode}}\in \sigma$); or if it is not the
% above case, then (i) $\heap(n)$ is defined, and (ii) for every heap
% location $n+i$ in the range of the pointer (if $\omega$ is a word
% type, range is $[0,1)$; if $\omega$ is an array type
%   ($\tarray{0}{b_h}{\tau'}$), range is $[0,b_h)$, if $\tau$ is a
%     NT-array type ($\tntarray{0}{b_h}{\tau'}$), range is $[0,b_h+1)$),
%       if $\heap(n+i)=\evalue{n_a}{\tau_a}$, then
%       $\evalue{n_a}{\tau_a}$ satisfies $\sigma \cup \{(n,\tau) \}
%       \vdash n_a : \tau_a$.
%  
%  
% \begin{figure}[t]
% {\small
% \text{Type Rules for Constants and Variables:}
% \begin{mathpar}
%   \inferrule[T-Var]
%       {x : \tau \in \Gamma}
%       {\Gamma;\Theta \vdash_m x : \tau}
%  
%   \inferrule[T-Const]
%       {\fv(\tau) = \emptyset \\ \emptyset \vdash n : \tau}
%       {\Gamma;\Theta\vdash_m \evalue{n}{\tau} : \tau}
% \end{mathpar}
%     
% \text{Rules for Checking Constant Pointers In Heap:}
% \begin{mathpar}
%   \inferrule
%       {}
%       {\sigma \vdash n : \tint}
%  
%   \inferrule
%       {}
%       {\sigma \vdash n : \tptr{\omega}{\umode}}
%  
%   \inferrule
%       {}
%       {\sigma \vdash 0 : \tptr{\omega}{\cmode}}
%  
%   \inferrule
%       {\evalue{n}{\tptr{\omega}{\cmode}}\in \sigma}
%       {\sigma \vdash n : \tptr{\omega}{\cmode}}
%  
%   \inferrule
%       {\forall i \in [0,\size(\omega)) .
%            \sigma \cup \{(n:\tptr{\omega}{\cmode}) \}\vdash \heap(n+i)}
%       {\sigma \vdash n : \tptr{\omega}{\cmode}}
% \end{mathpar}
% }
% \caption{Type Rules for Checking Constants/Variables}
% \label{fig:const-type}
% \end{figure}

% \review{
%  Theorem 1 refers to a program $e$ being well-formed. Unless I've missed
%   something, I didn't see such a definition in the paper.}
% \mwh{This was stale text (dropped); $e$'s well formedness follows from the
%   assumption of well typing; we have added more details about that.}

In this subsection, we focus on our main meta-theoretic results about
\lang: type soundness (progress and preservation),
non-exposure, and non-crashing.
These proofs have been conducted in our Coq model.
Type soundness relies on several \emph{well-formedness} given in \ref{li22checkedc} and \Cref{appx:add-type-sem}.
Progress states that a \lang program can always make a move:

\begin{thm}[Progress]\label{thm:progress}

For any \lang program $e$, heap $\heap$, stack
$\varphi$, type environment $\Gamma$, and variable predicate set $\Theta$
that are all are well-formed, consistent
($\Gamma;\Theta\vdash \varphi$ and $\heap \vdash \varphi$) and well
typed ($\Gamma;\Theta\vdash_{\cmode} e : \tau$ for some $\tau$),
one of the following holds:

\begin{itemize}

\item $e$ is a value ($\evalue{n}{\tau}$).

\item there exists $\varphi'$ $\heap'$ $r$, such that $(\varphi,\heap,e) \longrightarrow_m (\varphi',\heap',r)$.

\end{itemize}
\end{thm}
%{\em Proof:} By induction on the typing derivation.

\noindent
There are two forms of preservation regarding the checked and unchecked regions.
Checked Preservation states that a reduction step preserves both the
type and consistency of the program being reduced.
Unchecked Preservation states that any evaluation happens at unchecked region does not affect the checked heap.

\begin{thm}[Checked Preservation]
For any \lang program $e$, heap $\heap$, stack
$\varphi$, type environment $\Gamma$, and variable predicate set $\Theta$
that are all are well-formed, consistent
($\Gamma;\Theta\vdash \varphi$ and $\heap \vdash \varphi$) and well
typed ($\Gamma;\Theta\vdash_{\cmode} e : \tau$ for some $\tau$), if there exists $\varphi'$,
$\heap'$ and $e'$, such that $(\varphi,\heap,e)
\longrightarrow_{\cmode} (\varphi',\heap',e')$, then $\heap'$ is
checked region consistent with $\heap$ ($\heap \triangleright \heap'$) and there exists
$\Gamma'$ and $\tau'$ that are well formed, checked region consistent
($\Gamma';\Theta\vdash \varphi'$ and $\heap' \vdash \varphi'$) and
well typed ($\Gamma';\Theta \vdash_{\cmode} e: \tau'$), where
$\tau'\sqsubseteq_{\Theta} \tau$.
\end{thm}
%{\em Proof:} By induction on the typing derivation.
%\smallskip
\begin{thm}[Unchecked Preservation]
For any \lang program $e$, heap $\heap$, stack
$\varphi$, type environment $\Gamma$, and variable predicate set $\Theta$
that are all are well-formed and well
typed ($\Gamma;\Theta\vdash_{\cmode} e : \tau$ for some $\tau$), if there exists $\varphi'$,
$\heap'$ and $e'$, such that $(\varphi,\heap,e)
\longrightarrow_{\umode} (\varphi',\heap',e')$, then $\heap'(\cmode)=\heap(\cmode)$.
\end{thm}

Using the above theorem, we first show the non-exposure theorem,
where code in unchecked region cannot observe a valid checked pointer address.

\begin{thm}[Non-Exposure]
For any \lang program $e$, heap $\heap$, stack
$\varphi$, type environment $\Gamma$, and variable predicate set $\Theta$
that are all are well-formed and well
typed ($\Gamma;\Theta\vdash_{\cmode} e : \tau$ for some $\tau$), if there exists $\varphi'$,
$\heap'$ and $e'$, such that $(\varphi,\heap,e)
\longrightarrow_{\umode} (\varphi',\heap',e')$ and $e=E[\alpha(x)]$ and $\mode(E)=\umode$,
where $\alpha(x)$ is some expression (not $\echeckedtext$ nor $\euncheckedtext$) containing variable $x$; 
thus, it is not a checked pointer.
\end{thm}

We now state our main result, {\em non-crashing},
which suggests that a well-typed program can never be \emph{stuck} (expression
$e$ is a non-value that cannot take a step\footnote{Note that
  $\ebounds$ and $\enull$ are \emph{not} stuck expressions---they represent a
  program terminated by a failed run-time check. A program that tries to access $\heap{n}$
  but $\heap$ is undefined at $n$ will be stuck, and violates spatial
  safety.}).

% \review{- There appears to be a slight discrepancy between the blame theorem in Coq and the one in the paper: the paper mentions some e', which I believe should be r. Also, the Coq code has a further disjunct m=Unchecked in the conclusion.}
% \liyi{It is a typo. We will add the thing back that we show that either user uses a unchecked mode to evaluate $e$ or $e$ lives in a context that is an unchecked region. This is a bit due to the space limitation. The semantic rules allow users to input the mode $m$ of evaluating an expression, I just forgot to include the $m$ in the result of the proof statement. }

\begin{thm}[Non-Crashing]\label{thm:blame} For any \lang
  program $e$, heap $\heap$, stack
$\varphi$, type environment $\Gamma$, and variable predicate set $\Theta$
that are well-formed and consistent
($\Gamma;\Theta\vdash \varphi$ and $\heap \vdash \varphi$),
if $e$ is well-typed ($\varphi;\Theta\vdash_{\cmode} e :
\tau$ for some $\tau$) and there exists
$\varphi_i$, $\heap_i$, $e_i$, and $m_i$ for $i\in [1,k]$, such that
$(\varphi,\heap,e) \longrightarrow_{m_1} (\varphi_1,\heap_1,e_1)\longrightarrow_{m_2} ...\longrightarrow_{m_k} (\varphi_k,\heap_k,r)$, then $r$ can never be \emph{stuck}.
\end{thm}

% \review{ 
% My interpretation of section IV.C is that the authors have a pen-and-paper
%   proof of Theorem 4, and that random testing using the Redex models has been
%   used to gain confidence in such a proof. Is this correct?}
% \liyi{Yes, this is correct.}
% \mwh{No, not correct: We have no pen-and-paper proof, just the
%   model. Clarify here}
% showing that
% extensive random testing fails to falsify the bisimilarity
% property. \mwh{Better way to state the previous?}

In the \systemname compiler, we also restore the compiler simulation theorem, such that executing a \systemname program is simulated by executing the compiled LLVM program. The details are given in the \checkedc formalism \cite{li22checkedc} and \Cref{appx:comp1}. 

We formalize both the compilation procedure and the simulation
theorem in the semantic model we developed in Redex (same as our Coq model),
and then attempt to falsify it via Redex's support for random
testing. Redex allows us
  to specify compilation as logical rules (an extension
  of typing), but then execute it algorithmically to
  automatically test whether simulation holds. This process revealed
  several bugs in compilation and the theorem statement.
%
  % us gain confidence that our original pen and paper proof of
  % simulation remains true with the addition of variable bounds. }
We ultimately plan to prove simulation in the Coq model.

%Turning to the simulation theorem: We first introduce notation
%used to specify the theorem.
We use the notation $\gg$ to
indicate the \emph{erasure} of stack and heap---the rhs is the same as
the lhs but with type annotations removed:
\begin{equation*}
  \begin{split}
    \heap  \gg & \dot \heap \\
    \varphi \gg & \dot \varphi
  \end{split}
\end{equation*}
In addition, when $\Gamma;\emptyset\vdash
\varphi$ and $\varphi$ is well-formed, we write $(\varphi,\heap,e) \gg_m (\dot \varphi, \dot \heap,
\dot e)$ to denote $\varphi \gg \dot \varphi$, $\heap \gg \dot \heap$
and $\Gamma;\Theta;\emptyset \vdash_m e \gg \dot e : \tau$ for some $\tau$ respectively. $\Gamma$ is omitted from the notation since the well-formedness of $\varphi$ and its consistency with respect to $\Gamma$ imply that $e$ must be closed under $\varphi$, allowing us to recover $\Gamma$ from $\varphi$.
Finally, we use $\xrightarrow{\cdot}^*$ to denote the transitive closure of the
reduction relation of $\elang$. Unlike the $\lang$, the semantics of
$\elang$ does not distinguish checked and unchecked regions.

Fig.~\ref{fig:checkedc-simulation-ref} gives an overview of 
the simulation theorem.\footnote{We ellide the  possibility of $\dot e_1$ evaluating to $\ebounds$ or $\enull$ in the diagram for readability.} The simulation theorem is specified in a way
that is similar to the one by~\citet{merigoux2021catala}.

An ordinary simulation property would
replace the middle and bottom parts of the figure with the
following: \[(\dot \varphi_0, \dot \heap_0, \dot e_0) 
  \xrightarrow{\cdot}^* (\dot \varphi_1, \dot \heap_1, \dot e_1)\]
Instead, we relate two erased configurations using the relation $\sim$,
which only requires that the two configurations will eventually reduce
to the same state.

% The two theorems are translation preservation and simulation. We donate $\xrightarrow{c}$ as the transition semantics of CLight.
\begin{thm}[Simulation ($\sim$)]\label{simulation-thm}
For \lang expressions $e_0$, stacks $\varphi_0$, $\varphi_1$, and heap snapshots $\heap_0$, $\heap_1$, 
if $\heap_0 \vdash \varphi_0$, $(\varphi_0,\heap_0,e_0)\gg_c (\dot \varphi_0,\dot \heap_0, \dot e_0)$,
and if there exists some $r_1$ such that $(\varphi_0, \heap_0, e_0)
\rightarrow_c (\varphi_1, \heap_1, r_1)$, then the following facts hold:

\begin{itemize}

\item if there exists $e_1$ such that $r=e_1$ and $(\varphi_1, \heap_1, e_1) \gg (\dot \varphi_1, \dot \heap_1, \dot e_1)$, then there exists some $\dot \varphi$,$\dot \heap$, $\dot e$, such that
$(\dot \varphi_0, \dot \heap_0,\dot e_0) \xrightarrow{\cdot}^* (\dot
\varphi,\dot \heap,\dot e)$ and $(\dot
\varphi_1,\dot \heap_1,\dot e_1) \xrightarrow{\cdot}^* (\dot \varphi,
\dot \heap,\dot e)$.

\item if $r_1 = \ebounds$ or $\enull$, then $(\dot \varphi_0, \dot \heap_0,\dot e_0) \xrightarrow{\cdot}^* (\dot
\dot \varphi_1,\dot \heap_1, r_1)$ where $\varphi_1 \gg \dot
\varphi_1$, $\heap_1 \gg \dot \heap_1$.

\end{itemize}
\end{thm}


% when $r_1 = e_1$ for
% some $e_1$ and
% $(\varphi_1, \heap_1, e_1) \gg (\dot \varphi_1, \dot \heap_1, \dot e_1)$, then
% there exists some $\dot \varphi$,$\dot \heap$, $\dot e$, such that
% $(\dot \varphi_0, \dot \heap_0,\dot e_0) \xrightarrow{\cdot}^* (\dot
% \varphi,\dot \heap,\dot e)$ and $(\dot
% \varphi_1,\dot \heap_1,\dot e_1) \xrightarrow{\cdot}^* (\dot \varphi,
% \dot \heap,\dot e)$. When $r_1 = \ebounds$ or $\enull$, we have $(\dot \varphi_0, \dot \heap_0,\dot e_0) \xrightarrow{\cdot}^* (\dot
% \dot \varphi_1,\dot \heap_1, r_1)$ where $\varphi_1 \gg \dot
% \varphi_1$, $\heap_1 \gg \dot \heap_1$.

% \begin{thm}[Simulation ($\sim$)]\label{simulation-thm}
% For \lang expressions $e_0$, stacks $\varphi_0$, $\varphi_1$, and heap snapshots $\heap_0$, $\heap_1$, 
% if $\emptyset;\emptyset;\emptyset \vdash_\cmode e_0 \gg \dot e_0 :\tau_0$,
% and if there exists some $r_1$ such that $(\varphi_0, \heap_0, e_0)
% \rightarrow_\cmode (\varphi_1, \heap_1, r_1)$, when $r_1 = e_1$ for
% some $e_1$ and
% $\emptyset;\emptyset;\emptyset \vdash_\cmode e_1 \gg \dot e_1 :\tau_1$ where $\tau_1 \sqsubseteq \tau_0$
% , then
% there exists some $\dot \varphi$,$\dot \heap$, $\dot e$, such that
% $(\dot \varphi_0, \dot \heap_0,\dot e_0) \xrightarrow{\cdot}^* (\dot
% \varphi,\dot \heap,\dot e)$ and $(\dot
% \varphi_1,\dot \heap_1,\dot e_1) \xrightarrow{\cdot}^* (\dot \varphi,
% \dot \heap,\dot e)$. When $r_1 = \ebounds$ or $\enull$, we have $(\dot \varphi_0, \dot \heap_0,\dot e_0) \xrightarrow{\cdot}^* (\dot
% \dot \varphi_1,\dot \heap_1, r_1)$ where $\varphi_1 \gg \dot
% \varphi_1$, $\heap_1 \gg \dot \heap_1$.
% \end{thm}

As our random generator never generates
$\euncheckedtext$ expressions (whose behavior could be undefined), we can only test a the simulation theorem 
as it relates to checked code. This limitation makes it
unnecessary to state the other direction of the simulation theorem
where $e_0$ is stuck, because Theorem~\ref{thm:progress} guarantees
that $e_0$ will never enter a stuck state if it is well-typed in
checked mode.

The current version of the Redex model has been tested against $27500$
expressions with depth less than $15$. Each expression can
reduce multiple steps, and we test simulation between every two
adjacent steps to cover a wider range of programs, particularly the
ones that have a non-empty heap.

\begin{figure}[t]
{\small
\[
\begin{array}{c}
\begin{tikzpicture}[
            > = stealth, % arrow head style
            shorten > = 1pt, % don't touch arrow head to node
            auto,
            node distance = 3cm
        ]

\begin{scope}[every node/.style={draw}]
    \node (A) at (0,1.5) {$\varphi_0,\heap_0, e_0$};
    \node (B) at (4,1.5) {$\varphi_1, \heap_1 ,e_1$};
    \node (C) at (0,0) {$\dot \varphi_0, \dot \heap_0 ,\dot e_0$};
    \node (D) at (4,0) {$\dot \varphi_1, \dot \heap_1, \dot e_1$};
    \node (E) at (2,-1.5) {$\dot \varphi,\dot \heap ,\dot e$};
\end{scope}
\begin{scope}[every edge/.style={draw=black}]

    \path [->] (A) edge node {$\longrightarrow_\cmode$} (B);
    \path [<->] (A) edge node {$\gg$} (C);
    \path [<->] (B) edge node {$\gg$} (D);
    \path [dashed,<->] (C) edge node {$\sim$} (D);
    \path [dashed,->] (C) edge node {$\xrightarrow{\cdot}^*$} (E);
    \path [dashed,->] (D) edge node[above] {$\xrightarrow{\cdot}^*$} (E);
\end{scope}

\end{tikzpicture}
\end{array}
\]
}
\caption{Simulation between \lang and \elang }
\label{fig:checkedc-simulation-ref}
\end{figure}

%{\em Proof:} By induction on the number of steps of the \checkedc
%evaluation ($\longrightarrow_m^*$), using progress and preservation to
%maintain the invariance of the assumptions.

% \review{
%   add a paragraph that discusses what are the main changes from [21] in terms
%   of the technical development (if there are any), e.g., are there any new
%   challenges that needed to be solved while proving the blame theorem for this
%   paper's semantics?
% }
%  Compared to \citet{ruef18checkedc-incr}, proofs for
%  \lang were made challenging by the addition of dependently typed
%  functions and dynamic arrays, and the need to handle bounds widening for NT
%  array pointers. These features required changes in the runtime
%%  semantics (adding a stack, and dynamically changing bounds) and in
 % compile-time knowledge of them (to soundly typing widened bounds).

% \ignore{
% \begin{figure}[t!]
%   \begin{prooftree}
%     \hypo{\evalue{n}{\tau} \in \defscope}
%     \infer1[T-VConst]{\Gamma;\defscope \vdash_m \evalue{n}{\tau}  : \tau}
%   \end{prooftree}
%   \qquad


%   \begin{prooftree}
%     \hypo{
%       \begin{matrix}
%         \Gamma;\defscope \vdash_m e : \tau \\
%         \Gamma;\defscope \vdash_m e_1 : \tau_1 
%       \end{matrix}
%     }
%     \hypo{
%       \begin{matrix}
% \Gamma \vdash_m \tau_3 = \tau_1 \vee \tau_2 \\
%             \Gamma;\defscope \vdash_m e_2 : \tau_2
%           \end{matrix}
%         }
%     \infer2[T-If]{\Gamma;\defscope \vdash_m \eif{e}{e_1}{e_2} : \tau_3}
%   \end{prooftree} \\ \\

%   \begin{prooftree}
%     \hypo{
%       \begin{matrix}
%         \Gamma;\defscope \vdash_m x : \tptr[c]{(\tntarray{l}{0}{\tau})}, l \leq 0 \\
%         \Gamma, x : \tptr[c]{(\tntarray{l}{1}{\tau})};\defscope \vdash_m e_1 : \tau_1 \\
%         \Gamma;\defscope \vdash_m e_2 : \tau_2 \\
%         \Gamma \vdash_m \tau_3 = \tau_1 \vee \tau_2
%       \end{matrix}
%     }
%     \infer1[T-IfNT]{\Gamma;\defscope \vdash_m \eif{\estar{x}}{e_1}{e_2} : \tau_3}
%   \end{prooftree} \\ \\

%   \begin{prooftree}
%     \hypo{
%       \begin{matrix}
%         F(f) = \tau_{j}\;(x_0:\tau_0, \ldots, x_{j-1}:\tau_{j-1})\;e  \\
%         \Gamma; \defscope \vdash_m \tau_i[x_0,\ldots,x_{i-1} \mapsto e_0,\ldots,e_{i-1}] ~~~ 0 \leq i \leq j\\
%         \Gamma; \defscope \vdash_m e_i : \tau_i' ~~~ 0 \leq i \leq j\\
%         \Gamma; \defscope \vdash_m \subtype{\tau_i'}{\tau_i[x_0,\ldots,x_{i-1} \mapsto e_0,\ldots,e_{i-1}]}  ~~~ 0 \leq i < j
%       \end{matrix}
%     }
%     \infer1[T-VCall]{\Gamma; \defscope \vdash_m f(\overline{e}) : \tau_j[x_0,\ldots,x_{j-1} \mapsto e_0,\ldots,e_{j-1}]}
%   \end{prooftree} \\ \\
% % \inferrule*[lab=T-PtrC]
% % {
% %   \tau = \tptr[c]{\omega} \\
% %   \tau_0, ..., \tau_{j-1} = \mathrm{types}(D,\omega)\\\\
% %   \wt[\Gamma][\defscope, n^\tau]{H(n+k)}{\tau_k} ~~~ 0 \leq k < j
% % %  \Gamma,n^\tau \proves H(n+k) : \tau_k~~~0 \leq k < j
% % }



%   \begin{prooftree}
%     \hypo{
%           \Gamma; \defscope \vdash_m e_1 : \tau_1
%       }
%       \hypo{        \Gamma, x = e_1 : \tau_1; \defscope \vdash_m e_2 : \tau_2
% }
%     \infer2[T-Let]{\Gamma; \defscope \vdash_m \elet{x}{e_1}{e_2} : \tau[x \mapsto e_1]}
%   \end{prooftree} \\ \\


%   \begin{prooftree}
%     \hypo{
%       \begin{matrix}
%         \Gamma; \defscope \vdash_m \estrlen{y} : \tau_1\\
%         \Gamma; \defscope \vdash_m y : \tptr[c]{(\tntarray{le}{\_}{\tau_3})}\\
%         \Gamma, x = \estrlen{y} : \tau_1, \\ y : \tptr[c]{(\tntarray{le}{x}{\tau_3})} ;  \defscope  \vdash_m e_2 : \tau_2
%         \end{matrix}
%         }
%         \infer1[T-LetStr]{\Gamma; \defscope \vdash_m \elet{x}{\estrlen{y}}{e_2} : \tau[x \mapsto e_1]}
%   \end{prooftree} \\ \\

%   \begin{prooftree}
%     \hypo{
%         \Gamma; \defscope \vdash_m e : \tptr[c]{(\tntarray{le}{he}{\tau})}
%       }
%         \infer1[T-Str]{\Gamma; \defscope \vdash_m \estrlen{y} : \tint}
%       \end{prooftree} \\ \\

%       \begin{prooftree}
%         \hypo{
%           \begin{matrix}
%             \tau = \tint \vee \tau = \tptr[u]{\omega}~\vee  n=0~ \vee \\ \tau = \tptr[c]{(\tarray{0}{0}{\tau'})} \vee \\
%             \tau = \tptr[c]{(\tntarray{0}{0}{\tau'})}
%         \end{matrix}
%       }
%       \infer1[T-Base]{\Gamma; \defscope \vdash_m \evalue{n}{\tau}  : \tau}
%     \end{prooftree} \\ \\

%     \begin{prooftree}
%       \hypo{
%         \begin{matrix}
%         \tau = \tptr[c]{\omega} \\
%         \tau_0, ..., \tau_{j-1} = \mathrm{types}(D,\omega)\\
%         \Gamma;\defscope, \evalue{n}{\tau}  \vdash_m {H(n+k)} : {\tau_k} ~~~ 0 \leq k < j
%       \end{matrix}
%       }
%       \infer1[T-PtrC]{\Gamma; \defscope \vdash_m \evalue{n}{\tau}  : \tau}
%     \end{prooftree} \\ \\

%     \begin{prooftree}
%       \hypo{
%         \begin{matrix}
%             \Gamma; \defscope \vdash_m e :  {\tptr[m]{\tstruct{T}}} \\
%             % \Gamma \proves e : \tptr[m]{\tstruct{T}} \\\\
%             D(T) = ...; \tau_f~f; ...
%         \end{matrix}
%       }
%       \infer1[T-Amper]{\Gamma; \defscope \vdash_m \eamper{e}{f} : \tptr[m]{\tau_f}}
%     \end{prooftree} \\ \\


%     \todo[inline]{YL: how to express le - n as a metafunction?}
%     \begin{prooftree}
%       \hypo{
%         \begin{matrix}
%         \Gamma; \defscope \vdash_m e_1 : \tptr[m']{(\tgarray{\alpha}{le}{he}{\tau})}\\
%         \Gamma; \defscope \vdash_m \evalue{n}{\tau}  : \tint \\
%         le' = le - n, he' = he - n
%       \end{matrix}
%     }
%       \infer1[T-BinopInd]{\Gamma; \defscope \vdash_m (e_1 \plus \evalue{n}{\tau} ) : \tptr[m']{(\tgarray{\alpha}{le'}{he'}{\tau})}}
%     \end{prooftree}
% \caption{Typing}
% \label{fig:typing}
% \end{figure}

% \begin{figure}[t!]
%   \todo[inline]{no sizeof check because we don't know statically whether the allocation would be null}
%   \begin{prooftree}
%     \hypo{
%       \omega = \tgarray{\alpha}{le}{he}{\tau} \Rightarrow le = 0
%     }
%     \infer1[T-Malloc]{\Gamma; \defscope \vdash_m \emalloc{\omega} : \tptr[c]{\omega}}
%   \end{prooftree}\\\\

%   \begin{prooftree}
%     \hypo{
%       \Gamma; \defscope \vdash_u e : \tau
%     }
%     \infer1[T-Unchecked]{\Gamma; \defscope \vdash_m \eunchecked{e} : \tau}
%   \end{prooftree}\\\\

%   \todo[inline]{any constraints on m'?}
%   \begin{prooftree}
%     \hypo{\Gamma; \defscope \vdash_m e :  \tptr[m']{(\tgarray{\alpha}{le'}{he'}{\tau})} }
%     \infer1[T-DynCast]{\Gamma; \defscope \vdash_m \edyncast{\tptr[m']{(\tgarray{\alpha}{le}{he}{\tau})}}{e} : \tau}
%   \end{prooftree}\\\\
  
%   \begin{prooftree}
%     \hypo{\Gamma; \defscope \vdash_m e : \tau'}
%     \infer1[T-Cast]{\Gamma; \defscope \vdash_m \ecast{\tau}{e} : \tau}
%   \end{prooftree}\\\\



%   \begin{prooftree}
%     \hypo{
%       \begin{matrix}
%         \Gamma; \defscope \vdash_m e_1 : \tptr[m']{\tgarray{\alpha}{le}{he}{\tau}} \\
%         \Gamma; \defscope \vdash_m e_2 : \tint \\
%         m' = u \Rightarrow m = u \\
%       \end{matrix}
%     }
%     \infer1[T-Index]{\Gamma; \defscope \vdash_m \estar{(\ebinop{e_1}{e_2})} : \tau}
%   \end{prooftree} \\\\


%   \begin{prooftree}
%     \hypo{
%       \begin{matrix}
%         \Gamma; \defscope \vdash_m e_1 : \tptr[m']{\omega} \\
%         \Gamma; \defscope \vdash_m e_2 : \tau \\
%         m' = u \Rightarrow m = u \\
%         \omega = \tau \vee \omega = \tgarray{\alpha}{le}{he}{\tau}
%       \end{matrix}
%     }
%     \infer1[T-Assign]{\Gamma; \defscope \vdash_m \eassign{e_1}{e_2} : \tau}
%   \end{prooftree}\\\\


%   \begin{prooftree}
%     \hypo{
%       \begin{matrix}
%         \Gamma; \defscope \vdash_m e_1 : \tptr[m']{\omega} \\
%         \Gamma; \defscope \vdash_m e_2 : \tint \\
%         \Gamma; \defscope \vdash_m e_3 : \tau \\
%         m' = u \Rightarrow m = u \\
%         \omega = \tau \vee \omega = \tgarray{\alpha}{le}{he}{\tau}
%       \end{matrix}
%     }
%     \infer1[T-IndAssign]{\Gamma; \defscope \vdash_m \eassign{(e_1 \plus e_2)}{e_3} : \tau}
%   \end{prooftree}

% \caption{Typing Cont.}
% \label{fig:typing2}
% \end{figure}
% \ignore{





% \begin{figure}[t!]
%   \begin{prooftree}
%     \hypo{
%       \begin{matrix}
%         \Gamma \vdash x_0:\tau_0, \ldots, x_{j}:\tau_{j} \\
%         \Gamma, x_0 = \kw{none} : \tau_0, \ldots, x_{j-1} = \kw{none} : \tau_{j-1}; \defscope \vdash e : \tau_{r}
%       \end{matrix}
%     }
%     \infer1[WF-Fun]{\Gamma;\defscope \vdash \tau_{j}\;(x_0:\tau_0, \ldots, x_{j-1}:\tau_{j-1})\;e}
%   \end{prooftree} \\ \\

%   \begin{prooftree}
%     \infer0[WF-Nil]{\Gamma \vdash \circ}
%   \end{prooftree} \\\\

%   \begin{prooftree}
%     \hypo{
%       \begin{matrix}
%         \Gamma \vdash \tau_0 \\
%         \Gamma, x_0 = \kw{none} :\tau_0 \vdash x_1:\tau_{1}, \ldots, x_{j}:\tau_{j}
%       \end{matrix}
%     }
%     \infer1[WF-Cons]{\Gamma \vdash x_0:\tau_0, \ldots, x_{j}:\tau_{j}}
%   \end{prooftree} \\\\

%   \begin{prooftree}
%     \infer0[WF-Int]{\Gamma \vdash \tint}
%   \end{prooftree} \\\\

%   \begin{prooftree}
%     \hypo{   \Gamma \vdash le }
%      \hypo{   \Gamma \vdash he }
%       \hypo{  \Gamma \vdash \tau}

%     \infer3[WF-Array]{\Gamma \vdash \tptr[m]{(\tgarray{\alpha}{le}{he}{\tau})}}
%   \end{prooftree} \\\\

%   \begin{prooftree}
%     \hypo{T \in dom(D)}
%     \infer1[WF-Struct]{\Gamma \vdash \tptr[m]{\tstruct{T}}}
%   \end{prooftree} \\\\

%   \begin{prooftree}
%     \hypo{\Gamma \vdash \tau}
%     \infer1[WF-Ptr]{\Gamma \vdash \tptr[m]{\tau}}
%   \end{prooftree} \\\\
  
%   \begin{prooftree}
%     \infer0[WFB-Int]{\Gamma \vdash i}
%   \end{prooftree}\\\\
  
%   \begin{prooftree}
%     \hypo{x = e? : \tint \in \Gamma}
%     \infer1[WFB-Var]{\Gamma \vdash x \plus i}
%   \end{prooftree} \\\\

%   \caption{Well-Formedness}
%   \label{fig:wf}
% \end{figure}



% \begin{figure}[t!]
%   \begin{prooftree}
%     \infer0[Sub-Nt]{\subtype{\tptr[c]{(\tntarray{le}{he}{\tau})}}{\tptr[c]{(\tarray{le}{he}{\tau})}}}
%   \end{prooftree} \\ \\

%   \begin{prooftree}
%     \infer0[Sub-Refl]{\subtype{\tau}{\tau}}
%   \end{prooftree} \\ \\

%   \todo[inline]{what about ntarrays?}
%   \begin{prooftree}
%     \infer0[Sub-Ptr]{\subtype{\tptr[c]{\tau}}{\tptr{\tarray{0}{1}{\tau}}}}
%   \end{prooftree} \\ \\

%   \todo[inline]{the subtyping relation is not anti-symmetric..}
%   \begin{prooftree}
%     \infer0[Sub-Arr]{\subtype{\tptr[c]{\tarray{0}{1}{\tau}}}{\tptr{\tau}}}
%   \end{prooftree} \\ \\

%   \begin{prooftree}
%     \hypo{\rboundle{le_0}{le_1}}
%     \hypo{\rboundle{he_1}{he_0}}
%     \infer2[Sub-Subsume]{\subtype{\tptr{(\tgarray{\alpha}{le}{he}{\tau})}{\cmode}}{ \tptr{(\tgarray{\alpha}{ le_1} {he_1} {\tau})}}}
%   \end{prooftree} \\ \\

%   \begin{prooftree}
%     % check t-amper below
%     \hypo{D(T) = \tau_f~f; ...}
%     \hypo{\subtype{\tau_f}{\tau}}
%     \infer2{\subtype{\tptr[c]{\tstruct{T}}}{\tau}}
%   \end{prooftree}
%   \caption{Subtyping}
%   \label{fig:sub}
% \end{figure}
% }
% }

% \ignore{

% \begin{figure*}[t]
%   \begin{lstlisting}
% int foo(nt_array_ptr<char> p : count(0)) {
%   if (* p) {
%     dyn_bounds_cast<nt_array_ptr<char>>(p, count(1));
%   }
%   dyn_bounds_cast<nt_array_ptr<char>>(p, count(1));
%   return 0;
% }
%   \end{lstlisting}
% \caption{The example where clang Checked C fails at run-time}
% \label{fig:clangbad1}
% \end{figure*}

% \begin{figure*}[t]
%   \begin{lstlisting}
% /* nt_array_ptr<char> p : bounds(p,p) */
% size_t cnt = 0;

% while(*(p+cnt)) {
%   ++cnt;
% }
% dyn_bounds_cast<nt_array_ptr<char>>(p, count(cnt));
%   \end{lstlisting}
% \caption{A useful program that the Checked C spec doesn't allow at run-time}
% \label{fig:clangbad2}
% \end{figure*}
% }

% \ignore{
% The Clang CheckedC implementation uses statically determined
% bounds to insert run-time checks. In Fig.~\ref{fig:clangbad1}, the
% \code{dyn_bounds_cast} at line 3 will always succeed, because the
% compiler knows that within the scope of then branch, the pointer
% \code{p} must have at least one element. The same cast at line 5,
% however, will always fail, since there is no way to tell statically
% whether the program has entered the then branch before. The compiler
% will check whether the \code{count(1)} bounds specification is
% contained within the earlier \code{count(0)} specification, resulting
% in a run-time failure even when we pass in a non-empty string.

% Our formalization diverges from this run-time behavior and instead keeps
% track of the bounds on the stack. After entering the then branch, we
% increment the upper bound for \code{p}, effectively making the
% updated bounds information available even after we exit the if
% statement. The cast at line 5 will be checking the new bounds against
% the incremented bounds for non-empty strings.

% Fig.~\ref{fig:clangbad2} gives a more practical example of why keeping
% track of the bounds on the stack is useful. The program snippet
% implements the functionality of the \code{strlen} function using a
% while loop and a \code{cnt} variable. Even though the type system is
% unable to reason about the while loop, as long as the runtime system
% updates the bounds in-place, the user can apply a
% \code{dyn_bounds_cast} to soundly recover the more precise bounds
% information.
% }


% \inferrule*[lab=T-Amper]
% {
%   \Gamma \proves e : \tptr[m]{\tstruct{T}} \\\\
%   D(T) = ...; \tau_f~f; ...
% }
% {
%   \Gamma \proves \eamper{e}{f} : \tptr[m]{\tau_f}
% }
