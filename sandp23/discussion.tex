\section{Difference in the \checkedc Formalization}\label{sec:discussion}

\begin{figure}[t]
{\small
{\captionsetup[lstlisting]{margin = 8 mm}

 \begin{lstlisting}[xleftmargin=8 mm]
void memcpy (nt_array_ptr<int> a : count(0),
         nt_array_ptr<int> b : count(0)) {
   while (* a) {
     if (* b)
        * a = * b;
     else break;
     a++; b++;
   }
}
  \end{lstlisting}


}
}
\caption{Bound Widening Examples}
\label{fig:bound-widening}
\end{figure}

% In the left of program \textbf{(c)} (Fig.~\ref{fig:checkedc-example}), we 
% assume that the lower/upper bounds of \code{x} are given as \code{0} in the beginning of the program. 
% In a fat-pointer framework, this program might not reach failure because the \code{x} type is adjusted dynamically and the upper bound keeps growing.
% In a classical language that guarantees spatial safety statically, however, this program results in dynamic bound check errors, because once we access \code{x} and increase the pointer address by $1$, the pointer reaches the upper bound given by the type. 

There are some differences between our \checkedc formalization and the \checkedc specification/implementation. One of the examples is the use of $\Theta$ in the type system. The difference has been described in Sec.~\ref{subtype-type}.
The other difference is the bound widening scope of the branching operations.
Bound widening refers to that when we are accessing the final element in an NT-array indicated by the upper bound of the NT-array, we find that the element is not null; clearly, there is more elements after the element, so the upper bound can extended dynamically and we are able to access the elements after the final one.
In a fat-pointer framework, this is easy to implement. In a static bound system, however, there is impossible to implement the bound widening behaviors in any context because all bound information is statically guided. 
The \checkedc specification allows bound widening behaviors in a limited format. 
\code{strncat} in Fig.~\ref{fig:strncat-ex} is an example bound widening in \checkedc. Without the \texttt{strlen} usage in line 5, \code{b}'s upper bound is not extended to \code{x}. In that case, line 13 results in a $\ebounds$ error.

Our formalization is different from the \checkedc specification and implementation.
One cool thing that our formalization can do is a second version of the \code{memcpy} implementation in Fig.~\ref{fig:bound-widening}. 
This function takes two NT-array pointers, and copies the data in NT-array \code{b} to \code{\a} as mush as we can until we hit a null-character in either \code{a} or \code{b}.
The current \checkedc specification and compiler do not support this program, but they have difference. 
The \checkedc specification allows bound widening in the scope of the \code{true} branch, meaning that the \code{b}'s upper bound is $1$ in line 5 (in the \code{true} branch) in the first loop iteration; so does the pointer \code{a} in the while loop. In each iteration, \code{a}'s upper bound is extended by $1$. The problem is that the scope of the bound widening in the specification is lost after jumping out of the branching scope. After line 5, \code{b}'s upper bound is back to $0$. The \code{b++} in line 7 shifts \code{b}'s pointer address; thus, in the second iteration, the line 5 code results in a $\ebounds$ error because \code{b}'s dereference address is bigger than the upper bound. 
On the other hand, the \checkedc compiler results in a $\ebounds$ error for \code{a}'s pointer in line 3 in the second iteration, because the compiler's bound widening behavior in a \code{while} loop is only restricted to a local loop body, and does not extended to further loop iteration. 
Our formalization extends the bound widening behavior after the \code{true} branch scope; thus, executing the \code{memcpy} in Fig.~\ref{fig:bound-widening} does not reach an runtime error, and NT-array \code{a} copies \code{b}'s data as long as the length of \code{a} is greater than \code{b}. 

We implement this behavior different from both the \checkedc specification and compiler because of two reasons. First, the \checkedc specification and compiler disagree here so there is no formally "good" behavior. Second, we believe that this formalization permits more interesting programs, and sometimes, more efficient programs. 

In our compiler implementation, we utilize ghost variables, which is introduced in Sec.~\ref{sec:compilation}, to equalize the behavior in our formalization and implementation. Thus, the behavior of our formalization and implementation matches tightly (proved by Theorem~|ref{simulation-thm}).
The lack of shadow variables become more of an issue when we extend the
\checkedc specification with other bounds widening primitives such as \code{strlen}. 
For example, if users are not careful and move the \code{strlen} expression for pointer \code{y} in line 5 to later  are not able to Fig.~\ref{fig:strncat-ex}, it might causes the compiler ill-recognize the scope of the \code{strlen} bound widening effect; thus, the execution in line 13 fails.
With the shadow variables, it certainly allows us to mutate \code{y}'s bound widening scope and designs more effective compilation strategy to permit more intereting programs, like Fig.~\ref{fig:bound-widening}.
The use of shadow variables also allow an easy design of the compiler. 
complex static reasoning can be
difficult to implement. At the time of writing, the Clang \checkedc
implementation does not yet support the type of static reasoning to
allow functions like \code{memcpy} in Fig.~\ref{fig:bound-widening} to work properly.
Additionally, forcing the
user to declare a bound variable explicitly and guide the typechecker
to prove that it's a valid bound can result in extremely verbose
code.
Finally, shadow variables resides in the stack so they do not create any significant overhead for the execution of a program. 


\liyi{below are more wording from Yiyun that I think we can drop. Below means to the end of the section. }

The
 \code{strcat\_checked} function from Fig.~\ref{fig:bound-widening}
 takes an extra argument \code{n} that represents the maximum capacity
 of the NT-array \code{s0}. We first use the \code{strlen} function to
 locate the end of \code{s0}, and then start copying from \code{s1} to
 \code{s0} through a while loop. With the \code{strlen} widening
 extension introduced by the Clang \checkedc compiler, the bounds of
 \code{s0} will be casted into \code{count(i)} after
 line 14. This well-intended casting behavior can end up shrinking the
bounds of \code{s0} when \code{strlen(s0)} is smaller than \code{n},
which is almost always the case when the \code{strcat\_checked}
function is called. This will force the programmer to expand the
call to \code{strlen} into an explicit \code{while} loop to prevent
the \checkedc compiler from casting the bounds, adding unnecessary
complexity. On the other hand, with shadow variables, we ensure at
runtime the ghost variable representing the upper bound can only get
larger. In our formal semantics, the
ghost variable will be updated only if the return value of \code{strlen} is
greater than the variable's current value. The subsequent indexing
into the array will thus be checked against the upper bound that is equal
to \code{max(n,i)}, avoiding the runtime error.
Sadly, while our formal model is better-behaved at runtime, we do not
leverage the existence of shadow variables in our type
system. This is because the original \checkedc specification is
not designed around the idea of shadow variables. If we were to extend the
specification with shadow variables, we could expose special syntax for
programmers to read those variables so they don't have to rely on
runtime casts to indirectly retrieve that information. In the type
system, we can use the ghost variable to represent the largest known
bounds of an array and extend the subtyping rules to allow shrinking
to smaller bounds that can be inferred statically. 

The other significant difference is the use of shadow variables in our implementation of the \checkedc compilation.
The usefulness of shadow variables is not immediately obvious. In
this section, we provide examples that demonstrate the limitation of the
current \checkedc specification due to the lack of shadow variables.

In Fig.~\ref{fig:bound-widening}, we define the function \code{foo}
which takes an NT-array pointer of size $0$ as its argument. The
function initializes the variable \code{cnt} to \code{0} and
increments \code{cnt} only if the value pointed by \code{p} is not a
null character. Finally, after exiting the if statement, we attempt to
dynamically cast bounds of \code{p} to be \code{count(cnt)} before
returning.


It is also worth mentioning that the bookkeeping of shadow variables do
introduce hidden overhead, compared to an alternative approach that
relies on static proofs. In the \checkedc specification, it is
possible to take advantage of its data-flow analysis and assign more
precise bounds without relying on shadow variables. The function
\code{bar} from Figure.~\ref{fig:bound-widening} implements the same
functionality as \code{foo} but includes additional static bounds casting in the
\code{if} statement. The type system can prove that
\code{count(cnt)} is a valid bounds annotation after the if statement,
avoiding the overhead of an additional \code{dynamic_cast} as in \code{foo}.

However, we argue that there are strong reasons that justify
why we should include shadow variables. First,  Second,  \yiyun{I'm not familiar with the conversion tool. is this accurate?} This also
defeats the purpose of using \checkedc for gradually migrating legacy C code,
which, in practice, is more likely done by automatic
conversion tools that are unable to perform any sophisticated
theorem proving. Finally, the introduction of shadow variables can
always be made optional or simply used in conjunction with a static reasoning
system. \yiyun{I think I phrased this rather poorly. Would be helpful
  if someone could help me rephrase this a little} We expect to introduce
shadow variables to the \checkedc specification but defer some of the
more subtle design decisions that might have an impact on performance
until we receive more benchmark results from existing C to \checkedc
conversion tools \yiyun{should I cite the 3C paper here?}.
\ignore{
As we discuss in Section~\ref{sec:overview}, the \checkedc compiler
is able to statically determine that the upper bound of \code{p} must
be at least \code{1} in the then branch of the if statement. However,
once the program exits the if statement, the upper bound of \code{p}
is conservatively reverted back to \code{0}. Without using ghost
variables to keep track of the updated bounds, the \checkedc compiler
would attempt to check whether \code{count(cnt)} is included within
the statically determined imprecise bounds \code{count(0)} at line 6,
causing a runtime error when we pass in a non-empty string.
}




%\code{strcat_checked} function 

% This limitation can also affect programs that are more practical. For
% example, at Line 16, we compute \code{strlen(p)} manually through a
% \code{while} loop. The \code{dynamic_bounds_cast} fail for the reason
% as the one we see in \code{foo}.



% The Clang \checkedc implementation uses statically determined
% bounds to insert runtime checks. In the left of Fig.~\ref{fig:checkedc-example} \textbf{(c)}, the
% \code{dynamic_bounds_cast} at line 3 will always succeed, because the
% compiler knows that within the scope of then branch, the pointer
% \code{p} must have at least one element. The same cast at line 6,
% however, will always fail, since there is no way to tell statically
% whether the program has entered the then branch before. The compiler
% will check whether the \code{count(1)} bounds specification is
% contained within the earlier \code{count(0)} specification, resulting
% in a runtime failure even when we pass in a non-empty string.

% Our formalization diverges from this runtime behavior and instead keeps
% track of the bounds on the stack. After entering the then branch, we
% increment the upper bound for \code{p}, effectively making the
% updated bounds information available even after we exit the if
% statement. The cast at line 6 will be checking the new bounds against
% the incremented bounds for non-empty strings.

% In our formalization, every pointer literal is annotated with bounds
% information. During compilation, the bounds annotations are converted
% into shadow variables that are updated during widening and accessed
% for runtime checks. This allows our formal model to successfully
% run functions defined like \code{foo}. However, the \checkedc
% specification is designed based on the absence of ghost
% variables and therefore does not provide primitive operations that
% allow programmers to interact with them.




% Fig.~\ref{fig:checkedc-example} \textbf{(e)} gives a more practical example of why keeping
% track of the bounds on the stack is useful. The program snippet
% implements the functionality of the \code{strlen} function using a
% while loop and a \code{cnt} variable. Even though the type system is
% unable to reason about the while loop statically (which would require
% inductive reasoning), as long as the runtime system updates the bounds
% in-place, the user can apply a
% \code{dynamic_bounds_cast} to soundly recover the more precise bounds
% information.

% \yiyun{That's actually not right. The fresh bounds variables are generated during the creation of x, before strlen is called. I'll rewrite this later.}
% The \code{strlen} formalization is a special case of the branching operations with an NT-array pointer Boolean guard.
% In this case, \checkedc generates a new fresh local variable in the local stack in its type rule, which records the length from the current position of a pointer (\code{x}) to the first null-character. We do not know the value of \code{x} in the type checking stage; instead, in the compiler, \checkedc insert dynamic checks of \code{x} into necessary places that use the upper bound information of \code{x}.
% Fig.~\ref{fig:checkedc-example} \textbf{(b)} calls \code{strlen} to get the length of the NT-array pointer \code{x}. Since the input lower/upper bounds for \code{x} are \code{0}, the result of executing \code{strlen} is always the length between the current position of \code{x} and the first null character in the list because the length is always greater than \code{0}. In general, calling \code{strlen} on an NT-array pointer might not return the above size. If the the length between the current position of \code{x} and the first null character is less then the high-bound size of the pointer type, then the result is the greater of the two.

% \yiyun{strcmp would be a useful example here? It would be assigned a signature \lstinline{nt_array_ptr<char> p0:bounds(p0,p0)}, \lstinline{nt_array_ptr<char> p1:bounds(p1,p1)} }
% \paragraph{\textbf{Subtyping and Casting Operations}} 
% It is a common usage in a program to use a large size array as a smaller one, i.e., using a \code{array_ptr<int> : count(5)} typed array as a \code{array_ptr<int> : count(2)} typed one. \checkedc provides a static casting operation $\ecast{\tau}{x}$ to cast a subtype into its supertype. Every type checked casting operation is always executed successfully without causing any error states and with a "zero" cost. One example use of static casting is given in \code{strncat} in Fig.~\ref{fig:strncat-ex}.
% The \code{strlen(a)} in line 4 updates the upper-bound of \code{a} to be \code{x}. Then, the statement \code{(nt_array_ptr<int> : count(0)) a} casts the upper-bound of pointer \code{a} from \code{x} to \code{0}.

% \yiyun{TODO: Rewrite using the static bound cast syntax from page 39 of the checekdc spec. Also, this type of static reasoning that the n >= 0 is easy to do, but I don't think it's implemented in Redex. Maybe we should at least mention this strlen extension is not part of the checkedc spec?}
% \liyi{described in formal typing system.}

\mwh{HERE IS OLD TEXT FROM BACKGROUND. MIGHT GO HERE}

\yiyun{I rewrote this paragraph to express more explicitly why the static bounds are not precise.}
The bounds inferred at compile time are not always precise for null-terminated arrays. \yiyun{Should we rephrase this or remove the same example from the introduction?} Consider the statement \lstinline|if (*p) {..._0} {..._1} ..._2|
where \lstinline{p}'s type is \lstinline{nt_array_ptr<T> count(0)}, the type checker can infer that \code{p} has the bounds annotation \code{count(1)} in \lstinline{..._1} because the \lstinline{*p} cannot possibly be the null terminator in the then branch. However, what type should we assign to \lstinline{p} at \lstinline{..._2}? With the limited expressiveness of the \checkedc type system, it can only soundly infer the more conservative bounds annotation \lstinline{count(0)}.
We will later discuss how we keep track of the updated bounds information at runtime and also allow programmers to reflect the updated bounds to the type system via dynamic bounds casts.


\yiyun{move the following text somewhere else}
It is still possible to keep track of the updated bounds information at runtime.
To do so, we store the upper bound of the pointer \lstinline{p} as a stack variable. Inside the then branch, the stack variable will be incremented, effectively tracking the information even after the program exits the if statement. Later, we will see an practical example that makes use of this extra bit of information.

\yiyun{I want to justify our decision of using stack variables by pointing out the lack of clarity of the checkedc spec on the runtime behavior. Should I include this discussion in a later section?}
It is worth noting that the original \checkedc specification is very unclear ...
% The above pointer bound description discusses the case when we know exactly the lower/upper bounds of a pointer $p$. The pointer bound information of \checkedc is detected through the \checkedc type system and statically inserted bound checks. It is unlikely we can always know the exactly bounds for a pointer $p$. 
% For example,
% a given array/NT-array pointer \code{a} is declared as \code{array_ptr<int> a : count(x)}/\code{nt_array_ptr<int> a : count(x)}. The \code{count(x)} here means that the lower-bound of the pointer is \code{0}, and the upper-bound is \code{x}.
% For an array pointer, even though the bound values of the lower/upper bounds (\code{0} and \code{x}) can change during evaluation, the array length (\code{x-0}) is fixed.

% For a NT-array pointer, the situation is different. 
% The interpretation of the bounds of an NT-array pointer is similar to an array pointer, but the
% range can extend further to the right, until a null terminator
% is reached (i.e., the null is not within the bounds).
% During this process, it is possible that the upper bound \code{x} is extended and the NT-array length is "extended" due to the discovery of the upper-bound array position is not a null character.
