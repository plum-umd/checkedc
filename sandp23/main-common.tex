%\documentclass[conference,compsoc]{IEEEtran}
%
%\makeatletter
%\def\endthebibliography{%
%  \def\@noitemerr{\@latex@warning{Empty `thebibliography' environment}}%
%  \endlist
%}
%\makeatother
%
%\IEEEpeerreviewmaketitle

\documentclass[sigconf, anonymous]{acmart}

\fancyhf{} % Remove fancy page headers 
\fancyhead[C]{Anonymous submission \#9999 to ACM CCS 2023} % TODO: replace 9999 with your paper number
\fancyfoot[C]{\thepage}

\setcopyright{none} % No copyright notice required for submissions
\acmConference[Anonymous Submission to ACM CCS 2023]{ACM Conference on Computer and Communications Security}{}{}
\acmYear{2023}

\settopmatter{printacmref=false, printccs=true, printfolios=true} % We want page numbers on submissions

%\usepackage{filecontents}
%\usepackage[noadjust]{cite}
\usepackage{amsmath}
%\usepackage{amssymb}
\usepackage{ebproof}
\usepackage{todonotes}
\usepackage{amsfonts}
\usepackage{algorithmic}
\usepackage{tikz}
\usepackage{tikz-network}
\usepackage{fontawesome}
\usepackage{textcomp}
\usepackage{mathpartir}
\usepackage{color}
\usepackage{xcolor}
\DeclareMathAlphabet{\mathpzc}{OT1}{pzc}{m}{it}
\usepackage{cleveref} % must be loaded after hyperref and amsmath
%% Some recommended packages.
\usepackage{subcaption} %% For complex figures with subfigures/subcaptions
                        %% http://ctan.org/pkg/subcaption
\usepackage{epsfig}
\usepackage{listings}
\usepackage[override]{cmtt}
%\usepackage{amsthm}
\usepackage{graphicx}
\usepackage{balance}
\usepackage{xspace}
\usepackage{booktabs}
\usepackage{url}
% Optional argument magic
\usepackage{xparse}
\usepackage{enumitem}
%\input{header.tex}
%
\usepackage{minted}
\usepackage{fontawesome}
\usepackage{tikz}
\usepackage{tikz-network}
\usepackage{xspace}
\usemintedstyle{emacs}

\newcommand{\code}[1]{%
  \mintinline[fontsize=\footnotesize{},mathescape, escapeinside=||]{c}{#1}%
}

\definecolor{taintcolor}{rgb}{0.87, 0.36, 0.51}

\newcommand{\tbl}[1]{Table~\ref{#1}}
\newcommand{\sect}[1]{Section~\ref{#1}}
\newcommand{\fig}[1]{Figure~\ref{#1}}
\newcommand{\lst}[1]{Listing~\ref{#1}}
\newcommand{\algo}[1]{Algorithm~\ref{#1}}
\newcommand{\apdx}[1]{Appendix~\ref{#1}}

\newcommand{\realbug}{\textcolor{red}{\faBug}}
\newcommand{\complexcode}{\faChainBroken}
\newcommand{\entrypoint}{\faForward}
\newcommand{\rootcause}{\textcolor{orange}{\faWarning}}
\newcommand{\useradded}{\faUserPlus}
\newcommand{\usermods}{\faUser}

\newcommand{\ptr}{\ensuremath{ptr}}
\newcommand{\arr}{\ensuremath{arr}}
\newcommand{\ntarr}{\ensuremath{ntarr}}
\newcommand{\ptrT}[1]{\code{ptr<#1>}}
\newcommand{\arrT}[1]{\code{array_ptr<#1>}}
\newcommand{\ntarrT}[1]{\code{nt_array_ptr<#1>}}
\newcommand{\ucregion}{\ensuremath{uc-region}\xspace}
\newcommand{\cregion}{\ensuremath{c-region}\xspace}
\newcommand{\taintt}{\code{t_*}}

\newcommand{\aravind}[1]{\textcolor{green}{Aravind: #1}}

\usepackage{minted}
\usetikzlibrary{chains,fit,shapes,positioning,calc,arrows.meta,shapes.arrows}
\usepackage{xspace}
\usemintedstyle{emacs}
%\usepackage[numbers]{natbib}
%\bibliographystyle{plainnat}
%\bibliographystyle{IEEEtran}

\newcommand{\inlinecode}[1]{%
  \mintinline[fontsize=\footnotesize{},mathescape, escapeinside=||]{c}{#1}%
}

\lstdefinestyle{customc}{
  belowcaptionskip=1\baselineskip,
  breaklines=true,
  breakatwhitespace, %needed to avoid extra space after lstinline + linebreak
  numbers=left,
  xleftmargin=\parindent,
  language=C,
  columns=flexible,      
  showstringspaces=false,
  basicstyle=\small\sffamily,
  otherkeywords={uint,uintptr_t,let,assert},
  literate={{<-}{{$\leftarrow\,$}}2
            {->}{{$\rightarrow\,$}}2
            {<=}{{$\leq\,$}}2},
  numberstyle=\tiny\rmfamily,
  keywordstyle=\bfseries\color{green!40!black},
  commentstyle=\itshape\color{purple!40!black},
  identifierstyle=\color{blue!80!black},
}
\lstset{language=C,style=customc}

\newenvironment{DIFnomarkup}{}{}
\newcommand{\myparagraph}[1]{\noindent\textbf{#1}.\xspace}

\newcommand{\code}[1]{\lstinline|#1|}
\newcommand{\ignore}[1]{{}}
\newcommand{\kwchecked}{\lstinline|\_Checked|}
\newcommand{\kwunchecked}{\lstinline|\_Unchecked|}
\newcommand{\kwdynamiccheck}{\lstinline|\_Dynamic\_check|}

\newcommand{\var}[1]{#1}
\newcommand{\Arrayptr}[1]{\lstinline|_Array_ptr<|{#1}\lstinline|>|}
\newcommand{\ArrayptrT}{\Arrayptr{$\tau$}}
\newcommand{\Ptr}[1]{\lstinline|_Ptr<|{#1}\lstinline|>|}
\newcommand{\PtrName}{\lstinline|_Ptr|}
\newcommand{\PtrT}{\Ptr{$\tau$}}
\newcommand{\Ntarrayptr}[1]{\lstinline|_Nt_array_ptr<|{#1}\lstinline|>|}
\newcommand{\NtarrayptrT}{\Ntarrayptr{$\tau$}}
\newcommand{\Checkedarr}[2]{{#1}~\lstinline|_Checked[|{#2}\lstinline|]|}
\newcommand{\CheckedarrT}[1]{\Checkedarr{$\tau$}{#1}}


%\newcommand{\code}[1]{%
%  \mintinline[fontsize=\footnotesize{},mathescape, escapeinside=||]{c}{#1}%
%}

\definecolor{taintcolor}{rgb}{1.0, 0.65, 0.79}
%\definecolor{taintcolor}{rgb}{0.87, 0.36, 0.51}
%\definecolor{checkcolor}{rgb}{0.82, 0.89, 0.19}
\definecolor{checkcolor}{rgb}{1.0, 0.7, 0.0}
%\definecolor{checkcolor}{rgb}{0.0, 0.06, 0.54}
%\definecolor{checkcolor}{rgb}{0.38, 0.31, 0.86}

\newcommand{\tbl}[1]{Table~\ref{#1}}
\newcommand{\sect}[1]{Section~\ref{#1}}
\newcommand{\fig}[1]{Figure~\ref{#1}}
\newcommand{\lst}[1]{Listing~\ref{#1}}
\newcommand{\algo}[1]{Algorithm~\ref{#1}}
\newcommand{\apdx}[1]{Appendix~\ref{#1}}

\newcommand{\etal}{\textit{et al.,}\xspace}
\newcommand{\ie}{\textit{i.e.,}\xspace}
\newcommand{\eg}{\textit{e.g.,}\xspace}

\newcommand{\realbug}{\textcolor{red}{\faBug}}
\newcommand{\complexcode}{\faChainBroken}
\newcommand{\entrypoint}{\faForward}
\newcommand{\rootcause}{\textcolor{orange}{\faWarning}}
\newcommand{\useradded}{\faUserPlus}
\newcommand{\usermods}{\faUser}

\newcommand{\ptr}{\ensuremath{ptr}}
\newcommand{\arr}{\ensuremath{arr}}
\newcommand{\ntarr}{\ensuremath{ntarr}}
\newcommand{\ptrT}[1]{\code{ptr<}$#1$\code{>}}
\newcommand{\arrT}[1]{\code{array_ptr<}$#1$\code{>}}
\newcommand{\ntarrT}[1]{\code{nt_array_ptr<}$#1$\code{>}}
\newcommand{\ucregion}{\ensuremath{uc-region}\xspace}
\newcommand{\cregion}{\ensuremath{c-region}\xspace}
\newcommand{\taintt}{\code{t_*}}

\newif\ifsubmit\submitfalse

\ifsubmit
\newcommand{\mwh}[1]{}
\newcommand{\ashe}[1]{}
\newcommand{\dtarditi}[1]{}
\newcommand{\dvh}[1]{}
\newcommand{\leo}[1]{}
\newcommand{\liyi}[1]{}
\newcommand{\yiyun}[1]{}
\newcommand{\review}[1]{}
\newcommand{\aravind}[1]{}
\newcommand{\mz}[1]{}
\newcommand{\lc}[1]{}

\else
\newcommand{\mwh}[1]{\textcolor{red}{Mike: #1}}
\newcommand{\liyi}[1]{\textbf{\textcolor{orange}{Liyi: #1}}}
\newcommand{\yiyun}[1]{\textcolor{cyan}{Yiyun: #1}}
\newcommand{\review}[1]{\textbf{\textcolor{blue}{Review: #1}}}
\newcommand{\aravind}[1]{\textcolor{green}{Aravind: #1}}
\newcommand{\lc}[1]{\textcolor{teal}{Le: #1}}

\usepackage[normalem]{ulem}
\colorlet{MZ}{violet!80!pink}
\newcommand{\mz}[1]{{\color{MZ}\textbf{\textit{MZ: #1}}}}
\newcommand{\mzs}[1]{{\color{MZ} \sout{#1}}}
\newcommand{\mzu}[1]{{\color{MZ}\uline{#1}}}
\newcommand{\mzr}[1]{{\color{MZ}{#1}}}
\fi

\newcommand{\lang}{\textsc{CoreChkCBox}\xspace}
\newcommand{\elang}{\textsc{CoreC}\xspace}

\newtheorem{defi}{Definition}
\newtheorem{thm}{Theorem}
\newtheorem{prop}{Proposition}
% \newtheorem{lemma}{Lemma}
% \newcommand{\keyword}[1]{\texttt{#1}}

\newcommand{\CoreC}{\elang}
\newcommand{\CoreChkC}{\lang}
\newcommand{\checkedc}{\text{Checked C}\xspace }

 \newcommand{\rulelab}[1]{{\small \textsc{#1}}}
\newcommand{\steps}{\ensuremath{\longrightarrow}}
\newcommand{\tbool}{\texttt{bool}}
\newcommand{\tsizeof}{\texttt{sizeof}}
\newcommand{\vbool}{\texttt{Bool}}
\newcommand{\verror}{\texttt{Error}}
\newcommand{\vval}{\mathpzc{val}}
\newcommand{\tfixed}{\texttt{fixedp}}
\newcommand{\tnat}{\texttt{nat}}
\newcommand{\tarr}[2]{\texttt{array}~{#1}~{#2}}
\newcommand{\econst}[2]{({#1}){#2}}
\newcommand{\eindex}[2]{{#1}\texttt{[}{#2}\texttt{]}}
\newcommand{\sassign}[4]{{#1} \leftarrow {#3}~{#2}~{#4}}
\newcommand{\ssassign}[3]{{#1} \xleftarrow{#2} {#3}}
\newcommand{\sif}[3]{\texttt{if}~{#1}~{#2}~{#3}}
\newcommand{\sfor}[3]{\texttt{for}~{#1}~{#2}~{#3}}
\newcommand{\scall}[3]{{#1}\leftarrow {#2}~{#3}}
\newcommand{\sseq}[2]{{#1}\,\texttt{;}\,{#2}}
\newcommand{\sinv}[1]{\texttt{inv}~{#1}}
\newcommand{\inst}[3][ ]{\texttt{#2}^{#1}~{#3}}
\newcommand{\insttwo}[4][ ]{\texttt{#2}^{#1}~{#3}~{#4}}
\newcommand{\instthree}[5][ ]{\texttt{#2}^{#1}~{#3}~{#4}~{#5}}
\newcommand{\iskip}[1]{\inst{ID}{#1}}
\newcommand{\inot}[1]{\inst{X}{#1}}
\newcommand{\ictrl}[2]{\insttwo{CU}{#1}{#2}}
\newcommand{\irz}[3][ ]{\insttwo[#1]{RZ}{#2}{#3}}
\newcommand{\isr}[3][ ]{\insttwo[#1]{SR}{#2}{#3}}
\newcommand{\icnot}[2]{\insttwo{CNOT}{#1}{#2}}
\newcommand{\ilshift}[1]{\inst{Lshift}{#1}}
\newcommand{\irshift}[1]{\inst{Rshift}{#1}}
\newcommand{\irev}[1]{\inst{Rev}{#1}}
\newcommand{\iqft}[2][ ]{\inst[#1]{QFT}{#2}}
\newcommand{\ihad}[1]{\inst{H}{#1}}

\newcommand{\iseq}[2]{{#1}\,\texttt{;}\,{#2}}
\newcommand{\inval}[2]{\insttwo{Nval}{#1}{#2}}
\newcommand{\ihval}[3]{\instthree{Hval}{#1}{#2}{#3}}
\newcommand{\iqval}[2]{\insttwo{Qval}{#1}{#2}}

\newcommand{\instr}{\iota}

\newcommand{\kw}[1]{\ensuremath{\mathtt{#1}}}
\newcommand{\subtype}[2]{\ensuremath{\vdash{#1}\prec{#2}}}
\newcommand{\estrlen}[1]{\ensuremath{\kw{strlen}({#1})}}
\newcommand{\estrlentext}{\ensuremath{\kw{strlen}}}

\newcommand{\tarray}[3]{\tarrayb{({#1},{#2})}{#3}}
\newcommand{\tarrayb}[2]{\ensuremath{[{#1}~{#2}]}}
\newcommand{\tntarray}[3]{\tntarrayb{({#1},{#2})}{#3}}
\newcommand{\tntarrayb}[2]{\tarrayb{#1}{#2}_{nt}}
\newcommand{\tallarrayb}[2]{\ensuremath{[{#1}~{#2}]_{\kappa}}}
\newcommand{\tallarraybc}[2]{\ensuremath{[{#1}~{#2}]_{\textcolor{cyan}{\kappa}}}}
\newcommand{\tfun}[3]{\ensuremath{\forall \;#1.\;{#2}\to#3}}
\newcommand{\cupdot}{\mathbin{\mathaccent\cdot\cup}}

\newcommand{\tallarray}[3]{\tallarrayb{({#1},{#2})}{#3}}

\newcommand{\tptr}[2]{\ensuremath{\mathtt{ptr}^{#2}~{#1}}}
\newcommand{\tntptr}[4]{\ensuremath{[({#1},{#2})~{#3}]^{#4}_{nt}*}}


\newcommand{\tarrayptr}[4]{{\tptr{\tarray{#1}{#2}{#3}}{#4}}}
\newcommand{\tntarrayptr}[4]{{\tptr{\tntarray{#1}{#2}{#3}}{#4}}}
\newcommand{\tallarrayptr}[4]{{\tptr{\tallarray{#1}{#2}{#3}}{#4}}}


\newcommand{\tgarray}[4]{\ensuremath{\mathtt{#1}~{#2}~{#3}~{#4}}}
\newcommand{\tstruct}[1]{\ensuremath{\kw{struct}~{#1}}}

\newcommand{\evalue}[2]{\ensuremath{{#1}\!:\!{#2}}}

\newcommand{\emalloc}[2]{\ensuremath{\kw{malloc}({#1},{#2})}}
\newcommand{\emalloctext}{\ensuremath{\kw{malloc}}}
\newcommand{\ecall}[2]{\ensuremath{{#1}({#2})}}
\newcommand{\elcall}[3]{\ensuremath{{(#1,#2)}({#3})}}

\newcommand{\ret}[3]{\ensuremath{\kw{ret}({#1},{#2},{#3})}}
\newcommand{\rettext}{\ensuremath{\kw{ret}}}
\newcommand{\ecast}[2]{\ensuremath{\kw{(}{#1}\kw{)}{#2}}}
\newcommand{\edyncast}[2]{\ensuremath{\langle{#1}\rangle{#2}}}
\newcommand{\edcast}[2]{\ensuremath{\kw{(D},{#1}\kw{)}{#2}}}
\newcommand{\elet}[3]{\ensuremath{\kw{let}~#1\, \texttt{=}\, #2~\kw{in}\;{#3}}}
\newcommand{\elettext}{\ensuremath{\kw{let}}}
\newcommand{\ebinop}[2]{\ensuremath{#1 \plus #2}}
\newcommand{\eassign}[2]{\ensuremath{\texttt{*}~{#1}\,\texttt{=}\, {#2}}}
\newcommand{\elassign}[3]{\ensuremath{\texttt{*}{(#1,#2)}\,\texttt{=}\, {#3}}}

\newcommand{\eassignstack}[2]{\ensuremath{{#1}\,\texttt{=}\, {#2}}}
\newcommand{\efield}[2]{\ensuremath{{#1}\kw{\rightarrow}{#2}}}
\newcommand{\estar}[1]{\ensuremath{\texttt{*}~{#1}}}
\newcommand{\getstar}[2]{\ensuremath{\texttt{*}({#1},{#2})}}

\newcommand{\eamper}[2]{\ensuremath{\kw{\&}{#1}\kw{\rightarrow}{#2}}}
\newcommand{\eunchecked}[2]{\ensuremath{\kw{unchecked}({#1})\{#2\}}}
\newcommand{\euncheckedtext}{\ensuremath{\kw{unchecked}}}

\newcommand{\echecked}[2]{\ensuremath{\kw{checked}({#1}){\{#2\}}}}
\newcommand{\echeckedtext}{\ensuremath{\kw{checked}}}



\newcommand{\eif}[3]{\ensuremath{\kw{if\;}(#1)\;{#2}\;\kw{else}\;{#3}}}
\newcommand{\eifa}[3]{\ensuremath{\kw{IF\;}(#1)\;{#2}\;\kw{ELSE}\;{#3}}}
\newcommand{\eifatext}{\ensuremath{\kw{IF}}}

\newcommand{\eiftext}{\ensuremath{\kw{if}}}
\newcommand{\ethentext}{\ensuremath{\kw{then}}}
\newcommand{\eelse}{\ensuremath{\kw{else}}}
\newcommand{\ebreak}{\ensuremath{\kw{break}}}
\newcommand{\ebounds}{\ensuremath{\kw{bounds}}}
\newcommand{\enull}{\ensuremath{\kw{null}}}
\newcommand{\hole}{\ensuremath{\Box}}
\newcommand{\defscope}{\sigma}
\newcommand{\wf}[1]{wf \, #1}
\newcommand{\tint}{\ensuremath{\mathtt{int}}}
\newcommand{\efor}{\ensuremath{\mathtt{for}}}
\newcommand{\ewhile}{\ensuremath{\mathtt{while}}}
\newcommand{\etrue}{\ensuremath{\mathtt{true}}}
\newcommand{\efalse}{\ensuremath{\mathtt{false}}}
\newcommand{\edplus}{\ensuremath{\texttt{++}}}
\newcommand{\edminus}{\ensuremath{\texttt{--}}}

\newcommand{\ememcpy}{\ensuremath{\mathtt{memcpy}}}
\newcommand{\ereturn}{\ensuremath{\mathtt{return}}}
\newcommand{\emain}{\ensuremath{\mathtt{main}}}
\newcommand{\estrcat}{\ensuremath{\mathtt{strcat}}}
\newcommand{\estrcatbad}{\ensuremath{\mathtt{strcat\_b}}}
\newcommand{\heap}{\ensuremath{\mathpzc{H}}}
\newcommand{\heapup}[3]{\ensuremath{\mathpzc{H}}(#1)[#2 \mapsto #3]}
\newcommand{\eret}[3]{\ensuremath{\kw{ret}({#1},{#2},{#3})}}
\newcommand{\erettext}{\ensuremath{\kw{ret}}}

\newcommand{\plus}{\mathbin{\texttt{+}}}

% Typewriter font is for syntax, Metafunctions are italic.
\newcommand{\fv}{\mathit{FV}}
\newcommand{\dom}{\mathit{dom}}

\newcommand{\fm}{\mathit{FM}}
\newcommand{\size}{\mathit{size}}

\newcommand{\cmode}{\texttt{c}}
\newcommand{\umode}{\texttt{u}}
\newcommand{\tmode}{\texttt{t}}

\newcommand{\bvar}{\ensuremath{\beta}}
\newcommand{\mode}{\textit{mode}}
\newcommand{\type}{\textit{type}}
\newcommand{\funptr}{\textit{fun\_t}}

\newcommand{\tgez}{\texttt{ge}\_0}
\newcommand{\teq}[1]{\texttt{eq}\;{#1}}

% compilation
\newcommand{\cextend}[4]{\ensuremath{#4 = ~ \vdash_{extend} #1, #2, #3}}
\newcommand{\fresh}{\ensuremath{\kw{fresh}}}
\newcommand{\echecknull}[3]{\ensuremath{#3 = ~\vdash_{null}#1, #2}}
\newcommand{\echeckboundsdyn}[5]{\ensuremath{#5 = ~ \vdash_{boundsD}#1, #2, #3, #4}}
\newcommand{\echeckbounds}[5]{\ensuremath{#5 = ~ \vdash_{boundsR}#1, #2, #3, #4}}
\newcommand{\echeckboundsw}[5]{\ensuremath{#5 = ~ \vdash_{boundsW}#1, #2, #3, #4}}
\newcommand{\esizeof}[1]{\ensuremath{\kw{sizeof}(#1)}}
\newcommand{\ewidenstrlen}[5]{\ensuremath{#5 = ~ \vdash_{widenstr}#1, #2, #3, #4}}
\newcommand{\ewidenderef}[4]{\ensuremath{#4 = ~ \vdash_{widenderef}#1, #2, #3}}
\newcommand{\eleq}[2]{\ensuremath{#1 \leq #2}}
\newcommand{\systemname}{\textsc{CheckedCBox}\xspace}
\newcommand{\threec}{\textsc{3C}\xspace}



% metafunctions & relations
\newcommand{\rboundle}[2]{\ensuremath{\vdash_{bounds}#1 \leq #2}}

% eval numbers
\newcommand{\numprog}{XXX\xspace}


%%%%%%%%%%%%%%%%%%%%%%%%%%%%%%%%%%%%%%%%%%%%%%%%%%%%%%%%%%%%%%%%%%%%%%
%% Note: Authors migrating a paper from PACMPL format to traditional
%% SIGPLAN proceedings format must update the '\documentclass' and
%% topmatter commands above; see 'acmart-sigplanproc-template.tex'.
%%%%%%%%%%%%%%%%%%%%%%%%%%%%%%%%%%%%%%%%%%%%%%%%%%%%%%%%%%%%%%%%%%%%%%



%MWH: This line doesn't work on my Mac; what do I need to install?
%\setmonofont{inconsolata}

\def\codesize{\normalsize}
\definecolor{programs}{gray}{0.1}

\lstset{
  basicstyle=\ttfamily,  
}

% DVH: Hack to get lst to use up quotes instead of broken "smart"
% quotes.
\makeatletter
\lst@CCPutMacro
    \lst@ProcessOther {"22}{\lst@ifupquote \textquotedbl
                                     \else \char34\relax \fi}
    \@empty\z@\@empty
\makeatother

\lstloadlanguages{C}

\def\BibTeX{{\rm B\kern-.05em{\sc i\kern-.025em b}\kern-.08em
    T\kern-.1667em\lower.7ex\hbox{E}\kern-.125emX}}
\begin{document}

%% Title information
\iftr
\title{\systemname: Type Directed Program Partitioning with \checkedc for Incremental Spatial Memory Safety {\large (Extended Version)}}
\else
\title{\systemname: Type Directed Program Partitioning with \checkedc for Incremental Spatial Memory Safety}
\fi
%% [Short Title] is optional;
%% when present, will be used in
                                        %% header instead of Full Title.
% \titlenote{with title note}             %% \titlenote is optional;
                                        %% can be repeated if necessary;
                                        %% contents suppressed with 'anonymous'
% \subtitle{Subtitle}                     %% \subtitle is optional
% \subtitlenote{with subtitle note}       %% \subtitlenote is optional;
                                        %% can be repeated if necessary;
                                        %% contents suppressed with 'anonymous'


%% Author information
%% Contents and number of authors suppressed with 'anonymous'.
%% Each author should be introduced by \author, followed by
%% \authornote (optional), \orcid (optional), \affiliation, and
%% \email.
%% An author may have multiple affiliations and/or emails; repeat the
%% appropriate command.
%% Many elements are not rendered, but should be provided for metadata
%% extraction tools.

%% Author with single affiliation.
%%\author{Liyi Li, Yiyun Liu$^\dagger$, Deena Postol, Leonidas
%%  Lampropoulos, David Van Horn, and Michael Hicks\\
%%  University of Maryland $\quad\quad ~^\dagger$University of Pennsylvania}
  
%            David Van Horn (University of Maryland, College Park)
%            Michael W. Hicks (University of Maryland, College Park)

%   \IEEEauthorblockN{1\textsuperscript{st} Leonidas Lampropoulos}
% \IEEEauthorblockA{\textit{Department of Computer Science} \\
% \textit{University of Maryland}\\
% llampro@cs.umd.edu}
% \and
% \IEEEauthorblockN{1\textsuperscript{st} Michael Hicks}
% \IEEEauthorblockA{\textit{Department of Computer Science} \\
% \textit{University of Maryland}\\
% mwh@cs.umd.edu}
% }
% \author{Leonidas Lampropoulos \and
% Michael Hicks}

\maketitle



\iftr
\bigskip
\begin{small}
 This is an extended version of a paper that
  appears at the 2022 Computer Security Foundations Symposium.
\end{small}
\fi
  
\section{Introduction}\label{sec:intros}

Vulnerabilities due to memory corruption are still a major issue for C programs~\cite{cvetrend, microsoftmemsafe, Zeng:2013:SRF:2534766.2534798}  despite many efforts that tried to prevent them~\cite{song2019sanitizing}.
These vulnerabilities are not uniformly distributed in real programs~\cite{meng2021bran, du2019leopard},
as some components are likelier to have a certain class of vulnerabilities.
\eg{} spatial safety issues in string processing functions.

Program partitioning enables separating potentially vulnerable functions (untrusted region) from the rest of the program (trusted region).
Several program partitioning mechanisms~\cite{tan2017principles, brumley2004privtrans, bittau2008wedge, lind2017glamdring, liu2017ptrsplit,rlbox-paper,10.1145/3492321.3519582,privtrans,cqual-kernel-ptr,10.1145/3371100} exist.
These techniques usually combine program partitioning with SFI to guarantee robust safety \cite{10.1145/3371100}, such that code in the untrusted region cannot affect code in the trusted region,~\eg{} by executing untrusted partition in a separate process~\cite{liu2017ptrsplit} or executing in a trusted execution environment~\cite{lind2017glamdring}. 

Unfortunately, the robust safety statement is too vague in expressing the exact safety it provides, and it causes the systems based on it --- such as the ones above --- provide stronger guarantees than they should be and cause considerable performance overhead, but also provide ways where users can reduce overhead by essentially violating the robust safety.
For example, RLBox \cite{rlbox-paper} and PKRU-Safe \cite{10.1145/3492321.3519582} have a $100\%$ - $200\%$ overhead, and RLbox permits users to bypass its verification procedures that guarantee robust safety.
The ambiguity and its associated performance drawback decelerates the mechanism's development \cite{sandboxfu}.

In all these program partitioning mechanisms, the program partitioning parts are clear and similar.
They provided a type system or user defined annotations to permit the identification of untrusted data, perform some memory separation techniques (separate heaps or sandbox mechanism) to separate the allocations of trusted and untrusted data, and insert dynamic checks to restrict the communications of these two data.
The differences in these mechanisms appeared in the SFI part for separating completely the trusted and untrusted regions that respectively contain trusted and untrusted data, as well as the user defined verification procedures for the permission of some untrusted data usage in trusted regions, which are also the performance bottlenecks.
To balance safety guarantees and performance overhead, we first need to investigate the guarantees that program partitioning can provide, without the involvement of SFI.

In a program partitioning mechanism, a program is split into two regions: a \umode~\emph{region} representing the untrusted partition, and the \cmode~\emph{region} representing the trusted partition. Typically, there is a special types of pointers, which we named \textbf{tainted} (\tmode) mode pointers, being introduced to be used in communicating the \umode and \cmode regions, i.e.,
untainted pointers cannot be used in \umode region, while there are checks requiring for \tmode mode pointers to be used in \cmode region.
Users annotate pointer types and there is a static procedure, such as type checking, then creates two sets of source files:~\ucregion{} and~\cregion{}, containing tainted and untainted entities, respectively.
In addition, untainted pointers that are used in \cmode region are typically viewed as "good" pointers, named \emph{checked} (\cmode mode) pointers. Users classify \cmode mode pointers and \cmode region in a program because they consider these entities are safe to use, which indicates that the guarantee assumptions in \cmode and \umode regions are different and the safety property should reflect such differences.

We present \systemname, as a formalism for the above type-directed program partitioning mechanism based on \checkedc,
and show that the memory safety the mechanism provides is the \emph{non-crashing} property, i.e.,
\cmode region never experiences crashing due to memory safety issues regardless the execution of \umode region.
We also show that the system provides \emph{clean separation} property as a corollary of the non-crashing property,
i.e., any memory violation in \umode region does not affect the execution of \cmode region.


The system is based on \checkedc because of the third concern above. In privilege separation, it is obvious that trusted code is viewed as the part that does not have problem, while programs are in untrusted code, which is similar to the concept of checked and unchecked code regions in \checkedc and make it a perfect model to formalize type-directed program partitioning.
We make precise about the guarantees that a program partition mechanism can provide and have the following contributions.

\myparagraph{\systemname Type System and Formalism}
We present a type system that integrates tainted types with~\checkedc and provides additional guarentees---the~\emph{non-crashing} and \emph{non-exposure} guarantees,~\ie a well-typed~\systemname program can never crash due to spatial safety violations,
as well as \ucregion code cannot directly observe a checked pointer address.
%no \cregion pointer addresses will be leaked in
%\ucregion code
We extend the \checkedc compiler to support the type system and
formalize it by extending~\checkedc formalism~\cite{li22checkedc} with the non-crashing and non-exposure guarantees.
We formally prove theorems related to the two guarantees and use model-based randomized testing \cite{Pierce:SF4} to certify the simulation relation between the~\systemname semantics and its compiler formalism.
To the best of our knowledge,~\systemname is the first C(-like) language and compiler formalism with the program partitioning mechanism.


We designed and implemented a flexible program partition technique using tainted types.
Our type system guarantees that the tainted pointers cannot affect untainted pointers and partition the given code into~\ucregion (containing only tainted entities) and~\cregion (CITE SECTION).
We provide formal proof of the~\emph{clean separation theorem} that guarantees that code in ~\cregion will be separated from~\ucregion according to the target isolation mechanism (CITE SECTION).
We also integerated~\systemname{} with~\checkedc that provides additional~\emph{non-crashing} and \emph{non-exposure} guarantees (CITE SECTION).

\myparagraph{Sound and Better Compilation} Our second
contribution is a formalization of bounds-check insertion for array
accesses (Section~\ref{sec:compilation}). Our operational semantics
annotates each pointer with metadata that describes its bounds, and
the assignment and dereference rules have premises to confirm the
access is in bounds. An obvious compilation scheme (taken by
Cyclone~\cite{Jim2002,GrossmanMJHWC02}, CCured~\cite{Necula2005}, and
earlier works) would be to translate annotated pointers to multi-word
objects: one word for the pointer, and 1-2 words to describe its lower
and upper bounds. Inserted checks reference these bounds. While
convenient, such ``fat'' pointers are expensive, and break backward
binary compatibility with legacy pointers.

  To show that pointer annotations can be safely
  erased, and thus fat pointers are not needed, we formalize a
  translation of \lang to \elang, which is an 
  untyped version of \lang that drops metadata annotations, and
  lacks bounds/null checks in the semantics rules. Instead,
  the compilation process inserts null/bounds checks explicitly, leveraging
  compile-time type information. While we do not definitively prove
  it, we provide strong evidence that compilation is correct. We use PLT
  Redex~\cite{pltredex} to mechanize (a generalization of) \lang, 
  \elang, and compilation between the two, and we use randomized testing 
to validate that the compiled program \emph{simulates} the
original. In addition to demonstrating the technical point that metadata
annotations in the \lang formalism do not necessitate fat pointers,
compilation also sheds light on the actual Checked C compilation
process. 
% \review{ Beginning of page 2: "we show", then "give confidence". I am left confused as
%   to whether the CoreChkC --> CoreC compilation is shown correct (theorem with
%   proof), or carefully debugged and tested in plt-redex (which is fine too, I
%   just find the choice of terminology confusing)}
% \liyi{typo, should be "validate through randomized testing"}

As far as we are aware, \lang is the first formalism to cleanly
separate bounds-checking compilation from the core semantics; prior
work~\cite{Feng2006,Condit2007} merged the two, conflating
\emph{meaning} with \emph{mechanism}. In carrying out the
formalization, we discovered that our compilation approach for
null-terminated array pointers is more expressive than that proposed
in the Checked C specification~\cite{checkedc}
(Section~\ref{sec:disc}); we would not have discovered this
improvement had we not separated checks from semantics.

With the understanding of the true guarantees from program partitioning, we find a better compilation strategy.
Instead of sanitizing the entire \umode region, as many previous mechanisms did \cite{rlbox-paper},
we can only sanitize the \tmode pointers by providing extra dynamic checks in the compiler.
Compared to a typical $100\%$ - $200\%$ overhead in these works \cite{rlbox-paper} as our faithful RLBox representation \systemnamea,
we show that the new compilation implementation, named \systemnameh, only having an average $5\%$ overhead compared to the original \checkedc compiler.

\myparagraph{Extending \checkedc}
We formalize \systemname in \checkedc \cite{li22checkedc} because it has a formalism and compilation.
More importantly, \checkedc has checked and unchecked regions coinciding with the \cmode and \umode regions in program partitioning,
and it provides spatial safety for \cmode region and guarantees that any memory violation is from \umode region,
which is similar to the concept that \cmode region contains only "good" pointers.
Based on the \checkedc formalism, we provide extra \tmode mode pointers to communicate the \cmode and \umode regions in \checkedc, while keeping the original \cmode and \umode mode pointers to be only used in \cmode and \umode regions, respectively.
By upgrading the \checkedc formalism to \systemname, we not only build the foundation for program partitioning, but also extends \checkedc
to support safeties for third party libraries. Previously, users are required to rewrite C code to \checkedc completely to guarantee memory safeties. With \systemname, users can view C library functions as \umode region, and utilize \systemname to validate the \tmode mode pointers returning from these \umode region, without painfully rewriting the C library functions to \checkedc.


\begin{figure}
\includegraphics[width=0.5\textwidth]{relationship.pdf}
\caption{
    \lang models' relationship to \checkedc
}
  \label{fig:model-relation}
\end{figure}

\myparagraph{Summary Visualization} The
  relationship among our contributions is visualized in
  Fig.~\ref{fig:model-relation}. With the Coq model of \lang we prove
  soundness (and with it, \emph{blame}) of the \checkedc type system
  and semantics. With the Redex model, we use randomized testing to
  validate both type soundness and compilation correctness, where the
  latter shows how compilation need not output fat pointers despite
  the use of pointer annotations in the \lang model. The Redex \lang
  model is also the basis of randomized testing of the correctness of
  the \checkedc compiler implementation, both its type checker and the
  semantics of its emitted code, at least for the subset of the
  language in the Redex model. The Redex model's syntax
  is slightly richer than the Coq version: conditional guards and
  function arguments may be arbitrary expressions, where the Coq
  version limits them to constants and variables, making handling of
  dependent types a bit simpler. We find a useful synergy between the
  Coq and Redex models for carrying out a language development. The
  richer, executable Redex model is useful for quickly modeling and
  testing new features, both formally and against a real
  implementation. Once solidified, new features can be added to the
  Coq model (perhaps somewhat simplified) for final proofs of
  correctness.

We begin with a review of \checkedc (Section~\ref{sec:overview}),
present our main contributions
(Sections~\ref{sec:formal}--\ref{sec:evaluation}), and conclude with a
discussion of 
related and future work (Sections~\ref{sec:related},
\ref{sec:conclude}). All code and proof artifacts (both for Coq and
Redex) can be found at \url{https://github.com/plum-umd/checkedc}. 












\section{\checkedc Overview}\label{sec:overview}

This section describes Checked C, which extends C with new pointer
types and annotations that ensure spatial safety. Development of
Checked C was initiated by Microsoft Research in 2015 but starting in
late 2021 was forked and is now actively managed by the Secure Software
Development Project (SSDP). Details can be found in a prior
overview~\cite{Elliott2018} or the full specification~\cite{checkedc}.
Checked C is implemented as an extension of Clang/LLVM and the SSDP
fork is freely available at \url{https://github.com/secure-sw-dev}. 

\begin{figure}[t]
{\small
  \begin{lstlisting}[xleftmargin=4 mm]
nt_array_ptr<const char>
parse_utf16_hex(nt_array_ptr<const char> s,
                  ptr<uint> result) {
 int x1, x2, x3, x4;
 if (s[0] != 0) { x1 = hex_char_to_int(s[0]);
 if (s[1] != 0) { x2 = hex_char_to_int(s[1]);
 if (s[2] != 0) { x3 = hex_char_to_int(s[2]);
 if (s[3] != 0) { x4 = hex_char_to_int(s[3]);
 if (x1 != -1 && x2 != -1 && x3 != -1 && x4 != -1){
   *result = (uint)((x1<<12)|(x2<<8)|(x3<<4)|x4);
   return s+4;
  ...// several } braces
 } 
 return 0;
}
void parse(nt_array_ptr<const char> s,
            array_ptr<uint> p : count(n), 
            int n) {
 array_ptr<uint> q : bounds(p,p+n) = p;
 while (s && q < p+n) {
   array_ptr<uint> r : count(1) =
     dyn_bounds_cast<array_ptr<uint>>(q,count(1));
   s = parse_utf16_hex(s,r);
   q++;
 }
}
  \end{lstlisting}

% {\captionsetup[lstlisting]{margin = 8 mm}
%   \begin{lstlisting}[xleftmargin=8 mm]
% array_ptr<int> memcpy(int x,
%      array_ptr<int> a : count(x),
%      array_ptr<int> b : count(x)) : count(x) {
%  for(int i = 0; i < x; ++i) {
%     (*a = *b); a++; b++;
%  }
%  return a;
}

% static int idx = 0;
% int buf (int * reg,
%           nt_array_ptr<int> x,
%           nt_array_ptr<int> y) {
%   if (idx < n)
%     unchecked { * x = &(reg + idx); idx++; }
%   else bug();
%   memcpy(n,x,y);
%   return 0;
% }
% }
\caption{Parsing a String of UTF16 Hex Characters in \checkedc}
\label{fig:checkedc-example}
\end{figure}

% Fig.~\ref{fig:checkedc-example}, all dereferences of pointer \code{a}
% are within the range $[$\code{0}$,$\code{x}$)$ and therefore
% safe. Moreover, if we were to remove the condition \code{i<x} from the
% \code{for} loop, attempting to dereference an out-of-bounds memory
% location, the inserted checks guarantee that the system will raise an
% out-of-bound exception instead of crashing with a segmentation fault
% or, worse, allowing the access to
% happen. %if a out-of-bound error happens, the system causes an out-of-bound exception.

% To insert such checks, both array and NT-array pointers are associated
% with two values, their \emph{bounds}, defining the range of memory
% referenced by the pointers. An array pointer $p$ with
% bounds $b_l$ and $b_h$ allows access to the memory region
% $[p+b_l,p+b_h)$. 

\subsection{Checked Pointer Types}
%
\checkedc introduces three varieties of \emph{checked pointer}:
\begin{itemize}
\item \code{ptr<}$T$\code{>} types a pointer that is either null or
  points to a single object of type $T$.
\item \code{array_ptr<}$T$\code{>} types a pointer that is either null
  or points to an array of $T$ objects. The array width is defined
  by a \emph{bounds} expression, discussed below.
\item \code{nt_array_ptr<}$T$\code{>} is like
  \code{array_ptr<}$T$\code{>} except that the bounds expression
  defines the \emph{minimum} array width---additional objects may
  be available past the upper bound, up to a null terminator.
\end{itemize}
A bounds expression used with the latter two pointer types has three
forms:
\begin{itemize}
\item \code{count(}$e$\code{)} where $e$ defines the array's
  length. Thus, if pointer $p$ has bounds \code{count(n)} then the
  accessible memory is in the range $[p,p+$\code{n}$]$. Bounds
  expression $e$ must be side-effect free and may only refer to
  variables whose addresses are not taken, or adjacent \code{struct}
  fields.
\item \code{byte_count(}$e$\code{)} is like \code{count}, but
  expresses arithmetic using bytes, no objects; i.e.,
  \code{count(}$e$\code{)} used for \code{array_ptr<}$T$\code{>} is
  equivalent to \code{byte_count(}$e\times\texttt{sizeof}(T)$\code{)}
\item \code{bounds(}$e_l$,$e_h$\code{)} where $e_l$ and $e_h$ are
  pointers that bound the accessible region $[e_l,e_h)$ (the
  expressions are similarly restricted). Bounds
  \code{count(}$e$\code{)} is shorthand for
  \code{bounds(}$p, p + e$\code{)}. This most general form of bounds
  expression is useful for supporting pointer arithmetic.
\end{itemize}
  Dropping the bounds expression on an \code{nt_array_ptr} is equivalent
  to the bounds being \code{count(0)}.

The \checkedc compiler will instrument loads and stores of checked
pointers to confirm the pointer is non-null, and the access is within
the specified bounds. For pointers $p$ of type
\code{nt_array_ptr<}$T$\code{>}, such a check could spuriously fail if
the index is past $p$'s specified upper bound, but before the null
terminator. To address this problem, Checked C supports \emph{bounds
  widening}.
If $p$'s bounds expression is \code{bounds}$(e_l$,$e_h)$ a program may read from (but not
write to) $e_h$; when the compiler notices that a non-null character
is read at the upper bound, it will extend that bound to $e_h+1$.
% \liyi{solved: II.B.: LaTeX issue, your inline monospaced code snippets generate a space on
%   the following line (i.e. "a program may read from (but not write to)" is not
%   aligned with the left boundary of the paragraph).}

\subsection{Example}
\label{sec:example}

Fig.~\ref{fig:checkedc-example} gives an example \checkedc
program.\footnote{Ported from the Parson JSON
parser, \url{https://github.com/kgabis/parson}} 
The function \code{parse_utf16_hex} on lines 1-15 takes a 
null-terminated pointer \code{s} as its argument, from which it attempts to read four
characters. These are interpreted as hex digits and converted to an
\code{uint} returned via parameter \code{result}. At first,
\code{s}~has no specific bounds annotation, which we can interpret as
\code{count(0)}; this means that \code{s[0]} may be read on line
5. The true branch of the conditional (which extends all the way to
the brace on line 13) is thus type-checked with \code{s} given a
\emph{widened} bound of \code{count(1)}. Likewise, the conditionals on
lines 6-8 each widen it one further; the widened pointer
(\code{s+4}) is returned on success.

The \code{parse} function on lines 16-26 repeatedly invokes
\code{parse_utf16_hex} with its parameter \code{s}, and fills out
array \code{p} whose declared length is the parameter \code{n}. Writes
happen via pointer \code{q}, which is updated using pointer
arithmetic. We specify its bounds as
\code{bounds(p,p+n)} to support this: even as \code{q} changes, variables \code{p} and \code{n} (and therefore also \code{q}'s bounds) do not. Converting from an
\code{array_ptr<uint>} to a \code{ptr<uint>}, done for the call on
line 23, requires proving the array has size at least 1. While this is true
because of the loop condition \code{q < p+n}, which is \code{q}'s
upper bound,  the compiler is not smart enough to figure this
out. To convince it, we manually insert a \emph{dynamic cast} via
\code{dyn_bounds_cast}, which is trusted at compile-time but confirmed
with a dynamic check at run-time.

While bounds checks are \emph{conceptually} inserted on every array
load and store, many of these are eliminated by LLVM\@. For example,
all of the pointer accesses to \code{s} on lines 5-8 are proved safe
at compile-time, so no bounds checks are inserted for
them. \citet{Elliott2018}
reported average run-time overheads of 8.6\% on a pointer-intensive
benchmark suite (49.3\% in one case); \citet{duanrefactoring} measured
no overhead at all on a port of FreeBSD's UDP and IP stacks to \checkedc.

% For soundness, variables used in a bound expression may
% neither be modified nor have their address taken until the pointer
% containing the bound expression is out of the scope. As such, some
% legacy idioms may be unsupported. \yiyun{Mention how this restriction further motivates our dynamic bounds tracking?} In \checkedc, variables with
% checked pointer types or containing checked pointers must be
% initialized when they are declared. 

% \leo{The following is no longer needed, I think. It doesn't add anything new now}
% Fig.~\ref{fig:checkedc-example} \textbf{(a)} shows a usage of a \code{checked} NT-array pointers with dependent functions. We provide a special \code{memcpy} function to copy the data from NT-array pointer \code{b} to \code{a}. The execution of the function is guaranteed to the spatially safe because the argument types for \code{a} and \code{a} are set to \code{nt_array_ptr<int> : count(x)} (meaning that the address range is $[$\code{0}$,$\code{x}$)$), so it guarantees that the two input argument NT-arrays to have at least length \code{x}; thus, the \code{for} execution is valid all the time.
%  Sec.~\ref{sec:formal} discusses how we formally define these pointer bounds with slightly different syntax.
\subsection{Other features}

\checkedc has other features not modeled in this paper. Two in regular
use are \emph{interop types}, which ascribe checked pointer types to
unported legacy code, notably in libraries; and \emph{generic types}
on both functions and \code{struct}s, for type-safe polymorphism. More
details about these can be found in the language specification.

\subsection{Spatial Safety and Backward Compatibility}
%
\checkedc is backward compatible with legacy C in the sense that all
legacy code will type-check and compile. However, only code that
appears in \emph{checked regions}, which we call \emph{checked code},
is spatially safe. Checked regions can be designated at the level of
files, functions, or individual code blocks, the first with a
\code{#pragma} and the latter two using the \code{checked}
keyword.\footnote{You can also designate \emph{unchecked} regions
  within checked ones.}  Within checked regions, both
legacy pointers and certain unsafe idioms (e.g., \emph{variadic} function
calls) are disallowed. The code in Fig.~\ref{fig:checkedc-example}
satisfies these conditions, and will type-check in a checked region.

How should we think about code that contains both checked and legacy
components? \citet{ruef18checkedc-incr} proved, for a simple
formalization of \checkedc, that \emph{checked code cannot be blamed}:
Any spatial safety violation is caused by the execution of unchecked
code. In this paper we extend that result to a richer formalization of
\checkedc. 

% \paragraph*{\textbf{NT-array Pointer Formalization}} NT-array pointers point to NT-arrays whose endings are determined by a null (\code{'\0'}). The length of a NT-array is usually not fixed. The two uses of NT-array pointers in \checkedc are branching and \code{strlen} operations. \code{if (*x) }$e_1$\code{ else }$e_2$ branches to $e_1$/$e_2$ depending one the \code{x}'s data, while \code{strlen(x)} computes the length of the NT-array pointed to by \code{x}.
% \liyi{move to formal and later section.}
  
% \paragraph*{\textbf{Dependent Functions}} \checkedc allows users to declare dependent functions.  Fig.~\ref{fig:checkedc-example} shows an example usage of dependent functions. The integer argument \code{x} is used to specify the bounds of the two NT-array pointers. It means that when we call the \code{memcpy} function, the upper-bounds of the two NT-array pointers must be no less than \code{x}. If the programmer makes the mistake of writing \lstinline{i <= x} in the for loop, the runtime checks will ensure that the indexing of\lstinline|x| in either \code{a} or \code{b} results in an error.
% Because of the dependent function features, we are able to utilize the fact that the two remaining length of the NT-array pointers are greater than \code{x}, so that the \code{for} is guaranteed to execute properly.

% \ignore{
% \liyi{Good information about pointer types here:
% https://github.com/microsoft/checkedc/wiki/New-pointer-and-array-types}




% This code creates an array pointer of length $10$, and reads the pointer $(x+y)$. 
% If the integer value $y$ is greater than $10$, executing the program in C results in undefined behavior. 
% In \checkedc, however, we links every pointer in a program with their bound information statically, and inserts checks to dynamically verify if the usage of the pointer violates the spatial safety. For example, In compiling the above fragment, we inserts the bound's check for $x$ in the read operation $(x+y)$, and if $y$ results in a value $11$, the execution of the read operation $(x+y)$ results in an dynamic error. 

% There are many existing language development trying to guarantee the C spatial safety property through different approaches. For example, Cyclone \cite{Jim2002} and Cerberus \cite{cerberus} are the fat pointer approaches for guaranteeing the spatial safety and eliminating undefined behaviors in C. Rust \cite{Rust2016}, on the other hand, tries to restrict the possible way of writing programs to guarantee such property statically.
% The problem of these two approaches are effectiveness of executing programs and user practicality of writing programs. The fat-pointer approach completely relies on dynamic checks to guarantee the safety properties but it might pay a huge price on execution efficiency. In our experiment (Sec.~\ref{sec:evaluation}), we found that the execution of fat pointer approaches cost the speed a 50\% overhead. 
% On the other hand, Rust uses a static approach by restricting the usage of pointers to guarantee the properties, which results in a user UN-friendly system.  
% The development of \checkedc is an investigation on keeping the balance between execution efficiency and user-friendly systems. We develop a system to utilize static type systems to allow the \checkedc compiler to insert dynamic checks to guarantee the safety property. The efficiency of \checkedc is in the same level of Rust while we maintain a user-friendly system and inherited most of the operation features from C.

% Another feature of \checkedc is the division of the checked and unchecked blocks. When users try to rewrite their C programs into \checkedc programs, they can develop the code incrementally by rewriting sub-parts of their programs to \checkedc code fragments and placing them as the checked blocks, while keeping the rest C programs as unchecked C programs. \checkedc guarantees that if there is any safety violation, this comes from the unchecked blocks (the blame theory). 

% Here is the contributions of the paper. First, we develop the code formalization of \checkedc and show that the type system in \checkedc is type-sound and satisfies the blame theory with respect to the \checkedc semantics. Specifically, we formalize null-terminated array pointers. To our best knowledge, this is the first work defining such feature in C-related languages. 
% Second, in formalizing the \checkedc language, we investigate the balance between program execution efficiency and user practicality. We re-investigate the \checkedc type system with subtyping relations to make it more efficient and user-friendly.
% Second, we also develop random testing tools in Redex \cite{pltredex} to test the \checkedc compiler and make sure it is properly developed. We are able to find and reproduce many bugs/faults in the \checkedc compiler \liyi{numbers?}. The random testing tool is able to generate tens of thousands of programs to properly validate the \checkedc compiler. Third, we also spent a great amount of work in developing experiments to compare \checkedc and other existing C-like languages that guarantee the spatial safety and eliminating undefined behaviors. We found that \checkedc keeps the best balance between execution efficiency and user practicality.
% }




\section{Overview}
\label{sec:overview}
We need to explain the overall flow of our system here.

1) Start with a simple example and say what steps developer follows to convert the code to RLBoxed Checked C

2) Say what our system does with the converted code





% \ignore{
% \section{Null-Terminated Array Pointers}

% Key ideas, formalized:
% \begin{itemize}
% \item Can read one past the end: x[N] can be read, but only written if
%   the value being written is 0.
% \item NT arrays are a refinement of normal arrays: Can convert
%   x:(ntarray N t) to (array N t).
% \item NT arrays can expand their length flow sensitively: Can convert
%   x:(ntarray N t) to (ntarray N+1 t) assuming x[N] != 0.
% \item In general unsafe to convert x:(array N t) to (ntarray N-1 t)
%   even if x[N-1] == 0 because aliases of x could eliminate the NULL
%   terminator. Would need some sort of linearity/alias tracking to
%   support this idea.
% \item Question? What is the syntax of doing a malloc on a NT-array?
%   Should we just do malloc(NT-array l h type) as the malloc of a
%   normal array or should we also include something like int [] nt =
%   “abcd”, where the bound size is not specified?
% \end{itemize}

% See also the Checked C specification
% \ignore{
% https://github.com/microsoft/checkedc/tree/master/spec/bounds\_safety}
% }
%\section{\systemname Type System and Formalism}
\label{sec:checkcboxtypesystem}
\subsection{Typing}
--EXPLAIN CHECKED C BOX TYPE SYSTEM
\subsection{Semantics}
BRIEF OVERVIEW of Semantics
\section{Formalism}\label{sec:formal}

\begin{figure}
  \small \centering
  \[ \hspace*{-1.5em}
\setlength{\arraycolsep}{4pt}
\begin{array}{l}
\begin{array}{ll}
       \text{Variables:}~ x
& \quad\text{Integers:}~n::=\mathbb{Z} 
\end{array}
\\[0.5em]
\begin{array}{llcllcl}
\text{Context Mode:} & m & ::= & \cmode \mid \umode \\[0.2em]
\text{Memory Mode:} & \kappa & ::= & m \mid \tmodered \\[0.2em]
\text{Bound:} & b & ::= & n \mid x \plus n \\[0.2em]
              & \bvar & ::= & (b,b) \\[0.2em]
  
     \text{Word Type:}& \tau &::=& \tint\mid \tptr{\omega}{\kappa}
\\[0.2em]
\text{Type:}&\omega &::=& \tau \mid \tarrayb{\bvar}{\tau} \mid \textcolor{red}{\tfun{\overline{x}}{\overline{\tau}}{\tau}}
\\[0.2em]
\text{Expression:}& e & ::= & 
\evalue{n}{\tau} \mid x \mid \ebinop{e}{e}\mid \ecast{\tau}{e} \mid \estar{e}\mid\eassign{e}{e} 
  \\[0.2em]
&&\mid& \textcolor{red}{\emalloc{\kappa}{\omega}} \mid \textcolor{red}{\toa{m}{\overline{x}}{e}} \\[0.2em]
&&\mid&\eif{e}{e}{e} \mid  \elet{x}{e}{e} \mid \textcolor{red}{\ecall{e}{\overline{e}}}
\end{array}
    \end{array}
  \]
  \caption{\lang Syntax}
  \label{fig:checkc-syn}
\end{figure}

\begin{figure}
{
\begin{center}
\includegraphics[width=0.35\textwidth]{memory.png}
\end{center}
}
\caption{
    \systemname separate heap layouts
}
  \label{fig:checkc-memory}
\end{figure}

\begin{figure}
{
  \small \centering
\begin{tabular}{|c|c|c|c|c|}
\hline
\diagbox{\text{Context}}{\text{Memory}}
& $\texttt{malloc}$ & $\estar{e}$ & $\eassign{e}{e}$ & $\ecall{e}{\overline{e}}$\\[0.1em]
\hline
\cmode&$\cmodenor$,$\tmodenor$ & $\cmodenor$,$\tmodenor$ & $\cmodenor$,$\tmodenor$ & $\cmodefun$,$\tmodefun$\\[0.1em]
\hline
\umode&$\tmodenor$,$\umode$ & $\tmodenor$,$\umode$ & $\tmodenor$,$\umode$ & $\tmodefun$,$\umode$\\[0.1em]
\hline
    \end{tabular}
}
  \caption{Permitted Memory Modes in Context Regions}
  \label{fig:checkc-mode}
\end{figure}

\begin{figure}[t]
{\small
  \begin{mathpar}

  \inferrule[]
  {}
  {m \vdash \tint}
\qquad
  \inferrule[]
  {\kappa \wedge m\vdash \tau \\ \kappa \le m}
  {m \vdash \tptr{\tarrayb{\bvar}{\tau}}{\kappa}}
\qquad
  \inferrule[]
  {\kappa \wedge m \vdash \tau\\ \kappa \le m}
  {m \vdash \tptr{\tau}{\kappa}}

  \inferrule[]
  {\forall \tau'\in \overline{\tau}\cup\{\tau\}\,.\,\kappa \wedge m \vdash \tau'\\ \kappa \le m \\ \fv(\overline{\tau})\cup\fv(\tau)\subseteq \overline{x}}
  {m \vdash \tptr{(\tfun{\overline{x}}{\overline{\tau}}{\tau}}{\kappa})}
  \end{mathpar}
}
{\footnotesize
\[
\begin{array}{l} 
m \le m \qquad \tmode \le \kappa
\\[0.2em]
\tmode \wedge \cmode = \umode \qquad \kappa \wedge \umode = \umode
\qquad \cmode \wedge m = m 
\qquad  m_1 \wedge m_2 = m_2 \wedge m_1

\end{array}
\]
}
 \caption{Well-formedness for Types}
\label{fig:wftypes}
\end{figure}

This section describes \lang, the formal core model for both \systemname and \systemnamea.
We present its syntax, semantics, type system,
as well as \lang’s meta-theories, including the non-exposure, type soundness, and non-crashing theorems,
which are verified without the assumptions of the underlining sandbox structure.
\lang utilizes dynamic checks that are inserted in the compilation time to ensure memory safety, i.e.,
the \lang semantics (\Cref{sec:semantics}) describes the dynamic checks to perform in transitions,
while its type system (\Cref{sec:typechecking}) indicates the places needing for inserting such checks.

\subsection{Syntax}\label{sec:syntax}
\Cref{fig:checkc-syn} shows the syntax of~\lang, with marked red items being new added features different from the \checkedc model \cite{li22checkedc}.

\myparagraph{Types}
%
\systemname and \checkedc are similar in terms of type syntax, except for function types.
We classify types as word-size value, multi-word
value, or newly introduced function types (explained shortly below). A word-size value can be either an integer or pointer.
Every pointer type ($\tptr{\omega}{\kappa}$) includes a
memory mode (explained shortly below), and a type ($\omega$) denoting the valid value type it is pointed to.
A multi-word value type ($ \tarrayb{\bvar}{\tau}$) that ranges over 
arrays is constructed by the type of elements in the
array ($\tau$), and a bound ($\bvar$) comprised of an upper and
lower bound on the size of the array $(b_l,b_h)$.
Bounds $b$ are
limited to integer literals $n$ and expressions $x + n$.
An example representation of an array in~\lang is shown below:
% 
\[\hspace*{-0.5em}
\begin{array}{l}
\begin{array}{rcl}
$\code{t_array_ptr<}$\tau$\code{> : count(}$n$\code{)}$
&\Leftrightarrow& \tarrayptr{0}{n}{\tau}{\tmode}
\end{array}
\end{array}
\]

For simplicity, we write
$\tptr{\tarrayb{b}{\tau}}{\cmode}$ to mean $\tptr{\tarray{0}{b}{\tau}}{\cmode}$,
so the above examples could be rewritten as $\tptr{\tarrayb{n}{\tau}}{\tmode}$.
There are other syntactic categories for modeling null-terminated array and \code{struct} types in \Cref{app:main}.

\myparagraph{Modes and Type Well-formedness}
%
In \checkedc, modes can act as a context division to partition code into checked (\cmode) and unchecked (\umode) regions, and a pointer separation to divide pointers into checked (\cmode) and unchecked (\umode) pointers.
In \systemname, we keep the context division, while upgrade mode's pointer separation aspect to a memory partition mechanism that separates heap into three disjoint areas: checked (\cmode), tainted (\tmode), or unchecked (\umode) heap regions.

More specifically, we partition each of the above heap regions to a subrange (subscripted by $\texttt{f}$) storing static function code data, and a subrange (subscripted by $\texttt{n}$) storing normal heap data,
so function pointers and heap pointers point to different subranges conceptually.
Different operations in different context regions are allowed to access pointers with different modes, shown in \Cref{fig:checkc-mode}.
For example, \texttt{malloc} operations in \cmode region produce pointers accessing $\cmodenor$ or $\tmodenor$ memory subranges, while a \texttt{malloc} in $\umode$ region produces a pointer accessing $\tmodenor$ and $\umode$ subranges.
Function calls ($\ecall{e}{\overline{e}}$) in \cmode region are allowed to access pointers in $\cmodefun$ or $\tmodefun$ subranges,
while calls in \umode region are permitted to access pointers in the $\umode$ heap region.

\systemname nested pointers also has well-formedness type restrictions in~\Cref{fig:wftypes},
to prevent unsafe pointers, which might leak \cmode mode pointer address to \umode regions, from being constructed.
Consider the pointer type ~\code{t_array_ptr<ptr<int>>}, which describes a tainted array of checked pointers.
This is not well-formed in~\lang because it potentially exposes the checked pointer addresses in a $\umode$ region (a violation of the non-exposure property in \Cref{sec:theorem}), when the tainted ($\tmode$) array is used. Nevertheless, we can have a checked array whose elements are tainted pointers\mzs{,}\mzr{:}~\eg~\code{array_ptr<t_ptr<int>>} is a valid type.

Function types represent dependent functions,~\ie $\tfun{\overline{x}}{\overline{\tau}}{\tau}$,
where $\overline{x}$ represents a list of \tint{} type variables that bind type variables appearing in $\overline{\tau}$ and $\tau$.
An example of a function pointer type is shown below:
{\small
\[\hspace*{-0.5em}
\begin{array}{l}
$\code{t_ptr<(int)(t_array_ptr<}$\tau_1$\code{> : count(}$n$\code{),}$\\
\qquad\qquad$\code{t_array_ptr<}$\tau_2$\code{>: count(}$n$\code{), int}$\;n$\code{)>}$
\\[0.2em]
\Leftrightarrow\; $\tptr{(\tfun{n}{\tint \times \tarrayptr{0}{n}{\tau_2}{\tmode} \times \tarrayptr{0}{n}{\tau_1}{\tmode}}{\tint})}{\tmode}$
\end{array}
\]
}
The function type also has well-formedness definition (\Cref{fig:wftypes}), which disallows nesting checked pointers inside tainted pointers. Furthermore, the well-formedness also ensures that all variables in $\overline{\tau}$ and $\tau$ are bounded by $\overline{x}$.

\myparagraph{Expressions}
\lang expressions include common expressions such as addition ($\ebinop{e_1}{e_2}$), 
pointer dereference ($\estar{e}$) and assignment ($\eassign{e_1}{e_2}$),
along with expressions that require special handling, such as,
 static casts ($\ecast{\tau}{e}$),
memory allocations ($\emalloc{\kappa}{\omega}$), 
function calls ($\ecall{e}{\overline{e}}$).
We denote integer literals $n$ with a type $\tau$ (\ie $\tint$ or $\tptr{\omega}{\kappa}$), enabling the use of fixed addresses as pointers.
For example, $\evalue{0}{\tptr{\omega}{\kappa}}$ (for any $\kappa$ and $\omega$) represents a $\enull$ pointer.

Compared with \checkedc, \lang's \texttt{malloc} operation includes a mode $\kappa$ to model the allocations of pointers with different modes; $e$ in a function call $\ecall{e}{\overline{e}}$ represents a function pointer whose pointer mode indicates the memory range the function text locates, as well as the context-switch of context regions, explained shortly in \Cref{sec:semantics}.

\lang{} aims to be simple enough to work with but powerful enough to
encode realistic \systemname idioms. For example,
loops can be encoded as recursive function calls. 
More operations, such as \code{struct}s are given in \Cref{app:main}.

\subsection{Semantics}\label{sec:semantics}

\begin{figure}
{\small
$\hspace*{-1.2em}
    \begin{array}{l}
    \begin{array}{lll}
\mu & ::= & \evalue{n}{\tau} \mid \bot\\[0.1em]
e & ::= & \ldots \mid \ret{x}{\mu}{e}\\[0.1em]
r & ::= & e \mid \enull \mid \ebounds\\[0.1em]
E &::=& \Box \mid \ebinop{E}{e} \mid \ebinop{\evalue{n}{\tau}}{E}\mid \ecast{\tau}{E} \mid\estar{E}\mid\eassign{E}{e}
\mid\eassign{\evalue{n}{\tau}}{E}\\[0.1em]
&\mid& \eif{E}{e}{e}\mid \elet{x}{E}{e}\mid\ret{x}{\mu}{E}\\[0.1em]
&\mid& \textcolor{red}{\ecall{E}{\overline{e}}} \mid \textcolor{red}{\ecall{(\evalue{n}{\tau})}{\overline{E}}}
\mid \textcolor{red}{\toa{m}{\overline{x}}{e}}\\[0.1em]
\overline{E} &::=& E \mid  \evalue{n}{\tau},\overline{E} \mid \overline{E}, e
\end{array}
\\ \\
    \end{array} 
$
  \begin{mathpar}
    \inferrule[S-Frame]{ m=\mode(E) \\
      e=E[e'] \\
      (\varphi,\heap,e') \longrightarrow (\varphi',\heap',e'')}
    {(\varphi,\heap,e)\longrightarrow_{m} (\varphi',\heap',E[e''])}

\textcolor{red}{
    \inferrule[S-Recov]{ \umode=\mode(E) \\ \tau=\type(e)}
    {(\varphi,\heap,E[e])\longrightarrow_{\umode} (\varphi,\heap,E[\evalue{0}{\tau}])}
}

  \end{mathpar}
}
  \caption{\lang Semantics: Evaluation}
  \label{fig:c-context}
\end{figure}


The operational semantics for \lang is defined as a small-step
transition relation with the judgment $ (\varphi,\heap,e)
\longrightarrow_m (\varphi',\heap',r)$. Here, $\varphi$ is a
\emph{stack} mapping from variables to values $\evalue{n}{\tau}$ and
$\heap$ is a \emph{heap} that is partitioned into three parts ($\cmode$, $\tmode$, and $\umode$ heap regions), each of which
maps addresses (integer literals) to values $\evalue{n}{\tau}$; for all we ensure
$\fv(\tau)=\emptyset$.

While heap bindings can change, stack bindings are immutable---once
variable $x$ is bound to $\evalue{n}{\tau}$ in $\varphi$, that binding will not
be updated. 
We wrote $\heap(\kappa)$ to access the $\kappa$ heap region in $\heap$ and
$\heap(\kappa,n)$ to retrieve the $n$-location heap value in the $\kappa$ heap.
As mentioned, value $\evalue{0}{\tau}$
represents a $\enull$ pointer when $\tau$ is a pointer type.
Correspondingly, $\heap(m,0)$ should always be undefined.
For $\kappa=\cmode \vee \kappa = \tmode$, there is a bound value $\eta_{\kappa}$,
such that range $(0,\eta_{\kappa})$ in $\heap(\kappa)$ is reserved for function code storage, 
where we define function $\Xi(\kappa,n)$ to access the function code appearing in the $n$-th location of $\kappa$ heap region.

The relation steps to a \emph{result} $r$,
which is either an expression or a $\enull$ or $\ebounds$ failure,
representing a null-pointer dereference or out-of-bounds access,
respectively. Such failures are a \emph{good} outcome; stuck states
(non-value expressions that cannot transition to a result $r$)
characterize undefined behavior.
%
The context mode $m$ indicates whether the
stepped redex within $e$ was in a $\cmode$ or $\umode$ code region.

The rules for the main operational semantics
judgment---\emph{evaluation}---are given at the bottom of
Fig.~\ref{fig:c-context}.
The first rule takes an expression $e$, decomposes
it into an \emph{evaluation context} $E$ and a sub-expression $e'$
(such that replacing the hole $\Box$ in $E$ with $e'$ would yield
$e$), and then evaluates $e'$ according to the \emph{computation}
  relation $(\varphi,\heap,e') \longrightarrow (\varphi,\heap,e'')$,
whose rules are given in Fig.~\ref{fig:semantics}, discussed
shortly. 
Evaluation contexts $E$ define a
standard left-to-right evaluation order.
For example, context $\ecall{(\evalue{n}{\tau})}{\overline{E}}$ means that after the function pointer expression $e$ in $\ecall{e}{\overline{e'}}$ is evaluated to a function pointer $\evalue{n}{\tau}$, we now evaluate each argument expression, appearing in $\overline{e'}$, in sequence, such redex is marked as $\overline{E}$.

Compared with the \checkedc formalism, other than the additional contexts in \Cref{fig:c-context},
we add a new marked red rule \rulelab{S-Recov},
referring to that every $\umode$ region transition step might crash, because it represents the unsafe code,
and we have a mechanism to recover the crash as $\evalue{0}{\tau}$ \footnote{Function $\type$ represents the call of the type judgment in \Cref{sec:typechecking} to provide the type of the expression.}.
The $\mode$ function
determines the mode when evaluating $e'$ based on the context $E$:
if the $\Box$ occurs within $\toa{m}{\overline{x}}{E'}$ inside $E$ and $E'$ does not contain another $\totext$ expression, 
then the mode is $m$.
Expression $\ret{x}{\mu}{e}$ does not appear in source program syntax in \Cref{fig:checkc-syn}, are introduced in evaluating programs,
which is used for remembering $x$ was previously bound to
$\varphi(x)$ in a $\elettext$ binding, explained shortly below.

%The second rule in \Cref{fig:c-context} is newly introduced in \lang, to help the proof of the non-crashing theorem.
%In any given time, executing a unchecked region might non-deterministically cause the program to crash. We assume that there is a recovery mechanism for recover every such crashing, only in unchecked regions, to an initial state $\evalue{0}{\tau}$;
%thus, non-crashing shows that every unchecked region crashing does not affect the program evaluation in the checked regions.

\begin{DIFnomarkup}
\begin{figure*}[t]
{\footnotesize
  \begin{mathpar}
    \inferrule[S-DefArr]{\kappa \neq \tmode\\ n\ge\eta_{\tmode} \\\heap(\kappa,n)=\evalue{n_a}{\tptr{\tau_a}{\kappa}} \\ 0 \in [n_l,n_h)}
    {(\varphi,\heap,\estar{\evalue{n}{\tntarrayptr{n_l}{n_h}{\tau}{\kappa}}}) \longrightarrow (\varphi,\heap,\evalue{n_a}{\tau})}

        \inferrule[S-Def]{\kappa \neq \tmode\\n\ge\eta_{\tmode} \\\heap(n)=\evalue{n_a}{\tptr{\tau_a}{\kappa}} }
    {(\varphi,\heap,\estar{\evalue{n}{\tptr{\tau}{\kappa}}}) \longrightarrow (\varphi,\heap,\evalue{n_a}{\tau})}

    \inferrule[S-DefArrT]{n\ge\eta_{\tmode}\\\heap(\kappa,n)=\evalue{n_a}{\tptr{\tau_a}{\tmode}} \\ 0 \in [n_l,n_h)\\\heap;\emptyset \vdash \evalue{n_a}{\tau}}
    {(\varphi,\heap,\estar{\evalue{n}{\tntarrayptr{n_l}{n_h}{\tau}{\kappa}}}) \longrightarrow (\varphi,\heap,\evalue{n_a}{\tau})}

        \inferrule[S-DefT]{n\ge\eta_{\tmode}\\\heap(n)=\evalue{n_a}{\tptr{\tau_a}{\tmode}}\\\heap;\emptyset \vdash \evalue{n_a}{\tau}}
    {(\varphi,\heap,\estar{\evalue{n}{\tptr{\tau}{\kappa}}}) \longrightarrow (\varphi,\heap,\evalue{n_a}{\tau})}

    \inferrule[S-DefArrBound]{0 \not\in [n_l,n_h)}
     { (\varphi,\heap,\estar{\evalue{n}{\tallarrayptr{n_l}{n_h}{\tau}{\kappa}}}) \longrightarrow (\varphi,\heap,\ebounds)}

    \inferrule[S-DefNullF]{n<\eta_{\tmode}}{(\varphi,\heap,\estar{\evalue{n}{\tptr{\omega}{\kappa}}}) \longrightarrow (\varphi,\heap,\enull)}

    \inferrule[S-DefNull]{\heap(\kappa',n)=\evalue{n_a}{\tptr{\tau_a}{\kappa}}\\\kappa\neq\kappa'}{(\varphi,\heap,\estar{\evalue{n}{\tptr{\omega}{\kappa}}}) \longrightarrow (\varphi,\heap,\enull)}

    \inferrule[S-Cast]
              {}
              {(\varphi,\heap,\ecast{\tau}{\evalue{n}{\tau'}}) \longrightarrow (\varphi,\heap,\evalue{n}{\varphi(\tau)})}

  \inferrule[S-Malloc]{\varphi(\omega)=\omega_a \\ alloc(\heap,\kappa,\omega_a)=(n,\heap')}
   { (\varphi,\heap,\emalloc{\kappa}{\omega}) \longrightarrow (\varphi,\heap',\evalue{n}{\tptr{\omega_a}{\kappa}})}

    \inferrule[S-Switch]
              {}
              {(\varphi,\heap,\toa{m}{\overline{x}}{\evalue{n}{\tau}}) \longrightarrow (\varphi,\heap,\evalue{n}{\tau})}

        \inferrule[S-Let]{}{(\varphi,\heap,\elet{x}{\evalue{n}{\tau}}{e}) \longrightarrow (\varphi[x\mapsto \evalue{n}{\tau}],\heap,\ret{x}{\varphi(x)}{e})}

    \inferrule[S-Fun]{ \kappa\neq \umode\\n\in(0,\eta_{\kappa})\\\Xi(\kappa,n) = \tau\;(\evalue{\overline{x}}{\overline{\tau}})\;e}
        {(\varphi,\heap,\ecall{(\evalue{n}{\tptr{\tau}{\kappa})}}{{\evalue{\overline{n_a}}{\overline{\tau_a}}}}) \longrightarrow
   (\varphi,\heap, \mathtt{let}\;\overline{x}={\evalue{\overline{n}}{(\overline{\tau}[\overline{n} / \overline{x}])}}\;\mathtt{in}\;\ecast{\tau[\overline{n} / \overline{x}]}{e})}

    \inferrule[S-Ret]{}{(\varphi,\heap,\ret{x}{\mu}{\evalue{n}{\tau}}) \longrightarrow (\varphi[x\mapsto \mu],\heap,\evalue{n}{\tau})}

    \inferrule[S-FunU]{ n\in(0,|\heap(\umode)|)\\ \Xi(\umode,n) = \tau\;(\evalue{\overline{x}}{\overline{\tau}})\;e}
        {(\varphi,\heap,\ecall{(\evalue{n}{\tptr{\tau}{\umode})}}{{\evalue{\overline{n_a}}{\overline{\tau_a}}}}) \longrightarrow
   (\varphi,\heap, \mathtt{let}\;\overline{x}={\evalue{\overline{n}}{(\overline{\tau}[\overline{n} / \overline{x}])}}\;\mathtt{in}\;\ecast{\tau[\overline{n} / \overline{x}]}{e})}
\end{mathpar}
}
% {\footnotesize
% \begin{center}
% $
% \begin{array}{l}
% \tau[\overline{n} / \overline{x}]\texttt{(with types }\evalue{\overline{x}}{\overline{\tau}}\texttt{)}\triangleq \forall n_i\in\overline{n}\;x_i\in\overline{x}\;\tau_i\in\overline{\tau}\;.\;\tau_i = \tint \Rightarrow \tau[n_i / x_i]\\[0.2em]
% \mathtt{let}\;\overline{x}=\overline{e}\;\mathtt{in}...\triangleq \mathtt{let}\;x_0=e_0\;\mathtt{in}\;\mathtt{let}\;x_1=e_1\;\mathtt{in}...
% \end{array}
% $
% \end{center}
% }
\caption{\lang Semantics: Computation (Selected Rules)}
\label{fig:semantics}
\end{figure*}
\end{DIFnomarkup}

\ignore{
\begin{figure}[t]
{\small
{\captionsetup[lstlisting]{margin = 8 mm}
  \begin{lstlisting}[xleftmargin=8 mm]
nt_array_ptr<char> safe_strcat
   (nt_array_ptr<char> dst : count(n),
    nt_array_ptr<char> src : count(0), int n) {
  int x = strlen(dst);
  int y = strlen(src);
  nt_array_ptr<char> c : count(n) =
    dyn_bounds_cast
           <nt_array_ptr<char>>(dst,count(n));
    // sets c == dst with bound n (not x)
  if (x+y < n) {
    for (int i = 0; i < y; ++i)
      *(c+x+i) = *(src+i);
    *(c+x+y) = '\0';
    return dst;
  }
  return null;
}
  \end{lstlisting}
}
}
\caption{Implementation of safe \code{strcat}}
\label{fig:strcat-ex}
\end{figure}
}

\Cref{fig:semantics} shows selected rules for showing the key guarantees that \systemname and \systemnamea provide
; we explain them with the help of the example in
\Cref{sec:overview}.
\ignore{
  which defines a 
  safe version of \code{strcat} (using actual Checked C syntax).  The
  function takes a target 
  pointer \code{dst} of capacity \code{n}, where the first null
  character (determined by \code{strlen}) is at index \code{x} where
  $0 \leq $\code{x}$ \leq n$. It concatenates the \code{src} buffer to
  the end of \code{dst} as long as \code{dst} has sufficient space.
}
% Below, we introduce low-level transition semantics for some case operations. The design of the low-level individual operation semantics is carefully engineered to perform match our compiler's behavior, such as correctly characterizing the bound widening behaviors for NT-array pointers, even though it is written in terms of fat-pointer formalization.

\myparagraph{Pointer accesses, Casts, and Mallocs.}
%
We describe rules (\rulelab{S-DefArr}, \rulelab{S-Def}, \rulelab{S-DefArrBound}, \rulelab{S-DefNull}, and \rulelab{S-Cast}) for dereference and cast operations to illustrate how the dynamic checks are performed in \lang.
For every heap region $\heap(\kappa)$, every dereference and assignment operation (e.g., rule \rulelab{S-Def}) ensures that the pointer address is within the low and high bound range ($[\eta_{\kappa},|\heap(\kappa)|)$) of the heap region,
where the low bound refers to the bound $\eta_{\kappa}$ splitting the function code and heap data subranges
and the high bound is the cardinality of the heap ($|\heap(\kappa)|$).
If the check is not satisfied, the semantics produces a $\enull$ state in \rulelab{S-DefNull}.
When $\enull$ is returned by the
computation relation, the evaluation relation halts the entire
evaluation with $\enull$ (using a rule not shown in Fig.~\ref{fig:c-context}); it
does likewise when $\ebounds$ is returned (see below).

\rulelab{S-DefArr} dereferences an element in an array as long as $0$ (the point of
dereference) is within the bounds designated by the pointer's annotation
and strictly less than the upper bound ($0 \in [n_l,n_h)$),
while the location address $n$ should be within the bounds of the specific heap region ($n\in[\eta_{\kappa},|\heap(\kappa)|)$).
The former check is a $\ebounds$ check that verifies the element dereference is within the array bound,
while the latter is a $\enull$ check that ensures that the pointer exists in the correct heap region.
The dereference semantics and type rules are general enough to be used in both $\cmode$ and $\umode$ code regions.
As we mentioned in \Cref{sec:background}, $\umode$ code regions contain mostly C code that does not have array annotations.
In this case, a $\umode$ mode array pointer is cast and viewed as a normal C pointer, where we apply rule \rulelab{S-Def} to deal with its dereference.

Rules \rulelab{S-DefArrT} and \rulelab{S-DefT} define the dereference operations for $\tmode$ mode pointers.
Here, we perform an additional type verification $\heap;\emptyset \vdash \evalue{n_a}{\tau}$ making sure that the memory location context $n_a$ has the declared type $\tau$ appearing in the operation $\estar{\evalue{n}{\tptr{\tau}{\kappa}}}$.
Such check not only verifies if $n_a$ has type $\tau$, but also if $n_a$ is a nested pointer,
the check further verifies if all addresses within $n_a$'s legal scopes are well-typed.
In \Cref{lst:final}, function \code{read_msg} casts a $\umode$ mode pointer and assigns it to a tainted array pointer and returns to the checked function \code{process_req2}; such behavior is legal in a $\umode$ code region. 
\code{process_req2} catches the bug in line 10 through the type verification above, because pointer $x$ cannot have type $\tptr{\tarrayptr{0}{5}{\tint}{\tmode}}{\tmode}$ since it is illegal to cast type $\tptr{\tptr{\umode}{\tint}}{\tmode}$ to it.

Static casts of a literal $n\!:\!\tau'$ to a type $\tau$ are handled
by \rulelab{S-Cast}. In a type-correct program, such casts are
confirmed safe by the type system. To evaluate a cast, the rule
updates the type annotation on $n$. Before doing so, it must
``evaluate'' any variables that occur in $\tau$ according to their
bindings in $\varphi$. As an instance, if $\tau$ was
$\tarrayptr{0}{x+2}{\tint}{\cmode}$, then $\varphi(\tau)$ would
produce $\tarrayptr{0}{4}{\tint}{\cmode}$ if $\varphi(x) = 2$.
Rule \rulelab{S-Malloc} creates a new memory pointer $\tptr{\omega_a}{\kappa}$
in heap region $\heap(\kappa)$ through function $alloc$, which returns a neq heap $\heap'$ that creates a new memory space in $\heap(\kappa)$ and increments $\heap(\kappa)$'s cardinality, as $|\heap'(\kappa)|=|\heap(\kappa)|+$\code{sizeof}$(\omega_a)$;
$\varphi(\omega)$ evaluates potential bound variables in $\omega$ and returns a new type $\omega_a$ with no variables.

\myparagraph{Binding, Function Calls, and Context Switching}
We handle variable scopes using the special $\erettext$
operation. \textsc{S-Let} evaluates to a configuration whose stack
is $\varphi$ extended with a binding for $x$, and whose expression is
$\ret{x}{\varphi(x)}{e})$ which remembers $x$ was previously bound to
$\varphi(x)$; if it had no previous binding, $\varphi(x) =
\bot$. Evaluation proceeds on $e$ until it becomes a literal
$n\!:\!\tau$, in which case \textsc{S-Ret} restores the saved
binding (or $\bot$) in the new stack, and evaluates to
$n\!:\!\tau$. 

Function calls are handled by rules \rulelab{S-FunU} and \rulelab{S-Fun},
for $\umode$ and non-$\umode$ mode functions \footnote{A $\kappa$ mode function means the function pointer is labeled with mode $\kappa$ and its code is stored in $\kappa$ heap region.}, respectively.
When calling a function pointer $\evalue{n}{\tptr{\tau}{\kappa}}$,
we extra the $n$-th location code data from $\Xi(\kappa)$, referring to the function code fields in the $\kappa$ heap region and makes a memory address to $\tau\;(\evalue{\overline{x}}{\overline{\tau}})\;e$, where
$\tau$ is the return type, $(\evalue{\overline{x}}{\overline{\tau}})$
is the parameter list of variables and their types, and $e$ is the
function body.
We place a bound check in rule \rulelab{S-Fun} to 
ensure that $\cmode$ and $\tmode$ mode function calls only access addresses in the range $(0,\eta_{\kappa})$,
the code data field ranges for $\cmode$ and $\tmode$, respectively;
while $\umode$ function calls (\rulelab{S-FunU}) are allowed to access all $\umode$ heap region.
Similar to \checkedc, \lang function calls are dependent, i.e.,
array bounds in types may refer to in-scope variables; e.g., 
\code{process_req1}'s bound \code{count(m_1)} refers to parameter \code{m_1} in \Cref{lst:final}
The call is expanded into a \texttt{let} which binds
parameter variables $\overline{x}$ to the actual arguments
$\overline{n}$, but annotated with the parameter types
$\overline{\tau}$ (this will be safe for type-correct programs). The
function body $e$ is wrapped in a static cast
$(\tau[\overline{n} / \overline{x}])$ which is the function's return
type but with any parameter variables $\overline{x}$ appearing in that
type substituted with the call's actual arguments $\overline{n}$. 
\ignore{
To
see why this is needed, suppose that \code{safe_strcat} in
Fig.~\ref{fig:strcat-ex} is defined to return a
\code{nt_array_ptr<int>:count(n)} typed term, and assume that we
perform a \code{safe_strcat} function call as
\code{x=safe_strcat(a,b,10)}. After the evaluation of \code{safe_strcat}, the
function returns a value with type \code{nt_array_ptr<int>:count(10)}
because we substitute bound variable \code{n} in the 
defined return type with \code{10} from the function call's
argument list.
}
Note that the \textsc{S-Fun} rule replaces the
  annotations $\overline{\tau_a}$ with
  $\overline{\tau}$ (after instantiation) from the function's
  signature. Using $\overline{\tau_a}$ when executing the body of
the function has no impact on the soundness of \lang, but will violate
Theorem~\ref{simulation-thm}, which we introduce in Sec.~\ref{sec:compilation}.
Rule \rulelab{S-Swtich} defines the context mode switch block's semantics, which gets rid of the $\texttt{to}$ block once its context is a value.

\subsection{Typing}
\label{sec:typechecking}

\begin{DIFnomarkup}
\begin{figure*}[t]
{\small
  \begin{mathpar}
    \inferrule[T-DefArr]
              {\kappa\le m \\ \Gamma;\Theta \vdash_{m} e : \tptr{\tarrayb{\bvar}{\tau}}{\kappa}}
              {\Gamma;\Theta \vdash_m \estar{e} : \tau}

    \inferrule[T-Def]
              {\kappa \leq m \\\Gamma \vdash_m e : \tptr{\tau}{\kappa}}
              {\Gamma \vdash_m \estar{e} : \tau}

     \inferrule[T-Cast]
               {\Gamma \vdash_{\cmode} e : \tau' \\
                 \tau' \sqsubseteq \tptr{\tau}{\kappa}}
               {\Gamma \vdash_{\cmode} \ecast{\tptr{\tau}{\kappa}}{e} : \tptr{\tau}{\kappa}}

    \inferrule[T-CastU]
              {\kappa \le \umode \\
                \Gamma;\Theta \vdash_{\umode} e : \tau'}
              {\Gamma;\Theta \vdash_{\umode} \ecast{\tptr{\tau}{\kappa}}{e} : \tptr{\tau}{\kappa}}
                
    \inferrule[T-Malloc]
              {\kappa\le m}
              {\Gamma \vdash_m \emalloc{\kappa}{\omega} : \tptr{\omega}{\kappa}}

   \inferrule[T-Let]
    { x\not\in \fv(\tau') \\
        \Gamma \vdash_m e_1 : \tau \\\\
          \Gamma[x\mapsto \tau] \vdash_m e_2 : \tau'
             }
    {\Gamma \vdash_m \elet{x}{e_1}{e_2} : \tau'}

   \inferrule[T-LetInt]
    {\Gamma \vdash_m e_1 : \tint \\\\
           \Gamma[x\mapsto \tint] \vdash_m e_2 : \tau'\\\\
        x\in \fv(\tau') \Rightarrow e_1 \in \text{Bound} 
             }
    {\Gamma \vdash_m \elet{x}{e_1}{e_2} : \tau'[e_1 / x]}

\inferrule[T-Ret]
    {\Gamma(x)\neq \bot \\
          \Gamma;\Theta \vdash_m e : \tau}
    {\Gamma;\Theta \vdash_m \eret{x}{\mu}{e} : \tau}

    \inferrule[T-Switch]
              {\forall x\in\{\overline{x}\}.\,\neg\cmode(\Gamma(x))\\\neg\cmode(\tau)
                     \\\\\fv(e)\subseteq\overline{x}\\\Gamma \vdash_{m'} e : \tau}
              {\Gamma \vdash_m \toa{m'}{\overline{x}}{e} : \tau}

\inferrule[T-Fun]
    {\Gamma \vdash_m e : \tptr{\tfun{\overline{x}}{\overline{\tau}}{\tau}}{\kappa} \\
        \Gamma \vdash_m \overline{e} : \overline{\tau'} \\
         \overline{e'}=\{e'|(e',\tint)\in (\overline{e} : \overline{\tau'})\}\\\\
         \forall e'\;.\;e' \in \overline{e'} \Rightarrow e'\in \text{Bound}\\
             \overline{\tau'} \sqsubseteq
               \overline{\tau}[\overline{e'} / \overline{x}]}
    {\Gamma \vdash_m e(\overline{e}) : \tau[\overline{e'} / \overline{x}]}
  \end{mathpar}
}
% {\footnotesize
% \begin{center}
% $
% \begin{array}{l}
% \fm(e)\triangleq(\exists x\; n\; \tau. e=x+\evalue{n}{\tau}) \vee (\exists n\;\tau. e = \evalue{n}{\tau})
% \\[0.2em]
% \tau[\overline{e} / \overline{x}]\texttt{(with types }\evalue{\overline{x}}{\overline{\tau}}\texttt{)}\triangleq \forall e_i\in\overline{e}\;x_i\in\overline{x}\;\tau_i\in\overline{\tau}\;.\;\tau_i = \tint \wedge (x_i \in \fv(\tau) \Rightarrow \fm(e_i)) \Rightarrow \tau[e_i / x_i]
% \end{array}
% $
% \end{center}
% }
{\footnotesize
\begin{center}
$
\cmode(\tint)=\texttt{false}
\qquad
\cmode(\tptr{\omega}{\cmode})=\texttt{true}
\qquad
\cmode(\tptr{\omega}{\kappa})=\texttt{false}\;\;{[\emph{owise}]}
$
\end{center}
}
\caption{Selected type rules}
\label{fig:type-system-1}
\end{figure*}
\end{DIFnomarkup}

The \lang type system is a flow-sensitive, gradual type one that generates additional dynamic checks that are inserted in the typing checking stage and executed in the semantic evaluation stage.
Our type checker restricts the usage of tainted and checked pointer types to ensure that tainted pointers do not affect checked types, along with enforcing~\checkedc typing rules~\cite{li22checkedc}.

As partly shown in \Cref{fig:type-system-1} (labeled as \rulelab{T-}$X$),
each typing judgment has the form $\Gamma;\Theta\vdash_m e : \tau$,
which states that in a type environment $\Gamma$ (mapping variables to
their types) and a predicate environment $\Theta$ (mapping integer-typed
variables to Boolean predicates), expression $e$ will have type $\tau$ if evaluated
in context mode $m$, indicating that the code is in $m$ region.
The operational semantics for \lang is defined as a small-step
transition relation with the judgment $ (\varphi,\heap,e)
\longrightarrow_m (\varphi',\heap',r)$, as shown in \Cref{fig:c-context}.
 Here, $\varphi$ is a
\emph{stack} mapping from variables to values $\evalue{n}{\tau}$ and
$\heap$ is a \emph{heap} that is partitioned into two parts ($\cmode$ and $\umode$ heap regions), each of which
maps addresses (integer literals) to values $\evalue{n}{\tau}$.
The complete set of typing rules and special handling of (NT)-arrays are provided in~\Cref{rem-type,sec:rem-semantics}.

We wrote $\heap(m,n)$ to retrieve the $n$-location heap value in the $m$ heap,
and $\heapup{m}{n}{\evalue{n'}{\tau}}$ 
to update location $n$ with the value $\evalue{n'}{\tau}$ in the $m$ heap.
While heap bindings can change, stack bindings are immutable---once
variable $x$ is bound to $\evalue{n}{\tau}$ in $\varphi$, that binding will not
be updated. 
%We can model mutable stack variables as pointers into the mutable heap.
As mentioned, value $\evalue{0}{\tau}$
represents a $\enull$ pointer when $\tau$ is a pointer type.
Correspondingly, $\heap(m,0)$ should always be undefined.
% 
The relation steps to a \emph{result} $r$, which is either an
expression, a $\enull$ or $\ebounds$ failure, represent an expression right
  after the reduction, a null-pointer dereference or out-of-bounds access,
respectively.
% 
Such failures are a \emph{good} outcome; stuck states
(non-value expressions that cannot transition to a result $r$)
characterizing undefined behavior.
%The context mode $m$ (in $\longrightarrow_{m}$) indicates whether the stepped redex within $e$ was in a $\cmode$ or $\umode$ region.

The rules for the main operational semantics
judgment \emph{evaluation} are given at the bottom of
Fig.~\ref{fig:c-context}.
The first rule takes an expression $e$, decomposes
it into an \emph{evaluation context} $E$ and a sub-expression $e'$
(such that replacing the hole $\Box$ in $E$ with $e'$ would yield
$e$), and then evaluates $e'$ according to the \emph{computation}
  relation $(\varphi,\heap,e') \longrightarrow (\varphi,\heap,e'')$,
whose rules are given along with type rules in Fig.~\ref{fig:type-system-1} (labeled as \rulelab{S-}$X$), discussed
shortly.
The $\mode$ function in Fig.~\ref{fig:c-context}
determines the context mode, i.e., region, that the expression $e'$ locates based on the context $E$.
In \Cref{lst:humantaint}, the function call \code{handle_request} is in $\umode$ region since it is inside an unchecked function \code{server_loop}.
The second rule describes the exception handling 
for possible crashing behaviors in $\umode$ regions.
Operations in $\umode$ region can non-deterministically crash
and the \systemname sandbox mechanism recovers
the program to a safe point ($\evalue{0}{\tau}$)
and continues with the existing program state.
Evaluation contexts $E$ define a standard left-to-right evaluation order.
%(We explain the $\ret{x}{\mu}{e}$ syntax shortly.)
%There are other rules for describing the halts of evaluation to $\enull$ and $\ebounds$ states in \Cref{app:main}.

\myparagraph{Modes, Static Casting, and Subtyping}
In \lang, Context modes $m$ appearing in a type rule determine the code region 
that permits pointer dereferences and value-assignments, which also depends on the pointer modes.
We define a three point lattice $\kappa_1 \le \kappa_2$ \footnote{In typing rule, the lattice is usually used as $\kappa \le m$ as $m$ represents context modes.} to describe such permission, where $\tmode \le \kappa$ and $m \le m$.
This means that a $\tmode$ pointer can be dereferenced and value-assigned in any region, while $\cmode$ and $\umode$ pointers can only perform such operations in $\cmode$ and $\umode$ regions, respectively.
%Pointer modes are also useful in determining if a nested pointer has a valid type. For example, in a nested pointer $\tptr{(... \tptr{\tau}{\kappa_2} ...)}{\kappa_1}$, we require $\kappa_2\le \kappa_1$ to maintain non-exposure.

\lang also provides static casting operations. As described in rule \textsc{T-CastPtr} in \Cref{fig:type-system-1},
an pointer typed expression of type $\tptr{\tau_1}{\kappa_1}$ can be casted to another pointer type \tptr{\tau_2}{\kappa_2},
if \tptr{\tau_1}{\kappa_1} subtypes ($\sqsubseteq_{\Theta}$) to \tptr{\tau_2}{\kappa_2}, i.e., $\tptr{\tau_1}{\kappa_1} \sqsubseteq \tptr{\tau_2}{\kappa_2}$.
In \lang, except that we can cast a $\tmode$ mode pointer to a $\umode$ mode one, all subtyping relations are between two types with the same mode, meaning that $\kappa_1$ and $\kappa_2$ above are mostly the same and the above mode lattice ($\le$) has no business with subtyping.

\begin{minted}[xleftmargin=30pt, mathescape, escapeinside=||, fontsize=\footnotesize]{c}
//_Ptr<int> x; _t_Ptr<int> y; int *z;
z = (int *)y; // This is okay.
x = (_Ptr<int>)y; // Not allowed.
\end{minted}

In the above example, a $\tmode$ mode pointer can be cast to $\umode$ mode but casting $\tmode$ mode to $\cmode$ mode is disallowed.
The complete subtyping relation was described in \Cref{app:le}.
Notice that \texttt{let} statements are immutable in \lang, so the following code is not possible, because variables \code{x} and \code{y} must have the same type in \lang.

\begin{minted}[xleftmargin=30pt, mathescape, escapeinside=||, fontsize=\footnotesize]{c}
//_Ptr<int> x;  _t_Ptr<int> y;
x = y; // Not allowed.
\end{minted}

\myparagraph{Pointer Dereference}
The type and semantic rules for pointer dereference (\textsc{T-Def}, \textsc{S-DefC},
\textsc{S-DefT}, \textsc{S-DefNull} in \Cref{fig:type-system-1})
reflect the key \lang feature, where our type checker directs the insertions of dynamic checks executed in the evaluation stage.
The type rule (\textsc{T-Def}) ensures that pointers are used with the right modes in the right region ($\kappa \le m$).
With the dynamic checks inserted by the compiler, rule \textsc{S-DefNull} ensure that if a $\enull$ pointer is used,
\lang captures the runtime error.
Type and semantic rules for array types and pointer assignments are given in \Cref{rem-type,sec:rem-semantics}.

Rules \textsc{S-DefC} and \textsc{S-DefT} are for $\cmode$ and $\tmode$ mode pointer dereferences, respectively.
In addition to the no $\enull$ check in $\cmode$ mode pointer dereference,
any dynamic heap access of a tainted ($\tmode$) pointer requires a \textit{verification} ($\emptyset;\heap ; \emptyset \vdash_{\umode}\evalue{n_a}{\tau}$), which refers to that the pointer value $n_a$ is well-defined in $\heap(m,n_a)$ and has right type $\tau$.

\myparagraph{Unchecked and Checked Blocks}
The execution of a $\echeckedtext$ or $\euncheckedtext$ block represents 
the context switching from a $\cmode$ to an $\umode$ region, or vice versa,
with its type and semantic rules given in \Cref{fig:type-system-1}.
In this context switching, to guarantee the checked ($\cmode$) pointer non-exposure property, 
checked pointers are not allowed to go cross different regions, which is guaranteed by the predicates 
$\forall x\in\overline{x}\;.\;\neg\cmode(\Gamma(x))$ and $\neg\cmode(\tau)$,
as well as the check that all free variables in the block content $e$ are in $\overline{x}$.
For example, \code{StringAuth} in \Cref{subsub:gencregion} is a trampoline function that disallows
checked pointers as arguments and return values.
The use of the function in the following \code{_T_StringAuth}, which is in $\umode$ region,
cannot legally acknowledges any checked pointers; otherwise, we might expose a checked pointer address to unsafe code regions.
In \systemname, we actually permits the accesses of checked pointers inside \code{StringAuth},
since the function body of a trampoline function is in $\cmode$ region.
More information is given in \Cref{subsub:gencregion}.

\myparagraph{Dependent Function Pointers} %
Rule \textsc{T-Fun} (\Cref{fig:type-system-1}) states the type judgment for
dependent function pointer application, where we represent the result of
replacing all integer bound variables $\overline{x}$ in the type \(\tau\) with
with bound expressions $\overline{e'}$ by $\tau[\overline{e'} / \overline{x}]$
and write $\overline{\tau}[\overline{e'} / \overline{x}]$ to lift the
substitution to every type in \(\overline{\tau}\).
% 
Given an expression $e$ of function pointer type
($\tptr{\tfun{\overline{x}}{\overline{\tau}}{\tau}}{\kappa}$) and arguments
$\overline{e}$ of types $\overline{\tau'}$,
% 
the result of the application will
be of type $\tau[\overline{e'} / \overline{x}]$;
if for each pair of \(\tau'\) and \(\tau''\) in \(\overline{\tau'}\) and
$\overline{\tau}[\overline{e'} / \overline{x}]$, \(\tau'\) is a subtype of
\(\tau''\).
Consider the \code{process_req2} function in
Fig.~\ref{lst:final}, whose parameter type for \code{msg} 
depends on \code{m_1}.
Its function pointer type is 
$\tptr{\tfun{\code{m_1}}{\tint,\tntarray{0}{\code{m_1}}{\texttt{char}}}{\tint}}{\tmode}$.
In \code{handle_request}, the call \code{process_req2(buff, r_len)} binds variable \code{m_1} to \code{r_len}.
After the call returns, \code{m_1}'s scope is ended, so we need to substitute it with \code{r_len} in the final return type
because it might contain \code{m_1}.

\textsc{S-FunC} and \textsc{S-FunT} define the semantics
for $\cmode$ and $\tmode$ mode function pointers, respectively. 
A call to a function pointer $n$ retrieves
 the function definition in $n$'s location in the global function store $\Xi$,
which maps function pointers to
function data $\tau\;(\evalue{\overline{x}}{\overline{\tau}})\;(\kappa,e)$, where
$\tau$ is the return type, $(\evalue{\overline{x}}{\overline{\tau}})$
is the parameter list of variables and their types, 
$\kappa$ determines the mode of the function, and $e$ is the
function body. 
Similar to \heap, the global function store $\Xi$ is also partitioned into
two parts ($\cmode$ and $\umode$ store regions), each of which
maps addresses (integer literals) to the function data described above.
Rule \textsc{S-FunT} defines the tainted version of function call
with the verification process 
$\emptyset;\heap ; \emptyset \vdash_{\umode}\evalue{n}{\tptr{\tau}{\tmode}}$
makes sure that the function in the global store is well-defined and has the right type.

\subsection{Meta Theories}\label{sec:theorem}

% Before we present our main theorems, we need to first
% discuss the meaning what a pointer being well-typed in a given heap
% snapshot $\heap$ means, which is captured by rules in
% Fig.~\ref{fig:const-type}. The variable type rule ($\textsc{T-Var}$)
% simply checks if a given variable has the defined type in $\Gamma$;
% the constant rule ($\textsc{T-Const}$) is slightly more involved.
% First, it ensures that the type annotation $\tau$ does not contain any
% free variables. More importantly, it ensures that the pointer points
% to a location that makes sense in a given heap.
%  
%  
%  The $\size$ function in Fig.~\ref{fig:const-type}
% refers to the \code{sizeof} function in C computing the number of
% bytes for a type.
%  
%  
%  Second, we
% require that any constant ($\evalue{n}{\tau}$) should make sense in
% $\heap$. We develop a recursive predicate $\sigma \vdash n : \tau$ to
% verify if $n$ has $\tau$ in a heap snapshot $\heap$. $\sigma$ is a
% constant set containing the constants that have been verified by the
% relation. For every constant $\evalue{n}{\tau}$, it is either an
% integer $\tint$, an unchecked pointer $\tptr{\omega}{\umode}$,
% zero-valued number ($n=0$), checked in $\sigma$
% ($\evalue{n}{\tptr{\omega}{\cmode}}\in \sigma$); or if it is not the
% above case, then (i) $\heap(n)$ is defined, and (ii) for every heap
% location $n+i$ in the range of the pointer (if $\omega$ is a word
% type, range is $[0,1)$; if $\omega$ is an array type
%   ($\tarray{0}{b_h}{\tau'}$), range is $[0,b_h)$, if $\tau$ is a
%     NT-array type ($\tntarray{0}{b_h}{\tau'}$), range is $[0,b_h+1)$),
%       if $\heap(n+i)=\evalue{n_a}{\tau_a}$, then
%       $\evalue{n_a}{\tau_a}$ satisfies $\sigma \cup \{(n,\tau) \}
%       \vdash n_a : \tau_a$.
%  
%  
% \begin{figure}[t]
% {\small
% \text{Type Rules for Constants and Variables:}
% \begin{mathpar}
%   \inferrule[T-Var]
%       {x : \tau \in \Gamma}
%       {\Gamma;\Theta \vdash_m x : \tau}
%  
%   \inferrule[T-Const]
%       {\fv(\tau) = \emptyset \\ \emptyset \vdash n : \tau}
%       {\Gamma;\Theta\vdash_m \evalue{n}{\tau} : \tau}
% \end{mathpar}
%     
% \text{Rules for Checking Constant Pointers In Heap:}
% \begin{mathpar}
%   \inferrule
%       {}
%       {\sigma \vdash n : \tint}
%  
%   \inferrule
%       {}
%       {\sigma \vdash n : \tptr{\omega}{\umode}}
%  
%   \inferrule
%       {}
%       {\sigma \vdash 0 : \tptr{\omega}{\cmode}}
%  
%   \inferrule
%       {\evalue{n}{\tptr{\omega}{\cmode}}\in \sigma}
%       {\sigma \vdash n : \tptr{\omega}{\cmode}}
%  
%   \inferrule
%       {\forall i \in [0,\size(\omega)) .
%            \sigma \cup \{(n:\tptr{\omega}{\cmode}) \}\vdash \heap(n+i)}
%       {\sigma \vdash n : \tptr{\omega}{\cmode}}
% \end{mathpar}
% }
% \caption{Type Rules for Checking Constants/Variables}
% \label{fig:const-type}
% \end{figure}

% \review{
%  Theorem 1 refers to a program $e$ being well-formed. Unless I've missed
%   something, I didn't see such a definition in the paper.}
% \mwh{This was stale text (dropped); $e$'s well formedness follows from the
%   assumption of well typing; we have added more details about that.}

Here, we discuss our main meta-theoretic results for
\lang: type soundness (progress and preservation),
non-exposure, and non-crashing.
These proofs have been conducted in our Coq model.
Type soundness relies on several \emph{well-formedness} given in \cite{li22checkedc} and \Cref{sec:meta}.
The progress theorem below states that a \lang program can always make a move.

\begin{thm}[Progress]\label{thm:progress}

For any \lang program $e$, heap $\heap$, stack
$\varphi$, type environment $\Gamma$, and variable predicate set $\Theta$
that are all are well-formed, consistent
($\Gamma;\Theta\vdash \varphi$ and $\heap \vdash \varphi$) and well
typed ($\Gamma;\Theta\vdash_{\cmode} e : \tau$ for some $\tau$),
one of the following holds:

\begin{itemize}

\item $e$ is a value ($\evalue{n}{\tau}$).

\item there exists $\varphi'$ $\heap'$ $r$, such that $(\varphi,\heap,e) \longrightarrow_m (\varphi',\heap',r)$.

\end{itemize}
\end{thm}
%{\em Proof:} By induction on the typing derivation.

\noindent
There are two forms of preservation regarding the $\cmode$ and $\umode$ regions.
Checked Preservation states that a reduction step preserves both the
type and consistency of the program being reduced, while
unchecked Preservation states that any evaluation happens at $\umode$ region does not affect the $\cmode$ mode heap.

\begin{thm}[Checked Preservation]
For any \lang program $e$, heap $\heap$, stack
$\varphi$, type environment $\Gamma$, and variable predicate set $\Theta$
that are all are well-formed, consistent
($\Gamma;\Theta\vdash \varphi$ and $\heap \vdash \varphi$) and well
typed ($\Gamma;\Theta\vdash_{\cmode} e : \tau$ for some $\tau$), if there exists $\varphi'$,
$\heap'$ and $e'$, such that $(\varphi,\heap,e)
\longrightarrow_{\cmode} (\varphi',\heap',e')$, then $\heap'$ is
$\cmode$ region consistent with $\heap$ ($\heap \triangleright \heap'$) and there exists
$\Gamma'$ and $\tau'$ that are well formed, $\cmode$ region consistent
($\Gamma';\Theta\vdash \varphi'$ and $\heap' \vdash \varphi'$) and
well typed ($\Gamma';\Theta \vdash_{\cmode} e: \tau'$), where
$\tau'\sqsubseteq_{\Theta} \tau$.
\end{thm}
%{\em Proof:} By induction on the typing derivation.
%\smallskip
\begin{thm}[Unchecked Preservation]
For any \lang program $e$, heap $\heap$, stack
$\varphi$, type environment $\Gamma$, and variable predicate set $\Theta$
that are all are well-formed and well
typed ($\Gamma;\Theta\vdash_{\cmode} e : \tau$ for some $\tau$), if there exists $\varphi'$,
$\heap'$ and $e'$, such that $(\varphi,\heap,e)
\longrightarrow_{\umode} (\varphi',\heap',e')$, then $\heap'(\cmode)=\heap(\cmode)$.
\end{thm}

Using the above theorems, we first show the non-exposure theorem,
where code in $\umode$ region cannot observe a valid checked ($\cmode$) pointer address.

\begin{thm}[Non-Exposure]
For any \lang program $e$, heap $\heap$, stack
$\varphi$, type environment $\Gamma$, and variable predicate set $\Theta$
that are all are well-formed and well
typed ($\Gamma;\Theta\vdash_{\cmode} e : \tau$ for some $\tau$), if there exists $\varphi'$,
$\heap'$ and $e'$, such that $(\varphi,\heap,e)
\longrightarrow_{\umode} (\varphi',\heap',e')$ and $e=E[\alpha(x)]$ and $\mode(E)=\umode$,
where $\alpha(x)$ is some expression (not $\echeckedtext$ nor $\euncheckedtext$) containing variable $x$; 
thus, it is not a checked pointer.
\end{thm}

We now state our main result, {\em non-crashing},
which suggests that a well-typed program can never be \emph{stuck} (expression
$e$ is a non-value that cannot take a step\footnote{Note that
  $\ebounds$ and $\enull$ are \emph{not} stuck expressions---they represent a
  program terminated by a failed run-time check. A program that tries to access $\heap{n}$
  but $\heap$ is undefined at $n$ will be stuck, and violates spatial
  safety.}).

% \review{- There appears to be a slight discrepancy between the blame theorem in Coq and the one in the paper: the paper mentions some e', which I believe should be r. Also, the Coq code has a further disjunct m=Unchecked in the conclusion.}
% \liyi{It is a typo. We will add the thing back that we show that either user uses a unchecked mode to evaluate $e$ or $e$ lives in a context that is an unchecked region. This is a bit due to the space limitation. The semantic rules allow users to input the mode $m$ of evaluating an expression, I just forgot to include the $m$ in the result of the proof statement. }

\begin{thm}[Non-Crashing]\label{thm:blame} For any \lang
  program $e$, heap $\heap$, stack
$\varphi$, type environment $\Gamma$, and variable predicate set $\Theta$
that are well-formed and consistent
($\Gamma;\Theta\vdash \varphi$ and $\heap \vdash \varphi$),
if $e$ is well-typed ($\varphi;\Theta\vdash_{\cmode} e :
\tau$ for some $\tau$) and there exists
$\varphi_i$, $\heap_i$, $e_i$, and $m_i$ for $i\in [1,k]$, such that
$(\varphi,\heap,e) \longrightarrow_{m_1} (\varphi_1,\heap_1,e_1)\longrightarrow_{m_2} ...\longrightarrow_{m_k} (\varphi_k,\heap_k,r)$, then $r$ can never be \emph{stuck}.
\end{thm}

%{\em Proof:} By induction on the number of steps of the \checkedc
%evaluation ($\longrightarrow_m^*$), using progress and preservation to
%maintain the invariance of the assumptions.


\ignore{
\subsection{Semantics}\label{sec:semantics}

% The semantics
% gives an independent account of spatial safety in \lang by
% checking pointer bounds based on the annotations carried on types at
% run-time.  While this account makes clear that bounds checking occurs
% as expected, it suggests an implementation that uses fat pointers to
% carry bounds.  We resolve this tension in the subsequent section on
% compilation and show that an implementation faithful to the semantics
% can be obtained without fat pointers.  
% \review{repeat that the stack is immutable at this point?}
% \liyi{Is it? Is the stack immutable? What does the immutable mean? 
%   In a stack, the variable values can be changed? Right?
%   The pointer address itself cannot be changed once it is created, but the stack variable content can be updated?  }
% \mwh{It certainly seems to be immutable: Your create stack frames
%   using let binding, and the let-bound variables will always be bound
%   to the same things. I.e., stack cells are immutable.}

% \review{this raises a fair amount of questions regarding the treatment of the
%   NULL pointer at this stage of the paper... is it modeled as 0, as returned by
%   `malloc`? are dynamic checks inserted by CheckedC to guarantee that no NULL
%   pointer is dereferenced?}
% \mwh{Yes, it is modeled as 0, and the semantics checks for
%   dereferences of 0. }

Here, we discuss the \lang operational semantics (\Cref{fig:semantics}); 
mainly focusing on the new changes on top of \checkedc in \cite{li22checkedc} with function pointers, modes, and function calls.
The other type and semantic rules about (NT)-arrays are given in \cite{li22checkedc} and \Cref{sec:literal-pointer-typing}.

%The typing judgment has the form $\Gamma;\Theta\vdash_m e : \tau$,
%which states that in a type environment $\Gamma$ (mapping variables to
%their types) and a predicate environment $\Theta$ (mapping integer-typed
%variables to Boolean predicates), expression $e$ will have type $\tau$ if evaluated
%in context mode $m$. Key rules for this judgment are given in
%Fig.~\ref{fig:type-system-1},

The operational semantics for \lang is defined as a small-step
transition relation with the judgment $ (\varphi,\heap,e)
\longrightarrow_m (\varphi',\heap',r)$.
 Here, $\varphi$ is a
\emph{stack} mapping from variables to values $\evalue{n}{\tau}$ and
$\heap$ is a \emph{heap} that is partitioned into two parts ($\cmode$ and $\umode$ heaps), each of which
maps addresses (integer literals) to values $\evalue{n}{\tau}$.

\myparagraph{Pointers, Contexts, and Modes}
A $\cmode$ pointer is mapped to a heap location in the $\cmode$ heap, 
while a $\tmode$ and $\umode$ pointer represents a $\umode$ heap location.
We wrote $\heap(m,n)$ to retrieve the $n$-location heap value in the $m$ heap,
and $\heapup{m}{n}{\evalue{n'}{\tau}}$ 
to update location $n$ with the value $\evalue{n'}{\tau}$ in the $m$ heap.
It is worth noting that \systemname is not a fat-pointer system;
thus, in every heap update, the value type annotation remains the same through program executions.
% 
\mz{Does a ``non-fat pointer'' system cause the type preservation of heap
  values?
  % 
  The story seems to be that you design the system in a certain way s.t. heap
  value types are unchanged, which allows you to erase the type.
  % 
  And, from the type erasure property, we know that we don't need fat pointers.
}
% 
Additionally, for both stack and heap, 
we ensure $\fv(\tau)=\emptyset$ for all the value type annotations $\tau$.

While heap bindings can change, stack bindings are immutable---once
variable $x$ is bound to $\evalue{n}{\tau}$ in $\varphi$, that binding will not
be updated. 
We can model mutable stack variables as pointers into the
mutable heap.
As mentioned, value $\evalue{0}{\tau}$
represents a $\enull$ pointer when $\tau$ is a pointer type.
Correspondingly, $\heap(m,0)$ should always be undefined.
% 
The relation steps to a \emph{result} $r$, which is \mzs{either} \mzr{one of} an
expression, a $\enull$ or $\ebounds$ failure, represent \mzr{an expression right
  after the reduction}, a null-pointer dereference or out-of-bounds access,
respectively.
% 
Such failures are a \emph{good} outcome; stuck states
(non-value expressions that cannot transition to a result $r$)
characterizing undefined behavior.
%
% 
The context mode $m$ (in $\longrightarrow_{m}$) indicates whether the
stepped redex within $e$ was in a $\cmode$ or $\umode$ region.

The rules for the main operational semantics
judgment \emph{evaluation} are given at the bottom of
Fig.~\ref{fig:c-context}.
The first rule takes an expression $e$, decomposes
it into an \emph{evaluation context} $E$ and a sub-expression $e'$
(such that replacing the hole $\Box$ in $E$ with $e'$ would yield
$e$), and then evaluates $e'$ according to the \emph{computation}
  relation $(\varphi,\heap,e') \longrightarrow (\varphi,\heap,e'')$,
whose rules are given in Fig.~\ref{fig:semantics}, discussed
shortly.
The $\mode$ function  at the bottom of Fig.~\ref{fig:c-context}
determines the context mode that the expression $e'$ locates based on the context $E$.
In \Cref{lst:humantaint}, the function call expression \code{read_msg} has $\umode$ mode since it is inside a tainted function.
The second rule describes the exception handling 
for possible crashing behaviors in unchecked regions.
A $\umode$ mode operation can non-deterministically crash
and the \systemname sandbox mechanism recovers
the program to a safe point ($\evalue{0}{\tau}$)
and continues with the existing program state.
Evaluation contexts $E$ define a standard left-to-right evaluation order. (We explain the
$\ret{x}{\mu}{e}$ syntax shortly.)
%There are other rules for describing the halts of evaluation to $\enull$ and $\ebounds$ states in \Cref{app:main}.

Fig.~\ref{fig:semantics} shows selected rules for the computation relation.
The rules for pointer related operations---\textsc{S-DefC},
\textsc{S-DefT}, \textsc{S-DefNull}, and \textsc{S-Cast}.
The type rule for deference operations is given as rule \rulelab{T-Def} in \Cref{fig:type-system-1}.
The first three define the semantics of deference and assignment operations.
Rule \textsc{S-DefNull} transitions attempted null-pointer
dereferences to $\enull$, whereas \textsc{S-DefC} dereferences a $\cmode$-mode
non-null (single) pointer.
When $\enull$ is returned by the
computation relation, the evaluation relation halts the entire
evaluation with $\enull$ (using a rule not shown in Fig.~\ref{fig:c-context}); it
does likewise when $\ebounds$ is returned (see \Cref{sec:rem-semantics}).
%\textsc{S-AssignArrC} assigns to an array as long as 0 (the point of
%dereference) is within the bounds designated by the pointer's annotation
%and strictly less than the upper bound. 
\textsc{S-DefT} is similar to \textsc{S-DefC} for tainted pointers.
Any dynamic heap access of a tainted pointer requires a \textit{verification}.
Performing such a verification equates to performing a literal type check for a pointer constant in \Cref{fig:const-type}.
We explain this shortly below for \emph{constant validity checks}.
For now, the verification step, e.g. $\emptyset;\heap ; \emptyset \vdash_{\umode}\evalue{n_a}{\tau}$ in \textsc{S-DefC},
refers to that the value $n_a$ is well-defined in $\heap(m,n_a)$ and has type $\tau$, if $\tau$ is a pointer.
Static casts of a literal $n\!:\!\tau'$ to a type $\tau$ are handled
by \textsc{S-Cast}. In a type-correct program, such casts are
confirmed safe by the type system no matter
if the target is a $\tmode$ or $\cmode$ pointer. To evaluate a cast, the rule
updates the type annotation on $n$. Before doing so, it must
``evaluate'' any variables that occur in $\tau$ according to their
bindings in $\varphi$. For example, if $\tau$ was
$\tarrayptr{0}{x+3}{\tint}{\cmode}$, then $\varphi(\tau)$ would
produce $\tarrayptr{0}{5}{\tint}{\cmode}$ if $\varphi(x) = 2$.
%The full formalism, including \kw{struct}
%and null-terminated bound widening pointer operations, is given in \Cref{app:main}.

%\footnote{This approach is that of the PLT Redex model of \lang; the Coq
%development uses a slightly simpler syntax to achieve the same
%effect.}
% \review{the special case raises questions, e.g. why is this syntax-driven and
%   not type-driven? }
% \liyi{This describes the semantic transition rules. We are using context evaluation framework to define the transition rules as the $E$ definition in Fig.3. like $\frac{x \Rightarrow y}{x+z \Rightarrow y + z}$, I don't know how type-driven can help us define translation rules.  }
% \mwh{Don't follow the above. I don't see this ``context transition
%   rule'' anywhere, and I'm not sure how it would fire, if we had it.}
% \liyi{The comment seems to confuse the meaning of the text about the if-then-else rules. Making the rule specific will help. }



\begin{DIFnomarkup}
 \begin{figure}[t]
 {\small

 \begin{mathpar}
   \inferrule
       {}
       {\Theta;\heap;\sigma \vdash_m n : \tint}

   \inferrule
       {}
       {\Theta;\heap;\sigma \vdash_m 0 : \tptr{\omega}{\kappa}}

   \inferrule
       {(m = \cmode \Rightarrow \kappa \neq \cmode) \\\\ (m=\umode \Rightarrow \kappa = \umode)}
       {\Theta;\heap;\sigma \vdash_{\cmode} n : \tptr{\omega}{\tmode}}
  
   \inferrule
       {(\evalue{n}{\tptr{\omega}{\kappa}})\in \sigma}
       {\Theta;\heap;\sigma \vdash_m n : \tptr{\omega}{\kappa}}


   \inferrule
       {\tptr{\omega'}{\kappa'} \sqsubseteq_{\Theta} \tptr{\omega}{\kappa} 
            \\ \Theta;\heap;\sigma \vdash_m n : \tptr{\omega'}{\kappa'}}
       {\Theta;\heap;\sigma \vdash_m n : \tptr{\omega}{\kappa}}

   \inferrule
       { \kappa \le m 
     \\\Xi(m,n)=\tau\;(\evalue{\overline{x'}}{\overline{\tau}})\;(\kappa,e)
       \\  \overline{x} = \{x|(x:\tint) \in (\overline{x'}:\overline{\tau}) \}}
       {\Theta;\heap;\sigma \vdash_m n : \tptr{(\tfun{\overline{x}}{\overline{\tau}}{\tau})}{\kappa}}
  
   \inferrule
       {\neg\funptr(\omega)\\ \kappa \le m\\
        \forall i \in [0,\size(\omega)) \;.\;
            \Theta;\heap;(\sigma \cup \{(n:\tptr{\omega}{\kappa})) \}\vdash_m \heap(m,n+i)}
       {\Theta;\heap;\sigma \vdash_m n : \tptr{\omega}{\kappa}}
 \end{mathpar}
 }
{\footnotesize
\[
\begin{array}{l} 
\funptr(\tfun{\overline{x}}{\overline{\tau}}{\tau}) = \texttt{true}
\qquad
\funptr(\omega) = \texttt{false}\;\;{[\emph{owise}]}
\end{array}
\]
}
 \caption{Verification/Type Rules for Constants}
 \label{fig:const-type}
 \end{figure}
\end{DIFnomarkup}


% SKIPPING THIS
\iffalse

\myparagraph{Type Equality and Subtyping and Casting}
%
In \lang, the type equability ($=_{\Theta}$) and subtype ($\sqsubseteq$) relations are given in \Cref{fig:checkc-subtype}.
We provide some example descriptions here.
Type equality $\tau=_{\Theta}\tau'$
is a type construct equivalent relation defined by the bound equality ($=_{\Theta}$) in (NT-)array pointer types
and the alpha equivalence of two function types;
i.e., two (NT-)array pointer types $\tallarrayb{\bvar}{\tau} $ and $ \tallarrayb{\bvar'}{\tau'}$ are equivalent, if 
$\bvar =_{\Theta} \bvar'$ and $\tau=_{\Theta}\tau'$; two function types 
$\tfun{\overline{x}}{\overline{\tau}}{\tau} $ and $ \tfun{\overline{y}}{\overline{\tau'}}{\tau'}$
are equivalent, if we can find a same length (as $\overline{x}$ and $\overline{y}$) variable list $\overline{z}$ that is substituted for $\overline{x}$ and $\overline{y}$ in $\overline{\tau} \to {\tau}$ and $\overline{\tau'} \to {\tau'}$, resp.,
and the substitution results are equal.

The \textsc{T-CastPtr} rule in \Cref{fig:type-system-1}
permits casting from an expression of type $\tau'$ to a checked pointer when
$\tau' \sqsubseteq \tptr{\tau}{\cmode}$. This subtyping relation
$\sqsubseteq$ is built on the type equality ($\tau =_{\Theta} \tau'\Rightarrow\tau \sqsubseteq_{\Theta} \tau'$). 
The rule  ($0\le b_l \wedge b_h \le 1 \Rightarrow \tptr{\tau}{m}\sqsubseteq
\tarrayptr{b_l}{b_h}{\tau}{m}$) permits treating a singleton
pointer as an array pointer with $b_h\le 1$ and $0 \le b_l$.
Two function pointer types are subtyped ($\tptr{\tfun{\overline{x}}{\overline{\tau}}{\tau}}{\kappa} \sqsubseteq_{\Theta} \tptr{\tfun{\overline{x}}{\overline{\tau'}}{\tau'}}{\kappa}$), 
if the output type are subtyped ($\tau\sqsubseteq_{\Theta}\tau'$) and the argument types are reversely subtyped ($\overline{\tau'}\sqsubseteq_{\Theta}\overline{\tau}$).
%There is another casting rule in \Cref{app:main} stating that
% users are free to cast types in unchecked code regions, since unchecked regions can contain C code.

\begin{DIFnomarkup}
 \begin{figure}[t]
 {\small

 \begin{mathpar}
   \inferrule
       {}
       {\Theta;\heap;\sigma \vdash_m n : \tint}

   \inferrule
       {}
       {\Theta;\heap;\sigma \vdash_m 0 : \tptr{\omega}{\kappa}}

   \inferrule
       {(m = \cmode \Rightarrow \kappa \neq \cmode) \\\\ (m=\umode \Rightarrow \kappa = \umode)}
       {\Theta;\heap;\sigma \vdash_{\cmode} n : \tptr{\omega}{\tmode}}
  
   \inferrule
       {(\evalue{n}{\tptr{\omega}{\kappa}})\in \sigma}
       {\Theta;\heap;\sigma \vdash_m n : \tptr{\omega}{\kappa}}


   \inferrule
       {\tptr{\omega'}{\kappa'} \sqsubseteq_{\Theta} \tptr{\omega}{\kappa} 
            \\ \Theta;\heap;\sigma \vdash_m n : \tptr{\omega'}{\kappa'}}
       {\Theta;\heap;\sigma \vdash_m n : \tptr{\omega}{\kappa}}

   \inferrule
       { \kappa \le m 
     \\\Xi(m,n)=\tau\;(\evalue{\overline{x'}}{\overline{\tau}})\;(\kappa,e)
       \\  \overline{x} = \{x|(x:\tint) \in (\overline{x'}:\overline{\tau}) \}}
       {\Theta;\heap;\sigma \vdash_m n : \tptr{(\tfun{\overline{x}}{\overline{\tau}}{\tau})}{\kappa}}
  
   \inferrule
       {\neg\funptr(\omega)\\ \kappa \le m\\
        \forall i \in [0,\size(\omega)) \;.\;
            \Theta;\heap;(\sigma \cup \{(n:\tptr{\omega}{\kappa})) \}\vdash_m \heap(m,n+i)}
       {\Theta;\heap;\sigma \vdash_m n : \tptr{\omega}{\kappa}}
 \end{mathpar}
 }
{\footnotesize
\[
\begin{array}{l} 
\funptr(\tfun{\overline{x}}{\overline{\tau}}{\tau}) = \texttt{true}
\qquad
\funptr(\omega) = \texttt{false}\;\;{[\emph{owise}]}
\end{array}
\]
}
 \caption{Verification/Type Rules for Constants}
 \label{fig:const-type}
 \end{figure}
\end{DIFnomarkup}

\myparagraph{Constant Validity}
Rules \textsc{T-ConstU} and \textsc{T-ConstC} in \Cref{fig:type-system-1}
describe type assumptions for constants appearing in a program.
$\neg \cmode(\tau)$ judges that a constant pointer 
in an unchecked region cannot be of a checked type.
The restriction ensures that programmers 
cannot guess a checked pointer address and utilize it in an unchecked region in \systemname.
In rule \textsc{T-ConstC}, we requires a static 
verification procedure for validating a constant pointer in \Cref{fig:const-type}. 

The verification process $\Theta;\heap;\sigma \vdash_m n : \tau$
validates the constant $\evalue{n}{\tau}$, 
where $\heap(m)$ is the initial heap that the constant resides on and
$\sigma$ is a set of constant assumed to be checked.
A global function store $\Xi(m)$ is also required to check the validity of a function pointer.
A valid function pointer should appear in the right store region ($\cmode$ or $\umode$)
and the address stores a function with the right type.
The last rule in \Cref{fig:const-type} describes the validity check for a non-function pointer, 
where every element in the pointer range ($[0,\size(\omega))$) should be well
typed.
A checked pointer checks validity in type step as rule \textsc{T-ConstC},
while a tainted/unchecked pointer does not check for such during the type checking.
Tainted pointers are validated through the validity check in dynamic execution as we mentioned in rule \rulelab{S-DefT}.

% \review{Fig4: it was hard to tell which cases were stuck states, or could reduce
%   owing to a rule that was not shown}
% \mwh{Stuck states are those where the expression is a non-value and
%   not $\enull$ or $\ebounds$. Updated III-B and III-D.}
% \review{Fig4: can the rules be presented in the same order they were introduced in
%   the paper?}
% \liyi{ Reordered }

% \review{Fig4: S-FUN: $\vec\tau_a$ seems unused; why?}
% \liyi{the list of $\tau_a$ is a list of input argument types. These
%   are used during type checking, but not during evaluation (as is
%   typical).} 

% LEO: This has an overfull line...?



% Below, we introduce low-level transition semantics for some case operations. The design of the low-level individual operation semantics is carefully engineered to perform match our compiler's behavior, such as correctly characterizing the bound widening behaviors for NT-array pointers, even though it is written in terms of fat-pointer formalization.
% \review{- on page 6, section IIIC, paragraph "Pointer Access" mentions that checked pointers cannot be dereferenced in unchecked blocks - this looks funny, shouldn't it be the other way around? The Coq code contains the hypothesis m'=Unchecked -> m Unchecked in various rules of definition well-typed (BoundCheckedC, line 669; rule TyDeref in the code seems closest to the figure's T-DefArr and T-Def in the appendix, although it's a bit concerning that there's no 1-to-1 correspondence of the rules in the code and the paper).
% }
% \liyi{Yeah. It is another typo. It should be the other way around.  }

\myparagraph{Unchecked and Checked Blocks}
%
During the type checking,
Both $\echecked{\overline{x}}{e}$ and $\eunchecked{\overline{x}}{e}$
check all free variables in $e$ are within $\overline{x}$;
the types for $\overline{x}$ and the final return type $\tau$ of $e$ have no checked pointers.
Otherwise, it violates the non-exposure safety.
For example, \code{read_msg} in the \code{handle_request} function is tainted in \Cref{lst:humantaint},
if any argument for \code{handle_request} is a checked pointer, 
it means that we are exposing a checked pointer address to unsafe regions.
% 
\mzu{A $\echeckedtext$ or $\euncheckedtext$ block represents 
  the context switching from a checked to an checked region, or vice versa.
}
% 
\mz{An action + checked/unchecked block represents the switching.}
% 
We need to make sure no checked pointers are \mzs{information} exposed to unsafe code regions.
as rules \textsc{S-Unchecked} and \textsc{S-Checked} in \Cref{fig:semantics}.
% 
\mz{This paragraph is not that cohesive.}
% 

\myparagraph{Let Bindings and Dependent Function Pointers}
%
Rules \textsc{T-Let} and \textsc{T-LetInt} in in \Cref{fig:type-system-1} type a $\elettext$ expression, which also admits
type dependency. 
In particular, the result of evaluating a $\elettext$ expression
may have a type that refers to one of its bound variables (e.g., if
the result is a checked pointer with a variable-defined bound). 
If so, we must substitute away this variable once it goes out of scope (\textsc{T-LetInt}). 
Note that we restrict the expression $e_1$ to syntactically match the
structure of a Bounds expression $b$ (see Fig.~\ref{fig:checkc-syn}).
Rule \textsc{T-RetInt} types a $\erettext$ expression when $x$ is of type $\tint$.
$\erettext$ does not appear in source programs but is introduced by the semantics when
evaluating a let binding (rule \textsc{S-Let} in
Fig.~\ref{fig:semantics}). 
%\liyi{why? }
% After the evaluation of a let binding a variable $x$ concludes,
%we need to restore any prior binding of $x$, which is either
%$\bot$ (meaning that there is no $x$ originally) or some value
%$\evalue{n}{\tau}$.

Rule \textsc{T-Fun} in \Cref{fig:type-system-1} is the dependent function call rule. 
Given a function pointer type ($\tptr{\tfun{\overline{x}}{\overline{\tau}}{\tau}}{\kappa}$)
from a type-check for $e$ and the types $\overline{\tau'}$ from the argument type checks for $\overline{e}$,
we confirm that each of $\overline{\tau'}$ is
a subtype of the corresponding one in $\overline{\tau}[\overline{e'} / \overline{x}]$,
which replaces possible integer bound variables $\overline{x}$ with bound expressions $\overline{e'}$.
The final result type is the defined target type $\tau$ appearing in the function pointer type
also with such replacement, written as $\tau[\overline{e'} / \overline{x}]$.
Consider the \code{process_req2} function in
Fig.~\ref{lst:final}; its parameter type for \code{msg} 
depends on \code{m_1}. The \textsc{T-Fun} rule will substitute 
\code{n} with the argument at a call-site.
The semantics manages variable scopes using the special $\erettext$
form. \textsc{S-Let} evaluates to a configuration whose expression is
$\ret{x}{\evalue{n}{\tau}}{e})$. We keep $\varphi$ unchanged
and remember $x$ and its new value $\evalue{n}{\tau}$
in $e$'s scope that is defined by the $\erettext$ operation.
Every time when evaluation proceeds on $e$ (rule \textsc{S-RetCon}),
we install the stack value $\evalue{n}{\tau}$ for $x$ in $\varphi$ for the current scope.
After one-step evaluation is completed, 
we store $x$'s change in the result $\erettext$ operation $\ret{x}{\varphi'(x)}{e'})$,
and restore $x$'s outer score value $\varphi(x)$ in $\varphi'$. 
This procedure continues until $e'$ becomes a literal
$n\!:\!\tau$, in which case \textsc{S-RetEnd} removes the $\kw{ret}$ frame and returns
the literal. 

\textsc{S-FunC} and \textsc{S-FunT} are
for $\cmode$ and $\tmode$ mode function pointers, respectively. 
A call to a function pointer $n$ retrieves
 the function definition in $n$'s location in the global function store $\Xi$,
which maps function pointers to
function data $\tau\;(\evalue{\overline{x}}{\overline{\tau}})\;(\kappa,e)$, where
$\tau$ is the return type, $(\evalue{\overline{x}}{\overline{\tau}})$
is the parameter list of variables and their types, 
$\kappa$ determines the mode of the function, and $e$ is the
function body. 
Similar to \heap, the global function store $\Xi$ is also partitioned into
two parts ($\cmode$ and $\umode$ stores), each of which
maps addresses (integer literals) to the function data described above.

\systemname{} has dependent functions, whose semantic explaination is given in \Cref{appx:add-type-sem}.
Note that the \textsc{S-FunC} and \textsc{S-FunT} rules replace the
  annotations $\overline{\tau_a}$ with
  $\overline{\tau}$ (after instantiation) from the function's
  signature. Using $\overline{\tau_a}$ when executing the body of
the function has no impact on the soundness of \lang, but will violate
Theorem~\ref{simulation-thm}, which we introduce in Sec.~\ref{sec:compilation}.
Rule \textsc{S-FunT} defines the tainted version of function call semantics.
In such case, the verification process 
$\emptyset;\heap ; \emptyset \vdash_{\umode}\evalue{n}{\tptr{\tau}{\tmode}}$
makes sure that the function in the global store is well-defined and has the right type.

 %  For this rule and
% \textsc{S-StrWiden}, this widening persists in the current stack
% frame. When $x$ goes out of scope, .

% \textsc{S-IfNTF} does not widen when seeing null; rule
% \textsc{S-IfNTNot} sees a non-null character, but the pointer is not
% at its upper bound, so the bounds cannot be widened. 

% \ignore{
% Fig.~\ref{fig:semantics} provides the low-level semantic rules for operations involving NT-array pointers, mainly, the $\estrlentext$ and $\eiftext$ operations. The semantics has concurred the ambiguity in the \checkedc specification, e.g., we define the exact behavior of the $\estrlentext$ operation to return the length between the current pointer position and the first null-character.
% We also utilize new technique in our compiler so that the scope of the bound widening behavior in our formalization is a little longer. More details are in Sec.~\ref{sec:compilation}.

% The first rule defines the evaluation behavior of a $\estrlentext$ operation. Given a pointer $x$ with its type $\tntarrayptr{0}{n_h}{\tau}{m}$, the application of such operation takes the address of the pointer $x$, and search incrementally the heap positions next to the address $x$ until we find a $0$ value (representing a null character). We return the value $n_a$ as the length, and update the bound information in the stack for $x$. In the compilation, we use a ghost variable to record such bound changes without using fat-pointer implementations.

% The last three rules in Fig.~\ref{fig:semantics} describe the semantic behaviors of an $\eiftext$ branching operation when the Boolean guard is a dereference of an NT-array pointer. The first one states that if the type upper bound of the pointer $x$ is $0$, and the pointer data value $n_a$ is not $0$, we can conclude that the upper bound is not the last position of the NT-array pointer, so we can then update $1$ in the upper bound while jump to the $\etrue$ branch. The second rule describes that we do not extend the upper bound if the upper-bound of the type of $x$ is not zero because we know that we are not in the NT-array's last position. The third rule describes the behavior of jumping to the $\efalse$ branch when the pointer content is $0$. In this case, we also do not need to increase the upper-bound of the type of $x$.}

\fi
\subsection{Compilation}\label{sec:compilation}

As we have shown in \Cref{fig:overview}, the \systemname compiler utilizes the sandbox mechanism \cite{rul2009towards} and the \checkedc compiler \cite{li22checkedc} to compile programs. Here, we introduce how \systemname compiles a program into these two components.

\begin{figure}[t!]
{\small
\hspace*{-0.5em}
\begin{tabular}{|c|c|c|c|}
\hline
& \cmode & \tmode & \umode \\
\hline
& \textsc{CBox} / \textsc{Core} & \textsc{CBox} / \textsc{Core} & \textsc{CBox} / \textsc{Core} \\
\hline
\cmode & $\estar{x}$ / $\getstar{\cmode}{x}$ 
 & $\texttt{sand\_get}(x)$ / $\getstar{\umode}{x}$ &  $\times$ \\
\hline
\umode & $\times$
 & $\estar{x}$ / $\getstar{\umode}{x}$ &  $\estar{x}$ / $\getstar{\umode}{x}$ \\
\hline
\end{tabular}

}
\caption{Compiled Targets for Dereference}
\label{fig:flagtable}
\end{figure}

In \systemname, context and pointer modes determine the particular heap/function store that a pointer points to,
i.e., $\cmode$ pointers point to checked regions, while $\tmode$ and $\umode$ pointers point to unchecked regions.
Unchecked regions are associated with a sandbox mechanism that permits exception handling of potential memory failures.
In the compiled LLVM code, pointer access operations have different syntaxes when the modes are different. 
\Cref{fig:flagtable} lists the different compiled syntaxes of a deference operation ($\estar{x}$) for the compiler implementation (\textsc{CBox}, stands for \systemname) and formalism (\textsc{Core}, stands for \lang). The columns represent different pointer modes and the rows represent context modes.
For example, when we have a $\tmode$-mode pointer in a $\cmode$-mode region, we compile a deference operation to the sandbox pointer access function ($\texttt{sand\_get}(x)$) accessing the data in the \systemname implementation. In \lang, we create a new deference data-structure on top of the existing $\estar{x}$ operation (in LLVM): $\getstar{m}{x}$. If the mode is $\cmode$, it accesses the checked heap/function store; otherwise, it accesses the unchecked one.

We now show how \lang deals with pointer modes, mode switching and function pointer compilations, 
with no loss of expressiveness
as the \checkedc contains the erase of annotations in \cite{li22checkedc} and \Cref{appx:comp1}.
For the compiler formalism, 
we present a compilation algorithm that converts from
\lang to \elang, an untyped language without metadata
annotations, which represents an intermediate layer we build on LLVM for simplifying compilation. 
In \elang, the syntax for deference, assignment, malloc, function calls are: $\getstar{m}{e}$, $\elassign{m}{e}{e}$, 
$\emalloc{m}{\omega}$, and $\elcall{m}{e}{\overline{e}}$.
The algorithm sheds
  light on how compilation can be implemented in the real Checked C
  compiler, while eschewing many vital details (\elang has many 
  differences with LLVM IR).


%This section shows how \systemname deals with 
%annotations can be safely erased: using static information a compiler
%can insert code to manage and check bounds metadata, with no loss of
%expressiveness. We present a compilation algorithm that converts from
%\lang to \elang, an untyped language without metadata
%annotations. The syntax and semantics \elang
  %closely mirrors that of \lang; it differs only in that literals lack
  %type annotations and its operational rules perform no
  %bounds and null checks, which are instead inserted during
  %compilation. Our compilation algorithm is evidence that \lang's
  %semantics, despite its apparent use of fat pointers, faithfully
  %represents Checked C's intended behavior. The algorithm also sheds
  %light on how compilation can be implemented in the real Checked C
  %compiler, while eschewing many important details (\elang has many 
  %differences with LLVM IR).

Compilation is defined by extending \lang's
typing judgment as follows:
\[\Gamma;\Theta;\rho \vdash_m e \gg \dot e:\tau\]
There is now a \elang output $\dot e$ and an input $\rho$, which maps
each (NT-)array pointer variable to its mode and
each variable \code{p} to a pair of \emph{shadow
  variables} that keep \code{p}'s up-to-date upper and lower bounds. 
These may differ from the bounds in \code{p}'s type due to bounds
widening.\footnote{Since lower bounds are never widened, the
  lower-bound shadow variable is unnecessary; we include it for uniformity.} 

% When $\Gamma$,$\Theta$ and $\rho$ are all empty, we write $e \gg \dot e$ rather than the
% complete judgment, implicitly assuming that $e$ is a well-typed and closed
% term.

We formalize rules for this judgment in PLT Redex~\cite{pltredex},
following and extending our Coq development for \lang. To give
confidence that compilation is correct, we use Redex's property-based
random testing support to show that compiled-to $\dot e $ simulates
$e$, for all $e$.

\myparagraph{Checked and Unchecked Blocks}
%
In the \systemname implementation,
$\euncheckedtext$ and $\echeckedtext$ blocks 
are compiled as context switching functions provided by the sandbox mechanism.
We compile $\eunchecked{\overline{x}}{e}$ to 
$\texttt{sandbox\_call}(\overline{x},e)$, where we call the sandbox 
to execute expression $e$ with the arguments $\overline{x}$.
$\echecked{\overline{x}}{e}$ is compiled to 
$\texttt{callback}(\overline{x},e)$, where we perform 
a \texttt{callback} to a checked block code $e$ inside a sandbox.
In \systemname, we adopt an aggressive execution scheme that
directly learns pointer addresses from compiled assembly to make the $\texttt{callback}$ happen.
In the formalism, we rely on the type system to 
guarantee the context switching without creating the extra function calls for simplicity.

%Fig.~\ref{fig:compilationexample} shows how an invocation of
%\code{strlen} on a null-terminated string is compiled into C
%code. Each dereference of a checked pointer requires a null check
%(See \textsc{S-DefNull} in Fig.~\ref{fig:semantics}), which the
%compiler makes explicit: Line~$3$ of the generated code has the null
%check on pointer \code{p} due to the \code{strlen},
%  and a similar check happens
%  at line~$8$ due to the pointer arithmetic on \code{p}.
%Dereferences also require bounds checks: line~$2$ checks \code{p} is
%in bounds before computing \code{strlen(p)}, while line~$10$ does
%likewise before computing \code{*(p+1)}.

\myparagraph{Function Pointers and Calls}
%
Function pointers are managed similarly to normal pointers,
but we insert checks to check if the pointer address is not null in 
the function store instead of heap, and whether or not the type is correctly represented, 
for both $\cmode$ and $\tmode$ mode pointers 
\footnote{$\cmode$-mode pointers are checked once in the beginning and $\tmode$-mode pointers are checked every time when use}.
For example, in compiling the \code{read_msg} function in \Cref{lst:humantaint},
we place a check \code{verify_fun(read_msg, not_null(c, p_lo, p_hi) && type_match)},
The compilation of function calls (compiling to $\elcall{m}{e}{\overline{e}}$) 
is similar to the manipulation of pointer access operations in \Cref{fig:flagtable}.
The other compilation rules are given in \Cref{appx:add-type-sem}.
}

%\section{Compilation}\label{sec:compilation}

% \review{
% The first reviewer said:
% \begin{itemize}
% \item I had a really hard time understanding precisely the "formalized" compilation
%   from CoreChkC to CoreC. Specifically: is CoreC intended to model LLVM-IR, or
%   is it a subset of C? Is CheckedC compiling to C or is it a new frontend like
%   `clang`? See remark about undefined behavior (UB) below which got me even more
%   confused. \mwh{CoreC is meant to model CoreChkC but with
%   annotations, and checks removed; it is not meant to model C or LLVM-IR.}
% \item Is the compilation scheme from CoreChkC to CoreC is faithful to the actual
%   compilation scheme of the checkedc-clang compiler to ...? I feel like the
%   paper is missing an actual description of what happens in the compiler to
%   allow us to connect the dots and understand how the formalization illuminates
%   the implementation. \mwh{We did not model compilation on the real
%   implementation; our purpose was to show that annotations in CoreChkC
%   do not necessitate fat pointers in an implementation; that said,
%   our formalization does show how a real implementation can be
%   carried out}
% \item The paper doesn't seem upfront about what is *shown* (theorem in Coq) and what
%   is *tested* (via PLT-Redex), and thus remains a
%   conjecture. \mwh{Updated the intro and the individual sections}
% \item  Missing discussion of why Coq vs. PLT-Redex, effort involved, any plans to
%   formally prove compilation from CoreChkC to CoreC, any hopes of integrating
%   that in the official implementation, etc. \mwh{Added note at the
%   start of section 3.}
% \end{itemize} }

% \review{IV: ghost variables in other contexts (e.g. Why3, Dafny) are used for things
%   that do not exist at run-time, but this doesn't seem to be the case here.}
% \yiyun{Agreed: We changed the name to ``shadow variables'' to avoid confusion}
% \review{Do your bug report and github links break anonymity?}


% \review{From reviewer C:
% \begin{itemize}
% \item I'd like to have seen a bit more motivation for using PLT redex:
%   what aspects made the use of this tool preferable to formulating the
%   compilation in Coq and using Quickchick to do random testing of the
%   simulation result.
%   \mwh{Added some text to the end of III.A and IV.C}
% \yiyun{Some reasons I can think of:\begin{itemize}
%     \item Redex is highly optimized for specifying judgments that are algorithmic. By writing down a typing relation, we can immediately obtain a typechecker
%     that is executable. Same applies for the small-step evaluation relation. Translating the relations into functions in Coq is definitely doable but time-consuming, especially since compilation is embedded as part of the typing rules. It is also hard to see whether the function we define really corresponds to the relation unless we formally prove it. This issue is particularly relevant at the early stage of the development when the compilation rules were buggy and the simulation property was violated often as we added new generator cases.
%     In Redex, we don't have any formal guarantee either, but at least we can more easily see the correspondence because Redex is able to convert the relation into an executable version so we can specify the relation literally. This feature of Redex helped us speed up our development significantly when our compilation rules were constantly changing.
% \end{itemize} }
% \item there's a funny change of line spacing in column 2 of page 6, about 2/3 down.
% \end{itemize}
% }
  
The semantics of \lang uses annotations on pointer literals in order
to keep track of array bounds information, which is used in premises
of rules like \textsc{S-DefArray} and
  \textsc{S-AssignArr} to prevent spatial safety violations. However, in the real
implementation of \checkedc, which extends Clang/LLVM, these annotations
are not present---pointers are represented as a single
machine word with no extra metadata, and bounds
  checks are not handled by the machine, but inserted by the
  compiler.

This section shows how \lang
annotations can be safely erased: using static information a compiler
can insert code to manage and check bounds metadata, with no loss of
expressiveness. We present a compilation algorithm that converts from
\lang to \elang, an untyped language without metadata
annotations. The syntax and semantics \elang
  closely mirrors that of \lang; it differs only in that literals lack
  type annotations and its operational rules perform no
  bounds and null checks, which are instead inserted during
  compilation. Our compilation algorithm is evidence that \lang's
  semantics, despite its apparent use of fat pointers, faithfully
  represents Checked C's intended behavior. The algorithm also sheds
  light on how compilation can be implemented in the real Checked C
  compiler, while eschewing many important details (\elang has many 
  differences with LLVM IR).

Compilation is defined by extending \lang's
typing judgment thusly:
\[\Gamma;\Theta;\rho \vdash_m e \gg \dot e:\tau\]
There is now a \elang output $\dot e$ and an input $\rho$, which maps
each \code{nt_array_ptr} variable \code{p} to a pair of \emph{shadow
  variables} that keep \code{p}'s up-to-date upper and lower bounds; these
may differ from the bounds in \code{p}'s type due to bounds
widening.\footnote{Since lower bounds are never widened, the
  lower-bound shadow variable is unnecessary; we include it for uniformity.}
% When $\Gamma$,$\Theta$ and $\rho$ are all empty, we write $e \gg \dot e$ rather than the
% complete judgment, implicitly assuming that $e$ is a well-typed and closed
% term.

We formalize rules for this judgment in PLT Redex~\cite{pltredex},
following and extending our Coq development for \lang. To give
confidence that compilation is correct, we use Redex's property-based
random testing support to show that compiled-to $\dot e $ simulates
$e$, for all $e$.

% We developed a \checkedc compiler to compile a \checkedc program to a C program.
% \mwh{We formalized compilation from CoreChkC to a version of CoreChkC
%   but with the metadata removed, right? This is not a Checked C
%   compiler. You go on to see stuff about CompCert, CLight,
%   etc. This is confusing. We should be talking about how this relates
%   to what was just presented. Pick definitive names for things. }
% Given a \checkedc program $e$, we build a compilation process ($\gg$), such that $e \gg \dot e$, where $\dot e$ is the corresponding C program for $e$ in A-normal form (ANF). 
% The compilation process ($\gg$) relies on the type checking step $\Gamma;\Theta\vdash_m e:\tau$. 
% Especially, it relies on $\Gamma$ to provide the type information for variables in $e$. 
% We utilize CompCert/CLight syntax and semantics \cite{Leroy:2009:FVC:1666192.1666216,Blazy2009} as our translation target language.
%  Besides, we defined two data structures in the CompCert format for representing $\enull$ and $\ebounds$ states.
% We write $\xrightarrow{c}$ for the semantics of CLight. \mwh{Of our
%   target language, which I presume does not have all the features that
%   CLight has. Should mention up front that we did all of this in PLT Redex.}

\subsection{Approach}

Due to space constraints, we explain the rules for compilation by
example, using a C-like syntax; the complete rules are given in
\iftr
Appendix~\ref{appx:comp1}.
\else
the supplemental report~\cite{checkedc-tech-report}.
\fi
Each rule performs up to three tasks: (a) conversion of $e$ to
A-normal form; (b) insertion of dynamic checks; and (c) insertion of
bounds widening expressions.
%
A-normal form conversion is straightforward: compound expressions are
handled by storing results of subexpressions into temporary variables,
as in the following example.

{\vspace*{-0.5em}
{\small
\begin{center}
$
\begin{array}{l}
$\code{let y=(x+1)+(6+1)}$
\;
\begin{frame}

\tikz\draw[-Latex,line width=2pt,color=orange] (0,0) -- (1,0);

\end{frame}
\;
\begin{array}{l}
$\code{let a=x+1;}$\\
$\code{let b=6+1;}$\\
$\code{let y=a+b}$\\
\end{array}
\end{array}
$
\end{center}
}
}

This simplifies the management of effects from subexpressions. The
next two steps of compilation are more interesting.


\begin{figure}[t!]
  \begin{small}
\begin{lstlisting}[mathescape,xleftmargin=4 mm]
/* p : $\color{purple!40!black}\tntarrayptr{0}{0}{\tint}{\cmode}$  */
/* $\color{purple!40!black}\rho$(p) = p_lo,p_hi */
{
  let x = strlen(p);
  if (x > 1) putchar(*(p+1));
}
\end{lstlisting}
\begin{frame}

\tikz\draw[-Latex,line width=2pt,color=orange] (0,0) -- (1,0);

\end{frame}
\begin{lstlisting}[xleftmargin=4 mm]
{
  assert(p_lo <= 0 && 0 <= p_hi); // bounds check
  assert(p != 0); // null check
  let x = strlen(p);
  let p_hi_new = x;
  p_hi = max(p_hi, p_hi_new);
  if (x > 1) {
    assert(p != 0); // null check for p + 1
    let p_1 = p + 1;
    assert(p_lo <= 1 && // bounds check for p + 1
     1 <= p_hi);    
    putchar(*p_1);
  }
}
\end{lstlisting}
\end{small}
\caption{Compilation Example for Check Insertions
% \review{- Fig 8, line 1: did you mean `bounds` instead of `count`? I'm confused
% - Fig 8, line 8: why null-check `p` again?}
% \yiyun{Yes, fixed. The second null check is due to the pointer
%   arithmetic on p; our transformation does not attempt to optimize
%   away provably redundant checks.}
}
\label{fig:compilationexample}
\end{figure}


\begin{figure}[t!]
  \begin{small}
\begin{lstlisting}[mathescape,xleftmargin=4 mm]
int deref_array(n : int,
     p :  $\color{green!40!black}\tntarrayptr{0}{n}{\tint}{\cmode}$) {
  /* $\color{purple!40!black}\rho$(p) = p_lo,p_hi */
  if (* p)
    (* (p + 1))
    else 0
}
...
/* p0 : $\color{purple!40!black}\tntarrayptr{0}{5}{\tint}{\cmode}$ */
deref_array(5, p0);
    \end{lstlisting}
\begin{frame}

\tikz\draw[-Latex,line width=2pt,color=orange] (0,0) -- (1,0);

\end{frame}
\begin{lstlisting}[xleftmargin=4 mm]
deref_array(n, p) {
  let p_lo = 0;
  let p_hi = n;
  /* runtime checks */
  assert(p_lo <= 0 && 0 <= p_hi);
  assert(p != 0);
  let p_derefed = *p;
  if (p_derefed != 0) {
    /* widening */
    if (p_hi == 0) {
      p_hi = p_hi + 1;
    }
    /* null check before pointer arithmetic */
    assert(p != 0);
    let p0 = p + 1;
    assert(p_lo <= 1 && 1 <= p_hi);
    (* p0)
  }
  else {
    0
  }
}
...
deref_array(5, p0);
    \end{lstlisting}
\end{small}
    \caption{Compilation Example for Dependent Functions}
\label{fig:compilationexample1}
\end{figure}

% \review{Fig 9: if this is actual C code, then your null-check at line 6 will be
%   eliminated by the compiler. At line 3, you performed a pointer addition, which
%   is only defined when `p` is non-null. So, either `p` is non-null, and the
%   NULL-check can be eliminated; or, `p` is NULL, but line 3 was undefined
%   behavior, meaning the compiler is allowed to do anything, notably eliminate
%   the NULL-check. This is where I am super confused, and either:
%   - CoreC is not really the C language, and has different semantics...? but is
%     this well-defined in the context of LLVM?
%   - there is a problem that was not caught by the PLT-Redex-based testing.
% \yiyun{We have clarified at the start of IV that CoreC is an untyped
%   variant of CoreChkC, and does not aim to represent C per se, or LLVM
%   IR. We aimed to avoid confusion by rewriting the examples in a way that is more closely
%     related to the syntax presented in Fig 3. Pointer arithmetic between 0 and a non-zero
%     index is always valid because CoreC there is technically only
%     integer arithmetic.}}

During compilation, $\Gamma$ tracks the lower and upper bound
associated with every pointer variable according to its type. At each
declaration of a \code{nt_array_ptr} variable \code{p}, the
compiler allocates two \emph{shadow variables}, 
stored in $\rho(p)$; these are initialized to \code{p}'s declared bounds
and will be updated during bounds widening.\footnote{Shadow variables
  are not used for \code{array_ptr} types (the bounds expressions are)
  since they are not subject to bounds widening.} 
Fig.~\ref{fig:compilationexample} shows how an invocation of
\code{strlen} on a null-terminated string is compiled into C
code. Each dereference of a checked pointer requires a null check
(See \textsc{S-DefNull} in Fig.~\ref{fig:semantics}), which the
compiler makes explicit: Line~$3$ of the generated code has the null
check on pointer \code{p} due to the \code{strlen},
  and a similar check happens
  at line~$8$ due to the pointer arithmetic on \code{p}.
Dereferences also require bounds checks: line~$2$ checks \code{p} is
in bounds before computing \code{strlen(p)}, while line~$10$ does
likewise before computing \code{*(p+1)}.

For \code{strlen(p)} and conditionals \code{if(*p)}, the \lang
semantics allows the upper bound of \code{p} to be extended.
The compiler explicitly inserts statements to do so on \code{p}'s
shadow bound variables. For example,
Fig.~\ref{fig:compilationexample}~line~$6$ widens \code{p}'s upper
bound if \code{strlen}'s result is larger than the existing
bound. 
Lines 7--12 of the generated code in
Fig.~\ref{fig:compilationexample1}
show how bounds are 
widened when compiling expression \code{if(*p)}. If we find that the
current \code{p}'s relative upper bound is equal to $0$ (line 10),
and \code{p}'s content is not null (line 8), we then increase the
upper bound by $1$ (line 11).

Fig.~\ref{fig:compilationexample1} also shows a dependent function call.
Notice that the bounds for the array pointer \code{p} are not passed as
arguments. Instead, they are initialized according to \code{p}'s
type---see line~3 of the original \lang program at the top of the figure.
Line~$2$ of the generated code
sets the lower bound  to \code{0} and line~$3$ sets the
upper bound to \code{n}.

\subsection{Comparison with Checked C Specification}
\label{sec:disc}

\begin{figure}[t]
{\small
{\captionsetup[lstlisting]{margin = 8 mm}
  \begin{lstlisting}[xleftmargin=8 mm]
nt_array_ptr<char> safe_strcat_c 
   (nt_array_ptr<char> dst : count(n),
    nt_array_ptr<char> src : count(0), int n) {
  nt_array_ptr<char> tmp : count(n) = dst;
  int x = strlen(tmp);
  /* tmp now has x as its upper bound */
  /* dst still has n as its upper bound */
  int y = strlen(src);

  if (x+y < n) {
    for (int i = 0; i < y; ++i)
      *(dst+x+i) = *(src+i);
    *(dst+x+y) = '\0';
    return dst;
  }
  return null;
}
\end{lstlisting}
}
}
\caption{Safe \code{strcat} in Checked C that avoids a run-time error
  exhibited by \code{safe_strcat} (Fig.~\ref{fig:strcat-ex}) when
  compiled with the current Checked C compiler}
\label{fig:strcatc-ex}
\end{figure}

The use of shadow variables for bounds widening is a key novelty of our
compilation approach, and adds more precision to bounds checking at runtime
compared to the official specification and current implementation of
\checkedc~\cite[5.1.2, pg 85]{checkedc}.  For example, the
\code{safe_strcat} example of Fig.~\ref{fig:strcat-ex} compiles with the
current Clang \checkedc compiler but will fail with a runtime
error. The statement \code{int x = strlen(dst)} at line 4 changes the
statically determined upper bound of \code{dst} to \code{x}, which can be smaller than
\code{n}, the full capacity of \code{dst}. The attempt to recover
the full capacity of \code{dst} through a dynamic cast
at line 7 will always fail if the capacity \code{n} is checked
against the statically determined new upper bound \code{x}. % This problem can be worked around by
% effectively inlining the definition of \code{strlen} as 
% in \code{safe_strcat_c} in Fig.~\ref{fig:strcatc-ex} (lines
% 6-7).
This problem can be worked around by invoking \code{strlen} on a temporary
variable \code{tmp} instead of \code{dst}
as in \code{safe_strcat_c} in Fig.~\ref{fig:strcatc-ex} (lines 4-5).
Likewise, if we were to add line \code{putchar(*(p+1));} 
after line 6 in the original code at the top of
Fig.~\ref{fig:compilationexample}, the code will always fail: the Clang \checkedc
compiler (with the transliterated C code as its input) would 
check \code{p} against its \emph{original} bounds \code{(0,0)} since the
updated upper bound \code{x} is now out of
the scope. Shadow variables address these problems because
they retain widened bounds beyond the scope of variables that store
them (i.e., \code{x} in both examples).

To make it match the specification, our compilation definition could
easily eschew
shadow variables and rely only on the type-based
bounds expressions available in $\Gamma$ for checking. However, doing so would 
force us to weaken the simulation theorem, reduce expressiveness,
and/or force the semantics to be more awkward. We plan to work with
the \checkedc team to implement our approach in a future revision.

% \textcolor{purple}{Despite its name, \elang
%   does not model C. As mentioned at the beginning of this section,
%   \elang is an untyped version of \lang with the annotations on
%   literals removed. The lack of types make it impossible to capture
%   the undefinedness of certain operations such as pointer
%   arithmetic. This restriction is not an issue for establishing
%   the metatheory covered in the next section, but it does indicate
%   that we have to be careful if we want to embed \elang programs into C
%   programs.  As discussed in \cite[2.10, pg 24]{checkedc}, the C
%   standard can be overly
%   conservative when it comes to pointer arithmetic. For example,
%   it should be perfectly safe to increment a pointer out of its allocated
%   region, as long as the runtime guarantees that
%   the pointer is never dereferenced. However, a standard C compiler
%   would treat this type of incrementing operation as undefined, regardless of
%   whether the resulting pointer is dereferenced or not. Embedding
%   \elang naively into C would give us this type of undefined programs.
%   This can be mitigated through casting or rearranging the code during
%   the embedding process.
% }
% \mwh{Don't think all this was necessary. Added clarifying text at the
%   start at exactly what CoreC is. No need to go on at length about
%   what it is not.}

\subsection{Metatheory}
\label{sec:meta}

% \review{ 
% My interpretation of section IV.C is that the authors have a pen-and-paper
%   proof of Theorem 4, and that random testing using the Redex models has been
%   used to gain confidence in such a proof. Is this correct?}
% \liyi{Yes, this is correct.}
% \mwh{No, not correct: We have no pen-and-paper proof, just the
%   model. Clarify here}
% showing that
% extensive random testing fails to falsify the bisimilarity
% property. \mwh{Better way to state the previous?}

We formalize both the compilation procedure and the simulation
theorem in the PLT Redex model we developed for \lang (see Sec.~\ref{sec:syntax}),
and then attempt to falsify it via Redex's support for random
testing. Redex allows us
  to specify compilation as logical rules (essentially, an extension
  of typing), but then execute it algorithmically to
  automatically test whether simulation holds. This process revealed
  several bugs in compilation and the theorem statement.
%
  % us gain confidence that our original pen and paper proof of
  % simulation remains true with the addition of variable bounds. }
We ultimately plan to prove simulation in the Coq model.

%Turning to the simulation theorem: We first introduce notation
%used to specify the theorem.
We use the notation $\gg$ to
indicate the \emph{erasure} of stack and heap---the rhs is the same as
the lhs but with type annotations removed:
\begin{equation*}
  \begin{split}
    \heap  \gg & \dot \heap \\
    \varphi \gg & \dot \varphi
  \end{split}
\end{equation*}
In addition, when $\Gamma;\emptyset\vdash
\varphi$ and $\varphi$ is well-formed, we write $(\varphi,\heap,e) \gg (\dot \varphi, \dot \heap,
\dot e)$ to denote $\varphi \gg \dot \varphi$, $\heap \gg \dot \heap$
and $\Gamma;\Theta;\emptyset \vdash e \gg \dot e : \tau$ for some $\tau$ respectively. $\Gamma$ is omitted from the notation since the well-formedness of $\varphi$ and its consistency with respect to $\Gamma$ imply that $e$ must be closed under $\varphi$, allowing us to recover $\Gamma$ from $\varphi$.
Finally, we use $\xrightarrow{\cdot}^*$ to denote the transitive closure of the
reduction relation of $\elang$. Unlike the $\lang$, the semantics of
$\elang$ does not distinguish checked and unchecked regions.

Fig.~\ref{fig:checkedc-simulation-ref} gives an overview of 
the simulation theorem.\footnote{We ellide the  possibility of $\dot e_1$ evaluating to $\ebounds$ or $\enull$ in the diagram for readability.} The simulation theorem is specified in a way
that is similar to the one by~\citet{merigoux2021catala}.
An ordinary simulation property would
replace the middle and bottom parts of the figure with the
following: \[(\dot \varphi_0, \dot \heap_0, \dot e_0) 
  \xrightarrow{\cdot}^* (\dot \varphi_1, \dot \heap_1, \dot e_1)\]
Instead, we relate two erased configurations using the relation $\sim$,
which only requires that the two configurations will eventually reduce
to the same state. We formulate our simulation theorem differently
because the standard simulation theorem imposes a very strong
syntactic restriction to the compilation strategy. Very often, $(\dot
\varphi_0, \dot \heap_0, \dot e_0)$ reduces to a term that is
semantically equivalent to $(\dot \varphi_1, \dot \heap_1, \dot e_1)$,
but we are unable to syntactically equate the two configurations due
to the extra binders generated for dynamic checks and ANF
transformation. In earlier versions of the Redex model, we attempted
to change the compilation rules so the configurations could match
syntactically. However, the approach scaled poorly as we added
additional rules. 
This slight relaxation on the equivalence relation
between target configurations allows us to specify compilation more
naturally without having to worry about syntactic constraints.


% The two theorems are translation preservation and simulation. We donate $\xrightarrow{c}$ as the transition semantics of CLight.
\begin{thm}[Simulation ($\sim$)]\label{simulation-thm}
For \lang expressions $e_0$, stacks $\varphi_0$, $\varphi_1$, and heap snapshots $\heap_0$, $\heap_1$, 
if $\heap_0 \vdash \varphi_0$, $(\varphi_0,\heap_0,e_0)\gg(\dot \varphi_0,\dot \heap_0, \dot e_0)$,
and if there exists some $r_1$ such that $(\varphi_0, \heap_0, e_0)
\rightarrow_\cmode (\varphi_1, \heap_1, r_1)$, then the following facts hold:

\begin{itemize}

\item if there exists $e_1$ such that $r=e_1$ and $(\varphi_1, \heap_1, e_1) \gg (\dot \varphi_1, \dot \heap_1, \dot e_1)$, then there exists some $\dot \varphi$,$\dot \heap$, $\dot e$, such that
$(\dot \varphi_0, \dot \heap_0,\dot e_0) \xrightarrow{\cdot}^* (\dot
\varphi,\dot \heap,\dot e)$ and $(\dot
\varphi_1,\dot \heap_1,\dot e_1) \xrightarrow{\cdot}^* (\dot \varphi,
\dot \heap,\dot e)$.

\item if $r_1 = \ebounds$ or $\enull$, then we have $(\dot \varphi_0, \dot \heap_0,\dot e_0) \xrightarrow{\cdot}^* (\dot
\dot \varphi_1,\dot \heap_1, r_1)$ where $\varphi_1 \gg \dot
\varphi_1$, $\heap_1 \gg \dot \heap_1$.

\end{itemize}
\end{thm}


% when $r_1 = e_1$ for
% some $e_1$ and
% $(\varphi_1, \heap_1, e_1) \gg (\dot \varphi_1, \dot \heap_1, \dot e_1)$, then
% there exists some $\dot \varphi$,$\dot \heap$, $\dot e$, such that
% $(\dot \varphi_0, \dot \heap_0,\dot e_0) \xrightarrow{\cdot}^* (\dot
% \varphi,\dot \heap,\dot e)$ and $(\dot
% \varphi_1,\dot \heap_1,\dot e_1) \xrightarrow{\cdot}^* (\dot \varphi,
% \dot \heap,\dot e)$. When $r_1 = \ebounds$ or $\enull$, we have $(\dot \varphi_0, \dot \heap_0,\dot e_0) \xrightarrow{\cdot}^* (\dot
% \dot \varphi_1,\dot \heap_1, r_1)$ where $\varphi_1 \gg \dot
% \varphi_1$, $\heap_1 \gg \dot \heap_1$.

% \begin{thm}[Simulation ($\sim$)]\label{simulation-thm}
% For \lang expressions $e_0$, stacks $\varphi_0$, $\varphi_1$, and heap snapshots $\heap_0$, $\heap_1$, 
% if $\emptyset;\emptyset;\emptyset \vdash_\cmode e_0 \gg \dot e_0 :\tau_0$,
% and if there exists some $r_1$ such that $(\varphi_0, \heap_0, e_0)
% \rightarrow_\cmode (\varphi_1, \heap_1, r_1)$, when $r_1 = e_1$ for
% some $e_1$ and
% $\emptyset;\emptyset;\emptyset \vdash_\cmode e_1 \gg \dot e_1 :\tau_1$ where $\tau_1 \sqsubseteq \tau_0$
% , then
% there exists some $\dot \varphi$,$\dot \heap$, $\dot e$, such that
% $(\dot \varphi_0, \dot \heap_0,\dot e_0) \xrightarrow{\cdot}^* (\dot
% \varphi,\dot \heap,\dot e)$ and $(\dot
% \varphi_1,\dot \heap_1,\dot e_1) \xrightarrow{\cdot}^* (\dot \varphi,
% \dot \heap,\dot e)$. When $r_1 = \ebounds$ or $\enull$, we have $(\dot \varphi_0, \dot \heap_0,\dot e_0) \xrightarrow{\cdot}^* (\dot
% \dot \varphi_1,\dot \heap_1, r_1)$ where $\varphi_1 \gg \dot
% \varphi_1$, $\heap_1 \gg \dot \heap_1$.
% \end{thm}

Our random generator (discussed in the next section) never produces
$\euncheckedtext$ expressions (whose behavior could be undefined), so we
can only test a the simulation theorem 
as it applies to checked code. This limitation makes it
unnecessary to state the other direction of the simulation theorem
where $e_0$ is stuck, because Theorem~\ref{thm:progress} guarantees
that $e_0$ will never enter a stuck state if it is well-typed in
checked mode.

The current version of the Redex model has been tested against $20000$
expressions with depth less than $10$. Each expression can
reduce multiple steps, and we test simulation between every two
adjacent steps to cover a wider range of programs, particularly the
ones that have a non-empty heap.

\begin{figure}[t]
{\small
\[
\begin{array}{c}
\begin{tikzpicture}[
            > = stealth, % arrow head style
            shorten > = 1pt, % don't touch arrow head to node
            auto,
            node distance = 3cm
        ]

\begin{scope}[every node/.style={draw}]
    \node (A) at (0,1.5) {$\varphi_0,\heap_0, e_0$};
    \node (B) at (4,1.5) {$\varphi_1, \heap_1 ,e_1$};
    \node (C) at (0,0) {$\dot \varphi_0, \dot \heap_0 ,\dot e_0$};
    \node (D) at (4,0) {$\dot \varphi_1, \dot \heap_1, \dot e_1$};
    \node (E) at (2,-1.5) {$\dot \varphi,\dot \heap ,\dot e$};
\end{scope}
\begin{scope}[every edge/.style={draw=black}]

    \path [->] (A) edge node {$\longrightarrow_\cmode$} (B);
    \path [<->] (A) edge node {$\gg$} (C);
    \path [<->] (B) edge node {$\gg$} (D);
    \path [dashed,<->] (C) edge node {$\sim$} (D);
    \path [dashed,->] (C) edge node {$\xrightarrow{\cdot}^*$} (E);
    \path [dashed,->] (D) edge node[above] {$\xrightarrow{\cdot}^*$} (E);
\end{scope}

\end{tikzpicture}
\end{array}
\]
}
\caption{Simulation between \lang and \elang }
\label{fig:checkedc-simulation-ref}
\end{figure}

% \ignore
% {The two theorems can be best understood by the diagram in Fig.~\ref{fig:checkedc-simulation-ref}.
% % The diagram also indicates the simulation relation ($\sim$) among our formalization of the \checkedc Type System ($\vdash$), Semantics ($\longrightarrow$), Compilation ($\gg$), as well as the compiled C semantics ($\xrightarrow{c}$).
% On the top of the relation, we have type soundness theorem (Sec.~\ref{sec:theorem}) stating that every type-checked \lang program $e_0$ can transition via \lang semantics into a type-checked program $e_1$ by taking a checked step.
% The second line refers to that both $e_0$ and $e_1$ can be compiled to a C program $\dot e_0$ and $\dot e_1$, and such compilations always exist. The bottom two rewrites ($\xrightarrow{c}$) state that both $\dot e_0$ and $\dot e_1$ are transitioned finally to the same state $\dot e$. 

%  The simulation states that given any well-typed \lang expression $e_0$, its compiled C program as $\dot e_0$, if $e_0$ can transition one step to $e_1$, and there must exist the compiled C program $\dot e_1$, such that both $\dot e_0$ and $\dot e_1$ transition to the same final state $\dot e$ in C. The simulation relation $\sim$ is built on top of the "triangle" structure in Fig.~\ref{fig:checkedc-simulation-ref} stating that every \lang expression and its one step transitioned expression might be compiled to two different C program, but the C semantics always evaluates them to the same place.
% The translation preservation theorem is a support for the simulation theorem stating that for any well-typed \lang expression $e_0$,  its one step \checkedc transition expression $e_1$ must exist a compiled C program through the compilation process ($\gg$).

% A corollary of the simulation theorem states that for any \lang expression $e_0$ and its one step \lang transition expression $e_1$, such that $e_0 \longrightarrow e_1$, if $e_0$ contains a fault due to a unchecked block (the blame theorem tells us that if there is a fault in a \lang code, it must come from a unchecked block), and its compiled code $\dot e_0$ is evaluated to a fault state, the compiled code $\dot e_1$ for $e_1$ is always evaluated to the same fault state in C.
% This is an important property for a \lang compiler to preserve because any problematic program will be captured by running it long enough, so that enough random testing cases are able to capture all the bugs.
% In our formalization implementation, the type soundness theorem is proved through Coq, while the simulation theorem is validated through intensive random test case generation. We generate tens of thousands cases and cover all possible corner cases of the compilation. 
% }
% \ignore{
% \begin{figure*}[t]
%   \begin{subfigure}[b]{1\textwidth}
%     \begin{lstlisting}
% int deref_array (size_t n, nt_array_ptr<int> p : bounds(p, p + n)) {
%   return *p;
% }

% ...
% /* nt_array_ptr<int> p0 : bounds(p0, p0 + 5) */
% deref_array(5, p0);
%     \end{lstlisting}
%     \label{fig:chkcexamplederef}
%     \caption{\texttt{deref\_array} in \checkedc}
%   \end{subfigure}

%   \begin{subfigure}[b]{1\textwidth}
% \begin{lstlisting}
% int deref_array(size_t n, int *p) {
%   /* statically determine the definitions p_lo and p_hi */
%   int *p_lo = p;
%   int *p_hi = p + n;
%   /* runtime checks */
%   assert(p_lo <= p && p <= p_hi);
%   assert(p != NULL);
%   /* widening */
%   int p_derefed = *p;
%   if (p_derefed != '\0') {
%     if (p_hi == p) {
%       ++p_hi;
%     }
%   }
%   return p_derefed;
% }
% ...
% /* int *p0 */
% /* rho(p0) = p_lo, p_hi */
% deref_array(5, p0);
% \end{lstlisting}
% \caption{\texttt{deref\_array} in C}
%     \label{fig:cexamplederef}
%   \end{subfigure}

%   \begin{subfigure}[b]{\textwidth}
%     \begin{lstlisting}
% /* nt_array_ptr<int> p : bounds(e0, e1) */
% /* rho(p) = p_lo, p_hi */
% size_t x = strlen(p);
% if(x >= 1) {
%   *(p+1);
% }
%     \end{lstlisting}
%     \caption{\texttt{strlen} in \checkedc}
%     \label{fig:chkcexamplestrlen}
%   \end{subfigure}

%   \begin{subfigure}[b]{\textwidth}
%     \begin{lstlisting}
% /* nt_array_ptr<int> p : bounds(e0, e1) */
% /* rho(p) = p_lo, p_hi */
% /* runtime checks omitted */
% ...
% size_t x = strlen(p);
% int *p_hi_new = p + x;
% p_hi = max(p_hi, p_hi_new);
% if (x >= 1) {
%   /* null check for pointer arithmetic */
%   assert(p != NULL);
%   int *p_1 = p + 1;
%   /* null check for dereference */
%   assert(p_1 != NULL);
%   /* bounds check for dereference */
%   /* note how we are using p_hi rather than p + x */
%   assert(p_lo <= p_1 && p_1 <= p_hi);
%   ...
% }
%     \end{lstlisting}
%     \caption{\texttt{strlen} in C}
%     \label{fig:cexamplestrlen}
%   \end{subfigure}
% \caption{Compilation of null-terminated string dereference}
% \label{fig:compilationexample}
% \end{figure*}
% }

% \ignore{
% Fig.~\ref{fig:compilationexample} gives an example of how we compile the
% dereference of null-terminated strings by inserting explicit checks and
% bounds widening code. For readability, we present the example in real
% \checkedc and C syntax. At
% line 3-4 of Fig.~\ref{fig:cexamplederef}, we see how the upper and lower
% bounds are defined in terms of the arguments \code{p} and \code{n}
% according to the bounds annotations obtained during typechecking. This
% avoids the need to pass in bounds as extra arguments, thus maintaining
% compatibility with C code. When we call \code{deref_array} on
% \code{p0}, a string with size 5 (excluding the null character), there
% is no need to pass the bounds \code{p_lo} and \code{p_hi} stored on
% the stack. It is possible that \code{p_hi} has been widened
% before we reach line 21. The programmer will have to perform a
% \code{dynamic_bounds_cast} to recover the more precise bounds information. The if statement from line 10 to line 14
% attempts to widen the bound when the dereferenced result is not
% null. \todo[inline]{Show a full-fledged C program?} The widened upper bound will be available within the scope of
% \code{deref_array}, in contrast to the T-IfNT rule, which only
% remembers the widened bounds within the scope of the then branch of
% the if statement. 


% \todo[inline]{Widening can happen at every
%   dereference at runtime. Is that ok?} Fig.~\ref{fig:cexamplestrlen}
% shows how an invocation of \code{strlen} on null-termianted strings
% is compiled into C code. We perform the same runtime checks that
% happen during dereferencing. The widening code at line
% 8 updates \code{p}'s upper bound only if the result of
% \code{strlen} is larger than the value of the upper bound stored on
% the stack. This is another scenario where the runtime can be more precise
% than the statically determined bounds information.



% }

\section{Evaluation}\label{sec:evaluation}

% \review{While I found the idea behind section V very interesting, the current version
%   of this section lacks some details that would help in better understanding (1) 
%   how the approach works, and (2) the overall scope of the approach. 
%   $\\$
%   For instance, the authors state that, following [19], they try to "exercise
%   interesting patterns" by adding "admissible but redundant typing rules" like
%   G-ASTR. There are a few points that are unclear here: (1) are these rules
%   discovered manually or automatically (starting from the Redex semantics)?, (2)
%   are there any guiding principles for coming up with rules that lead to
%   interesting cases?
%   $\\$
%   Later, the authors refer to "generation rules modified to be slightly more
%   permissive" to generate "a little" ill-typed terms. Again, are these rules
%   obtained automatically or defined manually? If the latter, did you follow any
%   methodology to derive such rules? Are these rules the same as the "admissible
%   but redundant typing rules" from above?}
% \liyi{Deena? Leo? }
Our evaluation of \systemname consists of a series of tests that can be categorized into Micro-benchmarks.
MicroBenchmarking involves evaluating performance on fundamental operations involving tainted pointers, context switching between checked and sandboxed regions, and sandboxed execution of functions. We further go on to evaluate \systemname on six real-world programs pertaining to diversified domains to evaluate real-world run-time and memory performance. We use C's \<time.h\> library to evaluate the Runtime performance by placing each of the test-bench calls within the timing scope (Program scope within which the timer runs). Our timing scope excludes marshaling activities within the test-case environment with an intuition that marshaling is irrelevant if tainted\-ness of a pointer is propagated from the call site up until its declaration. We use Valgrind's "massif" memory profiler to benchmark the peak memory usage of the Heap. Unlike the Runtime performance, we record Peak Memory usage as a relative offset to original program because most programs are extremely small in comparison to the constant overhead from the sandbox (81 KiB approx). This can further be optimized away by choosing to only compile custom Tainted wrappers for those STDLib functions that are in use by the tainted pointers in the program. Valgrind's memory figures do not account for Sandboxed allocations which happen in the shadow memory. 
All of the evaluation was performed using 6-Core Intel i7-10700H with 40 GB of RAM, running Ubuntu 20.04.3 LTS and the benchmarks for every test were sampled as the mean of ten consecutive iterations.

\subsection{Micro-Benchmarks}
\subsubsection{Memory Access in SBX}
Memory access performance between the checked and the WASM region involves crafting a test case that involves a simple pointer arithmetic operation enclosed in a loop of 100k iterations. 156.6\% overhead with \systemname is caused by the code executing in WASM Sandbox which is comparatively inefficient as WASM compiler toolchain does not support code optimization like LLVM/GCC. 

\subsubsection{Indirect-Call}
This test involves evaluating the overhead involved in making indirect calls and is recorded by benchmarking the time taken for a sandboxed Callee (call from the checked region into the SBX) to return. This metric is evaluated against the time taken for a checked Callee (call within local scope) to return. Observed overhead is once again a consequence of code executing in the WASM Sandbox and call indirection.

\subsubsection{Pointer Access}
Pointer overhead during run-time between the pointers in the checked and Sandboxed regions is evaluated by benchmarking a test that involves 100k read/write/arithmetic operations on the pointer. 34\% in overhead is the result of "Offset To Pointer" conversion and sanity checks for Taintedness inserted by \systemname at every access to the tainted pointer.   

\begin{center}
\label{fig:micrbenchmarks}
\begin{tabular}{||c c||} 
 \hline
 Test-Name & \systemname overhead \\ [0.5ex] 
 \hline\hline
 Memory Access in SBX & +156\% \\
 Indirect-Call & +398\% \\ 
 Pointer Access & +34.16\% \\ [1ex]
 \hline 
\end{tabular}
\end{center}

\subsection{Program Run-time Benchmarks}
\myparagraph{Overview}
We evaluate \systemname on the basis of conversion efforts and performance on six programs with the use of a pre-included test suite for each program. We intend to demonstrate \systemname's capability of enforcing spatial safety in all of the scenarios leading to modern memory vulnerabilities by re-introducing corrected memory bugs and then annotating the relevant buggy code with tainted pointers. However, this requires the bugs to be predicted before they are discovered, which is chronologically impossible. Consequently, we also annotate large sections of relevant bug-free code to mimic the developer's intuition on making \systemname annotations without any input on the bug locality. The extent of conversion qualitatively dictates the converted program's conversion efforts required and the run-time performance. Consequently, each of the conversions for the six programs follows a varied approach. 

\myparagraph{Parsons}
Parsons is annotated comprehensively in two variants parsons\_wasm and parsons\_tainted. parsons\_wasm has most of its input parsing functions moved into the sandbox, whilst having all its pointers marked as tainted. These sandboxed functions interact with the checked region by making indirect calls through RLBOX's callback mechanism. However, with parsons\_tainted, we do not move any of the functions to the sandbox but still mark all the pointers as tainted. The test suite itself consists of 328 tests comprehensively testing the JSON parser's functionality. Benchmarks for both of these forks are recorded using the mean difference between the \systemname and generic-C/checked-C variants when executing 10 consecutive iterations of the test suite. parsons\_wasm expectedly shows 200/266\% runtime overhead when evaluated against checked-c and generic-c respectively due to the performance limitation of WebAssembly. However, evaluating parsons\_tainted against checked-c shows \systemname to be faster because \systemname by itself performs lighter run-time-instrumentation on tainted pointers as compared to the run-time bounds checking performed on checked pointers by checked-c. Furthermore, we only see an average peak memory of 9.5 KiB as compared to the anticipated 82 KiB overhead as Valgrind does not consider the WASM Shadow memory allocated to the tainted pointers.

\myparagraph{LibPNG}
\systemname changes for libPNG is narrow in scope and begins with the encapsulation CVE-2018-144550 and a buffer overflow in compare\_read(). However, we also annotate sections of Lib-png that involve reading, writing, and image processing (interlace, intrapixel, etc) on user-input image data as tainted. That is, rows of image bytes are read into tainted pointers and the taintedness for the row\_bytes is propagated throughout the program. All our changes extend to the png2pnm and pnm2png executables. To evaluate png2pnm, we take the mean of 10 iterations of a test script that runs png2pnm on 52 png files located within the libpng's pngsuite. To test pnm2png, we take the mean of 10 iterations of pnm2png in converting a 52MB 5184x3456 pixels large pnm image file to png. Valgrind's reported lower Heap space consumption for \systemname converted code is due to the discounted space consumed on the heap by the Sandbox's shadow memory. Consequently, when evaluating pnm2png, \systemname's heap consumption was 52 MB lower as the entire image was loaded onto the shadow memory.  

\myparagraph{ProFTPD}
\systemname changes for ProFTPD are further limited in extent and aimed at the exact changes required to encapsulate CVE-2010-4221. We mark the user input to the unsafe function "pr\_netio\_telnet\_gets()" as tainted (\_TPtr$l$char$g$) and propagate its tainted-ness to its callers and callees enclosed within its defined scope. Our changes for the above function were policed by 24 unique API tests, which we use to benchmark the performance. For each test, our benchmark samples the delta between the call and return time of pr\_netio\_telnet\_gets(). This sampling is repeated 10 times and its mean value is reflected in the below table. Although \systemname is shown to require 50\% more time in servicing this request, this delta is made insignificant during deployment where network bandwidth relatively accounts for a bigger metric to the overall performance.

\myparagraph{MicroHTTPD}
MicroHTTPD demonstrates the practical difficulties in converting a program to \systemname. Our conversion for this program was aimed at sandboxing memory vulnerabilities CVE-2021-3466 and CVE-2013-7039. CVE-2021-3466 is described as a vulnerability from a buffer overflow that occurs in an unguarded "memcpy" which copies data into a structure pointer (struct MHD\_PostProcessor pp) which is type-casted to a char buffer (char *kbuf = (char *) \&pp[1]). Our changes would require making the "memcpy" safe by marking this pointer as tainted. However, this would either require marshaling the data pointed by this structure (and its sub-structure pointer members) pointer or would require marking every reference to this structure pointer as tainted, which in turn requires every pointer member of this structure to be tainted. Marshalling data between structure pointers is not easy and demands substantial marshaling code due to the spatial non-linearity of its pointer members unlike a char*. This did not align with our conversion goals which were aimed at making minimal changes. Consequently, the above CVE stands un-handled by \systemname.  Our changes for CVE-2013-7039 involve marking the user input data arguments of this function as tainted pointers and in the interests of seeking minimal conversion changes, we do not propagate the tainted-ness on these functions. Following up on the chronological impossibility of sandboxing bugs before they are discovered and the general programmer intuiting, we moved many of the core internal functions (like MHD\_str\_pct\_decode\_strict\_() and MHD\_http\_unescape()) into the sandbox. 

\myparagraph{UFTPD}
\systemname changes for UFTPD were aimed at sandboxing CVE-2020-14149 and CVE-2020-5204. CVE-2020-14149 was recorded as a NULL pointer dereference in the handle\_CWD() which could have led to a DoS in versions before 2.12, thereby, requiring us to sandbox this function. CVE-2020-5204 was recorded as a buffer overflow vulnerability in the handle\_PORT() due to sprintf() which also required us to sandbox this function. Although we could have chosen to only mark the faulty pointers as tainted, we intended to keep our changes more generic. For evaluation, we manually write a script for 3 Tests that each trigger "quote CWD", "quote PORT", and FTP "get file" request 10 times in a loop. Following this we record an entry for each of these tests as a mean of recorded timestamps for 10 executions for \systemname and generic-c versions, following which we record the relative average latency across the three tests between both of these UFTPD versions as a percentage in the below table.   

\myparagraph{Tiny-bignum}
Motivated by its size and simplicity, we intended to make comprehensive \systemname changes for Tiny-bignum by marking all the pointers as tainted. Furthermore, in likes of an unsafe use of sprintf() that could possibly lead to a buffer overflow within the bignum\_to\_string(), we move this function into the WASM sandbox. Ideally, a major portion of conversion efforts to \systemname is attributed to understanding the codebase and finding the precise extent to which we choose to propagate the taintedness or to stop and give up and marshall the data, thereby, requiring only 4 hours to convert tiny-bignum to \systemname. The evaluation was performed on Tiny-bignum's test suite consisting of 4 Test cases, each of which test the functionality to scale on big numbers subject to all of the supported unary and binary operations.  

\begin{center}
\label{fig:prgrbenchmarks}
\begin{tabular}{||c | p{1.3cm} | p{1.8cm}||} 
 \hline
 Program & Run-Time & Avg Peak Memory \\ [0.5ex] 
 \hline\hline
 ProFTPD & +47.8\% & +82.1 KiB  \\
 MicroHTTPD & +171.7\% & +24.7 KiB\\ 
 UFTPD & +2.7\% & +86.3 KiB \\ 
 Tiny-Bignum (vs checked-c) & -23.4\% &  +81.1 KiB \\
 Tiny-Bignum &  +121.7\% &  +81.1 KiB \\
 parsons\_wasm (vs checked-c) & +200.2\% & +9.8 KiB\\ 
 parsons\_tainted (vs checked-c) & -8.7\% & +9.2 KiB\\
 parsons\_wasm & +266.5\% & +9.8 KiB\\ 
 parsons\_tainted & +10.45\% & +9.2 KiB\\ 
 LibPNG (png2pnm) & +11.41\% & -102.5 KiB\\ 
 LibPNG (pnm2png) & +46.5\% & -51.6 MiB\\ [1ex]
 \hline 

\end{tabular}
\end{center}


\begin{center}
\begin{tabular}{||p{2 cm} | p{1.2cm} | p{1.2cm} | p{1.0cm} | p{1.0cm}||} 
 \hline
 Program & Pointers Annotated  & Lines Sandboxed & CVEs fixed & Time To Port \\ [0.5ex] 
 \hline\hline
 ProFTPD & 6 & 0 & - & 1 Day  \\
 \hline
 MicroHTTPD & 139 & 650 & 1 & 4 Days \\
 \hline
 UFTPD & 146 & 90 & 2 & 3 Days \\ 
 \hline
 LibPNG & 248 &  0 & 1 & 8 Days \\
 \hline
 Tiny-Bignum &  69 &  30 & - & 4 Hours \\
 \hline
 parsons\_wasm & 364 & 800 &  - & 3 Days \\ 
 \hline
 parsons\_tainted & 378 & 0 & - & 8 Days \\ [1ex]
 \hline 
\end{tabular}
\end{center}
% 
%\leo{The following is extremely
%  weak. ``Most of them'', were there any that weren't? Which ones? For
%  the ones that were, mention github issues.}  The random generator,
%equipped with the conversion tool, successfully found a few minor
%errors in the clang compiler, most of them were already issues in the
%git bug reports. For example, we discovered that while the ternary
%operator is implemented in the compiler it cannot handle complex
%bounds types in the branches. The static analysis is not sophisticated
%enough to properly detect that both branches have the same type. While
%not precisely a bug, the clang compiler does not permit memory for
%null terminated arrays to be allocated with calloc. Although calloc
%fills all spaces in memory with null, the compiler does not recognize
%this and claims that it is an unsafe cast.


% Recall that our
% formal model makes liberal use of bounds annotations in literals and
% the heap. 


% In order to get a better understanding of the formalism we wrote it in
% redex. This allowed us to make sure that expressions were well-typed
% and evaluated to what we expected. It also was helpful for use in
% prototyping; new features could first be added to the redex model to
% see how they interacted with the existing language. This model was
% slightly larger than the Coq model and there are some differences in
% the type systems. We included top level functions and conditional
% expressions. All of these extra expressions are still expressible in
% the coq model, for example functions can be represented as nested let
% expressions. In the Coq model variables are stored on a stack while in
% the Redex model the variables are simply looked up in the context. In
% general the Redex model is easier to modify and slightly closer to the
% actual Checked-C specifications. Instead of using the model for a
% static proof, we used it to increase our certainty of the accuracy of
% the model.


% \item Describe the random testing generator setup and the properties
%   to test.
% \yiyun{Deena's description of the implementation details. I tried
%   integrating the ones that I find relevant/interesting to the text above. Maybe we can
%   add more if we have some space to fill in.}
% In order for our guarantee of safety to hold, we need to know that our
% model acurately reflects the CheckedC clang compiler. Safety is proved
% for the Coq model, but it is significantly smaller than the actual
% language. The Redex model is a combination of both. It is written in
% the same style as the formalism but has slightly more of Checked-C's
% extra features. If expressions from the Redex model display the same
% behavior as equivalent programs in Checked-C then we have greater
% certainty that our model is useful. We built a random testing
% generator to increase this certainty.

  % \item Describe the bug findings from the random testing against the Checked-C compiler.
%   \leo{This is now integrated above}
% The generator was helpful in finding bugs in the redex model. Several things failed to typecheck that should have been well typed, and the generator was able to catch them. The generated code also found a few minor errors in the clang compiler, most of them were already issues in the git bug reports. For example we discovered that while the ternary operator is implemented in the compiler it cannot handle complex bounds types in the branches. The static analysis is not sophisticated enough to properly detect that both branches have the same type. While not precisely a bug, the clang compiler does not permit memory for null terminated arrays to be allocated with calloc. Although calloc fills all spaces in memory with null, the compiler does not recognize this and claims that it is an unsafe cast. In the Redex model there is no issue with this. A few other minor things were brought to light in the implementation of the generator. The main use was to increase certainty that the behavior in the formal model accurately matched the clang compiler.
%  
% % \end{itemize}


% \begin{itemize}
%  
% \item Show that why the formal semantics/type-system defined for Checked-C is useful. 
% Since we have certainty that our model reflects the clang compiler the model is very useful. Proofs are easier on the smaller model, so we can show  that certain things are true for it. Since the Redex model is between the formalism and the clang version we can have certainty that properties we expect are actually true for the clang version.
%  
% \begin{itemize}
% \item Show some bug findings. 
% \item Show the properties that we can guarantee for Checked-C based on the type-system and blame theorem.
% \item Maybe other useful tools that can be extracted from the Redex model.
%  
% \end{itemize}
%  
% \end{itemize}

% \section{Difference in the \checkedc Formalization}\label{sec:discussion}

\begin{figure}[t]
{\small
{\captionsetup[lstlisting]{margin = 8 mm}

 \begin{lstlisting}[xleftmargin=8 mm]
void memcpy (nt_array_ptr<int> a : count(0),
         nt_array_ptr<int> b : count(0)) {
   while (* a) {
     if (* b)
        * a = * b;
     else break;
     a++; b++;
   }
}
  \end{lstlisting}


}
}
\caption{Bound Widening Examples}
\label{fig:bound-widening}
\end{figure}

% In the left of program \textbf{(c)} (Fig.~\ref{fig:checkedc-example}), we 
% assume that the lower/upper bounds of \code{x} are given as \code{0} in the beginning of the program. 
% In a fat-pointer framework, this program might not reach failure because the \code{x} type is adjusted dynamically and the upper bound keeps growing.
% In a classical language that guarantees spatial safety statically, however, this program results in dynamic bound check errors, because once we access \code{x} and increase the pointer address by $1$, the pointer reaches the upper bound given by the type. 

There are some differences between our \checkedc formalization and the \checkedc specification/implementation. One of the examples is the use of $\Theta$ in the type system. The difference has been described in Sec.~\ref{subtype-type}.
The other difference is the bound widening scope of the branching operations.
Bound widening refers to that when we are accessing the final element in an NT-array indicated by the upper bound of the NT-array, we find that the element is not null; clearly, there is more elements after the element, so the upper bound can extended dynamically and we are able to access the elements after the final one.
In a fat-pointer framework, this is easy to implement. In a static bound system, however, there is impossible to implement the bound widening behaviors in any context because all bound information is statically guided. 
The \checkedc specification allows bound widening behaviors in a limited format. 
\code{strncat} in Fig.~\ref{fig:strncat-ex} is an example bound widening in \checkedc. Without the \texttt{strlen} usage in line 5, \code{b}'s upper bound is not extended to \code{x}. In that case, line 13 results in a $\ebounds$ error.

Our formalization is different from the \checkedc specification and implementation.
One cool thing that our formalization can do is a second version of the \code{memcpy} implementation in Fig.~\ref{fig:bound-widening}. 
This function takes two NT-array pointers, and copies the data in NT-array \code{b} to \code{\a} as mush as we can until we hit a null-character in either \code{a} or \code{b}.
The current \checkedc specification and compiler do not support this program, but they have difference. 
The \checkedc specification allows bound widening in the scope of the \code{true} branch, meaning that the \code{b}'s upper bound is $1$ in line 5 (in the \code{true} branch) in the first loop iteration; so does the pointer \code{a} in the while loop. In each iteration, \code{a}'s upper bound is extended by $1$. The problem is that the scope of the bound widening in the specification is lost after jumping out of the branching scope. After line 5, \code{b}'s upper bound is back to $0$. The \code{b++} in line 7 shifts \code{b}'s pointer address; thus, in the second iteration, the line 5 code results in a $\ebounds$ error because \code{b}'s dereference address is bigger than the upper bound. 
On the other hand, the \checkedc compiler results in a $\ebounds$ error for \code{a}'s pointer in line 3 in the second iteration, because the compiler's bound widening behavior in a \code{while} loop is only restricted to a local loop body, and does not extended to further loop iteration. 
Our formalization extends the bound widening behavior after the \code{true} branch scope; thus, executing the \code{memcpy} in Fig.~\ref{fig:bound-widening} does not reach an runtime error, and NT-array \code{a} copies \code{b}'s data as long as the length of \code{a} is greater than \code{b}. 

We implement this behavior different from both the \checkedc specification and compiler because of two reasons. First, the \checkedc specification and compiler disagree here so there is no formally "good" behavior. Second, we believe that this formalization permits more interesting programs, and sometimes, more efficient programs. 

In our compiler implementation, we utilize ghost variables, which is introduced in Sec.~\ref{sec:compilation}, to equalize the behavior in our formalization and implementation. Thus, the behavior of our formalization and implementation matches tightly (proved by Theorem~|ref{simulation-thm}).
The lack of shadow variables become more of an issue when we extend the
\checkedc specification with other bounds widening primitives such as \code{strlen}. 
For example, if users are not careful and move the \code{strlen} expression for pointer \code{y} in line 5 to later  are not able to Fig.~\ref{fig:strncat-ex}, it might causes the compiler ill-recognize the scope of the \code{strlen} bound widening effect; thus, the execution in line 13 fails.
With the shadow variables, it certainly allows us to mutate \code{y}'s bound widening scope and designs more effective compilation strategy to permit more intereting programs, like Fig.~\ref{fig:bound-widening}.
The use of shadow variables also allow an easy design of the compiler. 
complex static reasoning can be
difficult to implement. At the time of writing, the Clang \checkedc
implementation does not yet support the type of static reasoning to
allow functions like \code{memcpy} in Fig.~\ref{fig:bound-widening} to work properly.
Additionally, forcing the
user to declare a bound variable explicitly and guide the typechecker
to prove that it's a valid bound can result in extremely verbose
code.
Finally, shadow variables resides in the stack so they do not create any significant overhead for the execution of a program. 


\liyi{below are more wording from Yiyun that I think we can drop. Below means to the end of the section. }

The
 \code{strcat\_checked} function from Fig.~\ref{fig:bound-widening}
 takes an extra argument \code{n} that represents the maximum capacity
 of the NT-array \code{s0}. We first use the \code{strlen} function to
 locate the end of \code{s0}, and then start copying from \code{s1} to
 \code{s0} through a while loop. With the \code{strlen} widening
 extension introduced by the Clang \checkedc compiler, the bounds of
 \code{s0} will be casted into \code{count(i)} after
 line 14. This well-intended casting behavior can end up shrinking the
bounds of \code{s0} when \code{strlen(s0)} is smaller than \code{n},
which is almost always the case when the \code{strcat\_checked}
function is called. This will force the programmer to expand the
call to \code{strlen} into an explicit \code{while} loop to prevent
the \checkedc compiler from casting the bounds, adding unnecessary
complexity. On the other hand, with shadow variables, we ensure at
runtime the ghost variable representing the upper bound can only get
larger. In our formal semantics, the
ghost variable will be updated only if the return value of \code{strlen} is
greater than the variable's current value. The subsequent indexing
into the array will thus be checked against the upper bound that is equal
to \code{max(n,i)}, avoiding the runtime error.
Sadly, while our formal model is better-behaved at runtime, we do not
leverage the existence of shadow variables in our type
system. This is because the original \checkedc specification is
not designed around the idea of shadow variables. If we were to extend the
specification with shadow variables, we could expose special syntax for
programmers to read those variables so they don't have to rely on
runtime casts to indirectly retrieve that information. In the type
system, we can use the ghost variable to represent the largest known
bounds of an array and extend the subtyping rules to allow shrinking
to smaller bounds that can be inferred statically. 

The other significant difference is the use of shadow variables in our implementation of the \checkedc compilation.
The usefulness of shadow variables is not immediately obvious. In
this section, we provide examples that demonstrate the limitation of the
current \checkedc specification due to the lack of shadow variables.

In Fig.~\ref{fig:bound-widening}, we define the function \code{foo}
which takes an NT-array pointer of size $0$ as its argument. The
function initializes the variable \code{cnt} to \code{0} and
increments \code{cnt} only if the value pointed by \code{p} is not a
null character. Finally, after exiting the if statement, we attempt to
dynamically cast bounds of \code{p} to be \code{count(cnt)} before
returning.


It is also worth mentioning that the bookkeeping of shadow variables do
introduce hidden overhead, compared to an alternative approach that
relies on static proofs. In the \checkedc specification, it is
possible to take advantage of its data-flow analysis and assign more
precise bounds without relying on shadow variables. The function
\code{bar} from Figure.~\ref{fig:bound-widening} implements the same
functionality as \code{foo} but includes additional static bounds casting in the
\code{if} statement. The type system can prove that
\code{count(cnt)} is a valid bounds annotation after the if statement,
avoiding the overhead of an additional \code{dynamic_cast} as in \code{foo}.

However, we argue that there are strong reasons that justify
why we should include shadow variables. First,  Second,  \yiyun{I'm not familiar with the conversion tool. is this accurate?} This also
defeats the purpose of using \checkedc for gradually migrating legacy C code,
which, in practice, is more likely done by automatic
conversion tools that are unable to perform any sophisticated
theorem proving. Finally, the introduction of shadow variables can
always be made optional or simply used in conjunction with a static reasoning
system. \yiyun{I think I phrased this rather poorly. Would be helpful
  if someone could help me rephrase this a little} We expect to introduce
shadow variables to the \checkedc specification but defer some of the
more subtle design decisions that might have an impact on performance
until we receive more benchmark results from existing C to \checkedc
conversion tools \yiyun{should I cite the 3C paper here?}.
\ignore{
As we discuss in Section~\ref{sec:overview}, the \checkedc compiler
is able to statically determine that the upper bound of \code{p} must
be at least \code{1} in the then branch of the if statement. However,
once the program exits the if statement, the upper bound of \code{p}
is conservatively reverted back to \code{0}. Without using ghost
variables to keep track of the updated bounds, the \checkedc compiler
would attempt to check whether \code{count(cnt)} is included within
the statically determined imprecise bounds \code{count(0)} at line 6,
causing a runtime error when we pass in a non-empty string.
}




%\code{strcat_checked} function 

% This limitation can also affect programs that are more practical. For
% example, at Line 16, we compute \code{strlen(p)} manually through a
% \code{while} loop. The \code{dynamic_bounds_cast} fail for the reason
% as the one we see in \code{foo}.



% The Clang \checkedc implementation uses statically determined
% bounds to insert runtime checks. In the left of Fig.~\ref{fig:checkedc-example} \textbf{(c)}, the
% \code{dynamic_bounds_cast} at line 3 will always succeed, because the
% compiler knows that within the scope of then branch, the pointer
% \code{p} must have at least one element. The same cast at line 6,
% however, will always fail, since there is no way to tell statically
% whether the program has entered the then branch before. The compiler
% will check whether the \code{count(1)} bounds specification is
% contained within the earlier \code{count(0)} specification, resulting
% in a runtime failure even when we pass in a non-empty string.

% Our formalization diverges from this runtime behavior and instead keeps
% track of the bounds on the stack. After entering the then branch, we
% increment the upper bound for \code{p}, effectively making the
% updated bounds information available even after we exit the if
% statement. The cast at line 6 will be checking the new bounds against
% the incremented bounds for non-empty strings.

% In our formalization, every pointer literal is annotated with bounds
% information. During compilation, the bounds annotations are converted
% into shadow variables that are updated during widening and accessed
% for runtime checks. This allows our formal model to successfully
% run functions defined like \code{foo}. However, the \checkedc
% specification is designed based on the absence of ghost
% variables and therefore does not provide primitive operations that
% allow programmers to interact with them.




% Fig.~\ref{fig:checkedc-example} \textbf{(e)} gives a more practical example of why keeping
% track of the bounds on the stack is useful. The program snippet
% implements the functionality of the \code{strlen} function using a
% while loop and a \code{cnt} variable. Even though the type system is
% unable to reason about the while loop statically (which would require
% inductive reasoning), as long as the runtime system updates the bounds
% in-place, the user can apply a
% \code{dynamic_bounds_cast} to soundly recover the more precise bounds
% information.

% \yiyun{That's actually not right. The fresh bounds variables are generated during the creation of x, before strlen is called. I'll rewrite this later.}
% The \code{strlen} formalization is a special case of the branching operations with an NT-array pointer Boolean guard.
% In this case, \checkedc generates a new fresh local variable in the local stack in its type rule, which records the length from the current position of a pointer (\code{x}) to the first null-character. We do not know the value of \code{x} in the type checking stage; instead, in the compiler, \checkedc insert dynamic checks of \code{x} into necessary places that use the upper bound information of \code{x}.
% Fig.~\ref{fig:checkedc-example} \textbf{(b)} calls \code{strlen} to get the length of the NT-array pointer \code{x}. Since the input lower/upper bounds for \code{x} are \code{0}, the result of executing \code{strlen} is always the length between the current position of \code{x} and the first null character in the list because the length is always greater than \code{0}. In general, calling \code{strlen} on an NT-array pointer might not return the above size. If the the length between the current position of \code{x} and the first null character is less then the high-bound size of the pointer type, then the result is the greater of the two.

% \yiyun{strcmp would be a useful example here? It would be assigned a signature \lstinline{nt_array_ptr<char> p0:bounds(p0,p0)}, \lstinline{nt_array_ptr<char> p1:bounds(p1,p1)} }
% \paragraph{\textbf{Subtyping and Casting Operations}} 
% It is a common usage in a program to use a large size array as a smaller one, i.e., using a \code{array_ptr<int> : count(5)} typed array as a \code{array_ptr<int> : count(2)} typed one. \checkedc provides a static casting operation $\ecast{\tau}{x}$ to cast a subtype into its supertype. Every type checked casting operation is always executed successfully without causing any error states and with a "zero" cost. One example use of static casting is given in \code{strncat} in Fig.~\ref{fig:strncat-ex}.
% The \code{strlen(a)} in line 4 updates the upper-bound of \code{a} to be \code{x}. Then, the statement \code{(nt_array_ptr<int> : count(0)) a} casts the upper-bound of pointer \code{a} from \code{x} to \code{0}.

% \yiyun{TODO: Rewrite using the static bound cast syntax from page 39 of the checekdc spec. Also, this type of static reasoning that the n >= 0 is easy to do, but I don't think it's implemented in Redex. Maybe we should at least mention this strlen extension is not part of the checkedc spec?}
% \liyi{described in formal typing system.}

\mwh{HERE IS OLD TEXT FROM BACKGROUND. MIGHT GO HERE}

\yiyun{I rewrote this paragraph to express more explicitly why the static bounds are not precise.}
The bounds inferred at compile time are not always precise for null-terminated arrays. \yiyun{Should we rephrase this or remove the same example from the introduction?} Consider the statement \lstinline|if (*p) {..._0} {..._1} ..._2|
where \lstinline{p}'s type is \lstinline{nt_array_ptr<T> count(0)}, the type checker can infer that \code{p} has the bounds annotation \code{count(1)} in \lstinline{..._1} because the \lstinline{*p} cannot possibly be the null terminator in the then branch. However, what type should we assign to \lstinline{p} at \lstinline{..._2}? With the limited expressiveness of the \checkedc type system, it can only soundly infer the more conservative bounds annotation \lstinline{count(0)}.
We will later discuss how we keep track of the updated bounds information at runtime and also allow programmers to reflect the updated bounds to the type system via dynamic bounds casts.


\yiyun{move the following text somewhere else}
It is still possible to keep track of the updated bounds information at runtime.
To do so, we store the upper bound of the pointer \lstinline{p} as a stack variable. Inside the then branch, the stack variable will be incremented, effectively tracking the information even after the program exits the if statement. Later, we will see an practical example that makes use of this extra bit of information.

\yiyun{I want to justify our decision of using stack variables by pointing out the lack of clarity of the checkedc spec on the runtime behavior. Should I include this discussion in a later section?}
It is worth noting that the original \checkedc specification is very unclear ...
% The above pointer bound description discusses the case when we know exactly the lower/upper bounds of a pointer $p$. The pointer bound information of \checkedc is detected through the \checkedc type system and statically inserted bound checks. It is unlikely we can always know the exactly bounds for a pointer $p$. 
% For example,
% a given array/NT-array pointer \code{a} is declared as \code{array_ptr<int> a : count(x)}/\code{nt_array_ptr<int> a : count(x)}. The \code{count(x)} here means that the lower-bound of the pointer is \code{0}, and the upper-bound is \code{x}.
% For an array pointer, even though the bound values of the lower/upper bounds (\code{0} and \code{x}) can change during evaluation, the array length (\code{x-0}) is fixed.

% For a NT-array pointer, the situation is different. 
% The interpretation of the bounds of an NT-array pointer is similar to an array pointer, but the
% range can extend further to the right, until a null terminator
% is reached (i.e., the null is not within the bounds).
% During this process, it is possible that the upper bound \code{x} is extended and the NT-array length is "extended" due to the discovery of the upper-bound array position is not a null character.

\section{Related Work}
\label{sec:related}

Our work is most closely related to prior formalizations of C(-like)
languages and program partitioning mechnism 
that aim to enforce memory safety, but it also touches on
C-language formalization in general.
%, and work that uses random
%testing to connect a formal development to a reference implementation.


\myparagraph{Formalizing C and Low-level code}
%
A number of prior works have looked at formalizing the semantics of C,
including CompCert~\cite{Blazy2009,leroy:hal-00703441},
\citet{ellison-rosu-2012-popl}, \citet{Kang:2015:FCM:2813885.2738005},
and \citet{10.1145/2980983.2908081, Memarian:2019:ECS:3302515.3290380}. These works also model
pointers as logically coupled with either the bounds of the blocks
they point to, or provenance information from which bounds can be
derived. None of these is directly concerned with enforcing
spatial safety, and that is reflected in the design. For example,
memory itself is not be represented as a flat address space, as in our
model or real machines, so memory corruption due to spatial safety
violations, which Checked C's type system aims to prevent, may not be
expressible. That said, these formalizations consider much more of the
C language than does \lang, since they are interested in the entire
language's behavior.

\myparagraph{Spatially Safe C Formalizations}
%
Several prior works formalize C-language transformations or C-language
dialects aiming to ensure spatial safety. 
%
\citet{10.1145/2813885.2737979} extends the formalization
of \citet{ellison-rosu-2012-popl} to produce a semantics that detects
violations of spatial safety (and other forms of undefinedness). It
uses a CompCert-style memory model, but ``fattens'' logical pointer
representations to facilitate adding side conditions similar to \lang's.
Its concern is bug finding, not compiling programs to
use this semantics.

CCured~\cite{Necula2005} and Softbound~\cite{softbound} implement
spatially safe semantics for normal C via program transformation. Like
\lang, both systems' operational semantics annotate pointers with
their bounds. CCured's equivalent of array pointers are compiled to be
``fat,'' while SoftBound compiles bounds metadata to a separate
hashtable, thus retaining binary compatibility at higher checking
cost. Checked C uses static type information to enable bounds checks
without need of pointer-attached metadata, as we show in
Section~\ref{sec:compilation}. Neither CCured nor Softbound models
null-terminated array pointers, whereas our semantics ensures that
such pointers respect the zero-termination invariant, leveraging
bounds widening to enhance expressiveness.

Cyclone \cite{Jim2002,GrossmanMJHWC02} is a C dialect that aims to
ensure memory safety; its pointer types are similar to
CCured. Cyclone's formalization~\cite{GrossmanMJHWC02} focuses on the
use of \emph{regions} to ensure temporal safety; it does not formalize
arrays or threats to spatial safety. Deputy~\cite{Feng2006,Condit2007}
is another safe-C dialect that aims to avoid fat pointers; it was an
initial inspiration for Checked C's design~\cite{Elliott2018}, though
it provides no specific modeling for null-terminated array
pointers. Deputy's formalization~\cite{Condit2007} defines its
semantics directly in terms of compilation, similar in style to what
we present in Section~\ref{sec:compilation}. Doing so tightly couples
typing, compilation, and semantics, which are treated independently in
\lang. Separating semantics from compilation isolates meaning 
from mechanism, easing understandability. Indeed, it was this
separation that led us to notice the 
limitation with Checked C's handling of bounds widening.

The most closely related work is the
formalization of \checkedc done by \citet{ruef18checkedc-incr} and \citet{li22checkedc}. They
present the type system and semantics of a core model of \checkedc,
mechanized in Coq, and prove a blame theorem. 
The difference between these works and \systemname is listed in \Cref{sec:intros} and \Cref{sec:overview}.

\myparagraph{Program Partitioning Mechanism}
%
The unchecked and checked code region separation in \systemname represents an isolation mechanism to ensure that code executed as part of unchecked regions does not violate the safety guarantees in checked regions,
which is typically called program partitioning~\cite{rul2009towards}, and there has been considerable work~\cite{tan2017principles, brumley2004privtrans, bittau2008wedge, lind2017glamdring, liu2017ptrsplit} in the area. Most of these techniques are~\emph{data-centric}~\cite{lind2017glamdring, liu2017ptrsplit}, wherein program data drives the partitioning. E.g., Given sensitive data in a program, the goal is to partition functions into two parts or partitions based on whether a function can access the sensitive data.
The performance overhead of these approaches is dominated by marshaling costs and depends on the usage of sensitive data.
The overhead of state-of-the-art approaches~\cite{lind2017glamdring, liu2017ptrsplit} is prohibitive and varies from 37\%-163\%.

\citet{rlbox-paper}, a.k.a RLBox, merged a type checker with a sandbox mechanism to better achieve the programming partitioning hechanism.
They allow tainted pointers to be shared among different code regions. There are several differences between their work with \systemname.

\begin{itemize}
\item There is no RLBox formalism.
\item RLBox implements program partition without spatial safety guarantee, while \systemname provides the non-crashing guarantee.
\item RLBox's type system is based on C++ templates, which might contain potential unknown faulty.
\item RLBox only provides program partition for third party library functions, while \systemname can context-switch between checked and unchecked code regions for arbitrary functions.
\item RLBox has no non-exposure guarantee, a checked pointer might leak its pointer address in a untrust third party library function, while \systemname ensures that checked pointers cannot be leaked to unchecked code regions.
\end{itemize}

% \item Lee \textsf{et al.} \cite{Lee:2017:TUB:3140587.3062343} proposed
%   a math model for defining some undefinedness in LLVM. Mainly,  they
%   provided a model in C to support the
%   $\mathtt{inttoptr}$/$\mathtt{ptrtoint}$ casting operations and
%   enabled the correctness proofs of many LLVM optimizations that rely
%   on certain memory provenance model features that the previous
%   CompCert model is not able to provide. \mwh{not related}
% \item CompCert is a verified
%   compiler for C. They do not have a type system for spatial safety. 
% \item Kang \textsf{et al.} \cite{Kang:2015:FCM:2813885.2738005} proposes a C memory model that supports integer-pointer casts.
% \item Ellison and Rosu \cite{ellison-rosu-2012-popl} defined a relatively complete C semantics with a simplified version of CompCert's model. They mainly focus on defining single-threaded front-end syntatic semantics.


% \begin{itemize}
% \item Lee \textsf{et al.} proposed a novel LLVM memory model including a data layout and memory pointer provenance model \cite{Lee:2018:RHO:3288538.3276495}. This model proposes an algorithm to guarantee the pointer provenance property in a high-level math model. Unlike our Check-C model, they do not have a concrete plan to apply spatial safety for the whole system. 
% % \item Memarian \textsf{et al.} \cite{Memarian:2019:ECS:3302515.3290380} provided two pointer provenance models for C/C++ languages and reconciled the ISO C standard. Similar to Lee \textsf{et al.}'s work, Memarian \textsf{et al.} focused on creating better pointer provenance models for C instead of investigating different C instruction behaviors through a concrete abstract machine. Without great effort, it is unclear how to build an abstract machine to support all LLVM IR instructions based on their model.
% \item Li \textsf{et al.} proposes a relatively complete semantics of LLVM \cite{DBLP:conf/ictac/LiG21}. The semantics focuses on defining a complete single-threaded/multi-threaded executable semantics of LLVM based on designing an abstract machine.
% % \item Memarian \textsf{et al.} \cite{10.1145/2980983.2908081} proposes a de Facto model for C to define some undefinedness in C in a relatively small math model.
%  \item Hathhorn \textsf{et al.} \cite{10.1145/2813885.2737979} identifies undefined behaviors in C and come up with a semantics to reject undefined programs.
% % \item Softbound \cite{10.1145/1543135.1542504} proposes a static type checking system for guaranteeing spatial safety, but they do not have a formal framework. They do not deal with null-terminated pointers.
% % \item CETS \cite{10.1145/1837855.1806657} is a static analysis tool for guaranteeing temporal safety, and they have formally defined their operational semantics and proved that the framework works for guaranteeing temporal safety, but not for spatial safety.
% \end{itemize}

% \myparagraph{Formal Language Design}
%  
% \mwh{TODO: lambdaJS, other stuff that does randomized testing? Maybe
%   not much to say here. No more than 1 para}

% systems attempt to overcome the limitations of
% fat pointers by storing the bounds information in a separate metadata
% space~\cite{Nagarakatte2009,mpx} or within unused bits in 64-bit
% pointers~\cite{Kwon:2013:LPC:2508859.2516713} (though this approach is
% unsound~\cite{gil18hole}). These approaches can add substantial
% overhead; e.g., Softbound's overhead for spatial safety checking is
% 67\%.
 
% \myparagraph{Security mitigations}
% There are extensive researches toward addressing the lack of spatial memory safety in C/C++ automatically.
% One approach is to attempt to prevent an attacker from exploiting a memory vulnerability.
% Approaches deployed in practice include stack canaries
% \cite{Steffen1992}, address space layout randomization (ASLR)
% \cite{PaX2003}, data-execution prevention (DEP), and control-flow
% integrity (CFI) \cite{Abadi2005}.
% Purify~\cite{Hastings1992} is a software allowing users to detect potential memory leaks and errors.
% These defenses have led to an escalating
% series of measures and counter-measures by attackers and defenders
% \cite{Szekeres:2013:SEW:2497621.2498101}. These approaches do not
% prevent data modification or data disclosure attacks, and they can be
% defeated by determined attackers who use those attacks. By contrast,
% enforcing spatial memory safety in the semantics level avoids these issues. 

% \myparagraph{Migratory Typing}
% %
% Checked C is closely related to work supporting migratory
% typing~\cite{tobinhochstadt_et_al:LIPIcs:2017:7120} (aka gradual 
% typing~\cite{siek06}). In this scheme, 
% portions of a program for a dynamically typed language
% can be annotated with static
% types. For Checked C, legacy C in the \code{unchecked} blocks
% plays the role of the dynamically typed
% language and checked regions play the role of statically typed
% portions. In migratory typing, one typically proves that a fully
% annotated program is statically type-safe.
% Mixed programs are given a semantics that checks static types at boundary
% crossings~\cite{matthews07}, e.g., calling a statically typed
% function from dynamically typed code might induce a dynamic check, where
% the passed-in argument has the specified type. When a function is
% passed as an argument, this check must be deferred until the function
% is called. The delay prompted research on proving the blame theorem: Even
% if a failure were to occur within static code, it could be blamed on
% bogus values provided by dynamic code~\cite{wadler09}. 
% This semantics is, however, slow~\cite{Takikawa:2016:SGT:2837614.2837630}, so many
% languages opt for what \citet{Greenman:2018:STS:3243631.3236766} term the
% \textbf{erasure semantics}: No checks are added and no notion of blame
% is proved, i.e., failures in statically typed code are not formally
% connected to errors in dynamic code. \checkedc also has erasure
% semantics, but Theorem~\ref{thm:blame} is able to lay the blame theorem with the
% code in \code{unchecked} blocks.

% \myparagraph{Rust}
% %
% Rust~\cite{Matsakis:2014:RL:2663171.2663188} is a programming
% language, like C, that supports zero-cost abstractions, but like
% \checkedc, aims to be safe. Rust programs may have designated
% \textbf{unsafe} blocks in which certain rules are relaxed, potentially
% allowing run-time failures. 
% There are two major differences between \checkedc and Rust/RustBelt~\cite{Jung:2017:RSF:3177123.3158154}.
% The obvious difference is that RustBelt does not have the formal semantics for many operations that \checkedc supports,
% such as the operations for null-terminated pointers and dependent functions.
% In addition, they do not have a subtyping system, like the \checkedc one, so that Rust/RustBelt do not allow many programs that \checkedc allows, such as the example programs in Sec.~\ref{sec:overview}.

% The less obvious difference is related to the unsafe code.
% As with \checkedc, the question is how to
% reason about the safety of a program that contains any amount of
% unsafe code. The RustBelt project~\cite{Jung:2017:RSF:3177123.3158154}
% proposes to use a semantic~\cite{milner78polymorphism}, rather than
% syntactic~\cite{wright94syntactic}, account of soundness, in which (1)
% types are given meaning according to what terms inhabit them; (2) type
% rules are sound when interpreted semantically; and (3) semantic well
% typing implies safe execution. With this approach, unsafe code can be
% (manually) proved to inhabit the semantic interpretation of its type, in which case
% its use by type-checked code will be safe.
% \checkedc can be viewed as complementary to that of RustBelt, perhaps
% constituting the first step in mixed-language safety assurance. In
% particular, we employ a simple, syntactic proof that checked code is
% safe and unchecked code can always be blamed for a failure---no proof
% about any particular unsafe code is required. Stronger assurance that
% programs are safe despite using mixed code could employ the (more involved
% and labor-intensive) RustBelt approach. 

% \ignore{


% Other work that might be relavent but for guaranteeing temporal safety, I have recorded some of them in the related work if they are relavent to spatial safety, but I might missed some.

% Purify operates on binaries, but only ensures the safety of heap-allocated objects.

% Safe C ensures complete temporal safety by using fat-pointers.

% MSCC uses a fat-pointer approach for handling the spatial and temporal safeties regarding stack/heap pointers. According to the experiment in MSCC, it is supposed to be faster than CCured, but it seems that they are in the same level or CCured might be faster in some cases according to.

% Fail-Safe C is an updated version of Safe C, and it maintains complete compatibility with ANSI C but incurs significant runtime overhead.

% Cyclone is a new dialect designed to minimise the changes required to port over C programs. Cyclone’s region-based memory management system provides a foundation for solving the temporal safety issues for stack pointers in Checked-C. The only difference between the Cyclone’s region-based system and the Checked-C will be time when dangling pointers cause errors. Dangling pointers should be banned at the time when they are used to access memory. If a program contains a function call with a stack pointer that is created and returned to the upper level function call and it is never used to access memory, the program should be considered as valid. 


% Here are several ideas of using metadata instead of fat-pointers, I (Liyi) think they are basically another form of fat-pointer approaches:

% Patil and Fischer address the spatial and temporal safeties by maintaining disjoint metadata and performing checks in a separate “shadow process,” but this requires an additional CPU.

% MemSafe provides a framework for checking spatial and temporal memory safety at runtime. 
% Their work is based on transferring an original C program to an updated CFG-based program with extra metadata. These metadata has similar functionality as fat pointer and there are extra
% checks allowing them to place extra checks during the program execution. For each pointer, the updated CFG stores information about all possible values of the pointer through possible aliasing. The efficiency of MemSafe can be as effective as CCured.

% Write integrity testing -- similar to regional memory management. The so-called colored memory space is essentially giving regional flags to different pointers, and then check if these pointers have potential security threats.

% The following approaches require hardware support:

% Clause et al. describe an efficient technique for detecting memory errors, but it requires custom hardware.

% Dhurjati and Adve describe a technique based on the Electric Fence [11] malloc debugger: Their system assigns a unique virtual page to every dynamically allocated object and relies on hardware page protection to detect dangling pointer dereferences.


% There are also few static approaches to the temporal safety in C. However, they are not complete and a lot of them are not even sound. There are four kinds of static approach methods: model checking, abstract interpretation, pattern matching, and pointer analysis. 

% Model Checking. CBMC is a bounded model checker that reasons about all the program paths for C/C++ programs as constraints that can be solved by an SMT solver. When used in finding UAF bugs, CBMC is sound (in a bounded manner) and highly precise but scales only to small programs whose “sizes are restricted” (according to its user manual).

% Thomas Ball and Sriram K. Rajamani developed a model checking framework (C2bp) for checking Temporal Safety Properties in C. The framework allows pointer analysis by using a pointer aliasing algorithm. The problem is that it only deals with small programs, and the pointer aliasing algorithm only detects a small part of aliasing pointers.

% MOPS (MOdel checking Program for Security Properties) makes use of the model checking technique to check for violation of security rules, that are defined as temporal safety properties. As for program representation, MOPS models the program in the form of Push Down Automaton (PDA), that contains all the possible execution paths. Push Down Automata are used as tools to analyze procedural sequential programs, and more specifically those having recursive procedures. As for automata, they are according to Schneider used in the objective of specifying security policies that can be enforced by mechanisms. MOPS makes use of this approach to model security properties in the form of Finite State Automata (FSA), that dictate the order of security-relevant operations sequence. The modularity of security properties was also proposed by MOPS; this approach allows the decomposition of complex security properties into simpler and reusable basic security properties that are easy to model and to extend (such as role based access). MOPS verifies that the security properties are properly respected in all the execution paths of the analyzed program, making use of the model checking on the PDA, and checks if risky states are reachable within the PDA. MOPS does not know that a pointer is an alias of another, so the property is not sound for this program.

% Abstract Interpretation. Clang Static Analyzer is an abstract interpreter for analyzing C/C++ programs. It adopts a highly unsound model by analyzing only a small subset of the program paths in order to achieve scalability and precision. To scale for large codebases with few false alarms, Clang limits its UAF-bug-finding ability by performing an intraprocedural analysis (with inlining). In general, such tools refrain from reporting too many false alarms, but at the expense of missing many UAF bugs.

% Pattern Matching. Coccinelle is a pattern-based tool for analyzing and certifying C programs. Coccinelle can find UAF bugs based on some patterns given. Due to the lack of the points-to information, Coccinelle can be both fairly unsound and imprecise but is highly scalable (due to its pattern-matching nature).

% Pointer Analysis. Supa is a state-of-the-art demand-driven pointer analysis that is field-, flow- and context-sensitive but path- insensitive for C programs. When used in finding UAF bugs, Supa can be regarded as reasoning about all the program paths with an extremely coarse abstraction, in order to achieve soundness and scalability.

% Yan et al. develops a Spatio-Temporal context reduction technique for  doing pointer-analysis for detecting use-after-free vulnerabilities, which are essentially temporal safety. They mainly focus on the heap pointer allocation-free detection to  check if a pointer is used after the free function call for the memory field the pointer points to. The technique is based on static analysis. The main idea is still based on intraprocedural control flow analysis. It relies on context sensitive reduction technique to reduce the search function calling path spaces for analyzing pointer aliasing.  The context sensitivity and their pointer analysis algorithm is applied in a state-of-the-art manner. The information for the algorithm of search pointer aliasing reduces the calling context path, while the context path information enhances the efficiency of the algorithm.  However, the technique is not complete. It will ban a large number of valid C programs due to false use-after-free vulnerabilities.

% It seems that they might not detect the use-after-free error in this case. In addition, the work does not handle stack pointer use-after-free vulnerabilities.

% ASAP: As Static As Possible memory management -- use data flow analysis, garbage collection and region memory management tools to ensure the security safety of C programs by static analysis. In order to guarantee the use-after-free, it requires the program written in a special way: in a linearly typed program, each value must be used exactly once. It does not include examples of dealing with stack pointers. It shows that in a pointer management world, a C program temporal safety cannot be fully analyzed by only static analysis.

% Jia-Ju Bai et al. develops a static method to detect use-after-free bugs in Linux device drivers. It is a UAF bug in the concurrent system. The basic algorithm is simple. It first goes through each single-threaded code to collect a read/write set storing point-to information. The information collection is through traditional intraprocedural CFG analysis. Then, it uses a trace analysis tool to discover UAF bugs. This is achievable only in dealing with linux device drivers, since usually, these drivers are simple and the concurrency is very formulated.  I believe if we apply the same process in dealing with C, the method will not be sound.

% Industrial tools:

% There are also some industrial level tools for static checking bugs in C. However, they are either not able to check UAF bugs or they only use heuristics to check very small sets of UAF bugs. They include CppCheck, Clang Core, Clang Alpha, CodeSonar, Facebook Infer,  Splint standard, Splint weak. For example, CodeSonar state-based checkers do not check temporal safety. They require insertion of extra code in order for the tool to check if a supposed deadcode will be reached or not. Coverity is a dataflow analysis tool that relies on inter-procedural analysis techniques. The analysis is neither sound nor complete, that is, there may be both defects which are not reported and there may be false alarms. Coverity Prevent implements an incremental analysis which means that the system automatically infers what parts of the source code that have to be re-analyzed after the code has been modified. This typically reduces the analysis time substantially, but may of course imply a complete re-analysis in the worst case. Klocwork does similar things to Coverity and Coverity Prevent.
% }


\section{Conclusion and Future Work}
\label{sec:conclude}

This paper presented \CoreChkC, a formalization of an extended core of
the Checked C language which aims to provide spatial memory
safety. \CoreChkC models dynamically sized and
null-terminated arrays with dependently typed bounds that can
additionally be widened at runtime. We prove, in Coq, the key safety
property of Checked C for our formalization, {\em blame}: if a mix of
checked and unchecked code gives rise to a spatial memory safety
violation, then this violation originated in an unchecked part of the
code. We also show how programs written in \CoreChkC (whose
semantics leverage fat pointers) can be compiled to \elang (which does
not) while preserving their behavior. We developed a version of \lang
written in PLT Redex, and used a custom term generator in conjunction with Redex's
randomized testing framework to give confidence that compilation is
correct. We also used this framework to cross-check \lang
against the \checkedc compiler, finding multiple inconsistencies
in the process. 

As future work, we wish to extend \CoreChkC to model more of Checked
C, with our Redex-based testing framework guiding the process. The
most interesting Checked C feature not yet modeled is \emph{interop
  types} (itypes), which are used to simplify interactions with
unchecked code via function calls. A function whose
parameters are itypes can be passed checked or unchecked pointers
depending on whether the caller is in a checked region. This feature
allows for a more modular C-to-Checked C porting process, but complicates reasoning
about blame. A more ambitious next step would be to extend an existing
formally verified framework for C, such as CompCert~\cite{compcert} or VeLLVM
\cite{Zhao:2012:FLI:2103621.2103709}, with Checked C features, towards
producing a verified-correct Checked C compiler. We believe that
\lang's Coq and Redex models lay the foundation for such a step, but
substantial engineering work remains.


%\bigskip

%\paragraph*{Acknowledgments}
%We thank the anonymous reviewers for their helpful, constructive
%comments. This work was supported in part by a gift from Microsoft. 

% \ignore{
% %% Acknowledgments
% \begin{acks}                            %% acks environment is optional
%                                         %% contents suppressed with 'anonymous'
%   %% Commands \grantsponsor{<sponsorID>}{<name>}{<url>} and
%   %% \grantnum[<url>]{<sponsorID>}{<number>} should be used to
%   %% acknowledge financial support and will be used by metadata
%   %% extraction tools.
%   This material is based upon work supported by the
%   \grantsponsor{GS100000001}{National Science
%     Foundation}{http://dx.doi.org/10.13039/100000001} under Grant
%   No.~\grantnum{GS100000001}{nnnnnnn} and Grant
%   No.~\grantnum{GS100000001}{mmmmmmm}.  Any opinions, findings, and
%   conclusions or recommendations expressed in this material are those
%   of the author and do not necessarily reflect the views of the
%   National Science Foundation.
% \end{acks}
% }

%% Bibliography
%\bibliography{bibfile}


%% Appendix
%%\appendix

%%Text of appendix \ldots
\begin{small}
%\balance
\bibliographystyle{ACM-Reference-Format}
\bibliography{IEEEabrv,paper,sources}
%%%%%%%%%%%%%%%%%%% Appendix %%%%%%%%%%%%%%%%%%%%%%%%%%%%%%%%%%%%
\end{small}

\iftr
\newpage
\appendix
\section{Appendix}\label{app:main}

\subsection{Converting C to \checkedc}
\label{app:convertctocc}
%
The safety guarantees of Checked C come with certain restrictions. For instance,
as shown below, Checked C programs cannot use address-taken variables in a
bounds expression as the bounds relations may not hold because of possible
modifications through pointers.
% 
\begin{minted}[xleftmargin=30pt, mathescape, escapeinside=||, fontsize=\footnotesize]{c}
...
_Array_ptr<int> p : count (n) = NULL;
|\textcolor{red}{\faTimes}|..,&n,.
\end{minted}
% 
Consequently, converting existing C programs to Checked C might require
refactoring,~\eg eliminate~\inlinecode{&n} from the program above without
changing its functionality.
% 
This might require considerable effort~\cite{duanrefactoring} depending on the
program's complexity.
% 
Recently, Machiry~\etal developed~\threec~\cite{machiry2022c} that tries to
automatically convert a program to Checked C by adding appropriate pointer
annotations.
However, as described in \threec, completely automated conversion
is \emph{infeasible}, and it requires the developer to convert some code regions
manually. 


\subsection{Well-formedness and Subtype}
\label{app:le}
  

\begin{figure}[h]
{\small
  \begin{mathpar}

  \inferrule[]
  {}
  {m \vdash \tint}

  \inferrule[]
  {\xi \wedge m\vdash \tau \\ \xi \le m}
  {m \vdash \tptr{\tallarrayb{\bvar}{\tau}}{\xi}}

  \inferrule[]
  {\xi \wedge m \vdash \tau\\ \xi \le m}
  {m \vdash \tptr{\tau}{\xi}}

  \inferrule[]
  {\xi \wedge m \vdash \tau\\ \xi \le m \\\\ \fv(\overline{\tau})\cup\fv(\tau)\subseteq \overline{x}}
  {m \vdash \tptr{(\tfun{\overline{x}}{\overline{\tau}}{\tau}}{\xi})}
  \end{mathpar}
}
{\footnotesize
\[
\begin{array}{l} 
\tmode \wedge \cmode = \umode \qquad \xi \wedge \umode = \umode
\qquad \cmode \wedge m = m 
\qquad  m_1 \wedge m_2 = m_2 \wedge m_1
\end{array}
\]
}
 \caption{Well-formedness for Nested Pointers}
\label{fig:wftypes}
\end{figure}

\begin{figure}[h]
{\small
  \begin{mathpar}

    \inferrule[]
    {}
    {\Gamma \vdash n}

    \inferrule[]
    {x:\tint \in \Gamma}
    {\Gamma \vdash x + n}

    \inferrule[]
    {\Gamma \vdash b_l\\
    \Gamma \vdash b_h}
  {\Gamma \vdash (b_l,b_h)}

  \inferrule[]
  {}
  {\Gamma \vdash \tint}

  \inferrule[]
  {\Gamma \vdash \bvar \\
  \Gamma \vdash \tau}
  {\Gamma \vdash \tptr{\tallarrayb{\bvar}{\tau}}{m}}

  \inferrule[]
  {\Gamma \vdash \tau}
  {\Gamma \vdash \tptr{\tau}{m}}

  \inferrule[]
  {T \in D}
  {\Gamma \vdash \tptr{\tstruct{T}}{m}}

  \inferrule[]
  {\Gamma \vdash \tau}
  {\Gamma \vdash \tptr{\tau}{m}}

  \inferrule[]
  {\forall \tau_i\in \overline{\tau}\,.\,\Gamma[\forall x\in\overline{x}\,.\, x\mapsto \tint] \vdash \tau_i\\\\
   \Gamma[\forall x\in\overline{x}\,.\, x\mapsto \tint] \vdash \tau }
  {\Gamma \vdash \tfun{\overline{x}}{\overline{\tau}}{\tau}}
  \end{mathpar}
}
 \caption{Well-formedness for Types and Bounds}
\label{fig:wftypesandbounds}
\end{figure}

\Cref{fig:wftypes} defines the well-formedness for nested pointers, guaranteeing that no tainted ($\tmode$) pointer has a checked ($\cmode$) element field.
Mainly, in a nested pointer $\tptr{(... \tptr{\tau}{\xi_2} ...)}{\xi_1}$, $\xi_2\le \xi_1$.
\Cref{fig:wftypesandbounds} defines the well-formedness for bounds appearing in a type, which requires that any bound must be $\tint$ type and has instance in $\Gamma$.
All bound variables appearing in a type must be either an instance of the type environment $\Gamma$,
or a bound variable appearing in a function pointer.

\begin{DIFnomarkup}
\begin{figure}
{\small
\[\hspace*{-1.2em}
\begin{array}{l}
\textcolor{blue}{\text{Bound Inequality and Equality:}}\\[0.3em]
  \begin{array}{r@{~}c@{~}l@{~}c@{~}l}
     n \le n' &\Rightarrow& n &\le_{\Theta} & n'\\
     n \le n' &\Rightarrow& x+n &\le_{\Theta} & x+n'\\
     n \le n' \wedge \Theta(x)=\tgez &\Rightarrow& n &\le_{\Theta} & x+n'\\
     \Theta(x)=\teq{b} \wedge b+n\le_{\Theta}b'  &\Rightarrow& x+n & \le_{\Theta} & b'\\
     \Theta(x)=\teq{b}\wedge b'\le_{\Theta}b+n  &\Rightarrow& b' & \le_{\Theta} & x+n\\
     b \le_{\Theta} b' \wedge b' \le_{\Theta} b  &\Rightarrow& b & =_{\Theta} & b'
    \end{array}
  \\[1.5em]
\textcolor{blue}{\text{Type Equility:}}\\
  \begin{array}{r@{~}c@{~}l@{~}c@{~}l}
     && \tint & =_{\Theta} & \tint\\
     \omega =_{\Theta} \omega' &\Rightarrow& \tptr{\omega}{\xi} & =_{\Theta} & \tptr{\omega'}{\xi}\\
     \bvar =_{\Theta} \bvar' \wedge  \tau =_{\Theta} \tau'
             &\Rightarrow& \tallarrayb{\bvar}{\tau} & =_{\Theta} & \tallarrayb{\bvar'}{\tau'}\\

    \textit{cond}(\overline{x},\overline{\tau}\to\tau,\overline{y},\overline{\tau'}\to\tau')

 &\Rightarrow& \tfun{\overline{x}}{\overline{\tau}}{\tau} & 
                         =_{\Theta} & \tfun{\overline{y}}{\overline{\tau'}}{\tau'}\\
    \end{array}
  \\[0.7em]
\textcolor{blue}{\text{Subtype:}}\\[0.3em]

  \begin{array}{r@{~}c@{~}l@{~}c@{~}l}
    \tau =_{\Theta} \tau'&\Rightarrow&\tau &\sqsubseteq_{\Theta}& \tau'\\[0.2em]

    && \tptr{\tau}{\tmode}&\sqsubseteq_{\Theta}& \tptr{\tau}{\umode}\\[0.2em]

    0\le_{\Theta} b_l \wedge b_h \le_{\Theta} 1 &\Rightarrow& \tptr{\tau}{m}&\sqsubseteq_{\Theta}& \tarrayptr{b_l}{b_h}{\tau}{m}\\[0.2em]
    b_l \le_{\Theta} 0 \wedge 1 \le_{\Theta} b_h &\Rightarrow& \tarrayptr{b_l}{b_h}{\tau}{m} &\sqsubseteq_{\Theta}& \tptr{\tau}{m}\\[0.2em]
    b_l \le_{\Theta} 0 \wedge 1 \le_{\Theta} b_h &\Rightarrow& \tntarrayptr{b_l}{b_h}{\tau}{m} &\sqsubseteq_{\Theta}& \tptr{\tau}{m}\\[0.2em]
    %% b_l \le b_l' \wedge b_h' \le b_h &\Rightarrow&  \tarrayptr{b_l}{b_h}{\tau}{m} &\sqsubseteq&  \tarrayptr{b_l'}{b_h'}{\tau}{m}\\[0.6em]
    b_l \le_{\Theta} b_l' \wedge b_h' \le_{\Theta} b_h &\Rightarrow& \tntarrayptr{b_l}{b_h}{\tau}{m} &\sqsubseteq_{\Theta}& \tarrayptr{b_l'}{b_h'}{\tau}{m}\\[0.6em]
    b_l \le_{\Theta} b_l' \wedge b_h' \le_{\Theta} b_h &\Rightarrow& \tallarrayptr{b_l}{b_h}{\tau}{m} &\sqsubseteq_{\Theta}& \tallarrayptr{b_l'}{b_h'}{\tau}{m}
\\[0.2em]
\overline{\tau'}\sqsubseteq_{\Theta}\overline{\tau}\wedge \tau\sqsubseteq_{\Theta}\tau' &\Rightarrow& \tptr{\tfun{\overline{x}}{\overline{\tau}}{\tau}}{\xi} &\sqsubseteq_{\Theta}& \tptr{\tfun{\overline{x}}{\overline{\tau'}}{\tau'}}{\xi}

    \end{array}
\end{array}
  \]
}
{\footnotesize
\[
\begin{array}{l}
n'+n = add(n',n)
\qquad
(x+n')+n = x+add(n',n)\\
\textit{cond}(\overline{x},\tau,\overline{y},\tau')
=\exists\overline{z}\;.\;\overline{x}\cupdot\overline{z}
  \wedge \overline{y}\cupdot\overline{z}
  \wedge \size(\overline{x})=\size(\overline{y})=\size(\overline{z})
\\\qquad\qquad\qquad\qquad\qquad
  \wedge \tau[\overline{z}/\overline{x}]= \tau'[\overline{z}/\overline{x}]
\end{array}
\]
}
  \caption{Type Equality and Subtyping}
  \label{fig:checkc-subtype}
\end{figure}
\end{DIFnomarkup}


In \lang, type equality $\tau=_{\Theta}\tau'$
is a type construct equivalent relation defined by the bound equality ($=_{\Theta}$) in (NT-)array pointer types
and the alpha equivalence of two function types in \Cref{fig:checkc-subtype};
i.e., two (NT-)array pointer types $\tallarrayb{\bvar}{\tau} $ and $ \tallarrayb{\bvar'}{\tau'}$ are equivalent, if 
$\bvar =_{\Theta} \bvar'$ and $\tau=_{\Theta}\tau'$; two function types 
$\tfun{\overline{x}}{\overline{\tau}}{\tau} $ and $ \tfun{\overline{y}}{\overline{\tau'}}{\tau'}$
are equivalent, if we can find a same length (as $\overline{x}$ and $\overline{y}$) variable list $\overline{z}$ that is substituted for $\overline{x}$ and $\overline{y}$ in $\overline{\tau} \to {\tau}$ and $\overline{\tau'} \to {\tau'}$, resp.,
and the substitution results are equal.

 The \textsc{T-CastPtr} rule
permits casting from an expression of type $\tau'$ to a checked pointer when
$\tau' \sqsubseteq \tptr{\tau}{\cmode}$. This subtyping relation
$\sqsubseteq$ is given in Fig.~\ref{fig:checkc-subtype} and is built on the type equality
($\tau =_{\Theta} \tau'\Rightarrow\tau \sqsubseteq_{\Theta} \tau'$). The many
rules ensure the relation is transitive. Most of the rules manage
casting between array pointer types. The rule 
($0\le b_l \wedge b_h \le 1 \Rightarrow \tptr{\tau}{m}\sqsubseteq
\tarrayptr{b_l}{b_h}{\tau}{m}$) permits treating a singleton
pointer as an array pointer with $b_h\le 1$ and $0 \le b_l$.
Two function pointer types are subtyped ($\tptr{\tfun{\overline{x}}{\overline{\tau}}{\tau}}{\xi} \sqsubseteq_{\Theta} \tptr{\tfun{\overline{x}}{\overline{\tau'}}{\tau'}}{\xi}$), 
if the output type are subtyped ($\tau\sqsubseteq_{\Theta}\tau'$) and the argument types are reversely subtyped ($\overline{\tau'}\sqsubseteq_{\Theta}\overline{\tau}$).
%There is another casting rule in \Cref{app:main} stating that
% users are free to cast types in unchecked code regions, since unchecked regions can contain C code.


The subtyping relation given in Fig.~\ref{fig:checkc-subtype} involves
dependent bounds, i.e., bounds that may refer to variables. To decide
premises $b \leq_{\Theta} b'$ in \Cref{fig:checkc-subtype}, we need a decision procedure that accounts for
the possible values of these variables. This process considers
$\Theta$, tracked by the typing judgment, and $\varphi$, the current
stack snapshot (when performing subtyping as part of the type
preservation proof).

Since bounds expressions may
contain variables, determining assumptions like $b_l \leq_{\Theta} b_l'$
requires reasoning about the probable values of these variables'. The type
system uses $\Theta$ to make such reasoning more precise.
$\Theta$ is a map from variables $x$ to
equation predicates $P$, which have the form $P ::= \tgez \;|\; \teq{b}$.
It maps variables to equations that are recorded along the type checking procedure.
If $\Theta$ maps $x$ to $\tgez$, that means that $x \ge 0$;
$\teq{b}$ means that $x$ is equivalent to the bound value $b$ in the current context, 
such as in the type judgment for $e_2$ in Rule \textsc{T-LetInt} and \textsc{T-RetInt}.

$\sqsubseteq$ is parameterized by
$\Theta$, which provides the range of allowed values for a bound
variable; thus, more $\sqsubseteq$ relation is provable. For example,
rule \rulelab{T-LetInt} inserts a predicate $\teq{e_1}$ for variable $x$.
Assume that $e_1$ is equal to $0$,
when $x$ is used as a type variable in type $\tntarrayptr{0}{x}{\tint}{\cmode}$ in $e_2$,
the subtyping relation $\tntarrayptr{0}{x}{\tint}{\cmode} \sqsubseteq
\tntarrayptr{0}{0}{\tint}{\cmode}$ is provable when we know
\code{x}$\teq{0}$.

To capture bound variables in dependent types, the \checkedc subtyping
relation ($\sqsubseteq$) is parameterized by a restricted stack
snapshot $\varphi|_{\rho}$ and the predicate map $\Theta$, where
$\varphi$ is a stack and $\rho$ is a set of
variables. $\varphi|_{\rho}$ means to restrict the domain of $\varphi$
to the variable set $\rho$. Clearly, we have the relation:
$\varphi|_{\rho} \subseteq \varphi$. $\sqsubseteq$
being parameterized by $\varphi|_{\rho}$ refers to that when we
compare two bounds $b \le_{\Theta} b'$, we actually do
$\varphi|_{\rho}(b) \le_{\Theta} \varphi|_{\rho}(b')$ by interpreting the
variables in $b$ and $b'$ with possible values in $\varphi|_{\rho}$.
Let's define a subset relation $\preceq$ for two restricted stack
snapshot $\varphi|_{\rho}$ and $\varphi'|_{\rho}$:


\begin{DIFnomarkup}
\begin{figure*}[t]
{\small
  \begin{mathpar}
   \inferrule[T-ConstU]
       { \neg \cmode(\tau)}
       {\Gamma;\Theta\vdash_u \evalue{n}{\tau} : \tau}
\quad
   \inferrule[T-ConstC]
       {\Theta;\heap;\emptyset \vdash_c n : \tau}
       {\Gamma;\Theta\vdash_c \evalue{n}{\tau} : \tau}
\quad      
   \inferrule[T-Let]
    { x\not\in \fv(\tau') \\
        \Gamma;\Theta \vdash_m e_1 : \tau \\\\
          \Gamma[x\mapsto \tau];\Theta \vdash_m e_2 : \tau'
             }
    {\Gamma;\Theta \vdash_m \elet{x}{e_1}{e_2} : \tau'}
\quad
   \inferrule[T-LetInt]
    { x\in \fv(\tau') \Rightarrow e_1 \in \text{Bound} \\
        \Gamma;\Theta \vdash_m e_1 : \tint \\\\
           \Gamma[x\mapsto \tint];\Theta[x\mapsto \teq{e_1}] \vdash_m e_2 : \tau'
             }
    {\Gamma;\Theta \vdash_m \elet{x}{e_1}{e_2} : \tau'[e_1 / x]}
\quad
   \inferrule[T-RetInt]
    { \Gamma[x\mapsto \tint];\Theta[x\mapsto \teq{n}] \vdash_m e : \tau}
    {\Gamma;\Theta \vdash_m \eret{x}{\evalue{n}{\tint}}{e} : \tau}

    \inferrule[T-Mac]
              {\xi\le m}
              {\Gamma; \Theta \vdash_m \emalloc{\xi}{\omega} : \tptr{\omega}{\xi}}

    \inferrule[T-Add]
              {\Gamma; \Theta \vdash_m e_1 : \tint \\
                \Gamma; \Theta \vdash_m e_2 : \tint}
              {\Gamma; \Theta \vdash_m (e_1 \plus e_2) : \tint }

    \inferrule[T-Ind] 
              {\Gamma; \Theta \vdash_m e_1 : \tptr{\tallarrayb{\bvar}{\tau}}{\xi} \\
                \Gamma; \Theta \vdash_m e_2 : \tint \\
                \xi \leq m}              
              {\Gamma; \Theta \vdash_m \estar{(\ebinop{e_1}{e_2})} : \tau}

    \inferrule[T-Assign]
              {\Gamma; \Theta \vdash_m e_1 : \tptr{\tau}{\xi} \\
                \Gamma; \Theta \vdash_m e_2 : \tau' \\\\
                \tau'\sqsubseteq_{\Theta} \tau \\
                \xi \leq m}
              {\Gamma; \Theta \vdash_m \eassign{e_1}{e_2} : \tau}

    \inferrule[T-AssignArr]
              {\Gamma; \Theta \vdash_m e_1 : \tptr{\tallarrayb{\bvar}{\tau}}{\xi}\\\\
                \Gamma; \Theta \vdash_m e_2 : \tau' \\
                \tau'\sqsubseteq_{\Theta} \tau\\
                \xi \leq m}
              {\Gamma; \Theta \vdash_m \eassign{e_1}{e_2} : \tau}              

   \inferrule[T-IndAssign]
              {\Gamma; \Theta \vdash_m e_1 : \tptr{\tallarrayb{\bvar}{\tau}}{\xi}\\
                \Gamma; \Theta \vdash_m e_2 : \tint \\\\
                \Gamma; \Theta \vdash_m e_3 : \tau' \\
                \tau'\sqsubseteq_{\Theta} \tau \\
                \xi \leq m}
              {\Gamma; \defscope \vdash_m \eassign{(e_1 \plus e_2)}{e_3} : \tau}


  \end{mathpar}
}
% {\footnotesize
% \begin{center}
% $
% \begin{array}{l}
% \fm(e)\triangleq(\exists x\; n\; \tau. e=x+\evalue{n}{\tau}) \vee (\exists n\;\tau. e = \evalue{n}{\tau})
% \\[0.2em]
% \tau[\overline{e} / \overline{x}]\texttt{(with types }\evalue{\overline{x}}{\overline{\tau}}\texttt{)}\triangleq \forall e_i\in\overline{e}\;x_i\in\overline{x}\;\tau_i\in\overline{\tau}\;.\;\tau_i = \tint \wedge (x_i \in \fv(\tau) \Rightarrow \fm(e_i)) \Rightarrow \tau[e_i / x_i]
% \end{array}
% $
% \end{center}
% }
\caption{Remaining \lang Type Rules (extends Fig.~\ref{fig:type-system-1})}
\label{fig:rem-type-system}
\end{figure*}
\end{DIFnomarkup}

\begin{defi}[Subset of Stack Snapshots]
  Given two $\varphi|_{\rho}$ and $\varphi'|_{\rho}$,
  $\varphi|_{\rho} \preceq \varphi'|_{\rho}$, iff for $x\in\rho$ and
  $y$,
  $(x,y) \in \varphi|_{\rho} \Rightarrow (x,y) \in \varphi'|_{\rho}$.
\end{defi}

For every two restricted stack snapshots $\varphi|_{\rho}$ and
$\varphi'|_{\rho}$, such that
$\varphi|_{\rho} \preceq \varphi'|_{\rho}$, we have the following
theorem in \checkedc (proved in Coq):

\begin{thm}[Stack Snapshot Theorem]
  Given two types $\tau$ and $\tau'$, two restricted stack snapshots
  $\varphi|_{\rho}$ and $\varphi'|_{\rho}$, if
  $\varphi|_{\rho}\preceq \varphi'|_{\rho}$, and
  $\tau \sqsubseteq \tau'$ under the parameterization of
  $\varphi|_{\rho}$, then $\tau \sqsubseteq \tau'$ under the
  parameterization of $\varphi'|_{\rho}$.
\end{thm}

Clearly, for every $\varphi|_{\rho}$, we have
$\emptyset \preceq \varphi|_{\rho}$. The type checking stage is a
compile-time process, so $\varphi|_{\rho}$
is $\emptyset$ at the type checking stage. Stack snapshots are needed
for proving type preserving, as variables in bounds expressions are
evaluated away.


\subsection{Other Type Rules and Literal Validity Checks}\label{rem-type}

Here we show the type rules for other \lang operations in Fig.~\ref{fig:rem-type-system}.
Rule \textsc{T-Mac} deals with
$\emalloctext$ operations. There is a well-formedness check to require
that the possible bound variables in $\omega$ must be in the domain of
$\Gamma$ (see Fig.~\ref{fig:wftypesandbounds}). This is similar to the well-formedness assumption of the type environment (Definition~\ref{type-wellformed}). Rule \textsc{T-Add} deals with binary operations whose sub-terms are integer expressions.
The other rules are explained as follows.

\myparagraph{Pointer Access}
%
The \textsc{T-AssignArr} rule examines array assignment operations, returning the type of
pointed-to objects. Rules for pointers for other object types are
similar, such that Rule \textsc{T-Assign} assigns a value to a non-array pointer location.
The condition $\xi\le m$ ensures that checked and unchecked pointers 
can only be dereferenced in checked and unchecked regions, respectively;
The type rules do not attempt to reason whether the access is in bounds;
such check is deferred to the semantics.
Rule \textsc{T-Ind} serves the case for pointer arithmetic. For simplicity, in the \checkedc formalization, we do not allow arbitrary pointer arithmetic. The only pointer arithmetic operations allowed are the forms shown in rules \textsc{T-Ind} and \textsc{T-IndAssign} in Fig.~\ref{fig:rem-type-system}.  The predicate $\tau'\sqsubseteq_{\Theta} \tau$ requires that the value being assigned is a subtype of the pointer type.
The \textsc{T-IndAssign} rule is an extended assignment operation for handling assignments for array/NT-array pointers with pointer arithmetic.

% Subtyping and casting operations are briefly introduced in
% Sec.~\ref{sec:intros}~and~\ref{sec:overview}.  Subtyping is useful in
% static casting operations that allow users to view a pointer in one
% type as another, such as casting an NT-array pointer to an array
% one. \checkedc provides a set of safe static casting operations that
% have no cost in execution.  Moreover, subtyping acts as oracles for
% bound widening and dynamic casting operations; thus, \checkedc is
% different from a complete static array pointer bound system.  For
% example, if $e$ has type $\tau'$ and $\varphi$ is the current stack
% snapshot, the semantics of $\edyncast{\tau}{e}$ does not transition to
% an error state when $\varphi(\tau')\sqsubseteq\varphi(\tau)$.  In a
% function call, for every argument, \lang permits users to input a
% subtype entity and we prove that this does not affect the correctness
% of the program.

\begin{DIFnomarkup}
 \begin{figure}[t]
 {\small

 \begin{mathpar}
   \inferrule
       {}
       {\Theta;\heap;\sigma \vdash_m n : \tint}

   \inferrule
       {}
       {\Theta;\heap;\sigma \vdash_m 0 : \tptr{\omega}{\xi}}

   \inferrule
       {(m = \cmode \Rightarrow \xi \neq \cmode) \\\\ (m=\umode \Rightarrow \xi = \umode)}
       {\Theta;\heap;\sigma \vdash_{\cmode} n : \tptr{\omega}{\tmode}}
  
   \inferrule
       {(\evalue{n}{\tptr{\omega}{\xi}})\in \sigma}
       {\Theta;\heap;\sigma \vdash_m n : \tptr{\omega}{\xi}}


   \inferrule
       {\tptr{\omega'}{\xi'} \sqsubseteq_{\Theta} \tptr{\omega}{\xi} 
            \\ \Theta;\heap;\sigma \vdash_m n : \tptr{\omega'}{\xi'}}
       {\Theta;\heap;\sigma \vdash_m n : \tptr{\omega}{\xi}}

   \inferrule
       { \xi \le m 
     \\\Xi(m,n)=\tau\;(\evalue{\overline{x'}}{\overline{\tau}})\;(\xi,e)
       \\  \overline{x} = \{x|(x:\tint) \in (\overline{x'}:\overline{\tau}) \}}
       {\Theta;\heap;\sigma \vdash_m n : \tptr{(\tfun{\overline{x}}{\overline{\tau}}{\tau})}{\xi}}
  
   \inferrule
       {\neg\funptr(\omega)\\ \xi \le m\\
        \forall i \in [0,\size(\omega)) \;.\;
            \Theta;\heap;(\sigma \cup \{(n:\tptr{\omega}{\xi})) \}\vdash_m \heap(m,n+i)}
       {\Theta;\heap;\sigma \vdash_m n : \tptr{\omega}{\xi}}
 \end{mathpar}
 }
{\footnotesize
\[
\begin{array}{l} 
\funptr(\tfun{\overline{x}}{\overline{\tau}}{\tau}) = \texttt{true}
\qquad
\funptr(\omega) = \texttt{false}\;\;{[\emph{owise}]}
\end{array}
\]
}
 \caption{Verification/Type Rules for Constants}
 \label{fig:const-type}
 \end{figure}
\end{DIFnomarkup}

\myparagraph{Literal Constant Validity}
Rules \textsc{T-ConstU} and \textsc{T-ConstC}
describes type assumptions for literals appearing in a program.
$\cmode(\tau)$ judges that a literal pointer 
in an unchecked region cannot be of a checked type,
which represents an assumption that programmers 
cannot guess a checked pointer address and utilize it in an unchecked region in \systemname.
In rule \textsc{T-ConstC}, we requires a static 
verification procedure for validating a literal pointer, 
which is similar to the dynamic verification process in \Cref{sec:typechecking}. 

The verification process $\Theta;\heap;\sigma \vdash_m n : \tau$ checks (\Cref{fig:const-type})
validate the literal $\evalue{n}{\tau}$, 
where $\heap(m)$ is the initial heap that the literal resides on and
$\sigma$ is a set of literals assumed to be checked.
A global function store $\Xi(m)$ is also required to check the validity of a function pointer.
A valid function pointer should appear in the right store region ($\cmode$ or $\umode$)
and the address stores a function with the right type.
The last rule in \Cref{fig:const-type} describes the validity check for a non-function pointer, 
where every element in the pointer range ($[0,\size(\omega))$) should be well
typed.


\begin{figure*}[t]
{\small
\begin{mathpar}

\inferrule [S-Var]{} {(\varphi,\heap,x)\longrightarrow (\varphi,\heap,\varphi(x))}

    \inferrule[S-DefArrayC]{\heap(\cmode,n)=\evalue{n_a}{\tau_a} \\ 0 \in [n_l,n_h)}
    {(\varphi,\heap,\estar{\evalue{n}{\tntarrayptr{n_l}{n_h}{\tau}{\cmode}}}) \longrightarrow (\varphi,\heap,\evalue{n_a}{\tau})}

    \inferrule[S-DefArrayT]{\heap(\umode,n)=\evalue{n_a}{\tau_a} \\ 0 \in [n_l,n_h) 
               \\  \emptyset;\heap ; \emptyset \vdash_{\umode}\evalue{n_a}{\tau}}
    {(\varphi,\heap,\estar{\evalue{n}{\tntarrayptr{n_l}{n_h}{\tau}{\tmode}}}) \longrightarrow (\varphi,\heap,\evalue{n_a}{\tau})}

    \inferrule[S-DefArrayBound]{0 \not\in [n_l,n_h)}
     { (\varphi,\heap,\estar{\evalue{n}{\tallarrayptr{n_l}{n_h}{\tau}{\xi}}}) \longrightarrow (\varphi,\heap,\ebounds)}
\quad
    \inferrule[S-DefNTArrayBound]{0 \notin [n_l,n_h]}
    {(\varphi,\heap,\estar{\evalue{n}{\tntarrayptr{n_l}{n_h}{\tau}{\xi}}}) \longrightarrow (\varphi,\heap,\ebounds)}
\quad
    \inferrule[S-Checked]{}{(\varphi,\heap,\echecked{\overline{x}}{\evalue{n}{\tau}}) \longrightarrow (\varphi,\heap,\evalue{n}{\tau})}

    \inferrule[S-RetEnd]{}{(\varphi,\heap,\ret{x}{\evalue{n}{\tau}}{\evalue{n'}{\tau'}}) \longrightarrow (\varphi,\heap,\evalue{n'}{\tau'})}

        \inferrule[S-Let]{}{(\varphi,\heap,\elet{x}{\evalue{n}{\tau}}{e}) \longrightarrow (\varphi,\heap,\ret{x}{\evalue{n}{\tau}}{e})}

    \inferrule[S-AssignNull]{}
      {(\varphi,\heap,\eassign{\evalue{0}{\tptr{\omega}{\xi}}}{\evalue{n_1}{\tau_1}}) \longrightarrow (\varphi,\heap,\enull)}

    \inferrule[S-RetCon]{ (\varphi[x\mapsto \evalue{n}{\tau}],\heap,e) \longrightarrow (\varphi',\heap',e')}{(\varphi,\heap,\ret{x}{\evalue{n}{\tau}}{e}) \longrightarrow (\varphi'[x\mapsto \varphi(x)],\heap',\ret{x}{\varphi'(x)}{e'})}

    \inferrule[S-AssignArrC]{\heap(\cmode,n)=\evalue{n_a}{\tau_a}\\ 0 \in [n_l,n_h) }
      {(\varphi,\heap,\eassign{\evalue{n}{\tallarrayptr{n_l}{n_h}{\tau}{\cmode}}}{\evalue{n_1}{\tau_1}}) \longrightarrow (\varphi,\heapup{\cmode}{n}{\evalue{n_1}{\tau_a}},\evalue{n_1}{\tau})}

    \inferrule[S-AssignArrT]{\heap(\umode,n)=\evalue{n_a}{\tau_a}\\ 0 \in [n_l,n_h) 
               \\ \emptyset;\heap ; \emptyset \vdash_{\umode}\evalue{n_a}{\tau}}
      {(\varphi,\heap,\eassign{\evalue{n}{\tallarrayptr{n_l}{n_h}{\tau}{\tmode}}}{\evalue{n_1}{\tau_1}}) \longrightarrow (\varphi,\heapup{\umode}{n}{\evalue{n_1}{\tau_a}},\evalue{n_1}{\tau})}

    \inferrule[S-AssignArrBound]{0 \not\in [n_l,n_h) }
      {(\varphi,\heap,\eassign{\evalue{n}{\tallarrayptr{n_l}{n_h}{\tau}{\cmode}}}{\evalue{n_1}{\tau_1}}) \longrightarrow (\varphi,\heap,\ebounds)}

    \inferrule[S-AssignC]{\heap(\cmode,n)=\evalue{n_a}{\tau_a} }
      {(\varphi,\heap,\eassign{\evalue{n}{\tptr{\tau}{\cmode}}}{\evalue{n_1}{\tau_1}}) \longrightarrow (\varphi,\heap[n \mapsto \evalue{n_1}{\tau}],\evalue{n_1}{\tau})}

  \inferrule[S-Malloc]{\varphi(\omega)=\omega_a \\ \mathtt{alloc}(\heap,\xi,\omega_a)=(n,\heap')}
   { (\varphi,\heap,\emalloc{\xi}{\omega}) \longrightarrow (\varphi,\heap',\evalue{n}{\tptr{\omega_a}{\xi}})}

  \inferrule[S-MallocBound]{\varphi(\omega)=\tallarray{n_l}{n_h}{\tau}\\ (n_l \neq 0 \vee n_h \le 0)}
    { (\varphi,\heap,\emalloc{\omega}) \longrightarrow (\varphi,\heap',\ebounds)}

  \inferrule[S-IfNTNotC]{\varphi(x)=\evalue{n}{\tntarrayptr{n_l}{n_h}{\tau}{\cmode}} \\ \heap(\cmode,n)\neq 0\\ 0 < n_h}
             {(\varphi,\heap,\eif{\estar{x}}{e_1}{e_2}) \longrightarrow (\varphi,\heap,e_1)}

    \inferrule[S-IfT]{n \neq 0 }
    {(\varphi,\heap,\eif{\evalue{n}{\tau}}{e_1}{e_2}) \longrightarrow (\varphi,\heap,e_1)}

    \inferrule[S-IfF]{}
    {(\varphi,\heap,\eif{\evalue{0}{\tau}}{e_1}{e_2}) \longrightarrow (\varphi,\heap,e_2)}

    \inferrule[S-Add]{n = n_1 + n_2}
    {(\varphi,\heap,\evalue{n_1}{\tint} \plus \evalue{n_2}{\tint}) \longrightarrow (\varphi,\heap, n)}

    \inferrule[S-AddArr]{n = n_1 + n_2\\ n_l' = n_l - n_2 \\ n_h' = n_h - n_2}
    {(\varphi,\heap,\evalue{n_1}{\tallarrayptr{n_l}{n_h}{\tau}{\xi}} \plus \evalue{n_2}{\tint}) \longrightarrow (\varphi,\heap, \evalue{n}{\tallarrayptr{n_l'}{n_h'}{\tau}{\xi}})}

n    \inferrule[S-AddArrNull]{}
    {(\varphi,\heap,\evalue{0}{\tallarrayptr{n_l}{n_h}{\tau}{\xi}} \plus \evalue{n_2}{\tint}) \longrightarrow (\varphi,\heap, \enull)}

\end{mathpar}

}
\caption{Remaining \lang Semantics Rules (extends Fig.~\ref{fig:type-system-1})}
\label{fig:rem-semantics}
\end{figure*}

A checked pointer checks validity in type step as rule \textsc{T-ConstC},
while a tainted/unchecked pointer does not check for such during the type checking.
Tainted pointers are verified through the validity check in dynamic execution as we mentioned above.

If the literal's type is an integer, an unchecked pointer, or a null
pointer, it is well typed, as shown by the top three rules in
Fig.~\ref{fig:const-type}. However, if it is a checked pointer
$\tptr{\omega}{\cmode}$, we need to ensure that what it points to in
the heap is of the appropriate pointed-to type ($\omega$), and also
recursively ensure that any literal pointers reachable this way are
also well-typed. This is captured by the bottom rule in the figure,
which states that for every location $n+i$ in the pointers' range
%
$[n, n+\size(\omega))$, where $\size$ yields the size of its argument,
  then the value at the location $\heap(n+i)$ is also well-typed.
  However, as heap snapshots can contain cyclic structures (which
  would lead to infinite typing deriviations), we use a scope $\sigma$
  to assume that the original pointer is well-typed when checking the
  types of what it points to. The middle rule then accesses the scope
  to tie the knot and keep the derivation finite, just like in
  \citet{ruef18checkedc-incr}.

\myparagraph{Let Bindings}
%
Rules \textsc{T-Let} and \textsc{T-LetInt} type a $\elettext$ expression, which also admits
type dependency. 
In particular, the result of evaluating a $\elettext$
may have a type that refers to one of its bound variables (e.g., if
the result is a checked pointer with a variable-defined bound). If so, we must substitute away this variable once it goes out of scope (\textsc{T-LetInt}). 
Note that we restrict the expression $e_1$ to syntactically match the
structure of a Bounds expression $b$ (see Fig.~\ref{fig:checkc-syn}).
Rule \textsc{T-RetInt} types a $\erettext$ expression when $x$ is of type $\tint$.
$\erettext$ does not appear in source programs but is introduced by the semantics when
evaluating a let binding (rule \textsc{S-Let} in
Fig.~\ref{fig:rem-semantics}). 

\subsection{Other Semantic Rules}\label{sec:rem-semantics}

Fig.~\ref{fig:rem-semantics} shows the remaining semantic rules for
$\lang$. We explain a selected few rules in this subsection.
% other few low-level semantic rules for variable and dereference and $\emalloctext$ operations in \checkedc. Other operations are defined in the same manner.

\myparagraph{Checked and Tainted Pointer Operations}
Rule \textsc{S-Var} loads the value for $x$ in stack $\varphi$.
Rules \textsc{S-DefArrayC} and \textsc{S-DefArrayT} dereference an $\cmode$ and $\tmode$ array pointer, respectively.
In \lang, the difference between array and NT-array dereference is that the range of $0$ is at $[n_l,n_h)$ not $[n_l,n_h]$, meaning that one cannot dereference the upper-bound position in an array.
Rules \textsc{DefArrayBound} and \textsc{DefNTArrayBound} describe an error case for a dereference operation.
If we are dereferencing an array/NT-array pointer and the mode is $\cmode$ (or $\tmode$), $0$ must be in the range from $n_l$ to $n_h$ (meaning that the dereference is in-bound); if not, the system results in a $\ebounds$ error. Obviously, the dereference of an array/NT-array pointer also experiences a $\enull$ state transition if $n\le 0$.

Rules \textsc{S-Malloc} and \textsc{S-MallocBound} describe the $\emalloctext$ semantics. Given a valid type $\omega_a$ that contains no free variables, $\mathtt{alloc}$ function returns an address pointing at the first position of an allocated space whose size is equal to the size of $\omega_a$ for a specific mode $\xi$, 
and a new heap snapshot $\heap'$ that marks the allocated space for the new allocation. 
If $\xi$ is $\cmode$, the new $\heap$ allocation is in the $\cmode$ region, while $\tmode$ and $\umode$ mode allocation creates new pieces in $\umode$ region.
The $\emalloctext$ is transitioned to the address $n$ with the type ${\tptr{\omega_a}{\xi}}$ and new updated heap. It is possible for $\emalloctext$ to transition to a $\ebounds$ error if the $\omega_a$ is an array/NT-array type $\tallarray{n_l}{n_h}{\tau}$, and either $n_l \neq 0$ or $n_h \le 0$. This can happen when the bound variable is evaluated to a bound constant that is not desired. 

Rules \textsc{S-AssignArrC} and \textsc{S-AssignArrT} are for assignment operations.
\textsc{S-AssignArrC} assigns to an array as long as 0 (the point of
dereference) is within the bounds designated by the pointer's annotation
and strictly less than the upper bound. 
Rule \textsc{S-AssignArrT} is similar to \textsc{S-AssignArrC} for tainted pointers.
Any dynamic heap access of a tainted pointer requires a \textit{verification}.
Performing such a verification equates to performing a literal type check for a pointer constant in \Cref{fig:const-type}.

\myparagraph{Let Bindings}
%
The semantics manages variable scopes using the special $\erettext$
form. \textsc{S-Let} evaluates to a configuration whose expression is
$\ret{x}{\evalue{n}{\tau}}{e})$. We keep $\varphi$ unchanged
and remember $x$ and its new value $\evalue{n}{\tau}$
in $e$'s scope that is defined by the $\erettext$ operation.
Every time when evaluation proceeds on $e$ (rule \textsc{S-RetCon}),
we install the stack value $\evalue{n}{\tau}$ for $x$ in $\varphi$ for the current scope.
After one-step evaluation is completed, 
we store $x$'s change in the result $\erettext$ operation $\ret{x}{\varphi'(x)}{e'})$,
and restore $x$'s outer score value $\varphi(x)$ in $\varphi'$. 
This procedure continues until $e'$ becomes a literal
$n\!:\!\tau$, in which case \textsc{S-RetEnd} removes the $\kw{ret}$ frame and returns
the literal. 

\subsection{Compilation Formalism}\label{appx:comp1}

% \review{
% The first reviewer said:
% \begin{itemize}
% \item I had a really hard time understanding precisely the "formalized" compilation
%   from CoreChkC to CoreC. Specifically: is CoreC intended to model LLVM-IR, or
%   is it a subset of C? Is CheckedC compiling to C or is it a new frontend like
%   `clang`? See remark about undefined behavior (UB) below which got me even more
%   confused. \mwh{CoreC is meant to model CoreChkC but with
%   annotations, and checks removed; it is not meant to model C or LLVM-IR.}
% \item Is the compilation scheme from CoreChkC to CoreC is faithful to the actual
%   compilation scheme of the checkedc-clang compiler to ...? I feel like the
%   paper is missing an actual description of what happens in the compiler to
%   allow us to connect the dots and understand how the formalization illuminates
%   the implementation. \mwh{We did not model compilation on the real
%   implementation; our purpose was to show that annotations in CoreChkC
%   do not necessitate fat pointers in an implementation; that said,
%   our formalization does show how a real implementation can be
%   carried out}
% \item The paper doesn't seem upfront about what is *shown* (theorem in Coq) and what
%   is *tested* (via PLT-Redex), and thus remains a
%   conjecture. \mwh{Updated the intro and the individual sections}
% \item  Missing discussion of why Coq vs. PLT-Redex, effort involved, any plans to
%   formally prove compilation from CoreChkC to CoreC, any hopes of integrating
%   that in the official implementation, etc. \mwh{Added note at the
%   start of section 3.}
% \end{itemize} }

% \review{IV: ghost variables in other contexts (e.g. Why3, Dafny) are used for things
%   that do not exist at run-time, but this doesn't seem to be the case here.}
% \yiyun{Agreed: We changed the name to ``shadow variables'' to avoid confusion}
% \review{Do your bug report and github links break anonymity?}


% \review{From reviewer C:
% \begin{itemize}
% \item I'd like to have seen a bit more motivation for using PLT redex:
%   what aspects made the use of this tool preferable to formulating the
%   compilation in Coq and using Quickchick to do random testing of the
%   simulation result.
%   \mwh{Added some text to the end of III.A and IV.C}
% \yiyun{Some reasons I can think of:\begin{itemize}
%     \item Redex is highly optimized for specifying judgments that are algorithmic. By writing down a typing relation, we can immediately obtain a typechecker
%     that is executable. Same applies for the small-step evaluation relation. Translating the relations into functions in Coq is definitely doable but time-consuming, especially since compilation is embedded as part of the typing rules. It is also hard to see whether the function we define really corresponds to the relation unless we formally prove it. This issue is particularly relevant at the early stage of the development when the compilation rules were buggy and the simulation property was violated often as we added new generator cases.
%     In Redex, we don't have any formal guarantee either, but at least we can more easily see the correspondence because Redex is able to convert the relation into an executable version so we can specify the relation literally. This feature of Redex helped us speed up our development significantly when our compilation rules were constantly changing.
% \end{itemize} }
% \item there's a funny change of line spacing in column 2 of page 6, about 2/3 down.
% \end{itemize}
% }

The main subtlety of compiling \checkedc to Clang/LLVM is to capture the annotations on pointer literals
that track array bound information, which is used in premises
of rules like \textsc{S-DefArrayC} and
  \textsc{S-AssignArrC} to prevent spatial safety violations.
The \checkedc compiler \cite{li22checkedc} inserted additional pointer checks 
for verifying pointers are not null and the bounds are within their limits.
The latter is done by introducing additional shadow variables for storing (NT-)array pointer bound information.

\begin{figure}[t!]
{\small
\hspace*{-0.5em}
\begin{tabular}{|c|c|c|c|}
\hline
& \cmode & \tmode & \umode \\
\hline
& \textsc{CBox} / \textsc{Core} & \textsc{CBox} / \textsc{Core} & \textsc{CBox} / \textsc{Core} \\
\hline
\cmode & $\estar{x}$ / $\getstar{\cmode}{x}$ 
 & $\texttt{sand\_get}(x)$ / $\getstar{\umode}{x}$ &  $\times$ \\
\hline
\umode & $\times$
 & $\estar{x}$ / $\getstar{\umode}{x}$ &  $\estar{x}$ / $\getstar{\umode}{x}$ \\
\hline
\end{tabular}

}
\caption{Compiled Targets for Dereference}
\label{fig:flagtable}
\end{figure}

In \systemname, context and pointer modes determine the particular heap/function store that a pointer points to,
i.e., $\cmode$ pointers point to checked regions, while $\tmode$ and $\umode$ pointers point to unchecked regions. 
Unchecked regions are associated with a sandbox mechanism that permits exception handling of potential memory failures.
In the compiled LLVM code, pointer access operations have different syntaxes when the modes are different.
\Cref{fig:flagtable} lists the different compiled syntaxes of a deference operation ($\estar{x}$) for the compiler implementation (\textsc{CBox}, stands for \systemname) and formalism (\textsc{Core}, stands for \lang). The columns represent different pointer modes and the rows represent context modes.
For example, when we have a $\tmode$-mode pointer in a $\cmode$-mode region, we compile a deference operation to the sandbox pointer access function ($\texttt{sand\_get}(x)$) accessing the data in the \systemname implementation. In \lang, we create a new deference data-structure on top of the existing $\estar{x}$ operation (in LLVM): $\getstar{m}{x}$. If the mode is $\cmode$, it accesses the checked heap/function store; otherwise, it accesses the unchecked one.

This section shows how \lang deals with pointer modes, mode switching and function pointer compilations, 
with no loss of expressiveness
as the \checkedc contains the erase of annotations in \cite{li22checkedc}.
For the compiler formalism, 
we present a compilation algorithm that converts from
\lang to \elang, an untyped language without metadata
annotations, which represents an intermediate layer we build on LLVM for simplifying compilation. 
In \elang, the syntax for deference, assignment, malloc, function calls are: $\getstar{m}{e}$, $\elassign{m}{e}{e}$, 
$\emalloc{m}{\omega}$, and $\ecall{e}{\overline{e}}$.
The algorithm sheds light on how compilation can be implemented in the real \systemname
  compiler, while eschewing many vital details (\elang has many differences with LLVM IR).

Compilation is defined by extending \lang's
typing judgment as follows:
\[\Gamma;\Theta;\rho \vdash_m e \gg \dot e:\tau\]
There is now a \elang output $\dot e$ and an input $\rho$, which maps
each (NT-)array pointer variable to its mode and
each variable \code{p} to a pair of \emph{shadow
  variables} that keep \code{p}'s up-to-date upper and lower bounds. 
These may differ from the bounds in \code{p}'s type due to bounds
widening.\footnote{Since lower bounds are never widened, the
  lower-bound shadow variable is unnecessary; we include it for uniformity.} 

% When $\Gamma$,$\Theta$ and $\rho$ are all empty, we write $e \gg \dot e$ rather than the
% complete judgment, implicitly assuming that $e$ is a well-typed and closed
% term.

We formalize rules for this judgment in PLT Redex~\cite{pltredex},
following and extending our Coq development for \lang. To give
confidence that compilation is correct, we use Redex's property-based
random testing support to show that compiled-to $\dot e $ simulates
$e$, for all $e$.

% We developed a \checkedc compiler to compile a \checkedc program to a C program.
% \mwh{We formalized compilation from CoreChkC to a version of CoreChkC
%   but with the metadata removed, right? This is not a Checked C
%   compiler. You go on to see stuff about CompCert, CLight,
%   etc. This is confusing. We should be talking about how this relates
%   to what was just presented. Pick definitive names for things. }
% Given a \checkedc program $e$, we build a compilation process ($\gg$), such that $e \gg \dot e$, where $\dot e$ is the corresponding C program for $e$ in A-normal form (ANF). 
% The compilation process ($\gg$) relies on the type checking step $\Gamma;\Theta\vdash_m e:\tau$. 
% Especially, it relies on $\Gamma$ to provide the type information for variables in $e$. 
% We utilize CompCert/CLight syntax and semantics \cite{Leroy:2009:FVC:1666192.1666216,Blazy2009} as our translation target language.
%  Besides, we defined two data structures in the CompCert format for representing $\enull$ and $\ebounds$ states.
% We write $\xrightarrow{c}$ for the semantics of CLight. \mwh{Of our
%   target language, which I presume does not have all the features that
%   CLight has. Should mention up front that we did all of this in PLT Redex.}

\myparagraph{Approach}
%
Here, we explain the rules for compilation by
examples, using a C-like syntax; the complete rules are given in
\cite{checkedc-tech-report}.
Each rule performs up to three tasks: (a) conversion of $e$ to
A-normal form; (b) insertion of dynamic checks and bound widening expressions; 
and (c) generate right pointer accessing expressions based on modes.
%
A-normal form conversion is straightforward: compound expressions are managed by storing results of subexpressions into temporary variables,
as in the following example.

{\vspace*{-0.5em}
{\small
\begin{center}
$
\begin{array}{l}
$\code{let y=(x+1)+(6+1)}$
\;
\begin{frame}

\tikz\draw[-Latex,line width=2pt,color=orange] (0,0) -- (1,0);

\end{frame}
\;
\begin{array}{l}
$\code{let a=x+1;}$\\
$\code{let b=6+1;}$\\
$\code{let y=a+b}$\\
\end{array}
\end{array}
$
\end{center}
}
}

This simplifies the management of effects from subexpressions. The
next two steps of compilation are more interesting.
We state them based on different \lang operations.

\begin{figure}[t!]
  \begin{small}
\begin{lstlisting}[mathescape,xleftmargin=4 mm]
int deref_array(n : int,
     p :  $\color{green!40!black}\tntarrayptr{0}{n}{\tint}{\cmode}$,
     q : $\color{green!40!black}\tntarrayptr{0}{n}{\tint}{\tmode}$) {
  /* $\color{purple!40!black}\rho$(p) = p_lo,p_hi,p_m */
  /* $\color{purple!40!black}\rho$(q) = q_lo,q_hi,q_m */
    * p;
    * q = 1;
}
...
/* p0 : $\color{purple!40!black}\tntarrayptr{0}{5}{\tint}{\cmode}$ */
/* q0 : $\color{purple!40!black}\tntarrayptr{0}{5}{\tint}{\tmode}$ */
deref_array(5, p0, q0);
    \end{lstlisting}
\begin{frame}

\tikz\draw[-Latex,line width=2pt,color=orange] (0,0) -- (1,0);

\end{frame}
\begin{lstlisting}[mathescape,xleftmargin=4 mm]
deref_array(int n, int* p, int * q) {
  //m is the current context mode
  let p_lo = 0; let p_hi = n; 
  let q_lo = 0; let q_hi = n; 
  /* runtime checks */
  assert(p_lo <= 0 && 0 <= p_hi);
  assert(p != 0);
  *(mode(p) $\wedge$ m,p);
  verify(q, not_null(m, q_lo, q_hi) 
             && q_lo <= 0 && 0 <= q_hi);
  *(mode(q) $\wedge$ m,q)=1;
}
...
deref_array(5, p0, q0);
    \end{lstlisting}
\end{small}
    \caption{Compilation Example for Dependent Functions}
\label{fig:compilationexample1}
\end{figure}

% \review{Fig 9: if this is actual C code, then your null-check at line 6 will be
%   eliminated by the compiler. At line 3, you performed a pointer addition, which
%   is only defined when `p` is non-null. So, either `p` is non-null, and the
%   NULL-check can be eliminated; or, `p` is NULL, but line 3 was undefined
%   behavior, meaning the compiler is allowed to do anything, notably eliminate
%   the NULL-check. This is where I am super confused, and either:
%   - CoreC is not really the C language, and has different semantics...? but is
%     this well-defined in the context of LLVM?
%   - there is a problem that was not caught by the PLT-Redex-based testing.
% \yiyun{We have clarified at the start of IV that CoreC is an untyped
%   variant of CoreChkC, and does not aim to represent C per se, or LLVM
%   IR. We aimed to avoid confusion by rewriting the examples in a way that is more closely
%     related to the syntax presented in Fig 3. Pointer arithmetic between 0 and a non-zero
%     index is always valid because CoreC there is technically only
%     integer arithmetic.}}

\myparagraph{Pointer Accesses and Modes}
%
In every declaration of a pointer,
if the poniter is an (NT-)array,
we first allocate two \emph{shadow variables}
to track the lower and upper bounds which are potentially changed for pointer arithmetic and NT-array bound widening.
Each $\cmode$-mode NT-array pointer variable is associated with its type information in a store.
Additionally, we place bounds and null-pointer checks, such as the line 6 and 7 in \Cref{fig:compilationexample1}.
In addition, in the formalism, before every use of a tainted pointer (\Cref{fig:compilationexample1} line 9 and 10), 
there is an inserted verification step similar to \Cref{fig:const-type},
which checks if a pointer is well defined in the heap (\code{not_null}) and the spatial safety.
Predicate \code{not_null} checks that every element in the pointer's range (\code{p_lo} and \code{p_hi}) is well defined in the heap.  
The modes in compiled deference (\code{*(mode(p) }$\wedge$\code{ m,p)})
 and assignment (\code{*(mode(q) }$\wedge$\code{ m,q)=1}) operations 
are computed based on the meet 
operation ($\wedge$) of the pointer mode (e.g. \code{mode(p)}) and the current context mode (\code{m}).

\myparagraph{Checked and Unchecked Blocks}
%
In the \systemname implementation,
$\euncheckedtext$ and $\echeckedtext$ blocks 
are compiled as context switching functions provided by sandbox.
$\eunchecked{\overline{x}}{e}$ is compiled to 
$\texttt{sandbox\_call}(\overline{x},e)$, where we call the sandbox 
to execute expression $e$ with the arguments $\overline{x}$.
$\echecked{\overline{x}}{e}$ is compiled to 
$\texttt{callback}(\overline{x},e)$, where we perform 
a \texttt{callback} to a checked block code $e$ inside a sandbox.
In \systemname, we adopt an aggressive execution scheme that
directly learns pointer addresses from compiled assembly to make the $\code{_Callback}$ happen.
In the formalism, we rely on the type system to 
guarantee the context switching without creating the extra function calls for simplicity.

%Fig.~\ref{fig:compilationexample} shows how an invocation of
%\code{strlen} on a null-terminated string is compiled into C
%code. Each dereference of a checked pointer requires a null check
%(See \textsc{S-DefNull} in Fig.~\ref{fig:semantics}), which the
%compiler makes explicit: Line~$3$ of the generated code has the null
%check on pointer \code{p} due to the \code{strlen},
%  and a similar check happens
%  at line~$8$ due to the pointer arithmetic on \code{p}.
%Dereferences also require bounds checks: line~$2$ checks \code{p} is
%in bounds before computing \code{strlen(p)}, while line~$10$ does
%likewise before computing \code{*(p+1)}.

\myparagraph{Function Pointers and Calls}
%
Function pointers are managed similarly to normal pointers,
but we insert checks to check if the pointer address is not null in 
the function store instead of heap, and whether or not the type is correctly represented, 
for both $\cmode$ and $\tmode$ mode pointers 
\footnote{$\cmode$-mode pointers are checked once in the beginning and $\tmode$-mode pointers are checked every time when use}.
The compilation of function calls (compiling to $\elcall{m}{e}{\overline{e}}$) 
is similar to the manipulation of pointer access operations in \Cref{fig:flagtable}.
For compiling dependent function calls,
\Cref{fig:compilationexample1} provides a hint.
Notice that the bounds for the array pointer \code{p} are not passed as
arguments. Instead, they are initialized according to \code{p}'s
type---see line~4 of the original \lang program at the top of the figure.
Line~$3$ of the generated code
sets the lower bound  to \code{0} and the
upper bound to \code{n}.

\subsection{Constraints and Metatheory}
\label{sec:meta}

Here, we first show some Well-formedness and consistency definitions that are required in \Cref{sec:theorem}, and then show the simulation theorem for the \lang compiler.
Type soundness relies on several \emph{well-formedness}:

\begin{definition}[Type Environment Well-formedness]\label{type-wellformed}
A type environment $\Gamma$ is well-formed if every variable mentioned as type bounds in $\Gamma$ are bounded by $\tint$ typed variables in $\Gamma$.
\end{definition}

\begin{definition}[Heap Well-formedness]
For every $m$, A heap $\heap$ is well-formed if (i) $\heap(m,0)$ is undefined, and
(ii) for all $\evalue{n}{\tau}$ in the range of $\heap(m)$, type $\tau$
contains no free variables. 
\end{definition}

\begin{definition}[Stack Well-formedness]
A stack snapshot $\varphi$ is well-formed if
for all $\evalue{n}{\tau}$ in the range of $\varphi$, type $\tau$
contains no free variables. 
\end{definition}

We also need to introduce a notion of
\emph{consistency}, relating heap environments before and after a
reduction step, and type environments, predicate sets, and stack
snapshots together.

\begin{definition}[Stack Consistency]
A type environment $\Gamma$, variable predicate set $\Theta$, and
stack snapshot $\varphi$ are consistent---written $\Gamma;\Theta\vdash
\varphi$---if for every variable $x$, $\Theta(x)$ is defined implies
$\Gamma(x) = \tau$ for some $\tau$ and 
$\varphi(x) =\evalue{n}{\tau'}$ for some $n,\tau'$ where $\tau' \sqsubseteq_{\Theta} \tau$. 
\end{definition}

\begin{definition}[Checked Stack-Heap Consistency]
A stack snapshot $\varphi$ is consistent with heap $\heap$---written $\heap \vdash \varphi$---if
for every variable $x$, $\varphi(x)= \evalue{n}{\tau}$ with $\mode(\tau)=\cmode$ implies $\emptyset;\heap(\cmode);\emptyset \vdash_{\cmode} n:\tau$.
\end{definition}

\begin{definition}[Checked Heap-Heap Consistency]
A heap $\heap'$ is consistent with $\heap$---written $\heap \triangleright \heap'$---if
for every constant $n$, $\emptyset;\heap;\emptyset \vdash_{\cmode} n:\tau$ implies $\emptyset;\heap';\emptyset \vdash_{\cmode} n:\tau$.
\end{definition}

We formalize both the compilation procedure and the simulation
theorem in the PLT Redex model we developed for \lang (see Sec.~\ref{sec:syntax}),
and then attempt to falsify it via Redex's support for random
testing. Redex allows us
  to specify compilation as logical rules (an extension
  of typing), but then execute it algorithmically to
  automatically test whether simulation holds. This process revealed
  several bugs in compilation and the theorem statement.
%
  % us gain confidence that our original pen and paper proof of
  % simulation remains true with the addition of variable bounds. }
We ultimately plan to prove simulation in the Coq model.

%Turning to the simulation theorem: We first introduce notation
%used to specify the theorem.
We use the notation $\gg$ to
indicate the \emph{erasure} of stack and heap---the rhs is the same as
the lhs but with type annotations removed:
\begin{equation*}
  \begin{split}
    \heap  \gg & \dot \heap \\
    \varphi \gg & \dot \varphi
  \end{split}
\end{equation*}
In addition, when $\Gamma;\emptyset\vdash
\varphi$ and $\varphi$ is well-formed, we write $(\varphi,\heap,e) \gg_m (\dot \varphi, \dot \heap,
\dot e)$ to denote $\varphi \gg \dot \varphi$, $\heap \gg \dot \heap$
and $\Gamma;\Theta;\emptyset \vdash_m e \gg \dot e : \tau$ for some $\tau$ respectively. $\Gamma$ is omitted from the notation since the well-formedness of $\varphi$ and its consistency with respect to $\Gamma$ imply that $e$ must be closed under $\varphi$, allowing us to recover $\Gamma$ from $\varphi$.
Finally, we use $\xrightarrow{\cdot}^*$ to denote the transitive closure of the
reduction relation of $\elang$. Unlike the $\lang$, the semantics of
$\elang$ does not distinguish checked and unchecked regions.



\begin{figure}[t]
{\small
\[
\begin{array}{c}
\begin{tikzpicture}[
            > = stealth, % arrow head style
            shorten > = 1pt, % don't touch arrow head to node
            auto,
            node distance = 3cm
        ]

\begin{scope}[every node/.style={draw}]
    \node (A) at (0,1.5) {$\varphi_0,\heap_0, e_0$};
    \node (B) at (4,1.5) {$\varphi_1, \heap_1 ,e_1$};
    \node (C) at (0,0) {$\dot \varphi_0, \dot \heap_0 ,\dot e_0$};
    \node (D) at (4,0) {$\dot \varphi_1, \dot \heap_1, \dot e_1$};
    \node (E) at (2,-1.5) {$\dot \varphi,\dot \heap ,\dot e$};
\end{scope}
\begin{scope}[every edge/.style={draw=black}]

    \path [->] (A) edge node {$\longrightarrow_{\cmode}$} (B);
    \path [<->] (A) edge node {$\gg$} (C);
    \path [<->] (B) edge node {$\gg$} (D);
    \path [dashed,<->] (C) edge node {$\sim$} (D);
    \path [dashed,->] (C) edge node {$\xrightarrow{\cdot}^*$} (E);
    \path [dashed,->] (D) edge node[above] {$\xrightarrow{\cdot}^*$} (E);
\end{scope}

\end{tikzpicture}
\end{array}
\]
}
\caption{Simulation between \lang and \elang }
\label{fig:checkedc-simulation-ref}
\end{figure}


Fig.~\ref{fig:checkedc-simulation-ref} gives an overview of 
the simulation theorem.\footnote{We ellide the  possibility of $\dot e_1$ evaluating to $\ebounds$ or $\enull$ in the diagram for readability.} The simulation theorem is specified in a way
that is similar to the one by~\citet{merigoux2021catala}.

An ordinary simulation property would
replace the middle and bottom parts of the figure with the
following: \[(\dot \varphi_0, \dot \heap_0, \dot e_0) 
  \xrightarrow{\cdot}^* (\dot \varphi_1, \dot \heap_1, \dot e_1)\]
Instead, we relate two erased configurations using the relation $\sim$,
which only requires that the two configurations will eventually reduce
to the same state.


% The two theorems are translation preservation and simulation. We donate $\xrightarrow{c}$ as the transition semantics of CLight.
\begin{thm}[Simulation ($\sim$)]\label{simulation-thm}
For \lang expressions $e_0$, stacks $\varphi_0$, $\varphi_1$, and heap snapshots $\heap_0$, $\heap_1$, 
if $\heap_0 \vdash \varphi_0$, $(\varphi_0,\heap_0,e_0)\gg_c (\dot \varphi_0,\dot \heap_0, \dot e_0)$,
and if there exists some $r_1$ such that $(\varphi_0, \heap_0, e_0)
\rightarrow_c (\varphi_1, \heap_1, r_1)$, then the following facts hold:

\begin{itemize}

\item if there exists $e_1$ such that $r=e_1$ and $(\varphi_1, \heap_1, e_1) \gg (\dot \varphi_1, \dot \heap_1, \dot e_1)$, then there exists some $\dot \varphi$,$\dot \heap$, $\dot e$, such that
$(\dot \varphi_0, \dot \heap_0,\dot e_0) \xrightarrow{\cdot}^* (\dot
\varphi,\dot \heap,\dot e)$ and $(\dot
\varphi_1,\dot \heap_1,\dot e_1) \xrightarrow{\cdot}^* (\dot \varphi,
\dot \heap,\dot e)$.

\item if $r_1 = \ebounds$ or $\enull$, then we have $(\dot \varphi_0, \dot \heap_0,\dot e_0) \xrightarrow{\cdot}^* (\dot
\dot \varphi_1,\dot \heap_1, r_1)$ where $\varphi_1 \gg \dot
\varphi_1$, $\heap_1 \gg \dot \heap_1$.

\end{itemize}
\end{thm}


% when $r_1 = e_1$ for
% some $e_1$ and
% $(\varphi_1, \heap_1, e_1) \gg (\dot \varphi_1, \dot \heap_1, \dot e_1)$, then
% there exists some $\dot \varphi$,$\dot \heap$, $\dot e$, such that
% $(\dot \varphi_0, \dot \heap_0,\dot e_0) \xrightarrow{\cdot}^* (\dot
% \varphi,\dot \heap,\dot e)$ and $(\dot
% \varphi_1,\dot \heap_1,\dot e_1) \xrightarrow{\cdot}^* (\dot \varphi,
% \dot \heap,\dot e)$. When $r_1 = \ebounds$ or $\enull$, we have $(\dot \varphi_0, \dot \heap_0,\dot e_0) \xrightarrow{\cdot}^* (\dot
% \dot \varphi_1,\dot \heap_1, r_1)$ where $\varphi_1 \gg \dot
% \varphi_1$, $\heap_1 \gg \dot \heap_1$.

% \begin{thm}[Simulation ($\sim$)]\label{simulation-thm}
% For \lang expressions $e_0$, stacks $\varphi_0$, $\varphi_1$, and heap snapshots $\heap_0$, $\heap_1$, 
% if $\emptyset;\emptyset;\emptyset \vdash_\cmode e_0 \gg \dot e_0 :\tau_0$,
% and if there exists some $r_1$ such that $(\varphi_0, \heap_0, e_0)
% \rightarrow_\cmode (\varphi_1, \heap_1, r_1)$, when $r_1 = e_1$ for
% some $e_1$ and
% $\emptyset;\emptyset;\emptyset \vdash_\cmode e_1 \gg \dot e_1 :\tau_1$ where $\tau_1 \sqsubseteq \tau_0$
% , then
% there exists some $\dot \varphi$,$\dot \heap$, $\dot e$, such that
% $(\dot \varphi_0, \dot \heap_0,\dot e_0) \xrightarrow{\cdot}^* (\dot
% \varphi,\dot \heap,\dot e)$ and $(\dot
% \varphi_1,\dot \heap_1,\dot e_1) \xrightarrow{\cdot}^* (\dot \varphi,
% \dot \heap,\dot e)$. When $r_1 = \ebounds$ or $\enull$, we have $(\dot \varphi_0, \dot \heap_0,\dot e_0) \xrightarrow{\cdot}^* (\dot
% \dot \varphi_1,\dot \heap_1, r_1)$ where $\varphi_1 \gg \dot
% \varphi_1$, $\heap_1 \gg \dot \heap_1$.
% \end{thm}

As our random generator never generates
$\euncheckedtext$ expressions (whose behavior could be undefined), we can only test a the simulation theorem 
as it relates to checked code. This limitation makes it
unnecessary to state the other direction of the simulation theorem
where $e_0$ is stuck, because Theorem~\ref{thm:progress} guarantees
that $e_0$ will never enter a stuck state if it is well-typed in
checked mode.

The current version of the Redex model has been tested against $21500$
expressions with depth less than $12$. Each expression can
reduce multiple steps, and we test simulation between every two
adjacent steps to cover a wider range of programs, particularly the
ones that have a non-empty heap.

\subsection{Additional Program evaluations}\label{appx:add-prog-eval}

Here, we provide the description of additional program evaluations.

\myparagraph{parsons}
Parsons is annotated comprehensively in two variants parsons\_wasm and parsons\_tainted. parsons\_wasm has most of its input parsing functions moved into the sandbox, whilst having all its pointers marked as tainted. These sandboxed functions interact with the checked region by making indirect calls through RLBOX's callback mechanism. However, with parsons\_tainted, we do not move any of the functions to the sandbox but still mark all the pointers as tainted. The test suite itself consists of 328 tests comprehensively testing the JSON parser's functionality. Benchmarks for both of these forks are recorded using the mean difference between the \systemname and generic-C/checked-C variants when executing 10 consecutive iterations of the test suite. parsons\_wasm expectedly shows 200/266\% runtime overhead when evaluated against checked-c and generic-c respectively due to the performance limitation of WebAssembly. However, evaluating parsons\_tainted against checked-c shows \systemname to be faster because \systemname by itself performs lighter run-time-instrumentation on tainted pointers as compared to the run-time bounds checking performed on checked pointers by checked-c. Furthermore, we only see an average peak memory of 9.5 KiB as compared to the anticipated 82 KiB overhead as Valgrind does not consider the WASM Shadow memory allocated to the tainted pointers.

\myparagraph{LibPNG}
\systemname changes for libPNG is narrow in scope and begins with the encapsulation CVE-2018-144550 and a buffer overflow in compare\_read(). However, we also annotate sections of Lib-png that involve reading, writing, and image processing (interlace, intrapixel, etc) on user-input image data as tainted. That is, rows of image bytes are read into tainted pointers and the taintedness for the row\_bytes is propagated throughout the program. All our changes extend to the png2pnm and pnm2png executables. To evaluate png2pnm, we take the mean of 10 iterations of a test script that runs png2pnm on 52 png files located within the libpng's pngsuite. To test pnm2png, we take the mean of 10 iterations of pnm2png in converting a 52MB 5184x3456 pixels large pnm image file to png. Valgrind's reported lower Heap space consumption for \systemname converted code is due to the discounted space consumed on the heap by the Sandbox's shadow memory. Consequently, when evaluating pnm2png, \systemname's heap consumption was 52 MB lower as the entire image was loaded onto the shadow memory.  

\myparagraph{MicroHTTPD}
MicroHTTPD demonstrates the practical difficulties in converting a program to \systemname. Our conversion for this program was aimed at sandboxing memory vulnerabilities CVE-2021-3466 and CVE-2013-7039. CVE-2021-3466 is described as a vulnerability from a buffer overflow that occurs in an unguarded "memcpy" which copies data into a structure pointer (struct MHD\_PostProcessor pp) which is type-casted to a char buffer (char *kbuf = (char *) \&pp[1]). Our changes would require making the "memcpy" safe by marking this pointer as tainted. However, this would either require marshaling the data pointed by this structure (and its sub-structure pointer members) pointer or would require marking every reference to this structure pointer as tainted, which in turn requires every pointer member of this structure to be tainted. Marshalling data between structure pointers is not easy and demands substantial marshaling code due to the spatial non-linearity of its pointer members unlike a char*. This did not align with our conversion goals which were aimed at making minimal changes. Consequently, the above CVE stands un-handled by \systemname.  Our changes for CVE-2013-7039 involve marking the user input data arguments of this function as tainted pointers and in the interests of seeking minimal conversion changes, we do not propagate the tainted-ness on these functions. Following up on the chronological impossibility of sandboxing bugs before they are discovered and the general programmer intuiting, we moved many of the core internal functions (like MHD\_str\_pct\_decode\_strict\_() and MHD\_http\_unescape()) into the sandbox. 

\myparagraph{Tiny-bignum}
Due to its small size and simplicity, \systemname changes for Tiny-bignum was chosen to be comprehensive. Furthermore, bignum\_to\_string() was moved to the sandbox due to a memory-unsafe use of sprintf(). Given that significant \systemname conversion efforts is attributed to understanding the source-code and finding the precise extent to which we choose to propagate the taintedness or to stop and give up to marshalling the data between regions, Tiny-bignum only required 4 hours. The evaluation was performed on Tiny-bignum's test suite consisting of 4 Test cases, each of which test the functionality to scale on big numbers subject to all of the supported unary and binary operations.  






\fi

%\newpage
%\input{remaining}
\end{document}
